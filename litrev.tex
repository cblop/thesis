\chapter{Literature Review}
\label{cha:literature-review}
This research covers a large number of fields of study, therefore an extensive
literature review covering these fields is needed. This section starts with a
look at the field of \emph{narratology}, or narrative theory, to gain some
insights into the themes and components that make up stories. Looking at
different formalisms that have been created for narrative and which themes and
motifs recur in stories should better inform the creation of techniques with
which to generate stories.

Additionally, computers have the potential to introduce a new type of storytelling. Though computer games offer increasingly immersive interactive worlds, the form of narrative they use remains much the same: linear. A player may be able to interact with elements of a game, but the story itself remains unchanged.
Chris Crawford refers to a new possible form of narrative as ``Interactive Storytelling'' \citep{crawford2012chris}. Though researchers and game designers have made great strides towards realising this new artform, little has been produced to capture the public imagination.

% why are you looking at classic narratology?
In order to better inform any implementation of interactive narrative, this
review begins with an examination of the field of classic narrative theory. Following from this is a look at the emerging research in interactive and generative narrative and their implementations.
As part of the examination of implementations of interactive narrative, this review especially focuses on agent-based systems. The section concludes with an overview of emotional models that can be used to model distinct characters using agents.

\section{Narratology}
\label{sec:narratology}
Narratology is a deep field with many sub-fields. This review examines the parts of it that might best inform the modelling of narrative by computers, as well as the construction of interactive narrative.

The first part of the overview of narratology examines research into categorising different types of narrative, both traditional and experimental.
This draws from classic narratological texts, as well as work done on ``cybertext'' and experimental narrative in the interactive age. This examination of recent research into non-linear narratives better informs how to better construct interactive narratives for games or simulations.

Structuralist formalisms of narrative attempt to explain how stories work by dividing them into commonly occuring themes and motifs. This is a natural fit to the modelling of narratives by computer, especially if using an ontology. This overview of narratology starts with structuralism for this reason.

After the overview of the structuralists follows a section on the use of formal logic for narrative modelling. The section ends with descriptions of other types of story components, taxonomies and ontologies used in the literature.

\subsection{Types of Narrative}
% Look at Cybertext, etc, and try to explain how best to divide different types
% of story
The rise of the Web in the 1990s brought with it great interest in the future of narrative in cyberspace. Aarseth's work, \emph{Cybertext} \citep{aarseth1997cybertext} describes the creation of a new form of narrative, for which he coins the term \emph{ergodic literature} (from the Greek words \emph{ergon} and \emph{hodos}, meaning `work' and `path'). In this new form of narrative, some amount of work or effort is required by the reader in order to traverse the path that the story takes.

% scriptons/discourse textons/fabula (Bal 1997)
Aarseth makes a distinction between the narrative as written by the author, and the way in which it is traversed by the reader, calling the former \emph{textons} and the latter \emph{scriptons}. In ergodic literature, the \emph{scripton} is produced by the effort that the reader goes through in interpreting the \emph{texton}. In the context of a game, it is as though the game interface is a gateway that allows access to the narrative at different times. Using classical music as a metaphor, the texton can be thought of as the \emph{score}, and the scripton the \emph{performance}.

% Make sure you actually made this assertion
In section \ref{sec:generative-and-interactive-narrative}, we assert that how generative a narrative is and its level of interactivity are two different variables in an experimental narrative. However, Aarseth identifies seven different methods of story traversal: \emph{dynamics, determinability, transiency, perspective, access, linking and user function}.

\paragraph{Dynamics} describe whether or not the content and number of scriptons changes. In a simple, static story with branching choices (such as in a \emph{Choose your own adventure} story), both the number of textons and scriptons are fixed, since all paths have been written out beforehand. A dynamic story would still have a fixed number of textons, but the scriptons would be generated as the user traverses the path of the narrative.
\paragraph{Determinability} is how deterministic the narrative is, whether or not the same interactions will result in the same scripton being produced.
\paragraph{Transiency} means to what extent scriptons are produced as time flows, or whether user interactions are required to produce them.
\paragraph{Perspective} is whether or not the user/reader plays a role as a character in the narrative.
\paragraph{Access:} if a user has access to all scriptons at any point in traversing the narrative, or whether their access is restricted.
\paragraph{Linking} means whether or not parts of the scripton are linked to other parts, and whether these links are conditional (if they rely on a user having already traversed part of the scripton).
\paragraph{User functions:} the functions the user uses to traverse the text. This could be interpretive (which is implicit in any traversal of the text), explorative (traversing the scripton according to whim) or configurative (specifying parts of the scripton in advance), for example.

% Ugh, this is all so arbitrary. Go on to describe Aarseth's PCA of these variables and explain why you don't think it's a good fit.

By performing correspondence analysis (a process similar to principle component analysis) on a diverse corpus of 23 texts ``\emph{ranging from ancient China to the Internet}'', Aarseth filters these seven variables down into two numerical axes which account for 49 percent of the variation between stories. Using these axes, he groups classic tales such as \emph{Moby Dick} and more experimental narratives such as William Gibson's \emph{Agrippa} and Michael Joyce's \emph{Afternoon}. By grouping these stories into categories, he intends to show how emerging media are enabling new types of story.

Chris Crawford's \emph{Chris Crawford on Interactive Storytelling} \citep{crawford2012chris} provides a scathing assessment of the relationship between narrative theory and computer science. A veteran of the games industry, he argues that `soft' science theories such as those of Aarseth et al are entirely removed from `hard' science, and are therefore an example of bubble intellectualism and impossible to implement. 

Crawford himself provides a useful examination of experimental narrative in computer games, defining interactivity as:

\begin{quote}
A cyclic process between two or more active agents in which each agent alternately listens, thinks, and speaks.
\end{quote}

He argues that for game narratives to be truly interactive, they must be more social. Characters in a story must be able to react with the player as though they were people in real life. In turn, the player should have some degree of freedom in the way in which they interact. Rather than presenting branching story points as choices, a better way to interact would be socially, through talking to agents in the game. This is the approach that Fa\c{c}ade takes \citep{mateas2003faccade}, which Crawford acknowledges as the most successful attempt at interactive storytelling to date. A detailed description of Fa\c{c}ade's implementation appears in section \ref{sec:modelling-agents}.

In order to determine whether Crawford's assertion that narratology research is too far removed from its practical implementations to be of use, we next provide an overview of these implementations and their underlying research. Has narrative theory research informed the creation of computer-generated or interactive narrative at all, or do they all take approaches grounded in computer science and artificial intelligence? If narrative theory has not been used, then we must ask another question: why not?


\subsection{Structuralist Formalisms of Narrative}
% Propp, etc
Attempts to organise recurring themes, roles and motifs of narrative go back at least a century. The Aarne-Thompson tale-type index \citep{aarne1987types}, first published in 1910 and later refined by Stith Thompson in 1928 and 1961, is well known amongst folklorists as a classification and analysis method for traditional folktales and myths. Aarne-Thompson's index is a taxonomy of tale themes, arranging tales into categories such as \emph{animal tales} and \emph{jokes and anecdotes}, and then sub-categories (\emph{tales of magic} and \emph{numskull [sic] stories} being two examples). This taxonomy is only two levels deep however, and only serves as a useful way to categorise individual stories or tales. In order to break down and analyse components of tales, we must dig deeper.

In \emph{Structural Anthropology}, Claude L\'{e}vi-Strauss seeks to discover why myths and legends are so similar across cultures and history \citep{levi2008structural}. He concludes that there are global laws that govern the way in which people create stories, therefore these laws can be modelled as a set of rules for describing myths.

His theory is that myths describe opposing forces which are resolved through mediation. The example he gives in \emph{Structural Anthropology} describes how Native American legends often contain `trickster' characters in the form of ravens or coyotes. As scavenging animals, these tricksters symbolically act as mediators between life and death.

Like much of early narrative theory, there is no rigorous evaluation of L\'{e}vi-Strauss' ideas, leaving them feeling a little too opinionated and arbitrary. While interesting, L\'{e}vi-Strauss' ideas bring us no closer to developing a formal model of narrative structure. For that, we must go even further back in time, and turn to Vladimir Propp.

\subsubsection{Propp's Morphology of the Folktale}
A notable narrative structuralist is Vladimir Propp, creator of \emph{The Morphology of the Folktale}~\cite{propp1968morphology}, a formalism for Russian folktales. Propp's formalism, though originally limited in scope, generalises well, and is still used by researchers to procedurally generate stories~\cite{grasbon2001morphological,gervas2005story,hartmann2005motif}. Drawing from a corpus of one hundred Russian folktales, Propp identifies thirty-one distinct \emph{story functions}, each of which is identified with a number and symbol. These functions are executed by characters following certain roles, each of which has a \emph{sphere of action} consisting of the functions that they are able to perform at any given point of the story. Stories are created by chaining story functions together, with subplots expressed as parallel chains of story functions.

In this formalism, characters have \emph{roles}, such as \emph{hero}, \emph{villain}, \emph{dispatcher}, \emph{false hero}, and more. Characters performing a certain role are able to perform a subset of \emph{story functions}, which are actions that make the narrative progress. For example, the \emph{dispatcher} might send the \emph{hero} on a quest, or the \emph{victim} may issue an \emph{interdiction} to the \emph{villain}, which is then \emph{violated}.

Propp defines a total of 31 distinct story functions, each of which is given a number and symbol in order to create a succinct way of describing entire stories. Examples of such functions are:

\begin{itemize}
  \item One of the members of a family absents himself from home: \emph{absentation}.
  \item An interdiction is addressed to the hero: \emph{interdiction}.
  \item The victim submits to deception and thereby unwittingly helps his enemy: \emph{complicity}.
  \item The villain causes harm or injury to a member of the family: \emph{villainy}.
\end{itemize}

Each of these functions can vary to some degree. For example, the \emph{villainy} function can be realised as one of 19 distinct forms of villainous deed, including \emph{the villain abducts a person}, \emph{the villain seizes the daylight}, and \emph{the villain makes a threat of cannibalism}.

These functions are enacted by characters following certain roles. Each role (or \emph{dramatis personae} in Propp's definition) has a \emph{sphere of action} consisting of the functions that they are able to perform at any point in the story. Propp defines seven roles that have distict spheres of action: \emph{villain}, \emph{donor}, \emph{helper}, \emph{princess}, \emph{dispatcher}, \emph{hero}, and \emph{false hero}.

Though Propp defines each \emph{dramatis personae} as being distinct (characters can only play one role at a time), it is simple to extend the idea to allow for overlapping roles. For example, a victim could also be a donor, so the set of functions they can perform would be the union of both the victim and donors' permitted function sets. Propp does not explore this possibility in \emph{The Morphology of the Folktale}, however.

\begin{figure}[!t]
\centerline{\includegraphics[height=0.4in]{propp1.png}}
\caption{One Propp function following another}\label{fig:propp1}
\end{figure}

\begin{figure}[!t]
\centerline{\includegraphics[height=0.6in]{propp2.png}}
\caption{Multiple simultaneous functions}\label{fig:propp2}
\end{figure}

In a typical story, one story function will follow another as the tale progresses in a sequential series of cause and effect (figure~\ref{fig:propp1}). However, Propp's formalism also allows for simultaneous story functions to be occuring at once (figure~\ref{fig:propp2}).

Though flexible, Propp's formalism is limited in its expressiveness. All story functions describe events at the same level of abstraction, describing one event after another. Also, Propp insists that the story functions occur in a prescribed order. Later French structuralists such as \citep{bremond1980logic}, \citep{greimas1983structural} and~\citep{todorov1969grammaire} address the latter problem by generalising Propp's work outside of Russian Folktales, though each represents only incremental improvements on Propp, lacking a means of nesting story functions to create abstractions. \citep{barthes1975introduction} broadly describe hierarchically composing \emph{narrative units} but, lacking implementation details, these can only be used as a template from which to build a new narrative model.
% What are the shortcomings of Propp? (i.e. lack of abstractability, etc)

\subsection{Describing Stories with Logic}
% Laure-Ryan did a bit of this, also include linear logic approaches

\subsection{Other Types of ``Story Component''}
Lehnert's \emph{plot units} are a more recent narrative formalism \cite{lehnert1981plot}. However, these plot units only describe stories as three types of event: positive, negative and mental. These events occur with respect to a single character in the story, so an author must always author story components with concrete characters in mind, making them difficult to re-use. Similarly to Propp's system, the order of composition must always be in a certain sequence, and plot units cannot refer to other plot units. Again, we are left without a means of creating abstractions for our story components. In the ``TropICAL: a DSL for Tropes'' section, we describe how tropes allow the nesting of components to allow story authors to create their own abstractions, addressing this issue.

\section{Implementations of Experimental Narrative}
\label{sec:implementations}
% I've plenty of material, but it really needs reworking and extending

\subsection{Story Generation}
% TaleSpin, etc

\subsubsection{Generative Grammar}

\subsubsection{Author Modelling}
% This would also use planners

% But this is limited in interactivity, so in order to have characters we can
% interact with, we must use...

\subsection{Intelligent Agents as Characters}
% Write an introduction

\subsubsection{Planner-based systems}
\citep{young1999notes} argues that planners are a good method for regulating plot, later creating an architecture to integrate a planner with character agents in an interactive game environment \cite{young2004architecture}. Young describes how narrative systems must re-plan when a player makes  narrative-breaking actions, by either restructuring the narrative mid-story (\emph{accommodating} the action) or preventing the action from executing (\emph{intervening} on the action).
Cavazza et al's \emph{I-Storytelling} system~\cite{cavazza2002character} implements Young's architecture with ideas from Barthes' narrative units, using characters with Hierarchical Task Networks (HTNs) to generate its stories. Each character has a main task, which is divided into subtasks to create a task hierarchy, with each task node having pre- and postconditions. The story emerges from the outcomes of each character's plans, and the narrative structure as a whole is not planned.
Mimesis~\cite{riedl2003managing}, another implementation of Young's architecture, allows the story author to create plans for the overall narrative in addition to its characters, in order to have more control over narrative structure. Rather than using a narrative model, Mimesis models the player to track their expected level of suspense while interacting with the story. Like the \emph{I-storytelling} system, its plans are hierarchical, using the Longbow hierarchical partial-order causal link planning system~\cite{young1994decomposition}.

A disadvantage of these planner-based systems is that they require the story author to think in a planner-oriented manner. They must consider the goals of both the story and the character, plans to achieve these goals, and re-planning when goals are not met or when situations change. This is a drastic change from the usual story writing methods of authors, where the focus is on structure, plot, themes and characters. Though graphical user interfaces such as Mimesis' Bowman system~\cite{thomas2006author} could be used to assist planner-driven story authoring, a complete shift in creative workflow is still required, making them inaccessible to non-technical writers.

\subsubsection{Drama Manager-based Systems}
\paragraph{The OZ project}
Carnegie Mellon University's OZ project \citep{mateas1999oz} is one of the first major research efforts towards creating interactive drama using agents as characters. A dramatic structure is given to the narrative by means of a \emph{drama manager}, which is able to see all of the actions occurring in the storyworld and can change anything in order to create a better experience for the user.

Ideas from the OZ project were later developed into what would become Fa\c{c}ade.
% Facade, etc
\paragraph{Fa\c{c}ade}
Mateas and Stern's \emph{Fa\c{c}ade} has players interact with the characters of the story through natural language. In this game, the player attends the party of a young couple (Grace and Trip) celebrating their wedding anniversary. As the course of events unfold however, the player learns that all is not as happy as it seems.

The player interacts with the characters by typing in natural language sentences, to which Grace and Trip respond. Though the characters are implemented through agents, the story is controlled using a drama manager. In all, their system consists of using NLP, a novel character authoring language and a novel drama manager to create an interactive narrative.

Several custom-designed languages were used to create the game, including a language called `A Behaviour Language' (ABL) for the agents and a special language for the sequencing of the beats. ABL represents situations as character goals, maintaining a tree of all the active goals and behaviours that are happening at any time.

In Fa\c{c}ade, the smallest unit of narrative action is called a \emph{story beat}, taken from McKee's book on authorial style for screenwriters \citep{mckee1997substance}. The simulation constantly monitors what the user is doing and how it may lead from the current story beat to another. Story beats have preconditions and effects on the state of the narrative, so it is the drama manager's job to work out when it makes sense to initiate a certain beat.

`Beats' have a very fine granularity, with 200 or so updating every minute of the simulation. They consist of a set of ABL behaviours, which advance the narrative yet still allow interaction to change to other beats. Only one beat can be active at a time.

A beat can have 5 types of goal:

\begin{enumerate}
  \item transition-in: characters express their intentions
  \item body: a dramatic question/situation is posed to the player
  \item local/global mix-in: react to the player before end of the beat
  \item wait-with-timeout: wait for the player's reaction
  \item transition-out: final reaction to the player's action in the beat
\end{enumerate}

A beat goal is a series of steps for an agent to perform, which can be:

\begin{itemize}
  \item staging (where to walk to, face)
  \item dialogue to speak
  \item where and how to gaze
  \item arm gestures to perform
  \item facial expression to perform
  \item head and face gestures to perform
  \item small arm and head emphasis motions triggered by dialogue (head nods, hand flourishes)
\end{itemize}

As an example, there is a behaviour called ``Fix\_Drinks'', which specifies a sequence of agent behaviours where the characters Grace and Trip have an argument while Trip asks the player what they would like to drink. If the player decides not to go along with the beat (in this case, by not choosing a drink), then the beat will be aborted and replaced with another.

Fa\c{c}ade has become popular as a game outside of academia, with playthroughs of the game reaching millions of views on Youtube. This shows the promise of interactive narrative as being a unique and engaging new form of entertainment. Unfortunately, no other implementation of interactive narrative seems to have captured the public imagination since the release of Fa\c{c}ade.

Fa\c{c}ade's popularity seems to reinforce Crawford's assertion (section \ref{sec:media}) that interactive narratives must be social in nature. The gameplay comes entirely from the conversations and interactions between Grace, Trip and the player. Much of the excitement comes from the social consequences of certain conversation paths or actions. By modelling characters as agents, Mateas et al have created a truly interactive experience. However, by also using a drama manager to manage the agents, they have used these agents to tell a story.

How might these agents be made more convincing? Outside of writing rules for their behaviour consisting of character goals and beliefs, how might an author create truly unique and idiosyncratic characters? To address the question, I next examine different types of emotional models in psychology, and how each might be used to model characters as agents.

\subsubsection{Character Modelling}

\subsubsection{Characters with Emotional Models}
% Intro: not done very much?

\subsubsection{Emotional models}\label{sec:emotional-models}
% How is this useful for narrative?
Usually it would seem odd to want to model emotion as part of a computational process. Emotion is such a seemingly irrational set of behaviours that they are easy to dismiss as `human imperfections'. However, as \citet{marsella2014} observe, emotions may have a useful role to play in communication, so long as they are displayed at appropriate times.

For example, anger prepares the human body to fight by increasing the manufacture of adrenaline. Fear similarly triggers the `fight or flight' response, alerting the senses for danger and preparing the body to react.

In order to model human emotions using agents, we must first find a suitable psychological model to use. Marsella et al describe three main types of emotional model:

\begin{enumerate}
 \item \textbf{Discrete} emotional models, which claim that humans have a set of innate, pre-defined emotional states which people may enter and leave.
 \item \textbf{Dimensional} models of emotion, describing the spectrum of emotions as being points somewhere in continuous space. Implementations typically use two or three dimensions for simplicity.
 \item \textbf{Appraisal} theories of emotion take an agent's mental processes into account. Their emotional state is derived from whether or not their goals have been achieved, and what effects current events are having on their circumstances, for example.
\end{enumerate}

% Give examples of concrete models for each type.
\subsubsection{`Basic' emotions}
Ekman first made a case for discrete, biologically-determined emotions, based on evidence from research into facial expressions \citep{ekman1992argument}. He describes emotions as being \emph{basic}, in two senses of the word: \emph{i.} that there are a number of distinct emotions, each with its own different characteristics, and \emph{ii.} that these emotions were developed through evolution for specific functions.

Ekman argues that these evolved emotions share nine characteristics:

\begin{enumerate}
  \item Distinctive universal signals
  \item Presence in other primates
  \item Distinctive physiology
  \item Distinctive universals in antecedent events
  \item Coherence among emotional response
  \item Quick onset
  \item Brief duration
  \item Automatic appraisal
  \item Unbidden occurrence
\end{enumerate}

These characteristics are shared by all of the `basic' emotions as observed in humans and primates.

Discrete models of emotion suggest that there is a neural basis for emotion. For example, Armony et al describe how the amygdala in the brain is responsible for conditioned fear responses  and create a neural network to model it \citep{armony1997computational}.

Using a discrete model of emotion for agent-based characters would be relatively simple. Each basic emotion could have its own distinct set of behaviours as postconditions, and triggering circumstances as preconditions.

However, a more fluid approach could be useful when modelling emotions with agents. It would be impossible to say that an agent is \emph{angry and approaching furious} using a discrete theory of emotion. Nuanced levels of emotion and even combinations of several emotions add an extra level of texture to a character. Dimensional and appraisal theories of emotion address this challenge.

\subsubsection{Russell's circumplex model of emotion}\label{sec:circumplex}
\begin{figure}[!t]
\centerline{\includegraphics[height=3in]{circumplex.png}}
\caption{Russell's circumplex model of emotion} \label{fig:circumplex}
\end{figure}

Russell's circumplex model of emotion is a well-known dimensional model \citep{russell1980circumplex}. In this case, the dimensional variables are \emph{valence} (how agreeable or otherwise a situation is to an agent) and \emph{arousal} (how excited an agent is).

Russell's original paper proposes a model similar to that shown in figure \ref{fig:circumplex}, where the $x$ axis is a person's valence level and the $y$ axis is their arousal level. He argues that the full range of human emotions lie as points along these axes. Eight such examples are shown in fig. \ref{fig:circumplex}.

This model is very easy to adapt to human-like agents. \citet{ahn2012nvc} adapt this model by adding a third dimension, dominance, to create conversational agents in a 3D environment. This `dominance' dimension was first proposed in Mehrabian and Russell's original work \citep{mehrabian1974approach}, but later removed due to being perceived as the consequences of the \emph{effects\/} of emotion \citep{russell1980circumplex}, rather than being a component of emotion itself. Like Ahn et al, I found it useful to add the dominance-submission dimension, and so left it in my emotional model. This is the approach I take in creating my Punch and Judy simulation, and so it is described in more detail in section \ref{sec:emotion}.

\subsubsection{Appraisal theory}
Appraisal theories of emotion lend well to simulation with agents, due to their taking a person's beliefs, desires and intentions into account with respect to external events. Emotions arise when an event occurs and a person internally \emph{appraises} its consequences with respect to their beliefs, desires and intentions. This fits well with the popular BDI architecture for intelligent agents.

Different methods of appraisal may be used in order to produce emotions. Gratch and Marsella use decision theoretic plans \citep{gratch2004domain}, but other approaches could include reactive plans, Markov-decision processes, or detailed cognitive models.

Though the Punch and Judy simulation described in section \ref{sec:punchjudy} uses a dimensional model of emotion, an appraisal-based model would be worth investigating due to its tight coupling with belief desire intention psychological models used in agents. I describe my intention to explore this area further in section \ref{sec:fappraisal}.


% But characters in a story need to follow some kind of underlying plot
% mechanism, so...

\subsection{Governing Narrative in a Multi-Agent System}
\subsubsection{Planner-based Systems}
% Riedl, Young, Carezza, etc
\subsubsection{Social Norms}
Versu~\cite{evans2014versu} is an interactive drama system that uses a multi agent system as characters. The characters' actions are coordinated with \emph{social practices}, which describe types of social situations and is described by the authors as a successor to the Schankian script. These social practices are implemented as reactive joint plans, which agents can choose whether to participate in or not. Rather than directly telling the agents what to do, these social practices merely \emph{suggest} courses of action, leaving each agent to decide for itself what to do based on its individual goals.

The authors decide against using a drama manager to control the agents' actions because they want to take the \emph{strong autonomy} approach to agent governance. This means that they prefer to give each agent some degree of autonomy by allowing it to make the final decision on which course of action to take, rather than blindly following a drama manager. Suggesting actions with social norms achieves this goal. Rather than describing typical story events in terms of social norms, however, in Versu the social norms \emph{are} the story. The gameplay revolves around the avoidance (or purposeful subvertion of) awkward social situations.

Each character has a role, which is governed by a social practice. For example, a \emph{greeting} practice involves characters with the \emph{greeter} and recipient roles. The greeting practice would tell the greeter in which manner they are to greet the recipient, and the recipient how to respond. It is noteworthy that these actions are merely suggested, and not enforced.

\emph{Exclusion logic}~\cite{evans2010introducing} is used to describe the social practices of the system. Exclusion logic allows the description of tree structures and includes an exclusion operator. For example, a description of a character called ``Brown'' is shown in listing \ref{lst:exclusive}. It describes the building up of character attributes as a tree structure.

Exclusion logic aims to address the frame problem. The frame problem is the uncertainty around whether predicates that change over time (fluents) change other predicate values. It aims to address this through use of an exclusion operator (``!''). Listing \ref{lst:exclusion} shows an example of the exclusion operator in use. The example specifies that an agent can have only one gender. The \emph{Praxis} language implementation of the exclusion logic has a type checker which ensures that no character can have multiple genders.

\begin{lstlisting}[float,label=lst:exclusion,caption=nextHopInfo: caption]
A(agent).sex!G(gender).
\end{lstlisting}

\begin{lstlisting}[label=lst:exclusion,caption=Description of ``Brown'' character.]
brown.sex!male;
brown.class!upper;
brown.in!dining_room;
brown.relationship.lucy.evaluation.attractive!40;
brown.relationship.lucy.evaluation.humour!20.
\end{lstlisting}

Exclusion logic's exclusion operator allows an author to express the fact that a variable can only have one value. For example, if `the `Brown'' character changes location from the dining room to the kitchen, \emph{brown.in!dining\_room} is terminated when \emph{brown.in!kitchen} holds.

Versu takes the \emph{constitutive} view of social practice, as opposed to the \emph{regulative} view. This means that rather than restricting an agent's possible actions based on its permissions and obligations, they participate in a certain social practice by taking an action. Their actions are only restricted by what is possible in the story world, and what the agent desires to do. This way, agents can choose whether or not to take part in certain social interactions.

Many of the components we aim to have in our story telling system appear in Versu: the use of social norms to gently encourage story-conforming behaviour rather than demanding it, and the use of formal logic to determine which behaviours are possible. However in Versu the social norms \emph{are} the story, rather than describing the story components that invisibly govern the behaviour of characters. In order for this kind of governance to occur, an institution-based solution is preferable, based on events, agent actions and standard deontic logic. Because character actions are constrained by the structure of a story, a \emph{regulative}  view of social practice is more suited to the expression of story components as social norms.

Much of the advantages of using exclusion logic can be gained by using an institution-based approach. Non-inertial fluents can be used to ensure that variables can only ever have one value. Standard deontic logic is enough to provide the rest of what is needed.

\subsection{Modelling Narrative with Logic}
\label{sec:model-logic}
Although most recent research focuses on the use of planners to manage the drama in a story, there is also much interesting work which makes use of formal logic to model narrative. Though often used for the generation of linear story text, it is increasingly being applied to non-linear narratives as well. Logic-based approaches are generally based on either temporal logic variants or some kind of linear logic.

Ceptre~\citep{martens2015ceptre} is a language for modelling generative interactive narratives using \emph{linear logic}, a formal logic designed to describe resource usage. 

A Ceptre story begins with an initial state $\Delta_0$. Each state iteration
$Delta_i$ is examined repeatedly, and a subset $S$ of it is updated with rule
$r$. The next state, $\Delta_{i+1}$, has the subset $S$ replaced with $S'$, the
new subset with the consequences of the applied rule $r$.

The rules are specified using the combination of logical statements with two
operators: $*$ (tensor) and $\text{-o}$ (lolli). The tensor operator is used to
concatenate statements, while the lolli operator expresses state transitions in
the form $S \mathrel{\text{-o}} S'$. The rules use \emph{replacement semantics},
which means that everything from state $S$ will disappear unless stated to be in
state $S'$. A $\$$ operator is used to mark facts in $S$ that the author wishes
to remain in $S$ without explicitly stating so.

Listing \ref{lst:ceptre-murder} shows an example from~\cite{martens2015ceptre}
that describes a ``murder'' rule and its consequences.

% This listing SERIOUSLY messes up my syntax highlighting!

% \begin{lstlisting}\label{lst:ceptre-murder}
% do/murder
%     : anger C C’ * anger C C’ * anger C C’ * anger C C’ *
%     $at C L * at C’ L * $has C weapon
%     -o !dead C’.
% \end{lstlisting}

In this case, four instances of the ``anger'' predicate with the same arguments
has a significant meaning: a character's emotion is treated as a resource. The
fact that \emph{anger $C C'$} appears four times means that a character is
\emph{four times} as angry at another character. Depending on how many time the
``anger'' statements appear in the new state, this anger can rise or fall at the
next step in the story.

\begin{comment}
  - delta initial state
  - state is examined, rule r is selected (which applies to a subset of the
  state), new state is generated
  - A <lolli> B means that whenever the predicates in A are present, they may be
  replaced with B
  - Tensor operator * just conjqins predicates
  - Uses replacement semantics, so everything in the new state must be on the
  right side hand of the rule
\end{comment}

% Versu

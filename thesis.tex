\documentclass[11pt]{report}
\usepackage{baththesis}
\usepackage{graphicx}
\usepackage{syntax}
% \usepackage[british]{babel}
\usepackage[T2A]{fontenc}
% \usepackage[T1]{fontenc}
\usepackage[utf8]{inputenc}
\usepackage{caption}
\usepackage{textcomp}
\usepackage{paralist}
\usepackage{amsmath}
\usepackage{latexsym}
\usepackage{multirow}
\usepackage{multicol}
\usepackage{subcaption}
\usepackage{verbatim}
\usepackage{hyperref}
% \usepackage[inline]{enumitem}
\usepackage{paralist}
%\usepackage{amsfonts}
\usepackage{url}
\usepackage{natbib}
\usepackage[hmargin=3cm,vmargin=2cm]{geometry}
%\usepackage{mathpazo}
%\usepackage{eulervm}
\usepackage[usenames,dvipsnames,svgnames,table]{xcolor}
\usepackage[]{todonotes}
\usepackage{tikz}
\usetikzlibrary{arrows}
\usepackage[]{todonotes}

\newcounter{myenumi}
\setcounter{myenumi}{0}
\newenvironment{myenumerate}{\begin{enumerate} \setcounter{enumi}{\themyenumi}}{ \setcounter{myenumi}{\theenumi}\end{enumerate}}

\def\mnote#1{\todo[color=Goldenrod,size=\scriptsize]{Matt: #1}}
\def\minote#1{\todo[inline,color=Goldenrod,size=\scriptsize]{Matt: #1}}
\def\jnote#1{\todo[color=CornflowerBlue,size=\scriptsize]{Julian: #1}}
\def\snote#1{\todo[color=WildStrawberry,size=\scriptsize]{Steve: #1}}
\definecolor{light-gray}{gray}{0.95}
\def\tropical{TropICAL}

\usepackage{listings}
\lstset{ %
   language=prolog,
%  frame=l,                     % adds a frame around the code
   basicstyle=\footnotesize\ttfamily,  % use courier
   breaklines=true,
   xleftmargin=0.5em,
   aboveskip=0.5em,
   belowskip=0.5em,
   upquote=true,
%  belowcaptionskip=5em,
   numbers=left,
   backgroundcolor=\color{light-gray},
   frame=single,
   framerule=0pt
}

\setlength{\jot}{0pt}
\def\mylabel#1{\tikz[remember picture]\node(#1){};}
\def\myref#1#2#3{\begin{tikzpicture}[remember picture]
\node[draw,rounded corners] (#2){\begin{minipage}{\textwidth}\raggedright#3\end{minipage}};
\draw[overlay,-triangle 45,thick,gray](#2.west)--(#1.west);
\end{tikzpicture}}

\title{Building Abstractable Story Components with Institutions and Tropes}
\author{Matt Thompson}
\degree{EngD Digital Media}
\department{Department of Computer Science}
\degreemonthyear{October 2016}
\norestrictions


\begin{document}

% ----------------------------------
\maketitle

\clearpage
\tableofcontents
\clearpage


% problem
% approach
% results
% conclusion

\begin{abstract}
  Though much research has gone into tackling the problem of creating
  interactive narratives, no software has yet emerged that can be used by
  story authors to create these new types of narratives without having to learn
  a programming language or narrative formalism. Two problems stand in the way of such software being created: researchers are
  still using inflexible and outdated formalisms to describe their story
  components, and most interactive narrative systems are tied to the paradigm of
  goals and planning.

  Widely-used formalisms in interactive narrative research, such as Propp's
  ``Morphology of the Folktale'' and Lehnert's
  ``Plot Units'' allow users to compose stories out of
  pre-defined components, but do not allow them to define their own story
  components, or to create abstractions by embedding components inside of other
  components. Current tools for interactive narrative
  authoring, such as those that use Young's \emph{Mimesis} architecture or
  \emph{Fa\c{c}ade's} drama manager approach, direct intelligent agents playing
  the roles of characters through use of planners. These systems tell the character
  agents exactly what actions to take, leaving them with no autonomy to produce
  interesting narrative outcomes.

  This thesis proposes two contributions that tackles these two problems. The
  first is the use of \emph{Story Tropes} to informally describe story
  components. We introduce \emph{TropICAL}, a controlled natural language system
  for the creation of tropes which
  allows non-programmer story authors to describe their story components informally. Crucially, this
  language allows for the creation of new story components and abstractions
  that allow existing components to be embedded inside of new ones.

  The second contribution replaces the paradigm of goals and planners for story
  direction with \emph{social institutions} that regulate,
  rather than regiment, the behaviour of character agents. Our language
  compiles to the input language for an Answer Set solver, which represents the
  story components in terms of a formal \emph{normative framework}, and hence allows for
  the automated verification of story paths. Due to the fact that
  story-following actions are conveyed to the agents in terms of
  \emph{permissions} and \emph{obligations}, the agents are able to break away
  from the story in extreme circumstances, such as when experiencing extreme
  stress due to an emotional model. This gives the character agents more
  autonomy than they would have with a drama manager or story planner
  ``director'', allowing for some unpredictable outcomes.

  We evaluate the suitability of these tools for interactive story construction through a
  thematic analysis of story authors’ completion of story-authoring tasks using the language.
  The participants complete tasks in which they have to describe stories with different degrees
  of complexity, finally requiring them to reuse existing tropes in their own trope abstractions.
  The thematic analysis identifies and examines the themes and patterns that emerge from the
  story authors’ use of the tool.\mnote{To be updated when the analysis is complete}

\end{abstract}

% \begin{abstract}
% Authoring an interactive narrative with branching story paths presents many
% challenges. Either an author must write out every single path through the story
% by hand, which is difficult and time-consuming, or they must find a way to
% automatically generate all possible story paths, which can result in a
% combinatorial explosion and lack of authorial control over the direction of the
% story.

% Current approaches, such as those that make use of Young's \emph{Mimesis} architecture for agent-based
% story planners, or drama manager-based systems such as Mateas and Stern's \emph{Fa\c{c}ade}, combine the use of narrative
% formalisms with a story planner
% to ``direct'' the actions of character agents in a story. Shortcomings exist in
% both components of this approach, however: current formalisms such as Propp's
% ``Morphology'' or Lehnert's ``Plot Units'', though they are flexible and
% expressive enough to describe almost any type of story, offer ways to piece
% together a story out of parts, but lack the means to define new story parts or
% organisations of story parts with which an author may create abstractions for stories. This would be achieved, for example, by creating a new story
% ``component'' as a combination of existing ``components''. The story planner
% requires the author adopt a goal-oriented style which is unfamiliar and
% relatively awkward for writers without experience of programming.

% Rather than having authors learn a specific formalism to describe the components of a story, the
% approach suggested in this thesis is to describe the parts of the story
% using a controlled natural language syntax. To the author, this should seem as
% though they are describing the trope informally. Once the parts of
% the story are described in this way, they are formalised by translation into
% InstAL, a language for social institutions, then AnsProlog, an Answer Set
% Programming language.

% We propose the use of
% \emph{story tropes} as a
% formalism with which authors describe composable, abstractable story components of their stories, and
% the use of a normative framework to
% regulate the behaviours of a multi-agent system so that their decisions are
% guided by the narrative described in the tropes. A trope is a recurring
% theme or motif in narrative, which abstracts
% certain sequence of events. Existing narrative formalisms have two major
% shortcomings: they do not allow authors to create abstractions (or usually even to change
% the formalism in any way), and they require that authors memorise the specific
% rules and components that make up the formalism.
% We show how tropes allow for the flexible description and abstraction of story
% components, a capability that existing systems lack. Additionally, most authors
% are already familiar with the concept of tropes, as
% we show through a questionnaire of interactive fiction authors. This means that
% they need not learn a new formalism in order to create their stories.

% The second contribution of this thesis is the translation of these tropes into
% \emph{social institutions}, describing the permissions and obligations of the
% characters at each point in the story. This means that narratives resulting from
% the tropes be enacted (validated) through their embedding inside a computational
% framework. We use this method to resolve the shortcomings of using drama
% managers and planners to direct the actions of character agents. Instead of the
% agents being strictly \emph{regimented} and told what to do, their behaviour is
% \emph{regulated} through social norms. This gives them some level of autonomy,
% so that they may break away from the narrative in special circumstances, such as
% when the characters enter extreme states of emotion.

% To address the shortcomings of existing narrative formalisms and the use of
% planners and drama managers to direct character agents, we have developed two
% tools that build on the formalisation of tropes to assist in the talk of
% interactive narrative creation. \emph{TropICAL} (the TROPe-oriented Interactive
% Chronical Action Language) and \emph{StoryBuilder}, an Interactive Development
% Environment (IDE) and visualisation tool to aid in the writing and combining of
% tropes in \emph{TropICAL}. \emph{TropICAL} is a language for the description of
% tropes. It has a controlled natural language
% syntax, and compiles to a representation that provides the means for bounded
% model checking against the author's desired properties for the story.

% We evaluate the suitability of these tools for interactive story construction
% through a thematic analysis of story authors' completion of story-authoring
% tasks using the language. The participants complete tasks in which they have to
% describe stories with different degrees of complexity, finally requiring them to
% reuse existing tropes in their own trope abstractions. The thematic analysis
% identifies and examines the themes and patterns that emerge from the story
% authors' use of the tool.\mnote{Will rewrite once analysis is complete}
% \end{abstract}

% ----------------------------------

% intro

\chapter{Introduction}
\label{cha:introduction}

In most media, stories are told linearly, where one event follows another
sequentially until the conclusion is reached. This is the format which most
printed works follow, due to the physical limitations of the medium. Though
these are limitations
that we have come to accept due to the popularity and wide proliferation of
printed media, they are not limitations inherent to the act of storytelling. Even
before written language, when stories were told around the camp fire, the
audience would have opportunities to interrupt the narrator in order to add
narrative details of their own. In its transition from the spoken to the printed
word, the medium of storytelling has lost its interactivity.

Innovative authors such as John Barth, and works such as the ``Choose Your Own Adventure'' and
``Fighting Fantasy'' series of books have attempted to expand the possibilities
of the printed word to allow for non-linear and interactive narratives. Working
within the limitations of the medium, these books offer the reader choices such
as ``turn to page 50 to drink the water'' or ``turn to page 14 to run away''.
These decision points create a branching narrative that can be described by a
tree structure. Though this method for describing branching narrative in a book
is innovative, it falls prey to the limitations of book size: the choices are
very often only the \emph{illusion} of choice, where the decision is usually
between continuing the story and death of the reader's character. If the tree of the narrative were to be
visualised, it would contain a great many ``dead end'' leaves.

Interactive storytelling is the method of telling a story
that a ``reader'' can meaningfully change through interaction. The ``Choose Your
Own Adventure'' example just described is not an example of meaningful change in
a story, as the only interactions are binary decisions that lead to a limited
number of paths through the tale. Characters are not able to react or change
their behaviour based on a sequence of actions the ``reader'' of the story has
taken, for example. The structure of the story cannot change in real time in
response to the
actions of the ``reader''. Even with new media such as computer games, crafting
a dynamically changing story is challenging, and narrative interaction is
limited to ``multiple endings'', where a single decision point at the end of the
game allows the player to choose between multiple outcomes.

Interactive storytelling researchers have attempted to challenge the limitations
of current computer game stories through the use of artificial intelligence
techniques. Early systems such as Talespin~\citep{meehan1977tale} and
Minstrel~\citep{turner1993minstrel} focus on story \emph{generation}, where a
story is created that changes each time the software program is run. Later
systems would then go on to use intelligent agents to simulate the characters
in a story, each with goals and plans that mimic their motivations. More recent
storytelling systems use planners (such as ones based on Young's
architecture~\citep{young2004architecture}) to direct the actions of these
character agents to fit the pre-determined path of a story.

Though the field of interactive narrative generation has many active researchers
and research centres, most of the tools produced from these projects are
targeted towards other computer science researchers to use. To learn esoteric
techniques such as how to use goal-based planner systems in order to describe
branching narratives is a challenge even for professional programmers.
Similarly, describing stories with declarative, logic-based languages such as
Prolog or Answer Set Programming languages requires authors to master new and
very different programming paradigms. If such methods are challenging to
programmers, they are far beyond the reach of non-programmers such as fiction
authors. The creation of interactive narrative authoring tools that can be used
by authors with no previous programming experience remains a challenge.

% Talk about components
Another challenge in the field is the identification of a suitable formalism for
narrative. Researchers still use structuralist formalisms created almost a
century ago, such as Aarne-Thompson's ``Tale-Type
Index''~\citep{aarne1987types}, or Propp's ``Morphology of the
Folktale''~\citep{propp1968morphology}, due to inertia, familiarity, and the fact
that they are ``good enough'' for most purposes.

The two contributions of this thesis are the use of a \emph{social institutions}
(also known as normative frameworks) for the governance of intelligent agent
characters in an interactive narrative, and the definitions of these institutions as story
\emph{tropes} using controlled natural language. The normative framework can be
thought of as the director that governs the behaviours of the actors in the
story, telling them what they are \emph{permitted} and \emph{obliged} to do as
part of the story. The tropes can be considered as a screenplay of sorts, a user interface that contains a high-level description of what
the director has to do. This thesis introduces the concept of \emph{tropes} as a
story formalism for
the first time into the interactive narrative literature, where tropes are formal, composable components
of stories. Tropes describe identifiable patterns that recur throughout multiple
stories, which become familiar to audiences through repeated exposure. Examples
include \emph{The Hero's Journey}, where the story's protagonist leaves home to
go on a quest, where they must overcome many challenges and obstacles before
returning triumphant, or the \emph{MacGuffin}, where the main characters of a
story are chasing after an object (such as the Ark in \emph{Indiana Jones and
  the Raiders of the Lost Ark}, or the falcon statue in \emph{The Maltese
  Falcon}). A story is composed of tropes that may describe its structure (such
as with the \emph{Three Act Structure} or \emph{Climax} tropes) as well as
specific scenes that must take place (such as the \emph{Deus Ex Machina} or
\emph{Betrayal} tropes). Describing the story in terms of such tropes gives an
author the ability to piece together a high-level version of the story
structure, while still leaving the small details of the story to the plans of
the character agents.
Additionally, tropes can be composed in different ways. As with formalisms such
as~\citet{propp1968morphology} and~\citet{lehnert1981plot}, tropes can be
composed one after another to tell a sequential series of scenes. However, the
real flexibility of tropes comes with their ability to be composed
\emph{hierarchically}, describing the structure of a narrative as well as its
phases. This is due to the fact that tropes can describe different levels of
abstraction. A top-level trope would be the \emph{Three Act Structure} mentioned
above, which can three main ``acts'' that consist of sequences of tropes.
However, different paths through the narrative could lead to different chains of
tropes within the three-act structure.
Another example of composing tropes hierarchically could be Vonnegut's ``Man in
Hole'' story shape~\citep{vonnegut2009palm}, where events get worse for the hero
until the very end of the story. At the start of this top-level trope, progressively
``worse'' tropes could be strung together, followed by progressively ``better''
tropes.
More detailed explanations appear in section~\ref{sec:tropes-intro} and chapter~\ref{cha:tropes}.

The next section of this introduction contains a high-level overview of
interactive narrative, describing its past and future use in and beyond entertainment.

\section{Interactive Narrative}
Though media such as computer games allow a player to influence events through
interaction with its characters and objects, the story of the game itself is
usually predetermined. In order to create media in which the narrative is shaped
by a person's interactions, one of two approaches are usually taken: emergent narratives or branching narratives.

\textbf{Emergent narratives} occur when a story (of sorts) unfolds as a result of a player's interactions with the rules of the game world. Examples of this appear in the games \emph{Minecraft}, \emph{Civilisation}, \emph{Dwarf Fortress} and \emph{The Sims}. In these cases, no author has sat down and written a branching narrative for every eventuality in the game. Instead, the narrative is created by the story world itself in response to a player's actions.

The advantage of this type of narrative is easily apparent from an author's point of view: it saves them the work of having to write all the branches of a story. In a story with any nontrivial amount of branching, the author would have to spend a great many hours writing down all of the alternative paths through it. In the end, a player might not even see the result of a fraction of the effort put into a game's writing. An emergent story solves this problem by procedurally generating a narrative.

However, it could be argued that these emergent stories are only ``stories'' in
the loosest possible sense --- they probably have no shape or structure, no recurring motifs, and nothing like a climax near the end. They do not take the player on an emotional ride, but merely describe a sequence of events. Indeed, it greatly decreases the influence of the writer as an artist, leaving a story totally in the hands of the player. If a game author has a certain story they want to tell in a non-linear fashion, or a message that they wish to convey, they must resign themselves to hoping that it somehow emerges from the unscripted interactions of a player with the game.

\textbf{Branching Narratives}, on the other hand, give authors total control over what happens in a non-linear story. They are able to structure each possible branch of the narrative, ensuring some branches terminate in climaxes while others lead to despair. They can insert artistic themes into their stories, and put their own personal flourishes into every single aspect of it. The price of this control is the amount of work it requires for the creation of each story branch, as well as the limited amount of branching an author's time constrains them to write.

Ideally, we could combine the best features of both types of non-linear
narrative: the labour-saving generative element of emergent narratives and the
structural control of the branching narratives, sharing control of the narrative
between author and actor.

\section{Interactive narrative for games}
Computer games are a new medium for artistic expression. The element of
interactivity, combined with visual art, music and storytelling, allows the
creation of ever more fantastic worlds. Game creators are now experimenting with
ways to make a narrative itself interactive. Imagine playing through a version
of Romeo and Juliet where there is a possibility that the characters could
escape their fates. Or playing a detective in a game where the story changes
depending on how quickly you can piece together clues. This is the new frontier that we are exploring with this research: the domain of \emph{interactive} storytelling.

We are defined by the choices we make throughout our lives. These choices form a
story that describes our own personal history. If a player can watch somebody's
life unfold and witness the choices that they made, and those that were forced
upon them, they would have a better understanding of how they came to be who they
are. This kind of deep understanding is only possible through experiencing a
story where the decisions made have real consequences. Traditional fiction, and
perhaps computer games, enable this to some extent. But the story is still
experienced passively by the consumer in these media. A truly interactive
narrative would go deeper, as though you had truly lived as another person,
forced to make the same decisions in the same situations. This has the potential
to enable a whole new level of narrative immersion.

Our initial attempt at creating an interactive narrative ``game'' uses
intelligent agents with emotional models to act out an interactive Punch and
Judy puppet show. As the audience boos or cheers at the actions of the puppets,
encouraging or scolding them, the agents' emotional state is changed, affecting
their decision-making process. Though this game may not offer much explanation
for the irrationally violent actions of ``Punch'', the audience can use the
system to explore
their expectations of what a Punch and Judy show should be. This work is
described in section~\ref{sec:emotional-pj}.

The Punch and Judy show in section~\ref{sec:emotional-pj} also makes use of a
\emph{social institution} (or \emph{normative framework}) to regulate the
actions of the agents playing the roles of the puppets. Because the normative
system only guides the actions of the agents, rather than forcing them to choose
specific courses of action, their emotional models may affect their decision
making in such a way that the agents choose to break away from the constraints
of the behaviour suggested by norms, at times of extreme emotional distress for
example.

\section{Normative Systems}
\label{sec:normative-intro}
In the field of multi-agent systems, a \emph{normative framework} or
\emph{social institution} coordinates the actions of the agents using the
language of deontic logic: permissions and obligations that describe what each
agent \emph{may} or \emph{must} do at each point in time.

Normative systems are often used to model legal contracts, so that the contracts
are described in terms of social norms. This way, the contract is described as a
social institution that each party adheres to. In the case of a contract
dispute, the party in the wrong is the one that has broken the social contract,
either by carrying out an unpermitted action, or by not executing an obliged
action before its deadline. As in a social contract, there is nothing that
physically stops either party from breaking the norms. The only barrier is the
prospect of a possible penalty for deviating from the agreed-upon rules.

In our system, we model the narrative as a social institution, which regulates
the behaviour of the character agents, rather than strictly regimenting their
actions. This means that each character is technically able to break away from
the story if they wish, but they would suffer grave consequences for doing so.
This makes for an interactive storytelling system with a greater degree of
unpredictability and variety, where an agent that is desperate to carry out a
plan, or caught in the darkest state of an emotional model, may do something
rash and spontaneous. This means that the simulated characters have more agency
to do what they want, even though they are encouraged towards certain actions by
the bounds of the story.

Describing a story in terms of permissions and obligations does not come
naturally to most story authors, however. For this reason, we use story tropes
as a kind of \emph{user interface} through which non-programmer artists and
authors may compose the social norms that govern the behaviour of the character agents.

\section{Story Tropes}
\label{sec:tropes-intro}


Tropes describe commonly-seen themes and elements of stories. In a sense, they are similar to clich\'es, in that an audience becomes familiar with them through repeated exposure. However, tropes are not clich\'es. Clich\'es are story elements that have been repeated ad nauseum, that are so often used and unoriginal that they evoke tedium from their audience. Tropes may perhaps become clich\'es if they are used to the point of becoming tiresome, though some are so embedded into the collective subconscious, that their presence is taken for granted. Joseph Campbell argues that \emph{The Hero's Journey} is a trope that transcends time and culture, representing a shared human blueprint for the rite-of-passage into adulthood~\citep{campbell2008hero}. In this way, stories and tropes could be considered a cultural language for describing common human themes. Clich\'es could not be described as such: they are simply overexploited ideas.

The themes, scenes, characters and structure of a story can be described in terms of tropes. Even individual lines of dialogue can be a trope, such as James Bond's witty put-downs when he defeats an enemy. Tropes describe parts of a story in an abstract way, which means that they can be easily identified in multiple stories.

Examples of tropes are:

\begin{itemize}
\item \textbf{The Hero's Journey}: A hero answers a call to adventure, and
  leaves home to go on a journey. The hero defeats the villain along the way, returning back home triumphant.
\item \textbf{The Evil Empire}: The villain is part of an empire, which tries to stop the hero at all costs.
\item \textbf{The MacGuffin}: There is an object that the hero needs, the search for which is used to drive the plot.
\item \textbf{Chekhov's Gun}: If a gun appears somewhere in the first act, it must be fired by the end of the third.
\end{itemize}

Tropes can describe a story at several layers of abstraction, meaning that tropes can contain other tropes as sub-tropes. For example, \emph{The Hero's Journey} can contain the \emph{Hero}, \emph{Quest}, and \emph{Call to Adventure} sub-tropes. This allows a degree of expressivity unavailable in other narrative models described in the \emph{Related Work} section.

In interactive narrative generation, where a player may take one of many
possible paths through a story, parts of the story are often procedurally
generated in order to save an author the time and effort of manually writing out
all the branches of the narrative. However, this means that the author must give
up some control over the structure and content of the story. By describing the
intended contents of the story in an abstract manner using tropes, an author is
still able to give it structure. In our case, the procedural generation occurs
through the use of intelligent agents to model story characters, which interact
with the player's actions. These character
agents are guided towards pursuing actions that fit the tropes in the authored
narrative, thus gently making sure they meet the expectations of the author
while allowing them some degree of freedom with which to pursue their own goals.
Additionally, the player's allowed actions are constrained according to the
defined tropes of the story, so that the narrative emerges as a result of the
interactions that are taken as a result of the allowed actions, along with the
way that the character agents respond to these interactions.

\section{Answer Set Programming for Story Solving}
\label{sec:trope-interface}
In section~\ref{sec:tropical}, we describe TropICAL, our programming language
for defining of story tropes. The requirements for this language are:

\begin{enumerate}
\item Easy to learn for authors who are unfamiliar with programming\label{req1}
  \item Able to express sequences of events and branching events\label{req2}
  \item Must have a mechanism for abstraction (embedding sub-tropes within tropes)\label{req3}
\end{enumerate}

In order to satisfy requirement~\ref{req1}, TropICAL uses a controlled
natural language syntax resembling that of the popular interactive fiction
programming language Inform 7, designed to allow authors with no previous
experience with programming to describe their story in terms of tropes. An
author uses this language to combine and sequence their tropes together, using
keywords such as ``X Then Y'' and ``X Or Y'' to specify sequences of events or
branches (requirement~\ref{req2}). Requirement~\label{req3}, the mechanism for
creating abstractions, is implemented by referring to other tropes by name (for
example, \emph{Then the ``Quest'' trope happens}).

When the tropes are compiled into the normative language InstAL~\citep{cliffe2007specifying}, they are defined
strictly in terms of permissions and obligations. InstAL allows the trope
descriptions to be compiled a second time into AnsProlog, an Answer Set
Programming (ASP) language, which can then be used with an Answer Set solver
such as Potassco's Clingo~\citep{gebser2011potassco} to generate all the different
possible actions of each character at any point in the story, given the
constraints expressed in the described tropes.

\section{Outline}


% Outline of sections goes here
The structure of this thesis is as follows:

\begin{enumerate}[{Chapter} 1:]
\setcounter{enumi}{1}
\item A \textbf{Literature Review}.
  We approach the literature survey from two perspectives: those of \emph{narratology},
  and \emph{interactive narrative}.
  In order to evaluate the current state of narrative formalisms, the literature review first covers narrative research
  (narratology) from social science (section~\ref{sec:narratology}), before
  examining existing implementations of interactive narrative and narrative
  generation systems from computer science
  (section~\ref{sec:implementations}).
\item The use of \textbf{Institutions as Story Worlds} is described in
  this chapter, exploring the theory and research behind
  social institutions. It discusses the merit of their use for describing
  stories, comparing their use against other logics and planner-based systems
  from the previous section. This chapter is a development of \emph{Governing
    Narrative Events with Institutional Norms}, which is listed as
  the \textbf{[CMN 2015]} conference paper in the list of papers below (section~\ref{sec:papers}).
\item Using \textbf{Tropes as Story Components}
  examines the use of \emph{tropes} as story components, analysing the process
  of turning an informal trope description into something more formal that can improve
  upon existing narrative formalisms. This is linked to the work in
  chapter~\ref{cha:institutions} by showing how tropes can be used as a
  ``user interface'' through which story authors would be able to
  describe the social institutions that govern story characters' behaviours.
  Just as a computer user interface exposes the useful parts of a piece of
  software, hiding complexity that is irrelevant to the user, our tropes allow
  authors to create their story components in controlled natural language,
  without having to worry about the complex code that results from the
  translation process.
  Work from this chapter appears in a legal
  context in \textbf{[JURIX2016]}.
\item Our specification of tropes as social institutions allows the character
  agents some degree of autonomy, with the possibility of them breaking away
  from the narrative in extreme circumstances.
  This chapter,
  \textbf{Intelligent Agents}, gives an example where the character
  agents are given emotional models that allow them to violate the social norms
  of the story in times of high emotional distress, describing work done as part of the \textbf{[AISB
    2015]} paper.
\item We use the requirements that emerge from Chapters~\ref{cha:institutions}
  to~\ref{cha:agents} to inform the design and implementation of our \textbf{TropICAL} programming
  language for tropes. The chapter concludes with examples of its use to build several stories.
\item
  Keeping a mental model of the branches of a story is simple when only one
  trope is involved, but can be difficult when several tropes are active during a story.
  Branching narratives can be difficult to check once they start becoming at all
  complex. Tool support is needed to help visualise the changes in a
  story. For
  this reason, this chapter describes the implementation of the \textbf{StoryBuilder} tool,
  designed for the interactive creation and visualisation of tropes, allowing
  the user to combine tropes to describe the story world for an interactive
  narrative. We have carried out a qualitative evaluation of the tool
  through a thematic analysis of eight participants building their own stories
  with the tool.
\item The final chapter, \textbf{Conclusions and Future Work}, reviews the
  research done, evaluating its potential impact and discusses possible future work.
\end{enumerate}

% litrev

\section{Published Works}
\label{sec:papers}

  The following list includes all conference presentations and papers published by the author which are related to this report.

  \paragraph{Conference Presentations:}
  \paragraph{[HSWI@ESWC 2014] Artfinder: A Faceted Browser for Cross-Cultural Art Discovery} Matt Thompson, Julian Padget and Steve Battle, Human Semantic Web Interaction Workshop, European Semantic Web Conference, May 2014, Crete, Greece.\label{pub:hswi}
  \paragraph{[DHIS 2015] Every Object Tells a Story} Matt Thompson, Julian Padget and Steve Battle, Digital Heritage Meets Interactive Storytelling 2015 (conference presentation), April 2015, York UK.\label{pub:dhis}
  \paragraph{[AISB 2015] An Interactive, Generative Punch and Judy Show Using Institutions, ASP and Emotional Agents} Matt Thompson, Julian Padget and Steve Battle, AISB AI \& Games Symposium, April 2015, Canterbury, UK.\label{pub:aisb}
  \paragraph{[CMN 2015] Governing Narrative Events with Institutional Norms} Matt Thompson, Julian Padget and Steve Battle, Sixth International Workshop on Computational Models of Narrative, May 2015, Atlanta, USA.\label{pub:cmn}
  \paragraph{Published Proceedings:}
  \paragraph{[COIN@IJCAI 2015] An Interactive, Generative Punch and Judy Show Using Institutions, ASP and Emotional Agents} Matt Thompson, Julian Padget and Steve Battle, International Workshop on Coordination, Organisation, Institutions and Norms in Multi-Agent Systems, July 2015, Buenos Aires, Argentina.\label{pub:coin}~\citep{thompson2015interactive}
  \paragraph{[ICIDS 2015] Telling Non-linear Stories with Interval Temporal Logic} Matt
  Thompson, Julian Padget and Steve Battle, International Conference on
  Interactive Digital Storytelling, December 2015, Copenhagen, Denmark.\label{pub:icids}~\citep{thompson2015telling}
  \paragraph{[JURIX 2016] Describing Legal Policies as Story Tropes in Normative
    Systems} Matthew Thompson, Julian Padget and Ken Satoh, JURIX 29th
  International Conference on Legal Knowledge and Information Systems, December
  2016, Nice, France.\label{pub:jurix}~\citep{thompson2016describing}


% HACK est. poms remaining: 

\chapter{Literature Review}
\label{cha:literature-review}

% TODO: section on controlled natural language?
% TODO: SIMPLE (from COIN 2015)

This research covers multiple fields of study such as narratology, interactive
storytelling, emotional modelling and artificial intelligence, therefore an extensive
literature review covering these fields is necessary. This section starts with a
look at the field of \emph{narratology}, or narrative theory, to gain some
insights into the themes and components that make up stories. Looking at
different formalisms that have been created for narrative and which themes and
motifs recur in stories should better inform the creation of techniques with
which to generate stories.

We also examine current approaches to interactive narrative generation,
focussing particularly on the use of planners, drama managers, social norms and
logical modelling for describing stories. Most of these implementations are
designed with multi-agent systems in mind, where the story forms a set of rules
which govern these agents. The start of this part of the literature survey also
examines non-interactive story generation through means of techniques such as
generative grammar, in order to provide a historical context for the research
that follows. 

As part of the examination of implementations of interactive narrative, this review especially focuses on agent-based systems. The section concludes with an overview of emotional models that can be used to model distinct characters using agents.

\section{Narratology}
\label{sec:narratology}
Narratology is a deep field with many sub-fields. This review examines the parts of it that might best inform the modelling of narrative by computers, as well as the construction of interactive narrative.

The first part of the overview of narratology examines research into categorising different types of narrative, both traditional and experimental.
This draws from classic narratological texts, as well as work done on
``cybertext'' and experimental narrative in the interactive age. This
examination of recent research into non-linear narratives is essential for the
construction of interactive narratives for games or simulations.

Structuralist formalisms of narrative attempt to explain how stories work by dividing them into commonly occuring themes and motifs. This is a natural fit to the modelling of narratives by computer. This overview of narratology starts with structuralism for this reason.

After the overview of structuralists' narrative models there follows a section on the use of formal logic for narrative modelling. The section ends with descriptions of other types of story components, taxonomies and ontologies used in the literature.

\subsection{Narrative Structure}
\label{sec:structure}
What is the difference between a narrative and a sequence of events? If we
recount the events that happened to us during the course of a day, would that
``count'' as a narrative? When is the retelling of events a simple listing of
facts, and when is it a story?

Narratologists from \citet{bal2009narratology} onwards refer to the
chronological ordering of events as \emph{fabula} (from the Russian фабула, ``scene''), and the retelling and
reordering of those events in a narrative as \emph{syuzhet} (сюжет, ``plot'').
In fact, Russian structuralists~\citet{propp1968morphology} and~\citet{shklovsky1991theory} were the first to use these terms in the
context of narrative, before their rediscovery by modern narrative theorists.

The concepts of \emph{fabula} and \emph{syuzhet} are key to the understanding of
narrative structure. Thanks to Aristotle's
\emph{Poetics}~\citep{halliwell1986aristotle}, we understand that a story must
have a beginning, a middle and and end. These three parts of a story describe
how the \emph{syuzhet} are organised, which can refer to its \emph{fabula}
(chronological) events in any order. For example, the beginning of a movie could
be set at the present day in the life of the protagonist, the middle could be a
flashback to an earlier time in her life, and the end could return to the events
following the start of the movie.

Though Aristotle identifies the beginning, middle and end as three key story
divisions, other theorists divide the \emph{syuzhet} of a story further. Joseph
Campbell's work \emph{The Hero with a Thousand Faces}~\citep{campbell2008hero}
describes how almost every story is a variation of the \emph{Hero's Journey}
(which Cambell also calls the \emph{monomyth}). Like Aristotle's beginning, middle and end,
the Hero's Journey has three acts: the Departure, the Initiation and the Return.
Campbell divides these acts further into seventeen distinct stages:

\paragraph{Departure}
\begin{myenumerate}
  \item The Call to Adventure
  \item Refusal of the Call
  \item Supernatural Aid
  \item Crossing the Threshold
  \item Belly of the Whale
\end{myenumerate}
\paragraph{Initiation}
\begin{myenumerate}
  \item The Road of Trials
  \item The Meeting with the Goddess
  \item Woman as Temptress
  \item Atonement with the Father
  \item Apotheosis
  \item The Ultimate Boon
\end{myenumerate}
\paragraph{Return}
\begin{myenumerate}
  \item Refusal of the Return
  \item The Magic Flight
  \item Rescue from Without
  \item The Crossing of the Return Threshold
  \item Master of Two Worlds
  \item Freedom to Live
\end{myenumerate}

In summary, the hero begins the story at home or in some otherwise familiar
setting, where she is called away on an adventure. After possibly rejecting this
call, the hero leaves home (crosses the threshold) and sets off for the land of adventure. After facing
many trials and possibly defeating an enemy, the hero returns home once again.
Having endured the many trials of the journey, our hero becomes stronger in spirit
and character.

As a classic example, consider the plot of Tolkien's \emph{The Hobbit}. At the
start of the tale, Bilbo Baggins is a comfortable but risk-averse hobbit who
refuses to leave his comfortable surroundings when Gandalf first visits to
send him on a quest. Of course, he does eventually leave home to fulfill the
quest, returning home a changed character.

Campbell argues that the \emph{monomyth} acts as a shared cultural memory of
sorts, being the universal template for the rite-of-passage tale. He says that
it is passed down as myth through many different cultures, replicating in much the same way
as genes (or memes) do and that it can be thought of as a \emph{metamyth}, or the
spiritual history of humanity.

% Polti 27
% Look at story shapes paper for inspiration
Other analysts have described several different types of commonly-seen plot.
\citet{harris1959basic} makes the case for just three types of plot, having
either a happy ending (where the protagonist is virtuous), an uphappy ending
(with a selfish protagonist) or being a tragedy (where the protagonist is struck
by fate).

\citet{booker2004seven} describes seven basic plots, where each may have either a
happy or unhappy ending:
\begin{itemize}
  \item Overcoming the monster
  \item Rags to riches
  \item The quest
  \item Voyage and return
  \item Comedy
  \item Tragedy
  \item Rebirth
\end{itemize}

Georges Polti famously divides narratives into the \emph{thirty-six dramatic
  situations}~\citep{polti1921thirty}, derived from his analysis of classical
Greek and both classical and contemporary French texts. Polti's thirty-six types
of plot are:

\begin{multicols}{2}
\begin{enumerate}
  \item Supplication
  \item Deliverance
  \item Crime pursued by vengeance
  \item Vengeance taken for kin upon kin
  \item Pursuit
  \item Disaster
  \item Falling prey to cruelty/misfortune
  \item Revolt
  \item Daring enterprise
  \item Abduction
  \item The enigma
  \item Obtaining
  \item Enmity of kin
  \item Rivalry of kin
  \item Murderous adultery
  \item Madness
  \item Fatal imprudence
  \item Involuntary crimes of love
  \item Slaying of kin unrecognized
  \item Self-sacrifice for an ideal
  \item Self-sacrifice for kin
  \item All sacrificed for passion
  \item Necessity of sacrificing loved ones
  \item Rivalry of superior vs. inferior
  \item Adultery
  \item Crimes of love
  \item Discovery of the dishonour of a loved one
  \item Obstacles to love
  \item An enemy loved
  \item Ambition
  \item Conflict with a god
  \item Mistaken jealousy
  \item Erroneous judgment
  \item Remorse
  \item Recovery of a lost one
  \item Loss of loved ones
\end{enumerate}
\end{multicols}

\begin{figure}[!t]
\centerline{\includegraphics[height=2in]{vonnegut.png}}
\caption{Two of Kurt Vonnegut's ``Shapes of Stories''}\label{fig:vonnegut}
\end{figure}

Author Kurt Vonnegut describes several different ``shapes'' of stories in his
rejected Master's thesis~\cite{vonnegut2009palm}. He plots the events of a story
as points in a 2D space with ``Beginning-End'' (time) on the X-axis and ``Ill
Fortune-Great Fortune'' (fortune of protagonist) on the Y-axis.
Figure~\ref{fig:vonnegut} shows two examples of story shapes. In the ``Man in
Hole'' example, things start out well for the story's protagonist, but then some
misfortune falls on them. They spend much of the story overcoming challenges
before finally ending the story with a triumph. In the ``Old Testament''
example, the protagonist(s) experience gradually increasing levels of fortune,
before finally being ``cast down'' to a grave level of misfortune at the end of
the story.

All of the above categorisations describe stories as a whole, sorting narratives
into one of several different ``types'' of story. This works well for
linear narratives within traditional media, but what about experimental and
non-linear narratives? In the next section, we review the literature relating to
these less traditional types of stories.

\subsection{Types of Narrative}
% Author Kurt Vonnegut
% Look at Cybertext, etc, and try to explain how best to divide different types
% of story
The rise of the Web in the 1990s brought with it great interest in the future of narrative in cyberspace. Aarseth's work, \emph{Cybertext} \citep{aarseth1997cybertext} describes the creation of a new form of narrative, for which he coins the term \emph{ergodic literature} (from the Greek words \emph{ergon} and \emph{hodos}, meaning `work' and `path'). In this new form of narrative, some amount of work or effort is required by the reader in order to traverse the path that the story takes.

% scriptons/discourse textons/fabula (Bal 1997)
Aarseth makes a distinction between the narrative as written by the author, and the way in which it is traversed by the reader, calling the former \emph{textons} and the latter \emph{scriptons}. In ergodic literature, the \emph{scripton} is produced by the effort that the reader goes through in interpreting the \emph{texton}. In the context of a game, it is as though the game interface is a gateway that allows access to the narrative at different times. Using classical music as a metaphor, the texton can be thought of as the \emph{score}, and the scripton the \emph{performance}.

% TODO Make sure you update the intro with this
In section~\ref{sec:generative-and-interactive-narrative}, we assert that how generative a narrative is and its level of interactivity are two different variables in an experimental narrative. However, Aarseth identifies seven different methods of story traversal: \emph{dynamics, determinability, transiency, perspective, access, linking and user function}.

\textbf{Dynamics} describe whether or not the content and number of scriptons changes. In a simple, static story with branching choices (such as in a \emph{Choose your own adventure} story), both the number of textons and scriptons are fixed, since all paths have been written out beforehand. A dynamic story would still have a fixed number of textons, but the scriptons would be generated as the user traverses the path of the narrative.

\textbf{Determinability} is how deterministic the narrative is, whether or not the same interactions will result in the same scripton being produced.

\textbf{Transiency} means to what extent scriptons are produced as time flows, or whether user interactions are required to produce them.

\textbf{Perspective} is whether or not the user/reader plays a role as a character in the narrative.

\textbf{Access:} if a user has access to all scriptons at any point in traversing the narrative, or whether their access is restricted.

\textbf{Linking:} whether or not parts of the scripton are linked to other parts, and whether these links are conditional (if they rely on a user having already traversed part of the scripton).
\textbf{User functions:} the functions the user uses to traverse the text. This could be interpretive (which is implicit in any traversal of the text), explorative (traversing the scripton according to whim) or configurative (specifying parts of the scripton in advance), for example.

% Ugh, this is all so arbitrary. Go on to describe Aarseth's PCA of these variables and explain why you don't think it's a good fit.

By performing correspondence analysis (a process similar to principle component analysis) on a diverse corpus of 23 texts ``\emph{ranging from ancient China to the Internet}'', Aarseth filters these seven variables down into two numerical axes which account for 49 percent of the variation between stories. Using these axes, he groups classic tales such as \emph{Moby Dick} and more experimental narratives such as William Gibson's \emph{Agrippa} and Michael Joyce's \emph{Afternoon}. By grouping these stories into categories, he intends to show how emerging media are enabling new types of story.

Chris Crawford's \emph{Chris Crawford on Interactive Storytelling} \citep{crawford2012chris} provides a scathing assessment of the relationship between narrative theory and computer science. A veteran of the games industry, he argues that `soft' science theories such as those of Aarseth et al are entirely removed from `hard' science, and are therefore an example of bubble intellectualism and impossible to implement. 

Crawford himself provides a useful examination of experimental narrative in computer games, defining interactivity as:

\begin{quote}
A cyclic process between two or more active agents in which each agent alternately listens, thinks, and speaks.
\end{quote}

He argues that for game narratives to be truly interactive, they must be more social. Characters in a story must be able to react with the player as though they were people in real life. In turn, the player should have some degree of freedom in the way in which they interact. Rather than presenting branching story points as choices, a better way to interact would be socially, through talking to agents in the game. This is the approach that Fa\c{c}ade takes \citep{mateas2003faccade}, which Crawford acknowledges as the most successful attempt at interactive storytelling to date. A detailed description of Fa\c{c}ade's implementation appears in section \ref{sec:modelling-agents}.

In order to determine whether Crawford's assertion that narratology research is
too far removed from its practical implementations to be of use, we provide an
overview of these implementations and their underlying research in section \ref{sec:implementations}. Has narrative theory research informed the creation of computer-generated or interactive narrative at all, or do they all take approaches grounded in computer science and artificial intelligence? If narrative theory has not been used, then we must ask: why not?


\subsection{Structuralist Formalisms of Narrative}
% Propp, etc
Attempts to organise recurring themes, roles and motifs of narrative go back at least a century. The Aarne-Thompson tale-type index \citep{aarne1987types}, first published in 1910 and later refined by Stith Thompson in 1928 and 1961, is well known amongst folklorists as a classification and analysis method for traditional folktales and myths. Aarne-Thompson's index is a taxonomy of tale themes, arranging tales into categories such as \emph{animal tales} and \emph{jokes and anecdotes}, and then sub-categories (\emph{tales of magic} and \emph{numskull [sic] stories} being two examples). This taxonomy is only two levels deep however, and only serves as a useful way to categorise individual stories or tales. In order to break down and analyse components of tales, we must dig deeper.

In \emph{Structural Anthropology}, Claude L\'{e}vi-Strauss seeks to discover why myths and legends are so similar across cultures and history \citep{levi1963structural}. He concludes that there are global laws that govern the way in which people create stories, therefore these laws can be modelled as a set of rules for describing myths.

His theory is that myths describe opposing forces which are resolved through mediation. The example he gives in \emph{Structural Anthropology} describes how Native American legends often contain `trickster' characters in the form of ravens or coyotes. As scavenging animals, these tricksters symbolically act as mediators between life and death.

Like much of early narrative theory, there is no rigorous evaluation of L\'{e}vi-Strauss' ideas, leaving them seeming opinionated and arbitrary. While interesting, L\'{e}vi-Strauss' ideas bring us no closer to developing a formal model of narrative structure. For that, we must go even further back in time, and turn to Vladimir Propp.

\subsubsection{Propp's Morphology of the Folktale}
A notable narrative structuralist is Vladimir Propp, creator of \emph{The Morphology of the Folktale}~\citep{propp1968morphology}, a formalism for Russian folktales. Propp's formalism, though originally limited in scope, generalises well, and is still used by researchers to procedurally generate stories~\citep{grasbon2001morphological,gervas2005story,hartmann2005motif}. Drawing from a corpus of one hundred Russian folktales, Propp identifies thirty-one distinct \emph{story functions}, each of which is identified by a number and symbol. These functions are executed by characters following certain roles, each of which has a \emph{sphere of action} consisting of the functions that they are able to perform at any given point of the story. Stories are created by chaining story functions together, with subplots expressed as parallel chains of story functions.

In this formalism, characters have \emph{roles}, such as \emph{hero}, \emph{villain}, \emph{dispatcher}, \emph{false hero}, and more. Characters performing a certain role are able to perform a subset of \emph{story functions}, which are actions that make the narrative progress. For example, the \emph{dispatcher} might send the \emph{hero} on a quest, or the \emph{victim} may issue an \emph{interdiction} to the \emph{villain}, which is then \emph{violated}.

Propp defines a total of 31 distinct story functions, each of which is given a number and symbol in order to create a succinct way of describing entire stories. Examples of such functions are:

\begin{itemize}
  \item One of the members of a family absents himself from home: \emph{absentation}.
  \item An interdiction is addressed to the hero: \emph{interdiction}.
  \item The victim submits to deception and thereby unwittingly helps his enemy: \emph{complicity}.
  \item The villain causes harm or injury to a member of the family: \emph{villainy}.
\end{itemize}

Each of these functions can vary to some degree. For example, the \emph{villainy} function can be realised as one of 19 distinct forms of villainous deed, including \emph{the villain abducts a person}, \emph{the villain seizes the daylight}, and \emph{the villain makes a threat of cannibalism}.

\begin{figure}[!t]
\centerline{\includegraphics[height=0.4in]{propp1.png}}
\caption{One Propp function following another}\label{fig:propp1}
\end{figure}

\begin{figure}[!t]
\centerline{\includegraphics[height=0.6in]{propp2.png}}
\caption{Multiple simultaneous functions}\label{fig:propp2}
\end{figure}

In a typical story, one story function will follow another as the tale progresses in a sequential series of cause and effect (figure~\ref{fig:propp1}). However, Propp's formalism also allows for simultaneous story functions to be occuring at once (figure~\ref{fig:propp2}).

Though flexible, Propp's formalism is limited in its expressiveness. All story functions describe events at the same level of abstraction, describing one event after another. Also, Propp insists that the story functions occur in a prescribed order. Later French structuralists such as \citet{bremond1980logic}, \citet{greimas1983structural} and~\citet{todorov1969grammaire} address the latter problem by generalising Propp's work outside of Russian Folktales, though each represents only incremental improvements on Propp, lacking a means of nesting story functions to create abstractions. \citet{barthes1975introduction} broadly describe hierarchically composing \emph{narrative units} but, lacking implementation details, these can only be used as a template from which to build a new narrative model.
% What are the shortcomings of Propp? (i.e. lack of abstractability, etc)

% \subsection{Describing Stories with Logic}
% Laure-Ryan did a bit of this, also include linear logic approaches

\subsection{Other Types of ``Story Component''}
Lehnert's \emph{plot units} are a more recent narrative formalism
\cite{lehnert1981plot}. However, these plot units only describe stories as three
types of event: positive, negative and mental. These events occur with respect
to a single character in the story, so an author must always author story
components with concrete characters in mind, making them difficult to re-use.
Similar to Propp's system, the order of composition must always be in a certain
sequence, and plot units cannot refer to other plot units. Again, we are left
without a means of creating abstractions for our story components. In the
``TropICAL: a DSL for Tropes'' section, we address this issue by describing how tropes allow the nesting of components to allow story authors to create their own abstractions.

\section{Implementations of Experimental Narrative}
\label{sec:implementations}
% I've plenty of material, but it really needs reworking and extending

\subsection{Story Generation}
% TaleSpin, etc
Inspired by Chomsky's theories of generative grammar \citep{chomsky1968sound}, researchers in generative narrative strive to build their own `universal grammar' for narrative.

\begin{figure}[!t]
  \begin{center}
  \begin{enumerate}
    \item $\texttt{Attempt}\rightarrow \texttt{Plan} + \texttt{Application}$\\
           $\qquad\Rightarrow\texttt{MOTIVATE(Plan, Application)}$
    \item $\texttt{Application}\rightarrow\texttt{(Preaction)*} + \texttt{Action} + \texttt{Consequence}$\\
           $\qquad\Rightarrow\texttt{Allow(AND(Preaction,Preaction,...),}$\\
           $\qquad\{\texttt{CAUSE | INITIATE | ALLOW}\}\texttt{(Action,Consequence)}$
  \end{enumerate}
  \end{center}

  \caption{Example rules from Rumelhart's story grammar}\label{fig:rumelhart}
\end{figure}

\citet{rumelhart1975notes} provides one early and influential model for the grammatical generation of natural language. Figure \ref{fig:rumelhart} shows two example rewrite rules from this grammar. The `+' symbol denotes items happening in a sequence, the `\textbar' showing possible alternatives. `*' denotes one or more item being generated. Capitalised words (such as ALLOW, MOTIVATE, etc) describe relationships between items. For example, MOTIVATE is a relationship between a character's thought and their reaction to that thought.

Other systems draw inspiration from generative grammar, such as GESTER \citep{pemberton1989modular}, which generates stories based on a grammar synthesised from old French epic tales. \citet{lang1999declarative} describes a declarative model for narrative, consisting of lists of first-order predicate calculus expressions. These expressions describe events, states, goals and beliefs which combine to form a narrative. More specifically, it combines:

\begin{itemize}
  \item A \textbf{grammar interpreter} to search for a sequence of grammar rewrites which would produce a convincing narrative.
  \item A set of \textbf{temporal predicates} to describe the occurence of events over time and enforce temporal constraints on story components.
  \item A \textbf{world model} which describes the set of actions that characters may perform and fluents that may alter over the course of the narrative.
  \item A \textbf{natural language output unit}, which takes the sequence of events produced by the story grammar and converts it into readable natural language sentences.
\end{itemize}

This combination of using a grammar interpreter, world model and natural language output unit is especially common amongst generative grammar approaches.

While generative grammar approaches may be effective for procedurally creating
prose, they are less well suited to the creation of \emph{interactive}
narratives. Once the grammar rewrite rules are specified, the user is entirely
passive, unable to affect the way in which the story is being generated. For
this to happen, the narrative needs to be part of a system that reacts to the
actions of the user, such as in a computer game.

\begin{figure}[!t]
\begin{quote}
  ONCE UPON A TIME GEORGE ANT LIVED NEAR A PATCH OF GROUND. THERE WAS A NEST IN AN ASH TREE. WILMA BIRD LIVED IN THE NEST. THERE WAS SOME WATER IN A RIVER. WILMA KNEW THAT THE WATER WAS IN THE RIVER. GEORGE KNEW THAT THE WATER WAS IN THE RIVER. ONE DAY WILMA WAS VERY THIRSTY. WILMA WANTED TO GET NEAR SOME WATER. WILMA FLEW FROM HER NEST ACROSS THE MEADOW THROUGH A VALLEY TO THE RIVER. WILMA DRANK THE WATER. WILMA WASN'T THIRSTY ANYMORE.

GEORGE WAS VERY THIRSTY. GEORGE WANTED TO GET NEAR SOME WATER. GEORGE WALKED FROM HIS PATCH OF GROUND ACROSS THE MEADOW THROUGH THE VALLEY TO A RIVER. GEORGE FELL INTO THE WATER. GEORGE WANTED TO GET NEAR THE VALLEY. GEORGE COULDN'T GET NEAR THE VALLEY. GEORGE WANTED TO GET NEAR THE MEADOW. GEORGE COULDN'T GET NEAR THE MEADOW. WILMA WANTED TO GET NEAR GEORGE. WILMA GRABBED GEORGE WITH HER CLAW. WILMA TOOK GEORGE FROM THE RIVER THROUGH THE VALLEY TO THE MEADOW. GEORGE WAS DEVOTED TO WILMA. GEORGE OWED EVERYTHING TO WILMA. WILMA LET GO OF GEORGE. GEORGE FELL TO THE MEADOW. THE END.
\end{quote}
\caption{Example TALE-SPIN output}\label{fig:tspin}
\end{figure}

% TODO elaborate on this
James Meehan's TALE-SPIN \citep{meehan1977tale} is an influential early approach to story generation using planning. In TALE-SPIN, the author describes a story domain, its characters and their goals, and a natural language story is produced as output. It works by using a problem-solver to resolve each character's goals over the story domain. Figure \ref{fig:tspin} is an example of TALE-SPIN's output.

TALE-SPIN's strong planning component is evident in the reading of its output. Sentences are terse, with one event leading directly to another in order to achieve some goal. One problem with this character-led approach is that the goals of the author are not necessarily taken into account. If the author intends for a character to die at some point in the story, it seems unnatural for a character to have the goal of dying to fulfill this intention.

 Turner criticises TALE-SPIN's stories as seeming ``pointless and somewhat boring'' \citep{turner1986thematic}, going on to create the MINSTREL system for story generation \citep{turner1993minstrel}. Using the legendary world of King Arthur's court as a story domain, MINSTREL strives to generate stories with an authorial purpose.

MINSTREL attempts to address TALE-SPIN's shortcomings by introducing two types of schema: author-schemas and character-schemas, both of which combine to represent the elements of a story. The author-schemas describe the goals of the author of the system, allowing story creators more control over the structure and content of their narrative. This allows authors to specify a `point' or moral to their story, something that is not possible to achieve with TALE-SPIN. Character-schemas describe character-level goals in a similar manner to those of TALE-SPIN's.

The comparison of TALE-SPIN and MINSTREL highlights a challenge that has
dominated story generation for decades: the balance of \emph{character} and
\emph{plot} (as~\citet{riedl2010narrative} highlights). Especially with approaches based on multi-agent systems, the regulation of character actions to conform to an underlying theme or structure is a challenging problem.

However, modelling characters with agents is a promising approach to take in order to achieve a story which is both generative and interactive. In such a system, an author can specify the story world and character models, creating the `big picture', and the agents would be able to fill in the details (such as dialogue, sub-plots, and relationships).
An apt metaphor would be that of animation. In large animation studios such as
Disney, the lead animator draws the key frames of a sequence, and a team of
other animators work to fill in the gaps in
between\footnote{This process is described at the following website:
  \url{http://www.justdisney.com/animation/animation.html}, accessed 20160805.}. This is what the combination of a managed narrative with agents could achieve: the author would be the `lead animator' in such a system, with the agents being the assistants.

The implementations until now have focused mainly on \emph{generation\/}, and little on \emph{interaction\/}. Character models have been mentioned, but these do not react in real time to a user. For that, we need a multi-agent system.

\subsection{Characters as Intelligent Agents}
Story worlds are usually populated with characters. Interactive story worlds
such as games contain characters with very basic scripted behaviour. At even the
highest level of game character simulation, AI techniques are usually used to
govern basic behaviour such as movements and actions. Governing character
behaviour to fit within the context of a narrative is a more challenging
problem. This section examines different approaches to tackling this challenge.

\subsubsection{Planner-based Systems}
The most prevalent approach to the generation and management of plot in
interactive narrative is to use planners. With a planner, an author sets the
goals for the story (certain situations that they would like to see happen), and
the planner tells the character agents what to do to make sure these ``story
goals'' are achieved. When a player that is interacting with the system takes an
action that compromises the story goals, the planner must re-plan to make sure
the goals can still be achieved, by restricting the actions of the player or
intervening in some other manner.

% This needs elaboration
\citet{young1999notes} argues that planners are a good method for regulating
plot, later creating the \emph{Mimesis} architecture for integrating a planner with character agents in an interactive game environment, \citep{young2004architecture}. Young describes how narrative systems must re-plan when a player makes  narrative-breaking actions, by either restructuring the narrative mid-story (\emph{accommodating} the action) or preventing the action from executing (\emph{intervening} on the action).

Given its influence over subsequent approaches to interactive narrative
generation, it is worth looking at the Mimesis architecture more closely.
It is designed to integrate into the \emph{Unreal
  Tournament} game engine, and has five components: the \emph{mimesis
  controller}, the \emph{story planner}, the \emph{discourse planner}, the
\emph{execution manager} and the \emph{MWorld}.

Figure \ref{fig:mimesis} shows how these components work together to form the
Mimesis architecture. Once an author has created the pre-defined libraries of
actions needed by the story planner, the following steps are taken to determine
the course of the story:

\begin{enumerate}
  \item All the components connect to the Mimesis Controller (MC) via socket
    connection. It then acts as a message router.
  \item The game initiates a plan request containing the state of the game
    world, a list of possible actions, and the goals for the plan.
  \item The \emph{story planner} responds with a \emph{story world plan}, a data
    structure that describes the actions (selected from the list) that must occur over time in order to
    meet the plan request's goals.
  \item Once the story world plan is created, it is sent to the \emph{discourse
      planner}, along with a list of actions that can occur in the game engine
    (such as camera movements, voice-overs and background
    music). The discourse planner then creates a sequence of these actions that
    best fit the story world plan.
  \item The discourse planner then sends the narrative plan to the
    \emph{execution manager}, which builds a directed acyclic graph (DAG), where
    the nodes are the actions within the plan and the edges are the temporal
    constraints between the orderings of the actions. The execution manager
    removes nodes from the DAG in order, sends the node's actions to the
    game engine, and updates the graph.
  \item The \emph{MWorld} is essentially the environment in which the story
    occurs, consisting of the game engine, coordination code, and class
    definitions for actions, and discourse planners. It is the MWorld that
    receives actions from the execution manager to be executed, and executes
    these actions in the game engine.
\end{enumerate}

Mimesis uses DPOCL (Decomposed Partial-Order Causal-Link Planner,~\citep{young1994decomposition}) plans for
its story planning. DPOCL plans are composed of \emph{steps} (the plan's
actions), \emph{ordering constraints}, \emph{decomposition links} describing the
hierarchical structure of a plan, and \emph{causal links} between pairs of
steps. It uses \emph{refinement search} \citep{kambhampati1995planning} as its
plan reasoning process, searching through the space of possible plans represented as a directed
graph, with each node in the graph being a plan or partial plan. Mimesis
specifies the initial planning problem for DPOCL using the current and goal
states of the story.

A key feature of Mimesis is its strategies for handling of user actions which potentially
interfere with the story plan, making its goals unachievable. In the intervention strategy, Mimesis simply prevents the
user's action from having any effect in the game world. With accommodation,
Mimesis replans the story events, restructuring the plan so that the interfering
actions are taken into account and worked around.

\begin{figure}[!t]
\centerline{\includegraphics[height=2in]{mimesis.png}}
\caption{The Mimesis architecture, from \cite{young2004architecture}}\label{fig:mimesis}
\end{figure}

% This needs elaboration
\citet{cavazza2002character} et al's \emph{I-Storytelling} system implements
\citet{young2004architecture}'s architecture with ideas from Barthes' narrative
units~\citep{barthes1975introduction}, using characters with behaviour described
by Hierarchical Task Networks (HTNs) to generate its stories. Each character has a main task, which is divided into subtasks to create a task hierarchy, with each task node having pre- and post-conditions. The story emerges from the outcomes of each character's plans, and the narrative structure as a whole is not planned.

% This needs elaboration
\citet{riedl2003managing} describes a further development of Young's architecture,
allowing the story author to create plans for the overall narrative in addition
to its characters.
Rather than using a narrative model, this version of Mimesis models the player to track their expected level of suspense while interacting with the story. Like the \emph{I-storytelling} system, its plans are hierarchical, using the Longbow hierarchical partial-order causal link planning system~\citep{young1994decomposition}.

A disadvantage of these planner-based systems is that they require the story
author to think in a planner-oriented manner. They must consider the goals of
both the story and the character, plans to achieve these goals, and re-planning
when goals are not met or when situations change. This is a drastic change from
the usual story writing methods of authors, where the focus is on structure,
plot, themes and characters. Though graphical user interfaces such as Mimesis'
Bowman system~\citep{thomas2006author} could be used to assist plan-driven
story authoring, a complete shift in creative workflow is still required, making
this approach inaccessible to non-technical writers.

% This needs elaboration
\subsubsection{Drama manager-based Systems}

Carnegie Mellon University's OZ project \cite{mateas1999oz} uses a \emph{drama manager} to structure its narrative, which observes the actions occurring in the storyworld and ``directs'' its characters to conform to shape the story. 

Mateas and Stern's \emph{Fa\c{c}ade} has players interact with the characters of the story through natural language. In this game, the player attends the party of a young couple (Grace and Trip) celebrating their wedding anniversary. As the course of events unfold however, the player learns that all is not as happy as it seems.

The player interacts with the characters by typing in natural language sentences, to which Grace and Trip respond. Though the characters are implemented through agents, the story is controlled using a drama manager. In all, their system consists of using NLP, a novel character authoring language and a novel drama manager to create an interactive narrative.

Several custom-designed languages were used to create the game, including a language called `A Behaviour Language' (ABL) for the agents and a special language for the sequencing of the beats. ABL represents situations as character goals, maintaining a tree of all the active goals and behaviours that are happening at any time.

In Fa\c{c}ade, the smallest unit of narrative action is called a \emph{story beat}, taken from McKee's book on authorial style for screenwriters \citep{mckee1997substance}. The simulation constantly monitors what the user is doing and how it may lead from the current story beat to another. Story beats have preconditions and effects on the state of the narrative, so it is the drama manager's job to work out when it makes sense to initiate a certain beat.

`Beats' have a very fine granularity, with 200 or so updating every minute of the simulation. They consist of a set of ABL behaviours, which advance the narrative yet still allow interaction to change to other beats. Only one beat can be active at a time.

A beat can have 5 types of goal:

\begin{enumerate}
  \item transition-in: characters express their intentions
  \item body: a dramatic question/situation is posed to the player
  \item local/global mix-in: react to the player before end of the beat
  \item wait-with-timeout: wait for the player's reaction
  \item transition-out: final reaction to the player's action in the beat
\end{enumerate}

A beat goal is a series of steps for an agent to perform, which can be:

\begin{itemize}
  \item staging (where to walk to, face)
  \item dialogue to speak
  \item where and how to gaze
  \item arm gestures to perform
  \item facial expression to perform
  \item head and face gestures to perform
  \item small arm and head emphasis motions triggered by dialogue (head nods, hand flourishes)
\end{itemize}

As an example, there is a behaviour called ``Fix\_Drinks'', which specifies a sequence of agent behaviours where the characters Grace and Trip have an argument while Trip asks the player what they would like to drink. If the player decides not to go along with the beat (in this case, by not choosing a drink), then the beat will be aborted and replaced with another.

Fa\c{c}ade has become popular as a game outside of academia, with playthroughs of the game reaching millions of views on Youtube. This shows the promise of interactive narrative as being a unique and engaging new form of entertainment. Unfortunately, no other implementation of interactive narrative seems to have captured the public imagination since the release of Fa\c{c}ade.

Fa\c{c}ade's popularity seems to reinforce Crawford's assertion (section \ref{sec:media}) that interactive narratives must be social in nature. The gameplay comes entirely from the conversations and interactions between Grace, Trip and the player. Much of the excitement comes from the social consequences of certain conversation paths or actions. By modelling characters as agents, Mateas et al have created a truly interactive experience. However, by also using a drama manager to manage the agents, they have used these agents to tell a story.

How might these agents be made more convincing? Outside of writing rules for their behaviour consisting of character goals and beliefs, how might an author create truly unique and idiosyncratic characters? To address the question, I next examine different types of emotional models in psychology, and how each might be used to model characters as agents.

\subsubsection{Social Norms}
Versu~\cite{evans2014versu} is an interactive drama system that uses a multi agent system as characters. The characters' actions are coordinated with \emph{social practices}, which describe types of social situations and is described by the authors as a successor to the Schankian script. These social practices are implemented as reactive joint plans, which agents can choose to participate in or not. Rather than directly telling the agents what to do, these social practices merely \emph{suggest} courses of action, leaving each agent to decide for itself what to do based on its individual goals.

The authors decide against using a drama manager to control the agents' actions because they want to take the \emph{strong autonomy} approach to agent governance. This means that they prefer to give each agent some degree of autonomy by allowing it to make the final decision on which course of action to take, rather than blindly following a drama manager. Suggesting actions with social norms achieves this goal. Rather than describing typical story events in terms of social norms, however, in Versu the social norms \emph{are} the story. The gameplay revolves around the avoidance (or purposeful subvertion of) awkward social situations.

Each character has a role, which is governed by a social practice. For example, a \emph{greeting} practice involves characters with the \emph{greeter} and recipient roles. The greeting practice would tell the greeter in which manner they are to greet the recipient, and the recipient how to respond. It is noteworthy that these actions are merely suggested, and not enforced.

\emph{Exclusion logic}~\cite{evans2010introducing} is used to describe the
social practices of the system. Exclusion logic manages the frame problem by
organising related fluents in a tree structure. A single update at the right
branch can change a set of related facts en masse. For example, a description of a character called ``Brown'' is shown in listing \ref{lst:exclusive}. It describes the building up of character attributes as a tree structure.

Exclusion logic aims to address the frame problem. The frame problem is the uncertainty around whether predicates that change over time (fluents) change other predicate values. It aims to address this through use of an exclusion operator (``!''). Listing \ref{lst:exclusion} shows an example of the exclusion operator in use. The example specifies that an agent can have only one gender. The \emph{Praxis} language implementation of the exclusion logic has a type checker which ensures that no character can have multiple genders.

\begin{lstlisting}[float,label=lst:exclusion,caption=nextHopInfo: caption]
A(agent).sex!G(gender).
\end{lstlisting}

\begin{lstlisting}[label=lst:exclusion,caption=Description of ``Brown'' character.]
brown.sex!male;
brown.class!upper;
brown.in!dining_room;
brown.relationship.lucy.evaluation.attractive!40;
brown.relationship.lucy.evaluation.humour!20.
\end{lstlisting}

Exclusion logic's exclusion operator allows an author to express the fact that a variable can only have one value. For example, if `the `Brown'' character changes location from the dining room to the kitchen, \emph{brown.in!dining\_room} is terminated when \emph{brown.in!kitchen} holds.

Versu takes the \emph{constitutive} view of social practice, as opposed to the
\emph{regulative} view~\citep{SJP:SJP658}. This means that rather than restricting an agent's possible actions based on its permissions and obligations, they participate in a certain social practice by taking an action. Their actions are only restricted by what is possible in the story world, and what the agent desires to do. This way, agents can choose whether or not to take part in certain social interactions.

Many of the components we aim to have in our story telling system appear in Versu: the use of social norms to gently encourage story-conforming behaviour rather than demanding it, and the use of formal logic to determine which behaviours are possible. However in Versu the social norms \emph{are} the story, rather than describing the story components that invisibly govern the behaviour of characters. In order for this kind of governance to occur, an institution-based solution is preferable, based on events, agent actions and standard deontic logic. Because character actions are constrained by the structure of a story, a \emph{regulative}  view of social practice is more suited to the expression of story components as social norms.

Many of the advantages of using exclusion logic can be gained by using an institution-based approach. Non-inertial fluents can be used to ensure that variables can only ever have one value. Standard deontic logic is enough to provide the rest of what is needed.

% Need at least one more example here
\subsection{Modelling Narrative with Logic}
\label{sec:model-logic}
Although most recent research focuses on the use of planners to manage the drama in a story, there is also much interesting work which makes use of formal logic to model narrative. Though often used for the generation of linear story text, it is increasingly being applied to non-linear narratives as well. Logic-based approaches are generally based on either temporal logic variants or some kind of linear logic.

Ceptre~\citep{martens2015ceptre} is a language for modelling generative interactive narratives using \emph{linear logic}, a formal logic designed to describe resource usage. 

A Ceptre story begins with an initial state $\Delta_0$. Each state iteration
$\Delta_i$ is examined repeatedly, and a subset $S$ of it is updated with rule
$r$. The next state, $\Delta_{i+1}$, has the subset $S$ replaced with $S'$, the
new subset with the consequences of the applied rule $r$.

The rules are specified using the combination of logical statements with two
operators: $*$ (tensor) and $\text{-o}$ (lolli). The tensor operator is used to
concatenate statements, while the lolli operator expresses state transitions in
the form $S \mathrel{\text{-o}} S'$. The rules use \emph{replacement semantics},
which means that everything from state $S$ will disappear unless stated to be in
state $S'$. A $\$$ operator is used to mark facts in $S$ that the author wishes
to remain in $S$ without explicitly stating so.

Listing \ref{lst:ceptre-murder} shows an example from~\cite{martens2015ceptre}
that describes a ``murder'' rule and its consequences.

\begin{lstlisting}[label=lst:ceptre-murder]
do/murder
    : anger C C' * anger C C' * anger C C' * anger C C' *
    $at C L * at C' L * $has C weapon
    -o !dead C'.
\end{lstlisting}

In this case, four instances of the ``anger'' predicate with the same arguments
has a significant meaning: a character's emotion is treated as a resource. The
fact that \emph{anger $C C'$} appears four times means that a character is
\emph{four times} as angry at another character. Depending on how many times the
``anger'' statements appear in the new state, this anger can rise or fall at the
next step in the story. In this case, the emotion is not only treated as a
pre-condition, but also as a resource. These resources can also be specified
through the addition of a number to the name of the resource.

Ceptre introduces a \emph{stages} feature that allow authors to structure a
program through the use of independent components. A stage is a unit of
computation that runs to quiescence, meaning that it terminates when no more
rules are able to fire. At this point, another stage may commence.

The central motivation behind Ceptre's design is its ability to use ``proofs as
traces'', or \emph{computation as proof search}~\cite{hodas1991logic}. Ceptre
uses a \emph{sequent calculus}, where a sequent $\Delta \vdash A$, $\Delta$ is a
state, and $A$ is a goal formula. If a complete proof tree can be formed with
that sequent as its root, then the sequent can be said to be \emph{derivable}.
Thus Ceptre takes a sequent as input and creates a proof as output, declaring
failure if a proof cannot be created. Ceptre looks at the left side of the
sequent, using \emph{forward chaining} to choose which rules to try in order to
reach the goal formula.

A key feature of Ceptre is its representation of resource management in stories.
Rules are able to either produce or consume resources. This has interesting
implications for the representation of causal structure in linear logic. For
example, if two rule applications consume different sets of resources from the
same state, they are occurring concurrently and independently. However, if one
rule produces resources that are consumed by another rule, then these rules have
a causal, dependent relationship.

This modelling of gameplay as proof search is similar to the technique we use in
section~\ref{sec:thn}, where we use Interval Temporal Logic and Kripke structures to
theorem-prove certain narrative states. The difference is that the system we
describe is more concerned with representing different temporal relations,
whereas Ceptre's focus is on resource management within a game.


\subsection{Character Modelling}

\subsubsection{Characters with Emotional Models}
% Intro: not done very much?

\subsubsection{Emotional models}\label{sec:emotional-models}
% How is this useful for narrative?
Usually it would seem odd to want to model emotion as part of a computational process. Emotion is such a seemingly irrational set of behaviours that they are easy to dismiss as `human imperfections'. However, as \citet{gratch2004domain} observe, emotions may have a useful role to play in communication, so long as they are displayed at appropriate times.

For example, anger prepares the human body to fight by increasing the manufacture of adrenaline. Fear similarly triggers the `fight or flight' response, alerting the senses for danger and preparing the body to react.

In order to model human emotions using agents, we must first find a suitable psychological model to use. Marsella et al describe three main types of emotional model:

\begin{enumerate}
 \item \textbf{Discrete} emotional models, which claim that humans have a set of innate, pre-defined emotional states which people may enter and leave.
 \item \textbf{Dimensional} models of emotion, describing the spectrum of emotions as being points somewhere in continuous space. Implementations typically use two or three dimensions for simplicity.
 \item \textbf{Appraisal} theories of emotion take an agent's mental processes into account. Their emotional state is derived from whether or not their goals have been achieved, and what effects current events are having on their circumstances, for example.
\end{enumerate}

% Give examples of concrete models for each type.
\subsubsection{`Basic' emotions}
Ekman first made a case for discrete, biologically-determined emotions, based on evidence from research into facial expressions \citep{ekman1992argument}. He describes emotions as being \emph{basic}, in two senses of the word: \emph{i.} that there are a number of distinct emotions, each with its own different characteristics, and \emph{ii.} that these emotions were developed through evolution for specific functions.

Ekman argues that these evolved emotions share nine characteristics:

\begin{enumerate}
  \item Distinctive universal signals
  \item Presence in other primates
  \item Distinctive physiology
  \item Distinctive universals in antecedent events
  \item Coherence among emotional response
  \item Quick onset
  \item Brief duration
  \item Automatic appraisal
  \item Unbidden occurrence
\end{enumerate}

These characteristics are shared by all of the `basic' emotions as observed in humans and primates.

Discrete models of emotion suggest that there is a neural basis for emotion. For example, Armony et al describe how the amygdala in the brain is responsible for conditioned fear responses  and create a neural network to model it \citep{armony1997computational}.

Using a discrete model of emotion for agent-based characters would be relatively simple. Each basic emotion could have its own distinct set of behaviours as postconditions, and triggering circumstances as preconditions.

However, a more fluid approach could be useful when modelling emotions with agents. It would be impossible to say that an agent is \emph{angry and approaching furious} using a discrete theory of emotion. Nuanced levels of emotion and even combinations of several emotions add an extra level of texture to a character. Dimensional and appraisal theories of emotion address this challenge.

\subsubsection{Russell's circumplex model of emotion}\label{sec:circumplex}
\begin{figure}[!t]
\centerline{\includegraphics[height=3in]{circumplex.png}}
\caption{Russell's circumplex model of emotion} \label{fig:circumplex}
\end{figure}

Russell's circumplex model of emotion is a well-known dimensional model \citep{russell1980circumplex}. In this case, the dimensional variables are \emph{valence} (how agreeable or otherwise a situation is to an agent) and \emph{arousal} (how excited an agent is).

Russell's original paper proposes a model similar to that shown in figure \ref{fig:circumplex}, where the $x$ axis is a person's valence level and the $y$ axis is their arousal level. He argues that the full range of human emotions lie as points along these axes. Eight such examples are shown in fig. \ref{fig:circumplex}.

This model is very easy to adapt to human-like agents. \citet{ahn2012nvc} adapt this model by adding a third dimension, dominance, to create conversational agents in a 3D environment. This `dominance' dimension was first proposed in Mehrabian and Russell's original work \citep{mehrabian1974approach}, but later removed due to being perceived as the consequences of the \emph{effects\/} of emotion \citep{russell1980circumplex}, rather than being a component of emotion itself. Like Ahn et al, I found it useful to add the dominance-submission dimension, and so left it in my emotional model. This is the approach I take in creating my Punch and Judy simulation, and so it is described in more detail in section \ref{sec:emotion}.

\subsubsection{Appraisal theory}
Appraisal theories of emotion lend well to simulation with agents, due to their taking a person's beliefs, desires and intentions into account with respect to external events. Emotions arise when an event occurs and a person internally \emph{appraises} its consequences with respect to their beliefs, desires and intentions. This fits well with the popular BDI architecture for intelligent agents.

Different methods of appraisal may be used in order to produce emotions. Gratch and Marsella use decision theoretic plans \citep{gratch2004domain}, but other approaches could include reactive plans, Markov-decision processes, or detailed cognitive models.

Though the Punch and Judy simulation described in section \ref{sec:punchjudy} uses a dimensional model of emotion, an appraisal-based model would be worth investigating due to its tight coupling with belief desire intention psychological models used in agents. I describe my intention to explore this area further in section \ref{sec:fappraisal}.

\subsection{Discussion}\label{sec:litrev-discussion}
Through this literature review, we have clearly identified these main issues in
need of attention:

\subsubsection{Character agents need some freedom to generate story details}
In most of the systems described in this literature review, the agents in a
story world are explicitly told what to do.

% Based on the analysis above, we hypothesise that using social institutions to
% govern the actions of character agents allows for more flexibility in the
% agents' actions. By 

\subsubsection{Story authors do not want to think in terms of goals}
The current dominant paradigm in interactive narrative creation is to
use planners to ``plan'' a narrative, and re-plan when a user interacts in
an unexpected manner. This does not align well with non-programmer story
authors, who are likely to be unfamiliar with planning systems. In order to
create these stories with planners, they would have to think of the story in
terms of story and character goals.
\subsubsection{Most narrative systems use outdated, inflexible story models}
There is an over-reliance on narrative formalisms such as
Propp~\citep{propp1968morphology} even in recent narrative generation
systems. Neither narrative nor AI research have produced a formalism for
narrative components that has endured. This is likely because Propp's formalism
is ``good enough'', and has worked for most researchers that have used it. There
is also likely to be a network effect, where Propp has become the formalism that
most researchers have heard of, and therefore the one that they end up using.

Though there have been attempts to create more expressive formalisms as
described in section~\ref{sec:formalisms}, none have been expressive enough to
overcome Propp's ``good enough'' properties. In order for a story model to
significantly improve upon Propp, it should add features that it and other
existing formalisms lack. We identify these features as:

\begin{itemize}
  \item A means of \emph{abstraction}
  \item Conceptually \emph{simple} enough for non-programmers to grasp
  \item A library of re-usable \emph{examples} for authors and researchers
\end{itemize}

\paragraph{A Means of Abstraction}
Current narrative formalisms lack a means of \emph{abstraction}. Propp,
for example, describes events in stories that all occur at the same level of
abstraction. This means that one is limited to describing events that occur one
after another, or in parallel, but not events that contain sub-events.

For example, suppose we define a ``Quest'' component of a story, which describes a
sequence of events that occur (such as the hero leaving home and then defeating
a monster). It is easy to think of other story components that could contain
this ``Quest'' component, such as ``Hero's Journey'', ``Rescue the World'' or
``Rite of Passage'' story components. These components could themselves be used
as part of other components. This gives us a means of creating abstractable,
reusable story components. Of all the story models described in
section~\ref{sec:formalisms}, none have any means of abstraction such as this.

The example we just described already hints at the use of story tropes that
contain other tropes. This is the means of abstraction and re-use that we use in
our system, described in section~\ref{sec:tropes}.

\paragraph{Conceptually Simple}
Story authoring tools are user interfaces which are used to write fiction. In
the computing world, user interfaces are often simplified through the use of an
\emph{analogy}. For example, a \emph{Desktop Environment} is a graphical user
interface analogy created by Xerox in the 1970s. Where computer users would
previously have had to learn and master a complicated command-line interface,
the desktop environment metaphor presented an interface that resembled something
that users were already familiar with: the top of a desk, with icons
representing files spread out over a ``desktop'' surface.

When creating a story model, it is easy to fall into the trap of creating
something complicated but arbitrary. As with Vonnegut's ``Shapes of Stories''
metaphor (described in section~\ref{sec:shapes}, sometimes the simplest
explanations are the easiest to grasp and most enduring. We believe that our
``trope'' analogy for modelling stories (section~\ref{sec:tropes}) is an
elegant, expressive and accessible way of describing parts of stories using a formalism.

Most story authors are already familiar with the concept of tropes. When
questioned about their familiarity with the idea, all the participants at an
interactive fiction meetup responded that they were familiar with the ``trope''
concept (sample size 19, see section~\ref{sec:tropes-simple}). This means that tropes are a
suitable analogy for the creation of a new narrative formalism.

\paragraph{A Library of Re-usable Examples}
Part of the problem of existing formalisms is that story authors that use them
need to write their own story components based on a formalism's description.
Though these formalisms are usually described in papers with one or two
examples, any story author would have to create their own formal descriptions,
even for commonly-occuring story components such as quests or \emph{The Hero's
  Journey}. What is needed is a library of pre-existing formal descriptions of
story components that authors can easily copy and paste into their stories.

As the concept of \emph{tropes} is one that is already well-known to story
authors and consumers of media, it is easy to find many examples of them in
reviews of books, films and computer games. In fact, there is a whole website
called ``TV Tropes'', which is dedicated entirely to the description of tropes
that recur throughout different types of media. This website takes the form of a
wiki, with a very active community who contribute trope descriptions along with
lists of the media they appear in.

For example, a trope called ``Karma Houdini'' has the following description:

\begin{quote}
The character has done a number of things that deserve a karmic comeuppance,
most importantly things that caused harm to the innocent.
But when the time comes for the hammer to fall, that's not what happens. At
least, not on him.
He doesn't get what he deserves. Instead, he gets away scot-free.
And he might even have reversed the Humiliation Conga that was being planned for him.

This is it. This is all there is to the story. The show is over. The book is finished. The author isn't going to write any more. The Word of God has been spoken. The guy has become a Karma Houdini.
\end{quote}

The site lists many examples from film and literature, including:
\begin{itemize}
  \item \textbf{Treasure Island:} Long John Silver escaped scot free with a chest of treasure, and was never caught. Not bad, for a month's murder and betrayal.
  \item \textbf{The Talented Mr Ripley:} Villain Protagonist Tom Ripley killed some people to assume a new identity and enrich himself thoroughly. In the sequels, he killed to protect his new life, and sometimes as favors for others. He never faced justice.
\end{itemize}

The \emph{TV Tropes} website is a Wikipedia of sorts for tropes. Since there are
already so many examples listed on its website, it reduces the work needed to
create a library of reusable formal descriptions of tropes. We have created such
a library, which is described in section~\ref{sec:library}.

These three omissions (means of abstraction, conceptual simplicity and a
re-usable library of components) from existing narrative systems are addressed
by our trope-based approach to interactive narrative generation, which is
described in further sections of this thesis. Section~\ref{sec:tropes} describes
the trope concept further, with examples. Section~\ref{sec:institutions}
explains how these tropes are used to govern agents in a multi-agent system. We
describe TropICAL, our domain-specific language for tropes, in
section~\ref{sec:tropical}. Section~\ref{sec:library} describes our creation of
a library of tropes and section~\ref{sec:policy} describe the application of our
\emph{tropes} concept to legal policies. The thesis concludes with an evaluation
in section~\ref{sec:evaluation} and conclusions in section~\ref{sec:conclusions}.

% insts

\chapter{Institutions as Story Worlds}
\label{cha:institutions}

% Intro about justification/inspiration for using institutions for stories

\section{Describing Stories With Logic}
Section~\ref{sec:model-logic} describes approaches that use formal logic to model interactive, non-linear narratives. Building on the work described in that section, we explore other possible ways with which to construct these kinds of stories.

% Describe why we will build stories with Propp and tropes (as a method of comparison)
\subsection{The Event Calculus}

\subsection{Modal Logic and Kripke Structures}

\subsubsection{Propp Example: Sausages and Crocodile Scene}\label{sec:pjexample}
The common elements of Punch and Judy are easily described in terms of Propp's story functions. Here we pick one scene to use as an example: the scene where Punch battles a Crocodile in order to safeguard some sausages.  In this scene, Joey the clown (our narrator) asks Punch to guard the sausages. Once Joey has left the stage, a Crocodile appears and eats the sausages. Punch fights with the Crocodile, but it escapes. Joey then returns to find that his sausages are gone.
The corresponding story functions are:
\begin{enumerate}
  \item Joey tells Punch to look after the sausages (\emph{interdiction}).
  \item Joey has some reservations, but decides to trust Punch (\emph{complicity}).
  \item Joey gives the sausages to Punch (\emph{provision or receipt of a magical agent}).
  \item Joey leaves the stage (\emph{absentation}).
  \item A Crocodile enters the stage and eats the sausages (\emph{violation}).
  \item Punch fights with the Crocodile (\emph{struggle}).
  \item Joey returns to find that the sausages are gone (\emph{return}).
\end{enumerate}

Some story functions map to Punch and Judy better than others (for example, it is debatable as to whether or not the sausages can be considered a ``magical agent''), but Propp's formalism seems well suited to Punch and Judy for the most part. The advantage of using Propp for the Punch and Judy story domain is that the story function concept maps well to the idea of internal events in institutional models.

\subsubsection{Combining Interval Temporal Logic and Modal Logic for Propp}\label{sec:propplogic}

Narrative construction can be described using two terms: \emph{fabula} and \emph{syuzhet}. Fabula is the events of the story as they occur in chronological order, but syuzhet refers to those events as they are ordered in the story's telling. Fabula describes one event following another, but syuzhet could describe events occuring out of order, branching sequences and events that occur at the same time.

The challenge faced here is how to find a way of not only describing the syuzhet of one story, but of all possible stories and paths through a story in a narrative world.

\paragraph{Modal Logic}
Modal logic extends classical propositional and predicate logic with modalities, which are operators that qualify a statement. For example, rather than simply stating `The Crocodile eats the sausages', we could instead say ``the Crocodile sometimes eats the sausages'', or ``it's possible that the Crocodile eats the sausages''.
Classic modal logic deals with \emph{alethic modality}, which describes whether a statement is \emph{possible} or \emph{necessary}. This is implemented using unary operators to qualify statements. For example, $\Diamond P$ states that $P$ is possible and $\Box P$ states that $P$ is necessary.
Naturally, we can use modalities beyond just possibility and necessity. In order to describe the syuzhet of a story, we need to be able to make statements such as ``The Crocodile eats the sausages at the beginning of the scene'', and ``Punch kills the baby while before his wife returns''. For this, we turn to \emph{temporal logic}.



\paragraph{Temporal Logics}
Arthur Prior is the first to employ modal logic as a way of describing sequences of time in his 1957 work \emph{Time and Modality} \cite{prior2003time}. Here he uses just two modal operators, $P$ and $F$, representing \emph{some time in the past} and \emph{some time in the future} respectively.
Hans Kamp adds two extra operators to Prior's logic, \emph{Since} and \emph{Until}, in his 1968 thesis \cite{kamp1968tense}, enabling it to describe spans of time in addition to the ordering of temporal events.
As Kripke later points out to Prior, this model lacks the expressiveness needed to describe all possible sequences of events. One major shortcoming is its restriction to describing only \emph{linear} events. 

Linear Temporal Logic (LTL), though still limited to the description of linear sequences of events, is an evolution of the work done by Prior and Kamp. Proposed by Amir Pnueli in 1977 \cite{pnueli1977temporal}, its original use is for the formal verification of computer programs.
Computational Tree Logic (CTL) \cite{ben1983temporal} is similar to LTL, but allows for the representation of non-linear time through the allowance of branches. Through CTL it is possible to describe several alternative pathways through time, though only one may ever be actualised. Like LTL, its original purpose is for formal verification of software, and in model checkers.
Both CTL and LTL are subsets of CTL* (Computational Tree Logic) \cite{emerson1986sometimes}, which can both describe both multiple branches of temporal paths and their durations. CTL* formulae must refer to a specific Kripke structure, however (a description of Kripke structures appears in section \ref{sec:kripke}).

\paragraph{Interval Temporal Logic}
In order to model fabula with modal logic, we employ Interval Temporal Logic (ITL), composed of the temporal intervals defined by Allen \cite{allen1983maintaining} and developed into modal operators by Halpern and Shoham \cite{halpern1991propositional}. This allows the expressiveness necessary to describe branching, parallel and nested paths through stories.

In most temporal logics (such as CTL* and its subsets), fixed \emph{time points} without duration are the basic unit of time. However, this can make it difficult to reason about the \emph{duration} of events that occur over a period of time. Temporal Interval Logic tackles this problem through the use of \emph{time intervals} or \emph{periods} as the basic temporal unit.

The version of ITL used here is that described by Della Monica et al. in their overview paper \cite{della2013interval}. Table \ref{tab:itl} lists the temporal intervals described by Allen, along with their modal operator equivalents in \emph{Halpern-Shoham logic}.

The operators defined by Halpern and Shoham are (a bar over an operator denotes its inverse):

\begin{itemize}
\item $\langle L \rangle / \langle \overline{L} \rangle$ (Later): The interval occurs at some point after another interval.
\item $\langle A \rangle / \langle \overline{A} \rangle$ (After): The interval occurs immediately after another interval.
\item $\langle O \rangle / \langle \overline{O} \rangle$ (Overlaps): The interval occurs both during and before or after another interval.
\item $\langle E \rangle / \langle \overline{E} \rangle$ (Ends): The interval ends at exactly the same time as another interval.
\item $\langle D \rangle / \langle \overline{D} \rangle$ (During): The interval both starts and ends inside the duration of another interval.
\item $\langle B \rangle / \langle \overline{B} \rangle$ (Begins): The interval begins at exactly the same time as another interval.
\end{itemize}
\begin{table}[!t]
  \centering
  \caption{Operators in the Interval Temporal Logic (adapted from \cite{della2013interval})}
  \label{tab:itl}
  \begin{tabular}{l|l|l}
    {\bf Interval} & {\bf Allen notation} & {\bf HS notation} \\
    \hline& &\\
    \multirow{7}{*}{\includegraphics[height=2.02in]{intervals.png}}&\emph{equals} \{=\} &\\
    & &\\
                   &\emph{before} \{\textless\} / \emph{after} \{\textgreater\} & $\langle L \rangle$ / $\langle \overline{L} \rangle$ \emph{(Later)}\\
    & &\\
                   &\emph{meets} \{\emph{m}\} / \emph{met-by} \{\emph{mi}\} &$\langle A \rangle$ / $\langle \overline{A} \rangle$ \emph{(After)}\\
    & &\\
    &\emph{overlaps} \{\emph{o}\} / \emph{overlapped-by} \{\emph{oi}\} &$\langle O \rangle$ / $\langle \overline{O} \rangle$ \emph{(Overlaps)}\\
    & &\\
    &\emph{finished-by} \{\emph{fi}\} / \emph{finishes} \{\emph{f}\} &$\langle E \rangle$ / $\langle \overline{E} \rangle$ \emph{(Ends)}\\
    & &\\
    &\emph{contains} \{\emph{di}\} / \emph{during} \{\emph{d}\} &$\langle D \rangle$ / $\langle \overline{D} \rangle$ \emph{(During)}\\
    & &\\
    &\emph{started-by} \{\emph{si}\} / \emph{starts} \{\emph{s}\} &$\langle B \rangle$ / $\langle \overline{B} \rangle$ \emph{(Begins)}
  \end{tabular}
\end{table}
\subsubsection{Propp Example with Punch and Judy}
In this example, we combine Halpern and Shoham's temporal operators with the possibility ($\Diamond$) and necessity ($\Box$) operators of modal logic. We follow the convention of writing possibility operators inside angle brackets: $\langle \, \rangle$ and necessity operators within square brackets: $[ \, ]$.

This example shows the ``sausages'' scene described in section \ref{sec:pjexample}, consisting of a set of situations $S$, containing Propp story functions $P$. The interval temporal logic operators used in this example are the set $T$. Figure \ref{fig:operators} shows the modal operators we use. $A, B$ and $C$ in formula \ref{eq:story} are variables that represent the characters and objects that appear in the story.
\begin{figure}[!t]
\begin{align}
    S &= \{S_0, S_1, S_2, S_{3a}, S_{3a_1}, S_4, S_{3b}, S_{3b_1}, S_4, S_5\}\\
    P &= \{\mathtt{interdiction(A, B, C), absentation(A), struggle(A, B),}\nonumber\\
  &\qquad\qquad\mathtt{victory(A), villainy(A, B), violation(A, B), return(A)}\}\label{eq:story}\\
  T &= \{D, \overline{D}, O, \overline{O}, A, \overline{A}, B, \overline{B}, L, \overline{L}, E, \overline{E}\}
\end{align}
\caption{Modal operators}\label{fig:operators}
\end{figure}
Hybrid logics allow worlds, time intervals, to be named. The hybrid logic nominal operator allows specific times to be uniquely referenced, allowing a logic to talk about specific states such as $S_0..S_5$. The nominal proposition is true of one specific time interval such that any two worlds with the same name represent co-extensive intervals of time. Anything true of one is true of the other.
We also make use of a nominal modal operator so that it is possible to make assertions about these named worlds, ``It is necessary that in state $S_1$ Joey absents himself.'' This technique enables us to associate story functions with specific, named intervals.
We use hybrid logic to identify nodes using the \emph{nominal} operator, shown as $@$. The nominal operator adds the capability of referring to possible worlds in formulas. In this way, each possible state of a system can be labelled and referenced from other states. As seen in figure \ref{fig:lotrec}, each nominal node must have its own name as a relation leading back to the root node, in order to be linked to and referred from the other nodes.
We can combine this with the Interval Temporal Logic to make statements such as ``An absentation starts with state $@S_1$ and ends with state $@S_5$.'' (formula \ref{eq:absentation}).
Figure \ref{fig:situations} describes the full sausages scene from Punch and Judy using the time intervals shown in figure \ref{fig:durations}.
\begin{figure}[!t]
  \centering
    \centerline{\includegraphics[width=0.9\textwidth]{durations.png}}
  \caption{Timings of the story functions in the sausages scene}\label{fig:durations}
\begin{align}
  &S_{0} \land \mathit{interdiction(Joey, Punch, Sausages)} \land\nonumber\\
  &\qquad\qquad\qquad\qquad\qquad\langle B \rangle @S_{1} \land \langle E \rangle @S_{4} \land \langle A \rangle @S_{5}\label{eq:interdiction}\\
  &[@S_{1}] \mathit{absentation(Joey)} \land \langle A \rangle @S_{2}\label{eq:absentation}\\
  &[@S_{2}] \mathit{struggle(Punch, Crocodile)} \land \langle E \rangle (@S_{3a} \lor @S_{3b})\label{eq:struggle}\\
  &[@S_{3a}] \mathit{victory(Crocodile)} \land \langle A \rangle @S_{3a_1}\\
  &[@S_{3a_1}] \mathit{villainy(Crocodile, Sausages)} \land \langle E \rangle @S_{4}\\
  &[@S_{3b}] \mathit{victory(Punch)} \land \langle A \rangle @S_{3b_1}\\
  &[@S_{3b_1}] \mathit{villainy(Punch, Sausages)} \land \langle E \rangle @S_{4}\\
  &[@S_{4}] \mathit{violation(Punch, Sausages)}\\
  &[@S_{5}] \mathit{return(Joey)}
\end{align}
\caption{Sausages scene with nominals and Interval Temporal Logic}\label{fig:situations}
\end{figure}

One notable feature of this approach is that it enables the building of reusable story components. For example, we have said that an interdiction must begin with an absentation and ends with a violation (formula \ref{eq:interdiction}). This pattern can be reused in different stories, or several times in the same story. Additionally, it allows for the abstraction and combination of story components in a more expressive way than Propp's original story functions. This is because Propp only describes narrative events at one level of abstraction. For example, he describes stories as a series of events containing an interdiction followed by an absentation, followed by a violation. But there is no mechanism for combining these three functions into a higher-level component (which could be called ``Don't do that, or else!'', for example). Our approach makes such abstraction and recombination possible.

\section{Describing Punch and Judy with Kripke Structures}\label{sec:kripke}
\begin{figure}[!t]
  \centering
    \centerline{\includegraphics[width=0.9\textwidth]{crocmodel.pdf}}
  \caption{One model from the sausages scene in LoTREC}\label{fig:lotrec}
\end{figure}

We use Kripke structures \cite{kripke1963semantical} as a method of interpreting the combination of modal logic with Interval Temporal Logic. A Kripke structure is a graph, the nodes of which represent a possible world consisting of a set of assertions, and the edges of which are the accessibility relations between worlds.

\subsubsection{LoTREC}
% Read the book
In order to build and visualise the Kripke structures, we use LoTREC \cite{del2001lotrec}, a generic tableaux prover for modal and description logics. It allows the user to build up Kripke models using a domain specific language and display those models in the form of a graph diagram.
LoTREC uses the tableau method for model checking. This method checks whether or not its input is satisfiable by attempting to build a model for it. If the model construction fails, then the input is unsatisfiable.
Building models in LoTREC consists of defining \emph{connectors}, \emph{rules} and \emph{strategies}. Connectors are the logical operators used in formulas, rules are the instructions LoTREC needs to expand them into new nodes and edges and strategies are ways of combining rules in order to form models.
LoTREC takes an initial set of formulas as input and expands each formula into its components. These formulas are input using a simple prefix notation domain specific language consisting of operators and arguments, described in the LoTREC instruction book \emph{Kripke's Worlds} \cite{gasquet2013kripke}.
In the expansion, if a $[\,]$ (necessary) operator is encountered, then the formulas that serve as its argument are propagated across to all subsequent nodes. In the case of a $\langle \, \rangle$ (possibility) operator, a new node (possible world) is created containing its arguments. In both cases, binary operators can be used where one argument is located inside the square or angle brackets. In these cases, the argument inside the brackets is the accessibility relation, and is used to label the edges leading to the subsequent nodes.
In the case of a disjunction ($\lor$) operator, the current model is copied in its entirety and extended with one of the operator's arguments. The other argument is added onto the current model. This is an important way of creating alternative routes through the narrative, all of which can be checked for consistency.

\paragraph{The Sausages Scene in LoTREC}
The narrative is captured as a system of interval temporal logic assertions over story functions. The  assertions are interpreted using a tableaux reasoner by unpacking them to form a Kripke structure.
Using the initial formulas from figure \ref{fig:situations} as input, figure \ref{fig:lotrec} shows the model for the case where Punch wins the fight with the Crocodile. One other model exists in this scenario, in which the Crocodile is instead the victor.

The example in figure \ref{fig:lotrec} describes the fabula of a branching story, where either Punch or the Crocodile may win the fight for the sausages. This corresponds to the disjunction in figure \ref{fig:situations}, formula \ref{eq:struggle}. This leads to the creation of two models: the one in which the Crocodile wins and then goes on to eat the sausages (situation $@S_{3a}$), and the one in which Punch wins (situation $@S_{3b}$).
Using a breadth first strategy like LoTREC we would build all models simultaneously. This enables them to be checked for internal consistency, eliminating potentially impossible temporal arrangements. For example, no interval may come after or later than itself.

The first node contains the logical description of the story, which is then unpacked into subsequent nodes. It starts with the initial situation, $S_0$, which contains an interdiction: $\mathit{interdiction(Joey, Punch, Sausages)}$, where Joey tells Punch to look after the sausages. The other situations, $S_1$ to $S_5$ are listed using the necessity operator, with their accessibility relations being each situation's nominal operator (such as $@S_1$). This means that these situations are all created as new nodes, linked to the root node using their nominal accessibility relation, and so can be referred to later using the nominal operator. In this way, situations can refer to other situations.

For example, in Propp's formalism, an interdiction must begin with the absentation of a character and end with the interdiction's violation. For this reason, the initial situation is connected to $@S_1$, ($\mathit{absentation(Joey)}$), with the $\langle B \rangle$ (begins) accessibility relation and to $@S_4$, ($\mathit{violation(Punch, Sausages)}$), with the $\langle E \rangle$ (ends) accessibility relation. As other situations are unpacked into nodes, they are linked to other situations in the same way. An example of this is shown in $S_1$, the absentation, which must end with Joey's return, $S_5$, and during which a struggle must occur ($S_2$). As the formulas inside $S_1$ are unpacked, $S_1$ is linked to the other situations with the appropriate temporal accessibility relations. This is all made possible with the hybrid logic, which allows nodes to be referred to by name and linked to.

By describing situations that are linked together with temporal relations, we can build narratives and check them as we go. If a narrative is inconsistent in some way (by stating that a character is both dead and alive, for example), then the model will be unsatisfiable. This is useful for ensuring the construction of consistent narratives.

Additionally, every time a story has the possibility of branching off, the model for each new branch can be checked to see if it is viable in the narrative world. This means that rather than relying on an author to write out each branch of a story, they can be generated automatically from a story description and checked for inconsistencies.

As mentioned previously, the combination of Interval Temporal Logic and a narrative formalism such as Propp's allows us to create abstracted story components. For example, we have described the interdiction as being started by an absentation and ended with a violation. This pattern can be called the ``interdiction pattern'', and so can be easily reused in other narratives.

The use of LoTREC has shown the utility of being able to visualise a branching narrative in its entirety. This suggests that our method enables the visual authoring of interactive narratives by non-technical creators. Rather than having to type in computer code, or logical formulas, a visual tool could be developed to create logically consistent narrative worlds.

This approach for describing narrative has many potential real-world uses. One obvious use would be for computer game narratives, but it could also be used to describe a repeatable training scenario, and the possible choices available to a user at any point during the simulation.

\subsection{Deontic Logic and Norms}

\section{Norms and Institutions}
\label{sec:norms-and-institutions}

An institution describes a set of `social' norms describing the permitted and obliged behaviour of interacting agents. Noriega's `Fish Market' thesis~\cite{noriega1999agent} describe how an institutional model can be used to regiment the actions of agents in a fish market auction. Several~\cite{artikis2009specifying,fornara2007agent,cardoso2007institutional} extend this idea to build systems where institutions actively regulate the actions of agents, while still allowing them to decide what to do. We build on the work of Cliffe et al.~\cite{cliffe2007specifying} and Lee et al.~\cite{lee2013decoupling} to adapt it for the world of narrative, using an institutional model to describe the story world of Punch and Judy in terms of Propp's story moves and character roles, through which the actors acquire powers and permissions appropriate to the character and the story function in which they are participating.

Institutional models use concepts from deontic logic to provide obligations and permissions that act on interacting agents in an environment. By combining this approach with Propp's concepts of \emph{roles} and \emph{story moves}, we describe a Propp-style formalism of Punch and Judy in terms of what agents are \emph{obliged} and \emph{permitted} to do at certain points in the story.

For example, in one Punch and Judy scene, a policeman enters the stage and attempts to apprehend Punch. According to the rules of the Punch and Judy world, Punch has an obligation to kill the policeman by the end of the scene (as this is what the audience expects to happen, having seen other Punch and Judy shows). The policeman has an obligation to try his best to catch Punch. Both agents have permission to be on the stage during the scene. The policeman only has permission to chase Punch if he can see him (Punch is obliged to hide from him at the start of the scene).

The permissions an agent has, on the one hand, constrain the choices of actions available to them at any given moment. Obligations, on the other hand, affect the goals of an agent. Whether or not an agent actively tries to fulfil an obligation depends on their emotional state.

\subsection{Institution example}
\label{sec:pjexample-insts}
To illustrate the application of institutional modelling, we here continue the `sausages and crocodile' scene example from section~\ref{sec:pjexample}, taking the Propp story functions and describing them in an institutional model.  We define our institution in terms of \emph{fluents}, \emph{events}, \emph{powers}, \emph{permissions} and \emph{obligations}, following~\cite{cliffe2007specifying}, to which the interested reader is referred for the full details of the formal model, including the generate ($\cal G$) and consequence ($\cal C$) relations, which are only described here in sufficient depth for the model being presented.

\subsubsection{Fluents}
These are properties that may or may not hold true at some instant in time, and that change over the course of time. \emph{Institutional events} are able to \emph{initiate} or \emph{terminate} fluents at points in time. A fluent could describe whether a character is currently on stage, the scene of the story that is currently being acted out, or whether or not the character is happy at that moment in time.
Domain fluents ($\mathcal{D}$) describe domain-specific properties that can hold at a certain point in time. In the Punch and Judy domain, these can be whether or not an agent is on stage, or their role in the narrative: % (equation~\ref{eq:domain}).
\begin{align*}
   \mathcal{D} &= \left\{\mathtt{onstage, hero, villain, victim, donor, item}\right\} %\label{eq:domain}
\end{align*}

Institutional fluents consist of (institutional) \emph{powers}, \emph{permissions} and \emph{obligations}.
% check your facts on this one
An \textbf{institutional power} ($\mathcal{W}$) describes whether or not an external event has the authority to generate a meaningful institutional event. Taking an example from Propp's formalism, an \emph{absentation\/} event can only be generated by an external event brought about by a \emph{donor\/} character (such as their leaving the stage). Therefore, any characters other than the donor character would not have the institutional power to generate an \emph{absentation\/} institutional event when they leave the stage.
The possible empowerments (institutional events) from Propp used in Punch and Judy are:
\begin{align*}
  \mathcal{W} =&\left\{\mathtt{pow(introduction), pow(interdiction), pow(give),}\right.\\ %\nonumber\\
               &\left. {} \mathtt{pow(absentation), pow(violation), pow(return)}\right\} %\label{eq:power}
\end{align*}

\subsubsection{Permissions} ($\mathcal{P}$) are associated with external actions that agents are permitted to do at a certain instant in time. These can be thought of as the set of \emph{socially permitted\/} actions available to an agent. While it is possible for an agent to perform other actions, societal norms usually discourage them from doing so.
% PJ examples
For example, it would not make sense in the world of Punch and Judy if Punch were to give the sausages to the Policeman. It is always Joey who gives the sausages to Punch. Also, it would be strange if Joey were to do this in the middle of a scene where Punch and Judy are arguing. We make sure agents' actions are governed so as to allow them only a certain subset of permitted actions at any one time. The set of permission fluents is:
\begin{align*}
\mathcal{P} =& \left\{\mathtt{perm(leavestage), perm(enterstage), perm(die), perm(kill),}\right.\nonumber\\
             &\left. {} \mathtt{perm(hit), perm(give), perm(fight)}\right\} %\label{eq:perm}
\end{align*}

\subsubsection{Obligations} ($\mathcal{O}$) are institutional facts that contain actions agents \emph{should} do before a certain deadline. If the action is not performed in time, a \emph{violation event} is triggered, which may result in a penalty being incurred. While an agent may be obliged to perform an action, it is entirely their choice whether or not they actually do so. They must weigh up whether or not pursuing other courses of action is worth accepting the penalty that an unfulfilled obligation brings.

% replace with sausages obligation
Anybody who has seen a Punch and Judy show knows that at some point Joey tells Punch to guard some sausages, before disappearing offstage. Joey's departure is modelled in the institution as the \emph{absentation\/} event. It could also be said that Joey has an obligation to leave the stage as part of the \emph{absentation} event, otherwise the story function is violated. This can be described in the institution as:
\begin{align*}
  \mathcal{O} =& \left\{\text{obl}(\mathtt{leavestage, absentation, viol(absentation)})\right\}%\label{eq:obl}
\end{align*}
The first argument is the external event that must be triggered according to the obligation, the second argument is the institutional deadline event, and the third argument is the violation event which is triggered if the obligation is not fulfilled before the deadline. 

\subsubsection{Events}
\label{sec:inst-events}
% actually 3 kinds, including violation events
Cliffe's model specifies three types of \textbf{event}: \emph{external events} (or `observed events', $\mathcal{E}_{obs}$), \emph{institutional events} ($\mathcal{E}_{instevent}$) and \emph{violation events} ($\mathcal{E}_{viol}$). Examples of each are given in Figure~\ref{fig:events}.
\emph{External events} are observed to happen in the agents' environment, which can \emph{generate} \emph{institutional events} which occur only within the institional model, leading to the \emph{initiation} or \emph{termination} of (domain) fluents, permissions, obligations or institutional powers.
An external event could be an agent leaving the stage, an agent hitting another, or an agent dying. Internal events include narrative events such as scene changes, or the triggering of Propp story functions such as \emph{absentation} or \emph{interdiction} (described in section~\ref{sec:propp}). \emph{Violation} is the name of a Propp story function, and is included as an internal event, although it has no relation to the violation events of an institution.
Violation events occur when an agent has failed to fulfil an obligation before the specified deadline. These can be implemented in the form of a penalty, by decreasing an agent's health, for example.

\begin{figure}[!t]
\begin{align}
  \mathcal{E}_{obs} =& \left\{\mathtt{startshow, leavestage, enterstage, die, give,}\right.\nonumber\\
  &\left. {} \mathtt{harmed, hit, fight, kill, escape}\right\}\label{eq:eobs}\\
  \mathcal{E}_{instevent} =& \left\{\mathtt{introduction, interdiction, receipt, absentation,}\right.\nonumber\\
                         &\left. {} \mathtt{violation, return, struggle, defeat, complicity,}\right.\nonumber\\
                         &\left. {} \mathtt{victory, escape}\right\}\label{eq:einst}\\
  \mathcal{E}_{viol} =& \left\{\mathtt{viol(introduction), viol(interdiction), viol(receipt),}\right.\nonumber\\
 &\left. {} \mathtt{viol(absentation), viol(violation), viol(return),}\right.\nonumber\\
 &\left. {} \mathtt{viol(struggle), viol(defeat), viol(complicity)}\right.\nonumber\\
 &\left. {} \mathtt{viol(victory), viol(escape)}\right\}\label{eq:viol}
\end{align}
\caption{External, institutional and violation events for Punch and Judy} \label{fig:events}
\end{figure}

% internal and external

\subsubsection{Event Generation and Consequences}
An \textbf{event generation} function, $\mathcal{G}$, describes how events
($\mathcal{E}$, usually external, but can also be internal) %
can generate other (usually institutional) events, conditional upon the current institutional state ($\cal X$). This is the counts-as relation.  For example, if an agent leaves the stage while the \emph{interdiction} event holds, they trigger the \emph{leavestage} event. This combination generates the \emph{absentation} institutional event (rule~\ref{eq:absentation}). Further examples appear in figure~\ref{fig:gen}.

Event generation functions follow a $\langle \mathtt{preconditions} \rangle \rightarrow \{\mathtt{postconditions}\}$ format. The preconditions consist of a set of fluents that hold at that time, along with an event to have occurred. The postconditions are the events that are generated. The generation functions are used to generate internal, institutional events from external events.

Consider the Punch and Judy scenario described in section~\ref{sec:pjexample}. There are seven institutional events (story functions) that occur during this scene: \emph{interdiction}, \emph{complicity}, \emph{receipt} (from Propp's \emph{receipt of a magical agent}) \emph{absentation}, \emph{violation}, \emph{struggle}, \emph{return}.
These institutional events are all generated by external events. The \emph{interdiction} is generated when Joey tells Punch to protect the sausages. Punch agreeing amounts to \emph{complicity}. Joey \emph{gives} punch the sausages (\emph{receipt}), then leaves the stage (\emph{absentation}). The crocodile eating the sausages is a \emph{violation} of Punch's oath, the agents fight (\emph{struggle}), then Joey enters the stage again (\emph{return}).

It is desirable that these story functions occur in this sequence in order for a satisfying narrative to emerge. Agents may decide to perform actions that diverge from this set of events, but the institution is guiding them towards the most fitting outcome for a \emph{Punch and Judy} world. For this reason, a currently active story function can be the precondition for event generation. For example, the \emph{receipt} event may only be triggered if an agent externally performs a \emph{give} action \textbf{and} if the \emph{complicity} event currently holds (rule~\ref{eq:receipt}).
Examples of event generation function for this scenario, complete with preconditions, are listed in rules~\ref{eq:gfirst}--\ref{eq:glast} (Figure~\ref{fig:gen}).

\begin{figure}[!t]
\abovedisplayskip=0pt
\abovedisplayshortskip=0pt
$\mathcal{G(X, E)}:\left\{\mbox{%
{\begin{minipage}[c]{0.85\textwidth}
% \vspace{-2.1em}\begin{align}
\begin{align}
\langle \emptyset,\mathit{tellprotect}\mathtt{(donor, villain, item)} \rangle%\nonumber\\
             %         &\qquad\qquad\qquad
& \rightarrow \left\{\mathit{interdiction}\right\}\label{eq:gfirst}\\
                      \langle \{\mathit{interdiction}\}, \mathit{agree}\mathtt{(villain)}) \rangle %\nonumber\\
            %          &\qquad\qquad\qquad
& \rightarrow \left\{\mathit{complicity}\right\}\\
                      \langle \emptyset, \mathit{give}\mathtt{(donor, villain, item)}) \rangle %\nonumber\\
        %              &\qquad\qquad\qquad
& \rightarrow \left\{\mathit{receipt}\right\}\label{eq:receipt}\\
                      \langle \{\mathit{interdiction}\}, \mathit{leavestage}(\mathtt{donor}) \rangle %\nonumber\\
              %        &\qquad\qquad\qquad
& \rightarrow \left\{\mathit{absentation}\right\}\label{eq:absentation}\\
                      \langle \{\mathit{interdiction}\}, \mathit{harmed}(\mathtt{item}) \rangle %\nonumber\\
         %             &\qquad\qquad\qquad
& \rightarrow \left\{\mathit{violation}\right\}\\
                      \langle \{\mathit{interdiction, absentation}\},
                      \mathit{enterstage}(\mathtt{donor}) \rangle %\nonumber\\
              %        &\qquad\qquad\qquad
& \rightarrow \left\{\mathit{return}\right\}\\
                      \langle \emptyset, \mathit{hit}(\mathtt{donor, villain}) \rangle %\nonumber\\
%                      &\qquad\qquad\qquad
& \rightarrow \left\{\mathit{struggle}\right\}\label{eq:glast}
\end{align}
\end{minipage}}}\right.$
\caption{Event generation in the sausage scene} \label{fig:gen}
\end{figure}

\textbf{Consequences} consist of fluents, permissions and obligations that are \emph{initiated} ($\mathcal{C}^{\uparrow}$) or \emph{terminated} ($\mathcal{C}^{\downarrow}$) by institutional events. For example, the institutional event \emph{receipt} initiates the donor agent's permission to leave the stage, triggering the \emph{absentation} event (rule~\ref{eq:initgive}). When the \emph{interdiction} event is currently active and a \emph{violation} event occurs, the interdiction event is terminated (\ref{eq:interm}). Rules~\ref{eq:cfirst}--\ref{eq:clast} in Figures~\ref{fig:init} and~\ref{fig:term} describe the initiation and termination of fluents in the Punch and Judy sausages scene detailed in section~\ref{sec:pjexample}.

\begin{figure}[!t]
\abovedisplayskip=0pt
\abovedisplayshortskip=0pt
$\mathcal{C^{\uparrow}(X, E)}:\left\{\mbox{%
\begin{minipage}[c]{0.85\textwidth}
\begin{align}
    \langle \emptyset, \mathtt{interdiction} \rangle %\nonumber\\
                                 % &\qquad\qquad
&\rightarrow \{\text{active}(\mathtt{interdiction}), \nonumber\\&\qquad\text{perm}(\mathtt{give(donor, villain, item)})\}\label{eq:cfirst}\\
                                 \langle \emptyset, \mathtt{receipt} \rangle % \nonumber\\
                                 % &\qquad\qquad
&\rightarrow \{\text{perm}(\mathtt{leavestage(donor)})\}\label{eq:initgive}\\
                                 \langle\{\mathit{active(absentation)}\}, \nonumber\\\mathtt{enterstage(villain)} \rangle %\nonumber\\
                                 %&\qquad\qquad
&\rightarrow \{\text{obl}(\mathtt{eat(villain, sausages),} \nonumber\\&\qquad\qquad\mathtt{return, viol(violation)})\}\label{eq:obl1}\\
                                 \langle\{\mathit{active(interdiction)}\}, \nonumber\\\mathtt{leavestage(donor)} \rangle % \nonumber\\
                                 % &\qquad\qquad
&\rightarrow \{\text{obl}(\mathtt{enterstage(donor),}\nonumber\\&\qquad\qquad\mathtt{eat(villain, sausages),}\nonumber\\&\qquad\qquad\mathtt{viol(return)})\}\label{eq:obl2}\\
                                 \{\mathit{active(interdiction)}\},\nonumber\\ \mathtt{violation} \rangle %\nonumber\\
                                 % &\qquad\qquad
&\rightarrow \{\text{perm}(\mathtt{enterstage(dispatcher)})\}\\
                                 \langle\{\mathit{active(absentation),}\nonumber\\\mathit{active(violation)}\},\nonumber\\ \mathtt{return} \rangle %\nonumber\\
                                 %&\qquad\qquad
&\rightarrow \{\text{perm}(\mathtt{hit(donor, villain)})\}
\end{align}
\end{minipage}}\right.$
\caption{Fluent initiation in the sausage scene} \label{fig:init}
\medskip
\abovedisplayskip=0pt
\abovedisplayshortskip=0pt
$\mathcal{C^{\downarrow} (X, E)}:\left\{\mbox{%
\begin{minipage}[c]{0.85\textwidth}
\begin{align}
\langle \emptyset, \mathtt{interdiction} \rangle %\nonumber\\
                                   %&\qquad\qquad
&\rightarrow \{\text{perm}(\mathtt{give(donor, villain, item)})\}\\
                                   \langle \{\mathit{active(interdiction)}\},\nonumber\\ \mathtt{absentation} \rangle %\nonumber\\
                                   %&\qquad\qquad
&\rightarrow \{\text{perm}(\mathtt{leavestage(donor)})\}\\
                                   \langle \{\mathit{active(interdiction)}\},\nonumber\\ \mathtt{violation} \rangle %\nonumber\\
                                   %&\qquad\qquad
&\rightarrow \{\mathit{active(interdiction)}\}\label{eq:interm}\\
                                   \langle \{\mathit{active(absentation),}\nonumber\\\mathit{ active(violation)}\},\nonumber\\ \mathtt{return} \rangle %\nonumber\\
                                   %&\qquad\qquad
&\rightarrow \{\mathit{active(absentation)}\}\label{eq:clast}
\end{align}
\end{minipage}}\right.$
\caption{Fluent termination in the sausage scene} \label{fig:term}
\end{figure}%

\section{The Bishop's Palace Interactive Story Project}
\label{sec:bishops-palace}

\section{Why Use Institutions for Interactive Narrative?}
\label{sec:why-use-institutions}
% Write about character freedom, regimentation vs regulation. Give story
% examples vs using a planner

\section{Modelling of Coordinated Instutions}
% Put bridge institution stuff here

% tropes

% TODO: explain that tropes are not covered academically, I am attempting to
% make tropes academically respectable
% TODO: make it clear at the start that we are choosing these examples as a way
% to demonstrate tropes ability to create abstractions
% TODO: parental abandonment might need extra characters
% TODO: connect character role (such as Comedic Sociopath) to our emotional model
% TODO: connect our idea of phases with landmarks in plans / knowledge
% representation literature
% do also need to demonstrate adequate knowledge of planners
% TODO: put TropICAL / instal fragments in column in appendix, pull out relevant
% features to describe in the chapter
\chapter{Tropes as Story Components}
\label{cha:tropes}
This chapter describes the foundations of our new formalism for narrative: story
tropes. Though our use of tropes for the description of stories has many
advantages, the main advantage that we will focus on throughout this thesis is
that they provide a means of creating new abstractions from existing tropes. For
more detail on how tropes allow us to do this, please see section~\ref{sec:abstractable}

Though story tropes may be a familiar concept to many outside of the academic
community, they do not appear in the literature in either fields Computer Science or
Narratology at the time of writing. Therefore this chapter contains a thorough
description of what a story trope is, along with several examples.

As previously described in chapter~\ref{cha:introduction}, tropes are
patterns that appear throughout various different media. Once one is familiar
with a trope, it becomes easy to identify its use in any story. Take, for
example, the \emph{Hero's Journey} trope first described in
chapter~\ref{cha:introduction}. It is a template which is repeated so often in
many different media, stories and contexts that it is instantly recognised even
by those that are completely unfamiliar with the concept of tropes.

% Bit about difference between tropes and cliches

In this section we examine the concept of a ``trope'', deconstructing examples
to demonstrate widely-recognised trope patterns, and exploring tropes that
operate at different levels of abstraction within a story. At the end of the
section we identify a formal definition of a trope, and how it fits within the
wider context of a story.

% TODO NUMBER
% TODO list of examples where tropes appear from jurisin paper
% TODO TVtropes screenshot figure
% TODO describe the periodic trope groups
\section{Tropes: a ``Folksonomy'' of Story Components}
The existence of a website called ``TV Tropes''~\citep{tvtropes} makes the discovery of example
tropes very simple. TV Tropes is a wiki for tropes, containing over 27,000
trope descriptions, along with the media that they appear in. For example, the
``The Empire'' trope appears in \emph{Star Wars}, Asimov's \emph{Foundation}
trilogy, the \emph{Hunger Games} books and films, and the \emph{Final Fantasy}
series of games, and a great many more stories in media.


\begin{figure}[!t]
\centerline{\includegraphics[height=5in]{evilEmpire.png}}
\caption{A screenshot of the ``The Empire'' wiki page from TV Tropes} \label{fig:evil-empire}
\end{figure}

Figure~\ref{fig:evil-empire} shows a screenshot of the ``The Empire'' page on the
website. It clearly shows the description of the trope at the top of the page,
and there are also instances of its use across different media at the bottom.

Tropes can also describe character archetypes. For example, this is how TV
Tropes describes \emph{Anti-Hero} characters:

\begin{quote}
An Archetypal Character who is almost as common in modern fiction as the Ideal Hero, an antihero is a protagonist who has the opposite of most of the traditional attributes of a hero. They may be bewildered, ineffectual, deluded, or merely apathetic. More often an antihero is just an amoral misfit. While heroes are typically conventional, anti-heroes, depending on the circumstances, may be preconventional (in a "good" society), postconventional (if the government is "evil") or even unconventional. Not to be confused with the Villain or the Big Bad, who is the opponent of Heroes (and Anti-Heroes, for that matter).
  \end{quote}

TV Tropes further clarifies that there are even further subdivisions of
Anti-Hero, depending on just how evil or cynical the character is. Batman, for
example, would be a highly cynical Anti-Hero who is nevertheless morally good.

Shakespeare's Macbeth is a character who becomes more and more of an evil Anti-Hero, until he is too morally evil
to still be a Hero and instead becomes a Tragic Villain.

Tropes can be very specific, referring to individual lines of dialogue.
One example is ``We Will Meet Again'':

\begin{quote}
The standard phrase when the villain finds that he has been defeated by the heroes and there is no point in staying around with the immediate Evil Plan foiled.
\end{quote}

Tropes can also be very abstract, referring to particular genres, types of
story, or events in a story that move the action forward. Other than the
previously mentioned ``Hero's Journey'' and ``The Empire'' tropes, another
example could be the ``Hilarity Ensues'' trope:

\begin{quote}
Actions that are dangerous or illegal often lead to injury, arrest, job dismissal, expulsion from school, deportation, or other dire consequences. Thankfully for our fictional friends, both the Rule of Cool and the Rule of Funny keep them safe (the latter more prominently).
\end{quote}

\emph{Metatropes} are tropes about tropes, often intended as a knowing wink to
the trope-savvy audience. One such example is ``Lampshade Hanging'', which TV
Tropes describes as:

\begin{quote}
...the writers' trick of dealing with any element of the story that threatens the audience's Willing Suspension of Disbelief, whether a very implausible plot development, or a particularly blatant use of a trope, by calling attention to it and simply moving on.
\end{quote}

In fact, even if an audience is unaware of the concept of tropes, they may be aware
of the recurring patterns and themes that they describe. This enables
genre-savvy (and especially postmodern) writers to play with the audience's
expectations. Ways to do this with tropes could include \emph{inversion}
(reversing the trope),
\emph{subversion} (making it look like the trope will happen, but then not using
it after all), \emph{parody} (using the trope in an over-the-top, exaggerated
manner) and \emph{deconstruction} (using the trope in a straightforward manner,
but in way which forces the audience to analyse the trope itself).

Take, for example, the well-known ``The Butler Did It'' trope from murder
mystery stories, where the butler of the house is revealed to be the murderer at
the end of the story. TV Trope describes ways that an author could ``play'' with this trope:

\begin{itemize}
  \item \textbf{Subvert} it: The butler is the prime suspect at the beginning, and is later found innocent.
  \item \textbf{Invert} it: The butler is the victim. Or the butler solved the crime. Or every suspect except the butler was part of the crime.
  \item \textbf{Parody} it: Butlers could learn their trade at butler college where they are taught cleaning, cooking, and murdering.
  \item \textbf{Deconstruct} it: The butler is a revolutionary serial killer, who purposely takes jobs as butlers to murder his rich masters. All the unfortunate implications of class warfare that this suggests are brought up and discussed.
\end{itemize}

Many other examples of using tropes in this way can be found on the ``Playing
with a Trope'' page of the TV Tropes website~\cite{playing-tropes}.


\begin{figure}[p!]
\centerline{\includegraphics[height=\textheight]{periodicTable.png}}
\caption{The ``Periodic Table of Storytelling'', original by James Harris
  (http://jamesharris.design/periodic/), poster format by Deviant Art user Dawn
  Paladin (http://dawnpaladin.deviantart.com/art/The-Periodic-Table-of-Storytelling-Second-Edition-425816342)} \label{fig:periodic-table}
\end{figure}

A large and highly active community of users and contributors exists around TV
Tropes. In addition to creating content for and curating the content on the
website, they also work to create useful ways to visualise the usage of tropes
in stories. For example, The Periodic Table of
Storytelling~\citep{periodicTableOfStorytelling} is a visualisation of tropes as
``elements'' in the ``molecules'' of a story. The table itself
(fig.~\ref{fig:periodic-table}) arranges the tropes into different ``groups''
according to the part of a story that they operate on. The leftmost groups
describe the story as a whole, describing its \emph{structure} (``Three Act
Structure'', ``MacGuffin'', ``Chekov's Gun''), \emph{story
  modifiers} (``Darker and Edgier'', ``Tear Jerker'', ``Jumping the Shark''),
and \emph{plot devices} (``Hand Wave'', ``Techno Babble'', ``Xanatos Gambit'').
In the centre of the table are different types of character such as
\emph{Heroes} (``Anti Hero'', ``Action Girl'', ``The Gunslinger''),
\emph{Archetypes} (``Mad Scientist'', ``The Fool'', ``Loveable Rogue'') and
\emph{Villains} (``Evil Twin'', ``The Empire'', ``Obstructive Bureaucrat''). The
right third of the table contains self-referential tropes such as
\emph{metatropes} (``Lampshade Hanging'', ``Subverted Trope'', ``The Fourth
wall''), and \emph{fandom and audience reactions} (``Fanon'', ``Fridge Logic'',
``Freud Was Right'').

The story is then visualised as a molecule composed from tropes, linked together as
atoms (shown at the bottom of fig.~\ref{fig:periodic-table}).

This visualisation demonstrates the core idea of our use of tropes as reusable
story components, but the ``molecule'' metaphor is unsuitable for a couple of
reasons. Firstly, linking tropes together as atoms in a molecule does not
communicate the different levels of abstraction at which tropes operate.
Considering that our main purpose for choosing tropes as our method of
describing narrative components, this means that the ``molecule'' metaphor used
by the author does not match our intentions. The
``Hero's Journey'' trope, for example, would describe the narrative arc as a
whole, while the ``Comeuppance'' trope would describe just a single scene in the
story. The metaphor is also not ideal because it presents orthogonal concepts
together in a story with no indication of which part of a narrative they affect.
A ``scoundrel sidekick'' could be linked together with a ``breaking the fourth
wall'' trope, even though one trope relates to a certain character, and the
other may describe a single line of dialogue or action that occurs at a specific
point in the story. 

Also, the arrangement and linking of the tropes in the example molecules is
quite arbitrary. The examples given on the \emph{periodic table} web site form
interesting shapes, but do not follow a consistent logic. In the \emph{Star
  Wars} molecule, for example, the ``Five Man Band'', ``Conflict'' and ``The
Empire'' tropes are linked in a straight line suggesting a linear sequence, but
three further tropes (``The Dragon'', ``The Chosen One'' and ``You Have Failed
Me'') are all linked to the ``The Empire'' trope in the molecule. While ``The
Dragon'' refers to the Death Star in the movie, the ``The Chosen One'' trope is
more closely linked to Luke Skywalker's Hero role. The ``You Have Failed Me''
trope refers to a specific scene where villain Darth Vader punishes an
under performing henchman with choking. It is not clear why the creator decided
to link these specific tropes to the ``The Empire''
trope.

Similarly, in the \emph{GhostBusters} example, the ``Five Man Band'' and ``Mad
Scientist'' tropes appear together in the same ``atom'', which are linked to
``Sealed Evil in a Can'' and ``Hilarity Ensues''. Again, it is not clear why
those particular tropes are arranged together into the same atoms, or why they
are linked together in this way. The most likely explanation is that this
visualisation of the way that tropes link together in a story is not intended as
a serious way to formalise stories, and is merely a ``fun'' example.

Taking this visualisation as inspiration, we develop our concept of tropes as
logical, reusable components for the formal description of stories. Importantly,
we develop a way to nest tropes within other tropes as subtropes as a way to
describe tropes acting at different levels of abstraction.

% TODO rip the whole bit from the lit review? Consider it at least
\section{Why Use Tropes?}
% Write about ability to abstract, give story examples vs Propp
% This is pretty much covered in the lit review

Returning to the shortcomings of existing narrative formalisms we describe in
section~\ref{litrev-discussion}, we now describe how tropes are suitable for use
as a narrative formalism that is able to overcome these limitations.

\subsection{A Means of Abstraction}
\label{sec:abstraction}
Most tropes exist in a hierarchy of tropes, with parent tropes such as the
\emph{Quest} containing child tropes such as \emph{Redemption Quest},
\emph{Sidequest} or a \emph{Quest for Identity}. These child-tropes inherit some
of the characteristics of their parents, but add subtle or major changes. For
example, a \emph{Quest for Identity} follows the \emph{Quest} format, but is
constrained so that the item the hero is questing after is the hero's own
identity. This is a mechanism of \emph{inheritance}, so one can imagine using
such a process to avoid duplication of effort when authoring new tropes by only
expressing how a trope differs from its parent.

\begin{figure}[!t]
\centerline{\includegraphics[height=1.5in]{freytag.png}}
\caption{Freytag's Pyramid} \label{fig:freytag}
\end{figure}

Another method of abstraction is to express tropes that are contained as parts
of larger tropes. The example we described in section~\ref{litrev-discussion}
describes how the ``Quest'' trope could form just one part of a larger trope
such as the ``Hero's Journey''. Another example of this would be the
\textbf{Three Act Structure} (also known as Freytag's Pyramid) shown in fig.~\ref{fig:freytag}, which describes the shape of a
story in terms of rising and falling levels of drama. This could be split into
five (or perhaps more) sub-tropes:

\begin{itemize}
  \item \textbf{Exposition}: The setting of the scene, providing any background
    information that is relevant to the story.
  \item \textbf{Rising Action}: A series of event drive the story forward, each
    increasing in dramatic intensity.
  \item \textbf{Climax}: The turning point of the story. Some fateful event
    occurs as a result of the rising action, which could be a battle between the
    hero and the villain, for example.
  \item \textbf{Falling Action}: The consequences of the climax play out, and
    the story shows how the characters are affected.
  \item \textbf{Denouement}: This is the final resolution, where all the ``loose
    ends'' of the story are tied up.
\end{itemize}

This means that if we already have trope definitions for the ``Exposition'',
``Rising Action'', ``Climax'', ``Falling Action'' and ``Denouement'' parts of a
story, and want to create a ``Three Act Structure'' trope, we can simply express
it in the following way:

\begin{itemize}
  \item The ``Exposition'' trope happens
  \item Then the ``Rising Action'' trope happens
  \item Then the ``Climax'' trope happens
  \item Then the ``Falling Action'' trope happens
  \item Then the ``Denouement'' trope happens
\end{itemize}

Returning to the concept of ``story structure'' described in
section~\ref{sec:structure}, we can use the ``subtropes'' we just identified to
describe other ``story shapes'' as defined by Vonnegut. For example, the ``man
in hole'' story shape could be simple described as a rearrangement of the
three-act structure:

\begin{itemize}
  \item The ``Exposition'' trope happens
  \item Then the ``Falling Action'' trope happens
  \item Then the ``Zenith'' trope happens
  \item Then the ``Rising Action'' trope happens
  \item Then the ``Climax'' trope happens
  \item Then the ``Denouement'' trope happens
\end{itemize}

Note that we added an additional trope, the ``Zenith'' trope to describe the
lowest point in the story. Otherwise, the rest of the story ``shape'' is easily
described in terms of tropes that we have defined already.

This re-use of existing trope definitions saves us the time and effort of the wasteful duplication of the steps
already described within them. This is why abstraction is such a powerful
and useful concept: it allows us to break down complicated stories into a series
of smaller sub-stories, rather than having to describe the whole thing in one go.

\subsection{Conceptually Simple}\label{sec:tropes-simple}

Most story authors are already familiar with the concept of tropes. In order to
evaluate the suitability of their use for the description of narrative
components, we presented a preliminary version of our trope-based TropICAL programming
language (described in section~\ref{}) for story authoring to the Oxford and London Interactive Fiction meetup group.

After a brief presentation on the concept of tropes and how we intend to use
them to create an authoring system for interactive narrative, participants were
given a questionnaire with the purpose of discovering their familiarity with
tropes, as well as finding out how suitable they thought tropes would be as a
new kind of formalism for narrative components.

There were 18 responses to the questionnaire. The questions and responses were as follows:

\textbf{What's your interest in Interactive Fiction?}
\begin{itemize}
  \item I'm an author: 5 (27.8\%)
  \item I'm a game developer: 10 (55.6\%)
  \item I write interactive fiction: 5 (27.8\%)
  \item It's my hobby: 6 (33.3\%)
  \item It's my job: 5 (27.8\%)
  \item Other: 2 (11.1\%)
\end{itemize}
\textbf{What tools do you use to create Interactive Fiction?}
\begin{itemize}
  \item Inform: 3 (16.7\%)
  \item Twinery: 7 (38.9\%)
  \item Unity or other IDE: 8 (44.4\%)
  \item Pure code: 4 (22.4\%)
  \item I don't create interactive fiction or games with narrative: 3 (16.7\%)
  \item Other: 6 (33.3\%)
\end{itemize}
\textbf{What kind of narratives are you interested in making?}
\begin{itemize}
  \item Linear: 2 (11.1\%)
  \item Non-linear: 11 (61.1\%)
  \item I'm not an author: 1 (5.6\%)
  \item Other: 4 (22.2\%)
\end{itemize}
\textbf{Are you familiar with the idea of ``tropes''?}
\begin{itemize}
  \item Yes, and I have visited the ``TV Tropes'' website: 15 (83.3\%)
  \item Yes, but I hadn't heard of ``TV Tropes'': 3 (16.7\%)
  \item No: 0 (0\%)
\end{itemize}
\textbf{How useful do you think tropes are for authoring interactive stories?
  (on a scale of 1 - 5)}
\begin{itemize}
  \item \textbf{1} (not useful): 0 (0\%)
  \item \textbf{2}: 2 (11.8\%)
  \item \textbf{3}: 9 (52.9\%)
  \item \textbf{4}: 2 (11.8\%)
  \item \textbf{5} (extremely useful): 4 (23.5\%)
\end{itemize}

The fact that all of the interactive fiction authors and games developers were
already familiar with the concept of tropes demonstrates that they are
conceptually simple enough for non-programmers to understand. Not only that, but
most of them were already familiar with the TV Tropes website. Compare this with
formalisms for narrative components such as Lehnert's Plot
Units~\citep{lehnert1981plot}, which would only be familiar to computer science specialists.

\subsection{A Library of Re-usable Examples}

One of the major strengths of Propp's system~\cite{propp1968morphology} is that
the Morphology is not only a theory: it is also a library of 31 story functions
that can be put together by story authors to create a narrative. The authors
need not create their own story functions, they can simply use the ones that
Propp has already created for them.

Our system shares the same strength due to the fact that the TV Tropes website
serves as our ``library'' of story components. In addition, it grants authors
the flexibility to create their own tropes which are not already listed on the
TV Tropes website.

TropICAL, our domain-specific programming language for trope-oriented story
authoring, makes the authoring of story tropes simple for non-programmer users.
In the same vein as Inform 7~\citep{reed2010creating}, our language uses a
constrained natural language syntax. Further details are described in
section~\ref{sec:tropical}, where we describe the language in detail. In the same manner as a wiki, once a number of authors have contributed tropes,
it will become a useful library of reusable tropes for future authors to use in
their stories.

To summarise: tropes are an ideal model to use for story components, and fulfil
the criteria we laid out in section~\ref{sec:litrev-discussion}: they provide a
\emph{means of abstraction} through subtropes as well as parent and child
tropes, they are \emph{conceptually simple} for authors to learn, given that
most authors are already familiar with them, and they enable us to easily create
a \emph{library of re-usable examples} from the tropes listed on the TV Tropes website.

% \section{Using Tropes with Modal Logic}
% In section~\ref{sec:pjexample}, we described the ``sausages'' scene from
% \emph{Punch and Judy} by combining Propp's story functions with modal operators
% to create Kripke structures to visualise the paths through the scene. In order
% to compare the expressiveness of tropes as a story formalism, this section
% describes the same scene, but instead using tropes in place of Propp functions.

% ACTUALLY DO THIS!

\section{Describing Tropes as Institutions} % 1/2
\label{sec:tropes-as-insts}
Rather than strictly telling our story characters what to do to conform to a
story arc, we govern their behaviours with \emph{social institutions}, as
described in Section~\ref{sec:institutions}. An institution describes a set of `social' norms describing the permitted and obliged behaviour of interacting agents. Noriega's `Fish Market' thesis~\cite{noriega1999agent} describes how an institutional model can be used to regiment the actions of agents in a fish market auction.~\cite{cliffe2007specifying},~\cite{lee2013decoupling} extend this idea to build systems where institutions actively regulate the actions of agents, while still allowing them to decide what to do. Adapting this idea to the world of narrative, we use an institutional model to describe the tropes that occur within a story world.

Institutional models use concepts from deontic logic to provide obligations and permissions that act on interacting agents in an environment. By combining this approach with the idea of tropes, we can create a narrative model in terms of what agents are \emph{obliged} and \emph{permitted} to do at certain points in the story. In this way, the tropes are described as social norms which govern the character agents of a story, where an institution describes the norms that govern a certain trope, and a story is a collection of tropes.

In order to describe story tropes in terms of social norms, we break them down into three components:

\begin{enumerate}
\item characters, which instantiate roles
\item objects, which instantiate types
\item places, which instantiate locations
\end{enumerate}

Characters' actions are described in terms of permissions and obligations. For example, a character in a certain role \emph{may} go to the cinema, or a character \emph{must} buy a ticket before the movie begins, otherwise they will not see it. Note that an obligation (which says that a character \emph{must} do something) can have a deadline (``before the movie begins'') and a consequence (``they will not see the movie''). These are both optional in our system.

Returning to the tropes described in the introduction, we can express them in terms of social norms:

\begin{itemize}
  \item \textbf{The Hero's Journey}: The hero \emph{must} leave home when they receive the call to adventure. Then the hero \emph{may} kill the villain. Once this is done, the hero \emph{may} return home.
  \item \textbf{The Evil Empire}: The villain has an empire, and \emph{may} kill the hero.
  \item \textbf{MacGuffin}: The hero \emph{must} search for an object. However, the hero \emph{may} find it.
  \item \textbf{Chekhov's Gun}: If a weapon appears in the beginning, it \emph{must} be used before the end of the story.
\end{itemize}

Describing tropes in terms of permissions and obligations is enough for us to be able to specify them as social norms, but also we need to be able to determine which norms hold at any point of a story. For this, we use an \emph{Answer Set Programming} (ASP) approach to describe our tropes in order to use an answer-set solver. We do this with the aid of \emph{InstAL}~\citep{cliffe2007specifying}, the Institution Action Language, a language for describing social institutions, which compiles to AnsProlog. This allows us to use trope models and an ASP solver to determine which norms hold after agent or player actions have occurred in the story world.

In InstAL, external events trigger institutional (internal) events. External events are the actions of the character agents in their environment. For example, an agent playing the role of Luke Skywalker in a Star Wars game may pick up a Lightsaber. Since Luke Skywalker is a hero character, and a Lightsaber is a type of weapon, this would trigger an institutional event where a hero has picked up a weapon. Institutional events initiate and terminate fluents inside the institution, which may describe the institutional state, and which permissions and obligations currently hold. So when Luke Skywalker picks up a Lightsaber, the institutional event could initiate his permission to use the weapon, or an obligation to go to the land of adventure.
For more details on InstAL, social institutions, and the formalism in figures~\ref{fig:events} and~\ref{fig:term}, refer to~\citep{cliffe2007specifying}.

Figure~\ref{fig:events} lists some external ($\mathcal{E}_{external}$) and institutional (internal, $\mathcal{E}_{internal}$) events for the \emph{Hero's Journey} trope. A wide range of external events such as \emph{go}, \emph{meet}, \emph{kill}, \emph{escape} generate the \emph{intHerosJourney} internal event, but only if the external event meets certain criteria. These criteria could be whether or not an agent fulfils a certain role, for example. Figure~\ref{fig:gen} shows examples of such internal event generation ($\mathcal{G}$). In the first example (rule~\ref{eq:tatooine}), the \emph{intHerosJourney} event is generated when Luke Skywalker goes to Tatooine, but only if Luke has the role of \emph{hero}, and Tatooine's location is \emph{home}.
Figure~\ref{fig:init} shows how internal events initiate ($\mathcal{C^{\uparrow}}$) fluents and norms (permissions and obligations) in a trope. Because the \emph{Hero's Journey} trope has several stages, this example only shows the first two phases of the trope (this is explained further in the ``Sequencing'' part of the ``TropICAL: a DSL for Tropes'' section). Rule~\ref{eq:phaseb} shows how the \emph{intHerosJourney} internal event initiates the hero's permission to kill the villain, an obligation for the hero to go to the Land of Adventure before the villain kills the victim, and the next phase (phase C) of the \emph{Hero's Journey} trope. These fluents are only initiated if the \emph{intHerosJourney} internal event happens while the trope is in phase B (when \emph{phase(herosJourney, phaseB)} holds), however.
Fluent termination ($\mathcal{C^{\downarrow}}$) works in a similar manner to initiation, with permissions and obligations for previous trope phases being terminated once the next phase of a trope has been entered. Examples for the first two phases of the \emph{Hero's Journey} trope are shown in figure~\ref{fig:term}.

While InstAL allows us to express tropes as social institutions, it would be difficult to use for non-programmers who are unfamiliar with logic programming paradigms. In order for story authors to be able to create their own tropes, a much more user-friendly language is needed. This is the motivation for TropICAL, the domain specific language we describe in the next section.

\begin{figure}[!t]\small
\begin{align}
  \mathcal{E}_{external} = & \left\{\begin{array}{c}
\mathtt{go(Agent, Place)},\\
\mathtt{meet(Agent, Agent)},\\
\mathtt{kill(Agent, Agent)}\\
\mathtt{escape(Agent)}
\end{array}
\right\}\label{eq:eobs}\\
  \mathcal{E}_{internal} = &\left\{\begin{array}{l}
\mathtt{intHerosJourney(Agent,}\\
\;\;\mathtt{Agent, Agent, Place, Place)}
\end{array}\right\}
\label{eq:einst}
\end{align}
\caption{External and institutional events ($\mathcal{E}$) for the \emph{Hero's Journey} trope} \label{fig:events}
\end{figure}

\begin{figure*}[!t]
\small
%------------------------------------------------------------------------
The generation relation $\mathcal{G}$ for trope state $\mathcal{X}$ and external event $\mathcal{E}$ in the \emph{Hero's Journey} trope:\\ 
$\mathcal{G(X, E)}:\left\{\mbox{%
\begin{minipage}[c]{0.85\textwidth}
\begin{align}
\begin{array}{r}
\langle \{\mathit{role(lukeSkywalker, hero)},\\
\mathit{location(tatooine, home)}\},\\
\mathit{go}\mathtt{(lukeSkywalker, tatooine)} \rangle
\end{array}
&\rightarrow
\begin{array}{l}
\{\mathit{intHerosJourney}\mathtt{(lukeSkywalker,}\\\;\;\;\;\mathtt{R, S, tatooine, T)}\}
\end{array}\label{eq:tatooine}\\
\begin{array}{r}
                                 \langle \{\mathit{role(lukeSkywalker, hero)},\\
  \mathit{role(obiWan, dispatcher)}\},\\
\mathit{meet}\mathtt{(lukeSkywalker, obiWan)} \rangle
\end{array}
&\rightarrow
\begin{array}{l}
\{\mathit{intHerosJourney}\mathtt{(lukeSkywalker,}\\\;\;\;\;\mathtt{obiWan, R, S, T)}\}
\end{array}\label{eq:hsnth}
\end{align}
\end{minipage}
}\right.$\smallskip

%------------------------------------------------------------------------
The fluent initiation relation $\mathcal{C^{\uparrow}}$ for trope state $\mathcal{X}$ and internal event $\mathcal{E}$ in the \emph{Hero's Journey} trope:\\
$\mathcal{C^{\uparrow}(X, E)}:\left\{\mbox{%
\begin{minipage}[c]{0.85\textwidth}
\begin{align}
\begin{array}{r}
                                 \langle \{\mathit{phase(herosJourney, phaseA)}\},\\
  \mathit{intHerosJourney(hero, dispatcher, villain},\\
\mathit{home, landOfAdventure)}\} \rangle
\end{array}
&\rightarrow
\left\{\begin{array}{l}
\text{perm}(\mathtt{meet(hero, dispatcher)})\\
\text{phase}(\mathtt{herosJourney, phaseB})
\end{array}\right\}\label{eq:cfirst}\\
\begin{array}{r}
                                 \langle\{\mathit{phase(herosJourney, phaseB)},\\
  \mathit{intHerosJourney(hero, dispatcher, villain},\\ \mathit{home, landOfAdventure)}\} \rangle\\
  \end{array}
&\rightarrow \left\{
\begin{array}{l}
  \text{perm}(\mathtt{kill(hero, villain)}) \\
\text{obl}(\mathtt{go(hero, landOfAdventure),} \\
\mathtt{kill(villain, victim)},\\
\mathtt{viol(story, end)})\\
\text{phase}(\mathtt{herosJourney, phaseC})
\end{array}
\right\}\label{eq:phaseb}
\end{align}
\end{minipage}
}\right.$\smallskip

%------------------------------------------------------------------------

The fluent termination relation $\mathcal{C^{\downarrow}}$ for trope state $\mathcal{X}$ and internal event $\mathcal{E}$ in the \emph{Hero's Journey} trope:\\
$\mathcal{C^{\downarrow} (X, E)}:\left\{\mbox{%
\begin{minipage}[c]{0.85\textwidth}
\begin{align}
\begin{array}{r}
                                 \langle \{\mathit{phase(herosJourney, phaseA)}\},\\
  \mathit{intHerosJourney(hero, dispatcher, villain},\\ \mathit{home, landOfAdventure)}\} \rangle
  \end{array}
&\rightarrow \left\{
\begin{array}{l}\text{perm}(\mathtt{go(hero, home)}),\\
\text{phase}(\mathtt{herosJourney, phaseA})
\end{array}
\right\}\label{eq:crt}\\
\begin{array}{r}
                                 \langle\{\mathit{phase(herosJourney, phaseB)},\\
  \mathit{intHerosJourney(hero, dispatcher, villain},\\ \mathit{home, landOfAdventure)}\} \rangle
  \end{array}
&\rightarrow \left\{
\begin{array}{l}
\text{perm}(\mathtt{meet(hero, dispatcher)})\\
\text{phase}(\mathtt{herosJourney, phaseB})
\end{array}
\right\}\label{eq:crst}
\end{align}
\end{minipage}
}\right.$
\caption{Generation events, and fluent initation and termination for the Hero's Journey trope}
\label{fig:gen}
\label{fig:init}
\label{fig:term}
\end{figure*}

\section{Punch and Judy as Tropes}
\label{sec:punchjudy-tropes}
In section~\ref{sec:pjexample}, we describe the \emph{sausages} scene of Punch
and Judy in terms of Propp's story functions. We begin this section in the same
way, by building up a scene description in terms of tropes. Then, as in
section~\ref{sec:pjexample-insts}, we create an institution out of the scene we
have described with our formalism. 

On the TV Tropes page for Punch and Judy, it lists the following tropes (quoted
directly from the site):

\begin{itemize}
  \item \textbf{Amusing Injuries}: People are often beaten up.
  \item \textbf{Audience Participation}: The children are expected to reply to Mr. Punch's Catch Phrase, "That's the way to do it" with a shout of "Oh no, it isn't!"
  \item \textbf{Black Comedy}: So black that many modern versions are often heavily censored compared to more historical stagings.
  \item \textbf{Catch Phrase}: ``That's the way to do it!'', ``HE'S BEHIND
    YOU!'', etc.
  \item \textbf{Comedic Sociopath}: Mr. Punch
  \item \textbf{Commedia dell'Arte}: Punch is based on the \emph{Pulcinella} character.
  \item \textbf{Crosses the Line Twice}: Good showings will definitely do this.
  \item \textbf{Hand Puppet}: All of the characters, except the baby, are
    puppets (though originally marionettes).
  \item \textbf{Head Bob}: Traditionally the puppets don't have articulated mouths, and use head bobbing to indicate which one is speaking.
  \item \textbf{Ironic Echo}: There's at least one rendition of the act where Punch ends up playing one trick too many on Snap the Crocodile, who promptly eats him (off-stage, of course) and returns repeating Mr Punch's "da-da-da" sound, culminating in a mock belch.
  \item \textbf{Karma Houdini}: In many versions, Punch is a psychopath who kills his own baby by throwing it out of a window, beats his wife to death with a stick, kills several other characters whom he encounters and finally outwits the devil himself to get away completely scot free.
  \item \textbf{Refuge in Audacity}: The entire show, especially the violence, is played as outrageous comedy.
  \item \textbf{Slapstick}: The style of the show, even named after the type of stick Punch uses.
  \item \textbf{Throw the Dog a Bone}: In some shows Judy will get her hands on Punch's stick and beat him with it. Though this is usually followed by Punch snatching it back and beating her with it.
  \item \textbf{Unsympathetic Comedy Protagonist}: Mr. Punch
  \item \textbf{Values Dissonance} Possibly the best example of this as Punch's domestic abuse of Judy is completely played for laughs.
  \item \textbf{Villain Protagonist}: Mr. Punch
\end{itemize}

Some of these tropes are useful in our construction of the story as a whole in
terms of tropes, while others are not. For example, the \emph{Commedia
  dell'Arte} trope would be difficult to express in a way that would influence
our interactive narrative. Instead of describing the entire story of Punch and
Judy, we will describe the sausages scene first mentioned in
section~\ref{sec:pjexample} in terms of tropes, using this one piece of the
story as an illustrative example. To help with this translation,
TV Tropes even has a page on Vladimir Propp which maps his story functions
directly into tropes~\cite{propp-tropes}:

\begin{enumerate}
  \item Joey tells Punch to look after the sausages (\emph{Rule Number One}).
  \item Joey has some reservations, but decides to trust Punch (\emph{Deal with
      the Devil}).
  \item Joey gives the sausages to Punch (\emph{Mentor Archetype}).
  \item Joey leaves the stage (\emph{Parental Abandonment}).
  \item A Crocodile enters the stage and eats the sausages (\emph{Don't Touch
      It, You Idiot!}).
  \item Punch fights with the Crocodile (\emph{Earn Your Happy Ending}).
  \item Joey returns to find that the sausages are gone (\emph{Where It All Began}).
\end{enumerate}

Here are the tropes mentioned above, described in more detail:

\textbf{Rule Number One} (interdiction)
TV Tropes actually describes two separate versions of this trope:

\begin{itemize}
  \item a situation where a character makes rules to govern a dangerous or
    uncomfortable situation (one such example being ``The first rule of Fight
    Club...'', which is in itself a trope of its own).
  \item when a Mentor Archetype conveys advice or admonishments to another
    character, such as ``Don't use the dangerous forbidden technique!'' or
    ``Always believe in yourself!''.
\end{itemize}

In the case of our scene, the interdiction seems to be the second type listed
here.

\textbf{Deal with the Devil} (complicity)
The classic incarnation of this trope is the $16^{th}$ century legend of Faust
selling his soul to Mephistopheles. It involves a desperate pawn (Faust) signing
a magically binding contract with a corrupt, exploitative trickster
(Mephistopheles, or any Satan-like character).

In this scene, Punch would be the corrupt exploiter, with Joey the Clown as his pawn.

\textbf{Mentor Archetype} (Receipt / provision of a Magical Agent)
TV Tropes describes this as ``A more experienced advisor or confidante to a
young, inexperienced character, particularly to a hero.''.

Due to our stretching of the original Propp function definition, this trope does
not fit what we want to express: the simple act of Joey giving the sausages to
Punch. Joey does not really fulfil the Mentor role. Additionally, TV Tropes'
translation of ``Receipt of a Magical Agent'' from Propp to ``Mentor Archetype''
is questionable: there is no mention of a Magical Agent in this trope, only of
the Mentor who provides it. We must look for a better-fitting trope in this case.

\textbf{Parental Abandonment} (absentation)
This trope is straightforward: the protagonist is abandoned by their parents
(emotionally or physically). In the case of the trope, it is described as
something that drives or influences the protagonist, such as in the case of
the character Bruce Wayne: the death of his parents early on forced him to
become the superhero Batman. This differs from Propp's function (and our Punch
and Judy example), in which the absence of parental or supervisory characters
leads to mischief from the protagonist, and the violation of the earlier
interdiction. This trope appears to be defined flexibly enough to fit this
interpretation as well, however.

\textbf{Don't Touch It, You Idiot!} (violation)
The title of this trope does not entirely convey the nuance of its meaning: TV
Tropes defines it as any order or interdiction that is inevitably violated at
some point later in the story. This actually fits well with Propp's original
definition, which stated that the ``interdiction'' and ``violation'' story
functions must always go together, as one inevitably leads to the other.

\textbf{Earn Your Happy Ending} (struggle)
This trope states that the characters in a story must face far more difficulty
than usual, overcoming more obstacles than most characters would have to face.
However, the characters get a happy ending as a result of their struggles.

Again, this does not fit our scene where Punch fights the crocodile for the
sausages. Though it does describe a struggle of sorts, it is more of a comedy
fight than anything arduous for the characters involved. We can probably find a
better match for this trope.

\textbf{Where It All Began} (return)
This trope does not match the definition of ``return'' that we have used from
Propp: in our case, it describes the return of a supervisory character some time
after they went away during the \emph{absentation} function. TV Tropes'
definition describes more the return of the protagonist to their hometown at the
end of the \emph{Hero's Journey}. In this case, we need to find a trope that
is the counterpart to the ``Parental Abandonment'' function from earlier, which
describes the return of the ``parents'' of that particular trope. The problem is
the slight mismatch in the definition of the trope against the Propp function we
are using. In the most literal case of the trope, the ``parents'' cannot return:
they are dead. Again, this indicates that perhaps we must find tropes
that more closely match Propp's definitions of \emph{absentation} and \emph{return}.

From our deeper analysis of TV Tropes' mappings of Propp story functions to
tropes, a number of issues have arisen:

\begin{itemize}
  \item Our use of Propp's story functions may be a little too ``flexible''.
    This means that the mappings of the functions we have used add an extra
    layer of interpretation to the translation, taking us away from the original
    intended meaning.
  \item Tropes, by their very nature, are a little ambiguous and open to
    interpretation. The same trope could even be expressed in multiple different
    ways, such as the ``Rule Number One'' trope.
\end{itemize}

In place of TV Trope's suggested ``struggle'' trope, \emph{Earn Your Happy
  Ending}, a more suitable trope to use would be the \emph{Chase Fight}:

\textbf{Chase Fight}: An X meets Y cross between a Chase Scene and a Fight Scene.

This is far more suited to our purposes, as the scene simply consists of the
crocodile fighting Punch by chasing him around the stage.

Similarly, in the place of the ``absentation'' and ``return'' tropes,
\emph{Parental Abandonment} and \emph{Where It All Began}, TV Tropes has a more
suitable pair to use:

\textbf{Put on a Bus}: a character is written out of a story so that they may
(possibly) return later.

\textbf{The Bus Came Back}: when one of the (main) characters returns back into
the story.

Though this trope pair better captures the essence of the leaving and return of
Joey, what if we also wanted some of the nuance of the \emph{Parental
  Abandonment} trope? The beauty of the capturing tropes as institutions is that
we can use both sets of tropes and let the player and character agents decide
the outcome, which could be a set of actions from a mixture of both tropes.

For simplicity, we can remove the ``Deal With The Devil'' trope, as well as
``Rule Number One''. The ``Don't Touch It, You Idiot'' trope includes the
interdiction that we wanted to express through the ``Rule Number One'' trope.
Also, as the ``Deal With The Devil'' trope involves a lot more than the simple
complicity with which Joey goes along with Punch's plans, it can be safely omitted.

This leaves us with just four tropes that describe our scene: \emph{Don't Touch
  It, You Idiot}, \emph{Put on a Bus}, \emph{Chase Fight}, and \emph{The Bus Came Back}.

\subsubsection{Trope Roles}
The character roles that appear in trope descriptions in TV Tropes differ
greatly from the \emph{Dramatis Personae} defined in Propp's morphology. For
each of the four tropes, we identify the following roles:

\begin{itemize}
\item \textbf{Don't Touch It, You Idiot}: the \emph{owner} (Joey) and the \emph{idiot} (Punch)
\item \textbf{Put on a Bus}: the \emph{absentee} (Joey)
\item \textbf{Chase Fight}: the \emph{chaser} (Crocodile) and the \emph{pursued} (Punch)
\item \textbf{The Bus Came Back}: the \emph{returnee} (Joey)
\end{itemize}

Clearly, this means that each character must adopt multiple roles throughout the
course of a narrative. In this scene alone, Joey the Clown plays the roles of
\emph{owner}, \emph{absentee} and \emph{returnee}. Punch himself must be an
\emph{idiot} and the \emph{pursued}.

These roles are not strictly defined in the descriptions found in TV Tropes, and
must be inferred by the reader. Furthermore, these roles do not describe
character archetypes such as \emph{Hero} or \emph{Comedic Sociopath}. One
interesting way to approach this issue could be to have an archetype inherit certain
character roles. For example, a \emph{Comedic Sociopath} could automatically
fill the roles of \emph{murderer}, \emph{idiot}, \emph{pursuer}, and
\emph{chaser}. This idea, while promising, is outside the scope of this thesis,
and so is discussed further only in the ``future work'' section~\ref{sec:future-work}.

\subsection{Return to the Sausages Scene}

Using the same process with which we described the \emph{Hero's Journey} in
terms of an institution in section~\ref{sec:tropes-as-insts}, this section shows
the translation of the \emph{Don't Touch It, You Idiot} into a formal institution.

First, we define the sequence of events that form the trope:

\begin{itemize}
\item The \emph{owner} has an \emph{object}
\item Then the \emph{owner} tells the \emph{idiot} to protect the \emph{item}
\item Then the \emph{owner} goes away
\item Then the \emph{idiot} breaks the \emph{item}
\item Then the \emph{owner} returns
\item Then the \emph{owner} fights the \emph{idiot}
\end{itemize}

This is the simplest possible interpretation of the trope. It is possible for
other interpretations to exist: for example, rather than the \emph{owner} returning and
fighting the \emph{idiot}, something bad might happen to the \emph{idiot}
instead. In our TropICAL language, alternative outcomes can be expressed using
the \emph{or} operator (sec.~\ref{tropical-or}) allowing the creation of more flexible tropes
which could be interpreted in multiple ways.

Now that we have determined the events that occur as part of the trope, we can describe its domain fluents, which include all the
actions of the characters, as well as their roles:
\begin{align*}
   \mathcal{D} &= \left\{\mathtt{owner, idiot, absentee, chaser, pursued, returnee, item, onstage, offstage}\right\} %\label{eq:domain}
\end{align*}

Also based on the above sequence events is the list of actions that may be
permitted to occur within the trope:
\begin{align*}
\mathcal{P} =& \left\{\mathtt{perm(go(offstage, absentee)), perm(go(onstage, returnee)),}\right.\nonumber\\
             &\left. {} \mathtt{ perm(chase(chaser, pursued)), perm(tellprotect(owner, idiot)),}\right.\nonumber\\
             &\left. {} \mathtt{perm(give(owner, idiot, object)), perm(break(idiot, item)),}\right.\nonumber\\
             &\left. {} \mathtt{perm(fight(owner, idiot))}\right\} %\label{eq:perm}
\end{align*}

\subsubsection{Trope Phases}

While arranging these four tropes in a linear sequence describes the
\emph{sausages} scene for the most part, our use of the \emph{Don't Touch It, You Idiot} trope in place of both of
Propp's \emph{interdiction} and \emph{violation} story functions introduces an
extra challenge to its implementation as an institution: it has two different
\emph{phases}. The first phase is triggered by one character warning another
character not to do something, the second is when the warned character performs
the forbidden action.

In fact, the \emph{Don't Touch It, You Idiot} trope can be divided into several
phases, or steps:

\begin{itemize}
\item The \emph{owner} has an \emph{object}
\item Then the \emph{owner} tells the \emph{idiot} to protect the \emph{item}
\item Then the \emph{owner} goes away
\item Then the \emph{idiot} breaks the \emph{item}
\item Then the \emph{owner} returns
\item Then the \emph{owner} fights the \emph{idiot}
\end{itemize}

This is the simplest possible interpretation of the trope. It is possible for
other interpretations to exist: for example, rather than the \emph{owner} returning and
fighting the \emph{idiot}, something bad might happen to the \emph{idiot}
instead. In our TropICAL language, alternative outcomes can be expressed using
the \emph{or} operator (sec.~\ref{tropical-or}) allowing the creation of more flexible tropes
which could be interpreted in multiple ways.

In order to properly describe the scene in terms of the four tropes, we must
first describe the mechanism by which we divide this trope into separate phases.
This is done through the addition of a \emph{phase} fluent which takes a trope
(institution) name and its current phase as parameters, such as:
$\mathtt(phase(intHerosJourney, phaseA))$.

At first, each trope is ``inactive'', reflected in the \emph{phase} fluent as
$\mathtt(phase(intTropeName, inactive))$ After each event that occurs in the
trope, its phase is updated, starting at \emph{phaseA}, then \emph{phaseB},
through to \emph{phaseZ}. Once a trope has finished (its final event has
happened), its \emph{phase} fluent is set to $\mathtt(phase(intTropeName,
done))$. It is set to \emph{done} rather than \emph{inactive}, because there may
be situations where we want to check whether or not a trope has already
appeared, so that it may not be repeated for example.

\begin{figure}[!t]
\abovedisplayskip=0pt
\abovedisplayshortskip=0pt
$\mathcal{C^{\uparrow}(X, E)}:\left\{\mbox{%
\begin{minipage}[c]{0.85\textwidth}
\begin{align}
  \langle \{\text{phase}(\mathtt{dontTouchItYouIdiot, inactive})\},\nonumber\\
  \mathtt{dontTouchItYouIdiot(owner, idiot, item)} \rangle %\nonumber\\
                                 % &\qquad\qquad
&\rightarrow \{\text{phase}(\mathtt{dontTouchItYouIdiot, phaseA}), \nonumber\\&\qquad\text{perm}(\mathtt{tellprotect(owner, idiot, item)})\}\label{eq:phase-inactive}\\
  \langle \{\text{phase}(\mathtt{dontTouchItYouIdiot, phaseA})\}, \nonumber\\
  \mathtt{dontTouchItYouIdiot(owner, idiot, item)} \rangle % \nonumber\\
                                 % &\qquad\qquad
&\rightarrow \{\text{phase}(\mathtt{dontTouchItYouIdiot, phaseB}), \nonumber\\&\qquad\text{perm}(\mathtt{go(owner, away)})
  \}\label{eq:phase-a}\\
  \langle\{\text{phase}(\mathtt{dontTouchItYouIdiot, phaseB})\},
  \nonumber\\\mathtt{dontTouchItYouIdiot(owner, idiot, item)} \rangle %\nonumber\\
                                 %&\qquad\qquad
&\rightarrow \{\text{phase}(\mathtt{dontTouchItYouIdiot, done}\}\label{eq:phase-done}
\end{align}
\end{minipage}}\right.$
\caption{Phase fluent initiation in the sausage scene} \label{fig:pj-phase-inits}
\medskip
\abovedisplayskip=0pt
\abovedisplayshortskip=0pt
$\mathcal{C^{\downarrow} (X, E)}:\left\{\mbox{%
\begin{minipage}[c]{0.85\textwidth}
\begin{align}
  \langle \{\text{phase}(\mathtt{dontTouchItYouIdiot, inactive})\},\nonumber\\
  \mathtt{dontTouchItYouIdiot(owner, idiot, item)} \rangle %\nonumber\\
                                 % &\qquad\qquad
&\rightarrow \emptyset\label{eq:phase-inactive}\\
  \langle \{\text{phase}(\mathtt{dontTouchItYouIdiot, phaseA})\}, \nonumber\\
  \mathtt{dontTouchItYouIdiot(owner, idiot, item)} \rangle % \nonumber\\
                                 % &\qquad\qquad
&\rightarrow \emptyset\label{eq:phase-a}\\
  \langle\{\text{phase}(\mathtt{dontTouchItYouIdiot, phaseB})\},
  \nonumber\\\mathtt{dontTouchItYouIdiot(owner, idiot, item)} \rangle %\nonumber\\
                                 %&\qquad\qquad
&\rightarrow \emptyset\label{eq:phase-done}
\end{align}
\end{minipage}}\right.$
\caption{Phase fluent termination in the sausage scene} \label{fig:pj-phase-terms}
\end{figure}

Returning to our \emph{Don't Touch It, You Idiot} example,
fig.~\ref{fig:pj-phase-inits} shows how a short version of the \emph{Dont Touch
  It You Idiot} trope with just two phases would be described with \emph{phase}
fluents that are initiated according to their corresponding events, with fig.~\ref{fig:pj-phase-terms}.

Extending this example to include all events, along with the corresponding
permissions and obligations that they grant our story characters, results in the
sets described in fig.~\ref{sausages-full}.

\section{TropICAL: a DSL for Tropes} % 1
\label{sec:tropical}

We propose \tropical\ (the TROPe Interactive Chronical Language) as a DSL for describing tropes in a constrained natural language, which we compile to InstAL~\cite{cliffe2007specifying}, through which process we capture the events that can occur and the consequent state changes, and from which a model is constructed in ASP.  The model, when given an event trace, delivers the evolution of the trope state, including crucially, the addition or removal of permission associations between actors and actions and the addition of obligations as consequences of actors' actions.  The syntax of \tropical\ is heavily influenced by the Inform 7~\cite{reed2010creating} language for interactive fiction, with the tropes being expressed in constrained natural language mostly conforming to Attempto Controlled English (ACE)~\cite{fuchs1996attempto}. \tropical\ shares similar aims to Inform~7 in that it aims to make interactive fiction authoring accessible to non-programmer story authors, however its focus is on authoring and combining tropes written in terms of roles, in contrast to complete stories in terms of actual characters. This section describes the syntax and semantics of \tropical, as well as sketching its compilation to InstAL.

The features of our trope description language are designed to be able to express the events, permissions and obligations of social institutions while addressing the shortcomings of planner and drama manager-based approaches:

\begin{enumerate}[R1.]
\item\label{perms} A way to express what certain characters are \textbf{permitted} to do at a given time.\footnote{An alternative approach would be to specify prohibitions, such that anything not prohibited is permitted, whereas we currently specify permissions, such that anything not permitted is prohibited.  This latter convention is the default semantics of InstAL, it is however straightforward from a technical point of view to adopt the alternative, as demonstrated in \cite{DBLP:conf/atal/KingLVDJPR15}.}
\item\label{obls} A way to express \textbf{obligations} on characters, with \emph{deadlines} and \emph{penalties} if the obligations are not fulfilled.
\item\label{state} A way to describe the state of a character or object at a
  given point in the story.
\item\label{seqs} A way of \textbf{sequencing} events in a trope.
\item\label{conds} A way to express \textbf{conditionals}, so that some events may occur only if others have.
\item\label{branches} A way to have \textbf{branches} in a trope, so that only one of two events may occur.
\item\label{embed} A way to \textbf{embed} sub-tropes inside of parent tropes.
\end{enumerate}
Requirements~R\ref{perms} to~R\ref{seqs} allow \tropical\ to describe the permissions, obligations and sequences of events that occur in social institutions, while R\ref{conds} and~R\ref{branches} add the ability to specify alternative paths through a story, as planner-based systems are able to do when combined with formalisms such as Propp's. Finally, requirement~R\ref{embed} enables us to nest tropes to go beyond the capabilities of structuralist formalisms of narrative, addressing the limitations described in the ``Related Work'' section of this paper. The TropICAL language satisfies these technical requirements, while being easy to learn for non-programmers, especially those familiar with the Inform 7 language (as is supported by the evaluation). It supports the expression of the above features in the following ways: 

% Replace with ``hostage situation''?
\begin{figure}[!t]
% \begin{lstlisting}[float=t!,caption={The ``Hero's Journey'' trope in TropICAL},label=lst:hero]
\begin{lstlisting}
"The Hero's Journey" is a Trope where:
  The Hero is at Home
  Then the Hero meets the Dispatcher
  Then the "Quest" trope happens
  Then the Hero returns Home
\end{lstlisting}%
\caption{The ``Hero's Journey'' trope in TropICAL\label{lst:hero}}
\medskip
% \begin{lstlisting}[float=t!,caption={The ``Evil Empire'' trope in TropICAL},label=lst:evil]
\begin{lstlisting}
"The Evil Empire" is a Trope where:
  The Villain has an Empire
  The Empire is a Weapon
  The Villain has a Victim
  The Villain may kill the Victim
  The Villain may kill the Hero
\end{lstlisting}%
\caption{The ``Evil Empire'' trope in TropICAL\label{lst:evil}}
\medskip
% \begin{lstlisting}[float=t!,caption={The ``Quest'' trope in TropICAL},label=lst:quest]
\begin{lstlisting}
"Quest" is a Trope where:
  Then the Hero must go to the Land Of Adventure before the Villain kills the Victim
    Otherwise, the Story ends
  When the Hero goes to the Land Of Adventure:
    The Hero may rescue the Victim
    The Hero may kill the Villain
      Or the Villain may escape
\end{lstlisting}
\caption{The ``Quest'' trope in TropICAL\label{lst:quest}}
\end{figure}

\begin{compactdesc}
% \subsection{Permissions}
\item[Permissions:]
Permissions on characters can be described by making statements in the simple present form, such as ``The Hero finds a weapon''. When compiled to InstAL code, these statements are equivalent to giving a character permission to do something. In this case, the Hero would have permission to find a weapon at that point in the trope.
The reason that this statement is translated to a permission is so that character agents can at any time have multiple permitted actions from several active tropes. It makes sense to make permission the ``default'' norm, rather than obligation, to allow the agents as much freedom as possible within the constraints of the story. If an author wants to make sure an agent carries out a particular action, they would specify it as an obligation instead (``The Hero \emph{must} find a weapon'').

% \subsection{Obligations}
\item[Obligations:]
Fig.~\ref{lst:quest} shows an example of an obligation with both a deadline and a violation event (both of which are optional). This obligation states that the hero must go to the Land of Adventure before the villain kills the victim, otherwise the story ends. In this case, the story ending is a particularly harsh penalty for the violation of the \emph{Quest} trope. Alternative violation events could be reduction of the hero's health, or the death of the victim.

\item[State:]
There are certain properties of each character or object in a story that will hold (or not
hold) at different points of time. For example, we want to be able to keep track
of the physical location of each character or object, what items a character possesses, or the
state of their health, for example. There are two keywords that we use to
express the state of a character: \emph{is} and \emph{has}.

For example, we can say that:
The \emph{sword} \textbf{is} in the \emph{castle}.
The \emph{hero} \textbf{is} at \emph{home}

Or:
The \emph{hero} \textbf{has} the \emph{sword}.
The \emph{villain} \textbf{has} the \emph{macguffin}.

These statements would be used to initiate and terminate fluents in their
corresponding institutions. The state of the story will progress according to
the combination of fluents that hold at a particular time.

% \subsection{Sequencing}
\item[Sequencing:]
As well as specifying permissions and obligations, it is frequently necessary to be able to express that certain events can only occur in a certain order. The \emph{Then} keyword means that the succeeding statement can only occur once the previous event has occurred. This is the means of implementing the \emph{phases} described in the ``Ordering Events in Tropes'' part of this section. In the \emph{Hero's Journey} (Fig.~\ref{lst:hero}) example, the Hero only has permission to return home (line 5) once everything in the \emph{Quest} trope has happened (Fig.~\ref{lst:quest}).
In some tropes, such as the \emph{Evil Empire} trope (Fig.~\ref{lst:evil}), the permissions and obligations described do not always need to occur in a certain sequence. In this case the trope serves the purpose of describing certain themes and characters in a part of a story, or events that may occur at any time, rather than in a specific order. The \emph{Evil Empire} trope only needs a villain with an empire to be present to fight the hero, so all of its permissions and obligations will apply from the beginning of the story.

% \subsection{Branching}
\item[Branching:]
Tropes may also contain branching paths where one or more events may take place. This is expressed in TropICAL with the \emph{Or} operator. Lines 6 and 7 of the \emph{Quest} trope in Fig.~\ref{lst:quest} express two alternatives: the Hero may kill the Villain or the Villain may escape. In this case, both permissions will hold at the same time, but both will be terminated once either permitted event has occurred. This makes it impossible for both events to happen in the story.

% \subsection{Conditionals}
\item[Conditionals:]
Conditionals are another way to create branching paths in a narrative, by allowing certain actions to occur if a particular trope state is reached. For this purpose we add the \emph{When} keyword, to express that when a specified state or event occurs, then some norms will be activated. In our \emph{Quest} example, we see an example of this on line 4, stating what the hero may do once in the Land of Adventure. This is similar to the ``Simple Present Statement'' example, except for the addition of the \emph{When} keyword. In much the same way, the statement after the \emph{When} keyword is a permission that holds on a character during the trope, except that it is also used to describe the consequences of certain events occurring. In this case, it states that when the Hero goes to the Land of Adventure, they may either kill the Villain, or the Villain may escape.

% \subsection{Embedding Tropes Within Tropes}
\item[Embedding Tropes Within Tropes:]
Tropes can be embedded inside other tropes by simply writing \emph{The X trope happens}.
Line 4 of the \emph{Hero's Journey} trope (Fig.~\ref{lst:hero}) shows an example of embedding one trope inside another. In this case, the \emph{Quest} trope occurs at a certain point in the \emph{Hero's Journey}, once the hero has met the dispatcher character. Because this is sequenced using the \emph{Then} keyword, these events must occur in the specified order, and the norms described in the \emph{Quest} trope cannot hold until the specified point in the trope. However, if the \emph{Then} keyword is omitted (\emph{The ``Quest'' trope happens}), this means that the norms contained inside the embedded trope apply from the start of its containing trope. In our example, this would mean that the Hero would be free to embark on a quest at any time, rather than waiting to first meet a dispatcher character.

Rather than compiling all the tropes into one institution, they are compiled
into separate institutions and coordinated using a ``bridge'' institution. This
technique, described by~\citep{bath45254}, allows institutions to generate
events inside of other institutions while keeping each one separate. The
cross-institution event generation logic is described in the ``bridge institution''
section of the next chapter (sec.~\ref{sec:bridges}).
\end{compactdesc}

This chapter has described the concept of tropes, breaking down several examples
and describing them as normative institutions. We then returned to our
\emph{Punch and Judy} example to describe a scene in terms of tropes rather than
Propp functions. Using this knowledge, we created a set of requirements for the
creation of a domain specific language for describing tropes that compile to
formal descriptions of institutions in terms of social norms, as well as a
high-level overview of the features of the language we have created (named \emph{TropICAL}).

The next chapter describes the actual implementation of \emph{TropICAL}, with snippets of code for example tropes and their
corresponding translations to \emph{InstAL}.

\chapter{Intelligent Agents as Story Characters}
\label{cha:agents}

% tropical

\section{Punch and Judy with Emotional Agents}
\label{sec:emotional-pj}

% HACK: predicted poms remaining: 73 / 79
\section{VAD emotional model} \label{sec:emotion}
\begin{figure}[!t]
% {\includegraphics[height=5in]{VAD.png}}
\begin{minipage}{0.6\textwidth}
\includegraphics[width=\textwidth]{VAD.png}
\end{minipage}\hfill%
\begin{minipage}{0.35\textwidth}\raggedright
The VAD model illustrates how valence, arousal and dominance values map to identifiable emotions. Valence, arousal and dominance can each have a value of low, medium or high. This allows the agents to have a total of 27 distinct emotional states.
The valence and arousal levels of each agent are affected by the actions of other agents. For example, a character being chased around the stage by Punch will see their valence level drop while their arousal increases. According to Russell's circumplex model of emotion~\cite{russell1980circumplex}, this would result in them becoming \emph{afraid\/} (if their dominance level is low).
\medskip

An agent's emotional state affects its ability to fulfil its institutional obligations. An agent that is \emph{furious\/} might have no problem carrying out an obligation that requires them to kill another agent. If that same agent is \emph{happy\/} or \emph{depressed}, however, they might not have the appropriate motivation to perform such a violent action.
% \rule{\textwidth}{2cm}
\end{minipage}
\caption{VAD emotional values (figure adapted from Ahn et al.)~\cite{ahn2012nvc}} \label{fig:vad}
\end{figure}
In order to make the agents acting out the Punch and Judy show more believable, we apply an emotional model to affect their actions and decisions. For this, we use the valence-arousal (circumplex) model first described by Russell~\cite{russell1980circumplex}.  To give each character its own distinct personality, we extend this model with an extra dimension: dominance, as used by Ahn et al.~\cite{ahn2012nvc} in their model for conversational virtual humans. This dominance level is affected by the reactions of the audience to the agents' actions. For example, Judy may become more dominant as her suggestions to hit Punch with a stick are cheered on by the audience, emboldening her into acting out her impulses.  A detailed description appears in the text in Figure~\ref{fig:vad}.

%% Figure~\ref{fig:vad} shows how valence, arousal and dominance values map to identifiable emotions. Valence, arousal and dominance can each have a value of low, medium or high. This allows the agents to have a total of 27 distinct emotional states.
%% The valence and arousal levels of each agent are affected by the actions of other agents. For example, a character being chased around the stage by Punch will see their valence level drop while their arousal increases. According to Russell's circumplex model of emotion~\cite{russell1980circumplex}, this would result in them becoming \emph{afraid\/} (if their dominance level is low).

%% An agent's emotional state affects its ability to fulfil its institutional obligations. An agent that is \emph{furious\/} might have no problem carrying out an obligation that requires them to kill another agent. If that same agent is \emph{happy\/} or \emph{depressed}, however, they might not have the appropriate motivation to perform such a violent action.

It is important to note that the emotional model is part of the agent belief state, and not held in the institution. We want to explore how the characters of the story might be able to choose actions based on their emotional state. While the institution could theoretically calculate the emotional state for each agent in turn and dictate this to them along with the norms of the narrative, it makes sense to decouple this feature from the narrative institution in order to separate the characters from the events of the story. %\jnote{review this paragraph}

Agents' emotional states change according to their interactions with the audience. This is unrelated to what is happening in the narrative, and so this underscores the decision not to include any emotional modelling in the institution. Also, we want the agents to have some degree of freedom within the narrative world. They should be allowed to determine their emotions themselves, so that in extreme emotional states they can perform `irrational' or `extreme' actions that may not necessarily fit into the narrative.

\subsection{VAD emotions in Jason}
Emotions are implemented as beliefs inside an agent. An agent believes it has a certain level of valence, arousal and dominance, and it works out its emotional state based on a combination of these three factors. When the audience cheers or boos them, this changes the belief holding the relevant emotional variable, and their emotional state as a whole is recalculated.

Valence, arousal and dominance values can take values of -1 (low), 0 (medium) or 1 (high). Listing~\ref{lst:emotions} shows the emotional belief rules for an agent with medium dominance (a dominance level of 0). Note that an agent maintains beliefs about both its current emotion label (such as sleepy or happy) and the separate valence, arousal and dominance values at the same time.  Similar sets of rules handle the belief emotion for the other dominance levels.  %We address the matter of how external stimuli affect dominance in section~\ref{sec:norms as percepts}.\jnote{still needs doing}

Every time an emotional variable (valence, arousal, or dominance) changes, an agent's emotion is changed according to the rules in listing~\ref{lst:emotions}. While an agent's valence, arousal and dominance belief values affect the way it makes decisions internally, the results of combinations of these values (sleepy, happy, etc) are broadcast as external actions. The reason for this is that an agent's emotional state may affect the way in which the character is animated: changing the speed at which they move or turning their smile into a frown, for example. For this reason, whenever an emotional change takes place, the new emotion is published as an external action of the agent so that observing entities may perceive it. The Bath sensor framework described in section~\ref{sec:arch} provides the means for this evidence of the agent's internal state change to be received by the animation system and reflected accordingly in the display.

\begin{lstlisting}[float=!t,caption={Emotional rules for a character with medium dominance},label=lst:emotions,escapechar=\%,basicstyle=\scriptsize\ttfamily]
emotion(sleepy) :- valence(0) & arousal(-1) & dominance(0).
emotion(neutral) :- valence(0) & arousal(0) & dominance(0).
emotion(surprised) :- valence(0) & arousal(1) & dominance(0).
emotion(anxious) :- valence(-1) & arousal(-1) & dominance(0).
emotion(unhappy) :- valence(-1) & arousal(0) & dominance(0).
emotion(embarrassed) :- valence(-1) & arousal(1) & dominance(0).
emotion(glad) :- valence(1) & arousal(-1) & dominance(0).
emotion(happy) :- valence(1) & arousal(0) & dominance(0).
emotion(delighted) :- valence(1) & arousal(1) & dominance(0).
\end{lstlisting}

\begin{lstlisting}[float=!t,caption={AgentSpeak rules for changing an agent's emotional values from audience responses},label=lst:response,basicstyle=\scriptsize\ttfamily]
+!changeMood
  <- ?emotion(Z);
     emotion(Z).
+response(_, boo) : asking
  <- -+valence(-1);
     -+dominance(-1);
     !changeMood.
+response(_, cheer) : asking
  <- -+valence(1);
     -+dominance(1);
     !changeMood.
\end{lstlisting}

Listing~\ref{lst:response} shows the AgentSpeak rules describing how an agent's
valence and dominance levels are changed by the audience cheering or booing
their actions. These AgentSpeak plans describe what the agent should do in
response to a goal addition (denoted by a $\texttt{+!}$ at the start of the plan
name) or a belief addition (prefixed by a simple $\texttt{+}$). In the case of
Listing~\ref{lst:response}, the $\texttt{+!changeMood}$ plan updates the agent's
emotional state based on its valence-arousal-dominance values and broadcasts the
result as an external action. The $\texttt{+response}$ plans raise or lower an agent's valence and dominance levels depending on whether the agent perceives a ``boo'' or ``cheer'' response from the audience.

An agent announces what they intend to do, then waits three seconds. During this time, they have the belief that they are `asking' the audience, and listen for a response. A boo reduces an agent's valence and dominance, while a cheer raises them. For each response, the \texttt{changeMood} goal is triggered, which looks up and broadcasts the agent's emotional state to the other agents and environment.

\section{Agent decision making} \label{sec:decisions}
The agents choose which goals to pursue according to three factors: their permitted actions, their obliged actions and their emotional state. Though obliged actions are given priority, and while agents' decisions are generally constrained by their permitted actions, an agent's emotional state has the final say in its decisions. In this way, an agent will follow the social norms of the narrative, but only according to their own mood.

\paragraph{Agent goals and plans}
% \subsection{Agent goals and plans}
The agents are implemented using a belief-desire-intention (BDI) psychological model using the Jason platform ~\cite{bordini2007programming}. An agent's knowledge about the state of their world and themselves are stored as \emph{beliefs}, with new information coming in from the environment getting added to their belief base as \emph{percepts}, which are ephemeral and only last for one reasoning cycle of an agent.

Agents are created with goals and plan libraries. Any goal that an agent is set on carrying out at any point is an \emph{intention}, whereas a goal that an agent has but is not yet pursuing is a \emph{desire}. Plan libraries describe the steps agents need to take in order to achieve goals, as well as how to react to changes in agents' environments.

\paragraph{Norms as percepts}\label{sec:norms as percepts}
% \subsection{Norms as percepts}\label{sec:norms as percepts}
When an event occurs, it is added to the event timeline, which is used to query the ASP (Answer Set Programming) solver to obtain the set of norms that hold after the new event has occurred. The new permissions and obligations are then added to each agent as \emph{percepts}. Each time this happens, the set of permitted and obliged actions that an agent sees is changed to be only those that apply at that instant in time, with the previous norms being discarded.

Agents choose between permitted and obliged actions based on their emotional state at the point of decision making. Obliged actions are given a higher priority over permitted ones for most of the emotional states that an agent can be in, though not always. If an agent is in a sulky mood, for example, they may decide to ignore what they are obliged to do by the narrative, even though they know there will be consequences.

For example, in the scene where Joey gives the sausages to Punch, Punch may see that he has permission to eat the sausages, drop them, fight the crocodile, run away (leave the stage) or shout for help at the crocodile or audience. His obligation for the scene, in accordance with the Punch and Judy narrative world, is to either eat the sausages himself, or let the crocodile have them. This ends Propp's \emph{interdiction\/} story function with a \emph{violation\/} function. Note that his obligation in this case is not to guard the sausages as asked to by Joey. While Joey's entrusting of the sausages is certainly an obligation in itself, Punch's main obligations are to the narrative. Lesser obligations towards characters in the story can be implemented as having a lower prority than those of the story itself.

Similarly, at times of extreme emotion, an agent may decide to disregard their set of permitted actions entirely, instead acting out their innermost desires. For example, an angry Punch might decide to just attack Joey instead of agreeing to look after the sausages, or he might just decide to give up and leave if he is depressed. The key point is that the norms act as the will of the \emph{narrative}, guiding the story forward, rather than a strict set of rules that the agents must follow at all costs.

\emph{Violation events\/} add percepts to the agents telling them that they are in violation of the narrative norms. Once an agent receives such a percept, an emotional variable is changed. Typically, their dominance will decrease. The reasoning behind this is that if agents are unwilling to participate in the story, they should have less influence in its course of events.

\section{Architecture}\label{sec:arch}

\paragraph{Multi-Agent System} \label{sec:inst}
% \subsection{Multi-Agent System} \label{sec:inst}
We use the JASON framework for belief-desire-intention (BDI) agents~\cite{bordini2007programming}, programming our agents in the AgentSpeak language.
The VAD emotional model is represented inside each agent as a set of beliefs. Each agent has beliefs for its \emph{valence}, \emph{arousal} and \emph{dominance} levels, each of which can take the value of low, medium or high, as discussed in section~\ref{sec:emotion}. This combination of VAD values creates one of the 27 emotional states shown in figure~\ref{fig:vad}, affecting whether or not an agent breaks from its permitted or obliged behaviour.


\chapter{TropICAL: A Language for Story Tropes}
\label{cha:tropical}
% TODO: describe pipeline
% parse -> data structure -> InstAL -> AnsProlog

This chapter describes the design and implementation of TropICAL (the TROPe
Interactive Chronical Action Language), our Domain Specific Language for
story tropes. The purpose of this language is to allow for the
creation of interactive narratives through the description of tropes in a
constrained natural language. The tropes created through the language are
designed to be reusable components that can go into a ``library'', from which
story authors can choose the tropes best suited to the particular story that they
are creating.

The motivation for the creation of TropICAL is the lack of any methods of
creating interactive narrative that are suitable for non-programmers to use. The
systems described in the literature review in
chapter~\ref{cha:literature-review}, such as those using the Mimesis
architecture~\citep{young2004architecture}, a drama manager and planner such as
in Fa\c{c}ade's system~\citep{mateas2003faccade}, or linear logic as in Ceptre's
system~\citep{martens2015ceptre}, all require the story author to be familiar with
planner-based systems or the description of formal logics. The purpose of
TropICAL is to greatly reduce the barrier to the creation of interactive stories
by allowing authors to describe the components of their story using constrained
natural language. In fact, many authors using our system should not even need to
write their own tropes using TropICAL, and will instead simply select the tropes
that they need for their story from a pre-created ``library'' of tropes. This
process is facilitated through the ``StoryBuilder'' user interface described in
chapter~\ref{cha:storybuilder}

The way in which our TropICAL language is used is as follows:

\begin{itemize}
  \item It is based around the idea of using \emph{story tropes} as reusable
    components that can be combined to create stories
  \item It uses \emph{controlled natural language} to describe these tropes
  \item The controlled natural language is then parsed and compiled to a data
    structure that describes the formal features of each trope
  \item The intermediate data structure is then used to generate InstAL code
  \item The InstAL code is compiled to AnsProlog, an Answer Set Programming
    language
  \item The AnsProlog code is then run through
    \emph{clingo}~\citep{gebser2011potassco}, an answer set solver, to generate
    all the possible sequences of events that can occur in the story.
\end{itemize}

This chapter begins by describing TropICAL's constrained natural
language syntax in section~\ref{sec:t-constrained}.
Section~\ref{sec:t-requirements} explores the use cases for our language, deriving
a set of requirements from them. Based upon these requirements, the features and
design of the language are described in section~\ref{sec:language-design}.
The translation of TropICAL to InstAL code is demonstrated in Section~\ref{sec:t-codegen}, with samples of generated InstAL
institutions, Answer Set
generation (section~\ref{sec:t-asp}), and finally its extension for the description of legal policies
(section~\ref{sec:t-legal}). The chapter concludes with a summary in
Section~\ref{sec:tropical-summary}, describing how some requirements from
Section~\ref{sec:requirements} were met, while others are addressed through the
\emph{StoryBuilder} tool described in Chapter~\ref{cha:storybuilder}.


\section{Controlled Natural Language Syntax}
\label{sec:t-constrained}
% DONE intro: 1
% DONE describe ACE: 2
% DONE describe Inform 7: 2

TropICAL uses a \emph{Controlled Natural Language} (also referred to as
\emph{Constrained Natural Language}) syntax, meaning that it superficially
resembles natural English, but is only a subset of the full language.

There are two types of Controlled Natural Language (CNL). The first type are \emph{naturalistic} languages
such as ASD Simplified Technical English~\citep{asd2007simplified}, designed to
be used in documentation for technical industries such as aerospace or defense,
or Ogden Basic English~\citep{ogden1944basic}, a simplified language for teaching
English as a second language. This type of Controlled Natural Language merely
describe a subset of the parent language. The other type of CNL are
\emph{formalistic}, with a formal
syntax and semantics, which can be mapped to rules in other formal languages
such as first-order logic. Attempto Controlled English~\citep{fuchs1996attempto}
is an example of this formal type of CNL, and the one which forms the basis of
our syntax for TropICAL.

\subsection{Attempto Controlled English}

Attempto Controlled English (ACE) is a controlled natural language that is also
a formal language, and was created by the University of Zurich in 1995. It is
still under development, with version 6.7 of the language announced in 2013. ACE
has been used in a wide variety of fields, such as software specifications,
ontologies, medical documentation and planning.

ACE's vocabulary has three components:

\begin{itemize}
  \item predefined function words
  \item predefined fixed phrases (\emph{there is}, \emph{it is false that}, ...)
  \item content words (nouns, proper nouns, verbs, adjectives, adverbs)
\end{itemize}

The Attempto Parsing Engine (APE) has a lexicon of content words, and users can
define their own content words. User-defined content words take precedence over
the built-in lexicon.

ACE's syntax is expressed as a set of construction rules (admissible ACE
sentence structures), with syntactically
correct sentences described as a set of interpretation rules, which contain the
actual semantics of the sentences.

The simplest ACE sentences follow a \emph{subject + verb + complements +
  adjuncts} structure:

\begin{lstlisting}
  A dragon sleeps.
  The sky is blue.
  A princess owns a castle.
  A king gives a sword to a knight.
\end{lstlisting}

In this structure, every sentence must have a subject and a verb. Sentences that
do not contain a verb are expressed with the \emph{there is} structure:

\begin{lstlisting}
  There is a kettle.
  There are 3 balls.
\end{lstlisting}

Details can be added to these sentences through the use of adjectives, and the
sentences can be joined together with the \emph{and} and \emph{then} keywords.

Other ways of modifying the sentences include \emph{negation}, adding
\emph{quantifiers}, or making the sentences \emph{interrogative} or
\emph{imperative}. We do not go into further details in this thesis, as we are
using ACE simply as an inspiration for our syntax, rather than following its
design entirely. For more details, see the ``ACE in a Nutshell'' page of its
documentation for an overview of the language~\citep{ace-nutshell}.

\subsection{Inform 7}
In addition to ACE, the other main inspiration for TropICAL's syntax is the
\emph{Inform 7} language.

Inform is a programming language for the creation of Interactive Fiction (also
called \emph{text adventure games}). All versions of Inform generate
\emph{Z-code}, which is interpreted by the \emph{Z-machine} virtual machine for
interactive fiction.

The syntax of the language has changed multiple times since its creation in
1993. The version of the language that we are most interested in is Inform 7~\citep{reed2010creating},
which is its most recent incarnation. Prior to Inform 7, the language used
traditional programming models such as procedural and object-oriented paradigms.
With version 7, however, its creator Graham Nelson redesigned its
syntax completely to allow authors to create their stories using controlled
natural language, so that the experience of story creation became closer to that
of writing a book.

The simplest possible game authored with Inform 7 would be:

\begin{lstlisting}[showstringspaces=false]
"Hello World" by "I.F. Author"

The world is a room.

When play begins, say "Hello, world."
\end{lstlisting}

The above code simply describes the name of the game (``Hello World''), declares
that there is one room in the game, and sets the game to say ``Hello World'' to
the player when the game starts.

An extended example, Will Crowther's cave exploration simulation, would be
described like this:

\begin{lstlisting}[showstringspaces=false]
"Cave Entrance"

The Cobble Crawl is a room. "You are crawling over cobbles in a low passage. There is a dim light at the east end of the passage."

A wicker cage is here. "There is a small wicker cage discarded nearby."

The Debris Room is west of the Crawl. "You are in a debris room filled with stuff washed in from the surface. A low wide passage with cobbles becomes plugged with mud and debris here, but an awkward canyon leads upward and west. A note on the wall says, 'Magic word XYZZY'."

The black rod is here. "A three foot black rod with a rusty star on one end lies nearby."

Above the Debris Room is the Sloping E/W Canyon. West of the Canyon is the Orange River Chamber.
\end{lstlisting}

The ``Cave Entrance'' example above is just one scene from the game. The code
begins with a declaration of the ``Cobble Crawl'' room, followed by the
description that appears when the player enters it. The next line declares that
a wicker cage is in the room, which is an object that the player can interact
with. The sentence after the declaration is the description of the item that
appears when the player looks around the room (``There is a small wicker cage
discarded nearby.'').
The ``Debris Room'' follows the same format described above, with a ``black
rod'' item inside the room. The final line of code simply describes how the
rooms are arranged in the game, using words like ``above'' and ``west of''.

Although Inform 7 is not based directly on ACE, we can see many of ACE's
features in it. For example, it uses the same syntax for simple statements such
as \emph{X is a Y}. Most statements in the listing above are pairs of sentences,
with the first in each pair declaring the existence of an object or room, and
the second being an uninterpreted string that is printed out as the description
of the defined object.

Before version 7, Inform was already widely used as a language for the creation
of Interactive Fiction. Given its continued popularity after the switch to
a controlled natural language syntax, it appears that the new programming
paradigm is successful.

The wide adoption of Inform 7 by story authors with no other programming
experience motivates our decision to create a controlled natural language syntax
for TropICAL, our own trope-based programming language. The intended audience is
the same, so an additional benefit would be that story authors who can code with
Inform 7 would also be able to use TropICAL.

\section{Requirements}
\label{sec:t-requirements}
% TODO intro: 1
% TODO use cases: 2
% TODO requirements: 2
% TODO list features: 1
% TODO tech used: 1
% TODO relate features to the requirements: 2

The TropICAL language is designed to be used by authors of interactive
narratives, enabling them to easily describe the ``rules'' of a story through
tropes. It is designed to be used in interactive storytelling systems where
intelligent software agents are acting out the roles of the characters in the
story. These agents, through an architecture such as BDI (Belief, Desire,
Intention), have their own goals and plans for how to achieve those goals.
TropICAL does not assist in the authoring of these characters and their plans or
lines of dialogue. Instead, its purpose is to direct the actions of these
``character'' agents so that their behaviour follows the rules of the author's
story. TropICAL is designed to be used with a system where the character agents have
already been authored. It guides them by allowing the characters to know what
they are permitted and obliged to do according to the story. This is achieved by
giving the characters new percepts so that they may \emph{believe} they are
allowed to take certain courses of action. For a description of how this is
implemented in practice, see section~\ref{sec:emotional-pj} of chapter~\ref{cha:agents}.

\subsection{Use Cases}
\label{sec:use-cases}

In order to design a set of tools that is suited to interactive story authors,
our intended end users, we derive our set of requirements from specific use
cases. These use cases come from the user stories of three users: Alice, Bob and
Charlie. In each case, we tell the user's story, highlighting their different
levels of technical experience, along with
their various needs for and ways of using tools for
interactive story creation.

\subsubsection{Alice}

Our first user story focuses on \emph{Alice}, a user with some programming knowledge and
specialist expertise in the field of intelligent agents:

\begin{quote}
  Alice is an interactive narrative researcher using the JASON intelligent agent
  framework to create the characters of a story. The story is a classic
  detective noir story, with a private investigator as the main character and
  several suspects, one of which is the killer. She has coded their behaviour
  and dialogue as a set of plans, which each agent will follow according to
  their beliefs, desires and intentions as they ``act out'' the story.

  Alice sets the simulation running, but sees that although the agents are interacting
  with each other as intended, there is no structure or sense of drama to the
  story. It merely seems to be one event happening after another.

  She decides to use the TropICAL framework to guide the actions of the
  character agents. The tropes that she chooses to describe her story are the
  ``Three Act Structure'', ``Chekov's Gun'', ``Murder Mystery'' and ``Rooftop
  Chase'' tropes. These tropes are used to not only give the story structure,
  but also to encourage the characters in it to behave in a way that makes the
  simulation ``feel'' like a murder-mystery genre novel.

  When using TropICAL, she selects the ``Three Act Structure'' and ``Chekov's
  Gun'' tropes from a pre-written library of tropes. However, she cannot find
  tropes that describe what a ``Murder Mystery'' story is, or what should happen
  in a ``Rooftop Chase'' scene. She creates these by writing her own tropes in TropICAL:

  \begin{lstlisting}[showstringspaces=false]
  ``Murder Mystery'' is a trope where:
    The Detective is a role
    The Killer is a role
    The Victim is a role
    The Killer kills the Victim
    The Detective investigates the Killer
    Then the Detective catches the Killer
      Or the ``Rooftop Chase'' happens

  ``Rooftop Chase'' is a trope where:
    The Detective is a role
    The Fugitive is a role
    The Rooftop is a place
    The Detective goes to the Rooftop
    The Fugitive goes to the Rooftop
    The Detective chases the Fugitive
    Then the Detective catches the Fugitive
      Or the Fugitive escapes
\end{lstlisting}
\end{quote}

From Alice's story, we can derive the following requirements:

\begin{itemize}
\item She needs a system for describing stories that can be used with a
  multi-agent system framework such as JASON
\item The system needs to direct the behaviour of the agents so that their
  actions fit within the description of a story
\item She needs to be able to write and re-use story components that fit the
  theme of her story (in this case, ``film noir'').
\item These story components must be able to express both sequences of events
  and diverging (``branching'') events
\item She wants to be able to select existing story components from a library or database
\item She also wants the ability to create her own story components and save
  them to the database
\end{itemize}

\subsubsection{Bob}

Our second hypothetical user, \emph{Bob}, has no technical expertise at all:

\begin{quote}
  Bob is a traditional story author interested in writing interactive narratives
  with branching plot lines. He has written a few stories in the style of the
  ``Choose Your Own Adventure'' books, using the traditional method of using paper and
  pen to write down the stories. He implements the story branches by labelling
  each scene with a name such as ``The Hero Sets Off on a Journey'', or ``The
  Villain Deceives a Victim'', and referring to them by name whenever a
  branching point occurs. For example, when the Hero of a story is tasked with a
  quest to complete, the decision could be to go to the ``Leave Home and Go on a
  Journey'' scene, or the ``Stay at Home and Avoid Adventure'' scene, which
  would be the names of scenes leading either to further choices, or the end of
  the story.

  While he is writing the story, he finds it difficult to keep a mental model of
  the whole plot, with all its branches, in his head. As he creates new choices
  and branching points in the story, he finds himself having to pin up notes on
  a board, and connect them with string.
  Frustrated, he searches for software to help him with the process of authoring
  non-linear stories, eventually finding a piece of software called ``Twine''.
  Twine's user interface uses the analogy of notes on a pinboard connected with
  string, the exact process he is used to. He quickly inputs the story he is
  working on into Twine, and it produces an HTML website with hyperlinks to
  pages representing the scenes of his story.
  Next, Bob wants to get more ambitious by abstracting away different parts of
  his story and re-using them in different places. He wants his story to be
  heavily quest-driven, with the player being given many different quests and
  missions over the course of the game. He decides that all of these quests
  follow a common pattern:

  \begin{lstlisting}[showstringspaces=false]
  The Hero is at Home
  Then the Hero meets the Dispatcher
  Then the Hero receives a Quest
  Then the Hero completes the Quest
    Or the Hero ignores the Quest
\end{lstlisting}

  In this case, the actual content of the ``Quest'' would change for each
  individual quest, but the way the player receives and embarks on the Quest
  remains the same. Bob wants to write this combination of nodes and branches
  once, and then re-use it as a single node labelled ``Quest''. Unfortunately,
  he finds no way to do this using Twine, and he is forced to copy and paste the
  existing nodes every time he wants to put a Quest into his story.
\end{quote}

Bob shares some requirements with Alice:

\begin{itemize}
  \item He wants to be able to create story components with sequences of events
    and branching events
  \item He wants to be able to save and re-use the components (to a database,
    for example)
  \item He also wants the ability to create his own story components to add to
    the database
\end{itemize}

However, Bob also has some different requirements from Alice:

\begin{itemize}
  \item He needs to embed tropes inside other tropes (as ``sub-tropes'')
  \item He needs to be able to visualise all of the story branches that result
    when one or more tropes are combined together
\end{itemize}

\subsubsection{Charlie}

Our final user, \emph{Charlie}, has technical expertise, and is also familiar with
existing research techniques for interactive narrative generation:

% Charlie needs to give agents autonomy to break away from the prescribed narrative
\begin{quote}
  Charlie is an experienced Interactive Narrative author that wants to
  automatically generate many of the small narrative details. By using
  intelligent agents to simulate the characters in the story, she hopes that
  their intelligence and autonomy will result in some surprising narrative arcs.

  First, she authors the character agents using the JASON library for
  intelligent agents. To do this, she must give them a library of plans to
  execute in order to carry out certain goals. At any moment in the story, each agent has a set of
  \emph{beliefs} about its environment, plans that they \emph{desire} to carry
  out, and plans that they \emph{intend} to execute.

  She sets the simulation running, but finds that the characters actions do not
  follow the structure of a story. She starts out by directing their actions using a planner-based system, but
  finds that the resulting stories don't make use of the agents' intelligence.
  The planner is simply telling the agents what to do, based on its goals. She
  thinks that this is fine most of the time, but nonetheless desires to
  occasionally see some unexpected (but plausible) actions from the characters.
  For example, if a character is close to finishing a plan to achieve a goal,
  she wants them to  choose to override what they are told to do by the director
  planner, and accept some penalty that comes with the violation of the story.

  She attempts to write this functionality into the agents' decision-making
  process, but this turns out to be too time-consuming, and she soon gives up.
  If only there were a better way.
\end{quote}

Charlie's only unique requirement is:

\begin{itemize}
  \item A way to describe the story such that the agents are told what to do
    next, but are able to break away from these commands when necessary.
\end{itemize}

\subsection{Requirement Specification}
\label{sec:requirements}

Based on the use cases above (section~\ref{sec:use-cases}), we can create the
following requirements:

\begin{enumerate}[R1.]
  \item\label{req:agents} The software must be able to integrate into a
    multi-agent framework such as JASON
  \item\label{req:direct} The system must direct the behaviour of agents to fit
    a story description
  \item\label{req:components} The user should be able to write and re-use
    existing story components (from a library)
  \item\label{req:sequences} The components must be able to express sequences of events
  \item\label{req:branches} The components must be able to express branching
    (diverging) events
  \item\label{req:subtropes} The user should have the ability to nest existing
    components inside new components
  \item\label{req:vis} The user should be able to visualise the branches of the
    story that result in the addition or modification of components to the story.
  \item\label{req:norm} The story must tell the character agents what to do, but
    they should be free to break away from it in extreme circumstances, to add a
    degree of unpredictability
\end{enumerate}

The next section describes the design of the TropICAL language, relating each
design decision back to these requirements derived from the use cases.

\section{Language Design and Features}
\label{sec:language-design}
In this section, we describe the main features of the TropICAL programming
language:

\begin{itemize}
  \item Entity Declarations (Section~\ref{sec:dec-code})
  \item Sequences of Events (Section~\ref{sec:seq-code})
  \item Obligations (Section~\ref{sec:obl-code})
  \item Branching Events (Section~\ref{sec:branch-code})
  \item Subtropes (Section~\ref{sec:subtrope-code})
\end{itemize}

Each part of this section contains code samples to demonstrate the use and
implementation of each language feature. Many of these samples are only extracts from
full trope definitions in TropICAL, however. More extensive examples of trope
definitions can be found in Appendix~\ref{appendix:tropes}.

This section begins by showing how the entities (roles, objects and places) of a
trope are defined by the author (Section~\ref{sec:dec-code}), then follows this with a description of how
these entities' actions are composed together to form sequences of events (Section~\ref{sec:seq-code}).
Normally, these events give the character entities \emph{permission} to perform
certain actions, so section~\ref{sec:obl-code} shows how an author may
describe events that \emph{must} happen in the story through obliging characters
to carry out specific actions.
In cases where multiple story paths may be taken, TropICAL allows for the
description of alternative narrative branches, the syntax for which is described
in section~\ref{sec:branch-code}. Finally, the means for creating trope
abstractions by embedding existing tropes into new ones is demonstrated in
section~\ref{sec:subtrope-code}.

\subsection{Entity Declarations}
\label{sec:dec-code}
A trope may contain three types of entities that take part in a trope:

\begin{itemize}
  \item Roles (characters acted out by the agents)
  \item Objects (items that are used or manipulated by the characters)
  \item Places (locations that the characters visit)
\end{itemize}

Before an author can use any characters, objects or places in their tropes, they
must first declare them at the top of the trope file. For example, to declare
that the \emph{Hero} and \emph{Evil Villain} are character roles, one would write:

\begin{lstlisting}[label={lst:dec-roles}, caption={Role declarations}]
The Hero is a role
Evil Villain is a role
\end{lstlisting}

In TropICAL, the ``The'' at the beginning of role, object or place names is
optional. If the name of an entity begins with ``The'', the parser
omits it from the name. Also, entity names can consist of multiple words, as is
the case for the \emph{Evil Villain} character. However, each word in a name must
always start with a capital letter.

Similarly, to declare that our trope contains \emph{Sword} and \emph{Magic Potion} objects,
the author would write:

\begin{lstlisting}[label={lst:dec-objs}, caption={Object declarations}]
The Sword is an object
The Magic Potion is an object
\end{lstlisting}

Finally, \emph{Home} and \emph{The Land of Adventure} locations in the
trope are declared in the same way:

\begin{lstlisting}[label={lst:dec-places}, caption={Place declarations}]
Home is a place
The Land of Adventure is a place
\end{lstlisting}

Once the author has declared the entities of their trope at the top of the trope
definition, they are able to use them inside the events of the trope itself.

\subsubsection{Entity Instances}
In cases where multiple characters fulfil the same role (a story could have two
hero characters, for example), or there are multiple objects or locations of the
same type, these separate entity instances can be passed to the compiler in
separate files. An example with multiple Heroes, Swords and Evil Lairs would be:

\begin{lstlisting}[label={lst:dec-instances}, caption={Entity instances}]
Harry Potter is a Hero
Luke Skywalker is a Hero
Excalibur is a Sword
Oathbreaker is a Sword
The Volcano is an Evil Lair
The Secret Cave is an Evil Lair
The Underground Bunker is an Evil Lair
\end{lstlisting}

If these instances are not input into the compiler, then each entity has just
one instance with the same name as its type (a hero called ``Hero'', a sword
called ``Sword'', an evil lair called ``Evil Lair'', for example).

\subsection{Sequences of Events}
\label{sec:seq-code}
TropICAL is an event-based language: it describes events that happen as part of
a story, where the events involve actions taken by roles (the agents in a
multi-agent system) interacting with other roles as well as objects and places.
The simplest possible statement in the language, given the declarations above,
would be to write:

\begin{lstlisting}[label={lst:seq1}, caption={An event}]
The Hero goes to the Land of Adventure
\end{lstlisting}

This specifies that the first event that occurs in the trope is that the \emph{Hero}
character goes to a place called \emph{The Land of Adventure}. Due to the fact
that this statement is compiled to a norm, it states only what is possible in
the story, giving \emph{permission} for events to happen, not mandating that
they occur. We do this to fulfil requirement~\ref{req:norm}, so that we guide
the actions of character agents to fit our story, rather than regimenting their behaviour. It is
possible to tell the agents what to do in a stronger fashion using
\emph{obligations}, which we describe in section~\ref{sec:obl-code}.

Typically, tropes tend to consist
of a sequence of several events, as is stated in requirement~\ref{req:sequences}. So we can
extend this trope further to meet this requirement:

\begin{lstlisting}[label={lst:seq2}, caption={Two consecutive events}]
The Hero goes to the Land of Adventure
Then the Hero finds the Sword
\end{lstlisting}

This series of events can be visualised graphically (using a prefix notation for
the verbs and objects) thus:

\begin{figure}[h]
\caption{Two consecutive events}
\label{fig:seq1}
\vspace{7mm}
\centerline{\includegraphics[width=0.7\textwidth]{seq1.png}}
\vspace{7mm}
\end{figure}

The \emph{Then} keyword denotes that the event is the next stage in a sequence
of events. However, this keyword is syntactical sugar, and can be omitted with the same effect:

\begin{lstlisting}[label={lst:seq3}, caption={``The'' instead of ``Then''}]
The Hero goes to the Land of Adventure
The Hero finds the Sword
\end{lstlisting}

Events can be strung together in a sequence that is as long as the author needs:

\begin{lstlisting}[label={lst:seq4}, caption={A sequence of events}]
The Hero goes to the Land of Adventure
Then the Hero finds the Sword
Then the Hero meets the Villain
Then the Hero kills the Villain
Then the Hero returns Home
\end{lstlisting}

\vspace{7mm}
\centerline{\includegraphics[width=\textwidth]{seq2.png}}
\vspace{7mm}

The author can use any verb they wish to describe an event. The verbs are
stemmed using \emph{WordNet}~\citep{miller1995wordnet} so that, for example,
\emph{goes} becomes \emph{go} and \emph{wanted} becomes \emph{want} when the
tropes are compiled.

\subsection{Obligations}
\label{sec:obl-code}
All of the code examples until this point express events that \emph{may} occur
in the story. When compiled to InstAL, and expressed in terms of norms, they
describe the \emph{permitted} behaviour of the agents in a story. This way, the
character agents are aware of the actions available to them that follow the path
of the author's narrative. However, there are other actions that an author may
want to strongly encourage an agent to take, so that they can direct their
actions with more control (as per requirement~\ref{req:direct}). The syntax for this type of action
is expressed with the \emph{must} keyword, and is shown in
listing~\ref{lst:obl1}. In this case, the first action (\emph{The Hero goes
  Home}) is expressed as a permission, but the second action (\emph{Then the
  Hero must go to the Land of Adventure}) is an obligation.

\begin{lstlisting}[label={lst:obl1}, caption={An obliged event within a trope}]
The Hero goes Home
Then the Hero must go to the Land of Adventure
\end{lstlisting}

\subsubsection{Deadline Events and Consequences}
Obligations may have optional deadline events and consequences. The consequence
event is triggered if the obligation has not been fulfilled before the deadline
event has occurred. Listing~\ref{lst:obl2} shows an example trope where the Hero
is obliged to go to the Land of Adventure before the Villain character kills the
Mentor character. If the Villain kills the Mentor and the Hero has not yet gone
to the Land of Adventure, then a possibility opens up for the Villain to kill
the Hero in the story.

\begin{lstlisting}[label={lst:obl2}, caption={An obliged event with a deadline
and a consequence}]
The Hero must go to the Land of Adventure before the Villain kills the Mentor
  Otherwise, the Villain kills the Hero
\end{lstlisting}

This mechanism could be used as a method of creating branches in the story. If the
specified event happens before the deadline event, the path of the trope goes
one way. If it doesn't, the story follows the path of the consequence event.
Section~\ref{sec:instal-obl} explains in detail how obligations, deadlines and
consequences are compiled to InstAL code.

\subsection{Branching Events}
\label{sec:branch-code}
As requirement~\ref{req:branches} states, a trope author may want to describe alternative possibilities in their story,
where either the player or a character in the story may decide from one of
multiple actions to take. This is implemented using the \emph{Or} keyword in
TropICAL, along with a single indentation level of two spaces:

\begin{lstlisting}[label={lst:branch1}, caption={Two branches}]
The Hero goes to the Land of Adventure
  Or Hero finds the Sword
\end{lstlisting}

\vspace{7mm}
\centerline{\includegraphics[width=0.5\textwidth]{branch1.png}}
\vspace{7mm}

The above code describes two alternative events: either the Hero can go to the
Land of Adventure, or the Hero can find the Sword.
Multiple events can be chained together on the same level of indentation to create
multiple possible alternatives:

\begin{lstlisting}[label={lst:branch2}, caption={Five branches}]
The Hero goes to the Land of Adventure
  Or the Hero finds the Sword
  Or the Hero meets the Villain
  Or the Hero kills the Villain
  Or the Hero returns Home
\end{lstlisting}

\vspace{7mm}
\centerline{\includegraphics[width=0.5\textwidth]{branch2.png}}
\vspace{7mm}

In the above example, the trope begins with five different possible alternative
events, representing five paths through the story.
These branching events can be combined with the event sequences described
previously to create more complex tropes:

\begin{lstlisting}[label={lst:branch3}, caption={A combination of branches and sequences}]
The Hero goes Home
Then the Hero finds a Sword
  Or the Hero goes to the Land of Adventure
  Or the Hero kills the Villain
Then the Hero meets the Mentor
  Or the Hero goes to the Realm of Mystery
\end{lstlisting}

\vspace{7mm}
\centerline{\includegraphics[width=\textwidth]{branch3.png}}
\vspace{7mm}\mnote{How can we improve this graph layout?}

In the previous example, only one event occurs in each branching story path
before merging back into the main course of the story. In order to extend each
story path further, one must indent an extra level:

\begin{lstlisting}[label={lst:branch4}, caption={Extending branches}]
The Hero goes Home
Then the Hero finds a Sword
  Or the Hero goes to the Land of Adventure
    Then the Hero finds a Treasure
    Then the Hero drowns
  Or the Hero kills the Villain
    Then the Hero runs Home
      Or the Hero cries
Then the Hero meets the Mentor
  Or the Hero goes to the Realm of Mystery
    Then the Hero meets the Wizard
    Then the Wizard casts a Spell
\end{lstlisting}

\vspace{7mm}
\centerline{\includegraphics[width=\textwidth]{branch4.png}}
\vspace{7mm}

It should be noted that all branches will merge back into the ``main'' story
path (the bottom level of indentation) once they have completed. To prevent this
from happening, and terminate the story, the author may write \emph{``Then the
  Story ends''}. In this example, the story will always end unless the Hero goes
to the Way Out or escapes the Dark Room:

\begin{lstlisting}[label={lst:branch5}, caption={Terminating branches}]
The Hero goes to the House
Then the Hero meets the Villain
Then the Hero kills the Villain
  Or the Villain kills the Hero
    Then the Story ends
Then the Hero goes to the Way Out
  Or the Hero goes to the Dark Pit
    Then the Crocodile eats the Hero
    Then the Story ends
  Or the Hero goes to the Dark Room
    Then the Hero escapes
      Or the Grue eats the Hero
        Then the Story ends
Then the Hero finds the Treasure
\end{lstlisting}

\vspace{7mm}
\centerline{\includegraphics[width=\textwidth]{branch5.png}}
\vspace{7mm}

Using this technique, branches can be extended indefinitely through increasing
levels of indentation.
Deeply indented code is often undesirable, however, and is usually a symptom
that the code needs to be subdivided into modules. We can achieve this by
embedding subtropes inside of other tropes.

\subsection{Subtropes}
\label{sec:subtrope-code}
Requirement~\ref{req:subtropes} states that we need a mechanism for the
nesting of tropes, so that we may create new abstractions by combining old
tropes into new ones. We do this through \emph{Subtropes}, which are simply previously-written tropes that are embedded inside a
new trope. Take this simple trope example:

\begin{lstlisting}[showstringspaces=false, label={lst:subtrope1}, caption={Subtrope
te be embedded}]
"Item Search" is a trope where:
   The Macguffin is an object
   The Hero is a role
   Home is a place

   The Hero chases the Macguffin
   Then the Hero finds the Macguffin
     Or the Hero goes Home
\end{lstlisting}

\vspace{7mm}
\centerline{\includegraphics[width=0.7\textwidth]{subtrope1.png}}
\vspace{7mm}

To embed this trope inside another trope, we refer to it by name by writing
\emph{Then the ``Item Search'' trope happens}. It is important to put the name
of the trope inside quotation marks. An example of putting this trope at the end
of another trope is shown in the following example:

\begin{lstlisting}[showstringspaces=false, label={lst:subtrope2}, caption={Trope containing a subtrope}]
"Kill then Search" is a trope where:
  Away is a place
  The Hero is a role
  The Villain is a role

  The Hero goes Away
  Then the Hero kills the Villain
  Then the "Item Search" trope happens
\end{lstlisting}

\vspace{7mm}
\centerline{\includegraphics[width=\textwidth]{subtrope2.png}}
\vspace{7mm}

The ``Item Search'' subtrope is thus embedded inside the ``Kill then Search''
trope, with all of the events and branches of ``Item Search'' being appended to
the end of ``Kill then Search''. Rather than only being embedded at the end of a
trope however, subtropes can appear at any point in a trope, or even at several
points in a trope. Take this example that uses the ``Item Search'' trope inside
nested branches of its story:

\begin{lstlisting}[showstringspaces=false, label={lst:subtrope3}, caption={Subtrope in multiple places}]
"Futile Search" is a trope where:
  The Hero is a role
  The Villain is a role
  The Mentor is a role
  Away is a place
  Home is a place
  The Land of Adventure is a place
  The Realm of Mystery is a place

  The Hero goes Away
  Then the Hero meets the Villain
    Or the Hero meets the Mentor
      Then the "Item Search" trope happens
  Then the Hero kills the Villain
  Then the "Item Search" trope happens
\end{lstlisting}

\vspace{7mm}
\centerline{\includegraphics[width=\textwidth]{subtrope3.png}}
\vspace{7mm}\mnote{This is a good demonstration of why a graph won't work as a
  visualisation: we don't want \emph{kill_hero_villain} to happen after the
  second time the \emph{Item Search} trope happens, just the first time. Will
  redo as a tree visualisation instead.}

An author can create complex branching narrative by combining these tropes together. To see
visualisations of the story paths of multiple tropes combined, see
section~\ref{sec:story-vis} of chapter~\ref{cha:storybuilder}.

For technical details of the parser implementation, including the EBNF grammar
of the TropICAL language, refer to Appendix~\ref{appendix:t-grammar}.
Appendix~\ref{appendix:parse-tree} shows visualisations of the parse tree output for some
of the code examples described in this section.

\section{InstAL Code Generation}
\label{sec:t-codegen}
% TODO intro: 1
% TODO tech discussion: 1
Once the tropes written in TropICAL have been parsed and converted into an
intermediate hash map, this data structure is then used to generate InstAL code.
This section describes how specific language features translate to
InstAL, with fully translated examples of InstAL code appearing in Appendix~\ref{appendix:instal}.

As described in section~\ref{sec:inst-events}, InstAL features three types of
events: \emph{external}, \emph{institutional} and \emph{violation}. We begin by
examining the translation of trope events to institutional events.

\subsection{Initial Conditions}
% TODO before / after snippets: 2
% TODO explanation: 2
% TODO explain how the domain works in InstAL
The \emph{initially} clause in InstAL specifies the fluents that initially hold
when the institution is effected. Role, object and place declarations are
translated into fluent declarations as part of this clause. For example, for a
trope with the following declarations:

\begin{lstlisting}
The Hero is a role
The Sword is an object
The Land of Adventure is a place
\end{lstlisting}

And the following entity instances:

\begin{lstlisting}
Luke Skywalker is a Hero
The Lightsaber is a Sword
The Death Star is a Land of Adventure
\end{lstlisting}

The following InstAL code is produced, with type and fluent declarations
automatically appearing at the top of the file:

\begin{lstlisting}
type Agent;
type Role;
type Place;
type PlaceName;
type Object;
type ObjectName;

fluent role(Agent, Role);
fluent place(PlaceName, Place);
fluent object(ObjectName, Object);

initially:
  role(lukeSkywalker, hero),
  object(lightSaber, sword),
  place(deathStar, landOfAdventure);
\end{lstlisting}

The \emph{initially} clause also specifies the permitted and obliged events that
occur at the start of the trope. Say our trope contains a series of events (as in
listing~\ref{lst:seq4}):

\begin{lstlisting}
The Hero goes to the Land of Adventure
Then the Hero finds the Sword
Then the Hero meets the Villain
Then the Hero kills the Villain
Then the Hero returns Home
\end{lstlisting}

In this case, the first event of the trope (\emph{The Hero goes to the Land of
  Adventure}) must be permitted to happen at the very beginning of the
institution, inside the \emph{initially} clause. In combination with the above
entity declarations, this results in the following code being generated
(omitting the type and fluent declarations this time):

\begin{lstlisting}
initially:
  perm(go(X, Y)) if role(X, hero), place(Y, landOfAdventure),
  role(lukeSkywalker, hero),
  object(lightSaber, sword),
  place(deathStar, landOfAdventure);
\end{lstlisting}

This means that the Luke Skywalker character is able to go to the Death Star at the beginning
of the institution, as is specified in the trope.

\subsection{Generation}
% TODO before / after snippets: 2
% TODO explanation: 2

Tropes usually consist of multiple events, so it is not enough to have one event
permitted at the start of a trope. Once this first event has occurred, the next
event in the sequence must have permission to happen.

The institution watches for events to happen as \emph{external} events in the
environment, and then has these events generate \emph{institutional} events that
occur within the institution and create permissions or obligations for further
external events to occur. Returning to listing~\ref{lst:seq4} above, the
external events that happen are:

\begin{itemize}
  \item The Hero finds the Sword
  \item The Hero meets the Villain
  \item The Hero kills the Villain
  \item The Hero returns Home
\end{itemize}

These are translated into \emph{generates} statements in InstAL so that they
generate the required institutional events:

\begin{lstlisting}
find(R, T) generates
    intSequence3(R, S, T, U, V) if
        role(R, hero),
        object(T, sword);
kill(R, S) generates
    intSequence3(R, S, T, U, V) if
        role(S, villain),
        role(R, hero);
meet(R, S) generates
    intSequence3(R, S, T, U, V) if
        role(S, villain),
        role(R, hero);
go(R, V) generates
    intSequence3(R, S, T, U, V) if
        role(R, hero),
        place(V, landOfAdventure);
return(R, U) generates
    intSequence3(R, S, T, U, V) if
        role(R, hero),
        place(U, home);
\end{lstlisting}

Each institution only has one institutional event that is triggered by the
external events, which is named after the trope itself (in this case, the event
is named \emph{intSequence3} after the trope name ``Sequence 3''). The
permissions and obligations that this institutional event initiates depend on
the \emph{phase} of the trope that is currently active (see
section~\ref{sec:event-phases}) below.

\subsection{Initiation}
% TODO before / after snippets: 2
% TODO explanation: 2

External events generate an internal event for the institution, which permits
the next event or events in a sequence to occur. To ensure that the events of
the trope are permitted or obliged to occur one after another, we use a
mechanism called \emph{event phases}.

\subsubsection{Event Phases}
\label{sec:event-phases}
An \emph{event phase} is a fluent that holds the state of the current
institution. At the beginning of the trope, its state is simply \emph{active}.
This is represented in the institution through the fluent \emph{phase(tropeName,
active)}.
Once the first event has occurred, the trope enters \emph{phase A}, which means
that the fluent \emph{phase(tropeName, phaseA)} then holds, and the
\emph{phase(tropeName, active} fluent is terminated (this is described in
section~\ref{sec:termination}). This initiation and termination process then
repeats through all phases of the trope, through phases B, C and D if they
exist, until the final event occurs and the last phase is terminated.

In the case of our example sequence of events from listing~\ref{lst:seq4}, there
are five phases in total:

\begin{itemize}
  \item The Hero goes to the Land of Adventure (active)
  \item The Hero finds the Sword (phase A)
  \item The Hero meets the Villain (phase B)
  \item The Hero kills the Villain (phase C)
  \item The Hero returns Home (phase D)
\end{itemize}

The events and phases of this trope are initiated in InstAL through the
following generated code:

\begin{lstlisting}[label={lst:initiates}, caption={Institutional event
initiation code for listing~\ref{lst:seq4}}]
intSequence3(R, S, T, U, V) initiates
    phase(sequence3, phaseA),
    perm(find(R, T)) if
        phase(sequence3, active),
        role(R, hero),
        object(T, sword);
intSequence3(R, S, T, U, V) initiates
    phase(sequence3, phaseB),
    perm(meet(R, S)) if
        phase(sequence3, phaseA),
        role(S, villain),
        role(R, hero);
intSequence3(R, S, T, U, V) initiates
    phase(sequence3, phaseC),
    perm(kill(R, S)) if
        phase(sequence3, phaseB),
        role(S, villain),
        role(R, hero);
intSequence3(R, S, T, U, V) initiates
    phase(sequence3, phaseD),
    perm(return(R, U)) if
        phase(sequence3, phaseC),
        role(R, hero),
        place(U, home);
\end{lstlisting}


\subsection{Termination}
\label{sec:termination}
% TODO before / after snippets: 2
% TODO explanation: 2

The generated code that terminates the phases and permissions corresponding to
those initiated in listing~\ref{lst:initiates} is shown in
listing~\ref{lst:terminates} below. Once an external event occurs, its
permission to occur again is terminated along with the phase in which it occurred:

\begin{lstlisting}[label={lst:terminates}, caption={Institutional event
termination code for listing~\ref{lst:seq4}}]
intSequence3(R, S, T, U, V) terminates
    phase(sequence3, active),
    perm(go(R, V)) if
        phase(sequence3, active),
        role(R, hero),
        place(V, landOfAdventure);
intSequence3(R, S, T, U, V) terminates
    phase(sequence3, phaseA),
    perm(find(R, T)) if
        phase(sequence3, phaseA),
        role(R, hero),
        object(T, sword);
intSequence3(R, S, T, U, V) terminates
    phase(sequence3, phaseB),
    perm(meet(R, S)) if
        phase(sequence3, phaseB),
        role(S, villain),
        role(R, hero);
intSequence3(R, S, T, U, V) terminates
    phase(sequence3, phaseC),
    perm(kill(R, S)) if
        phase(sequence3, phaseC),
        role(S, villain),
        role(R, hero);
intSequence3(R, S, T, U, V) terminates
    phase(sequence3, phaseD),
    perm(return(R, U)) if
        phase(sequence3, phaseD),
        role(R, hero),
        place(U, home);
\end{lstlisting}

\subsection{Branches}
% TODO before / after snippets: 2
% TODO explanation: 2

To represent branching events in InstAL, multiple events are permitted to occur
during one phase. Take as an example the combination of event sequences and
branches shown in listing~\ref{lst:branch3}:

\begin{lstlisting}
The Hero goes Home
Then the Hero finds a Sword
  Or the Hero goes to the Land of Adventure
  Or the Hero kills the Villain
Then the Hero meets the Mentor
  Or the Hero goes to the Realm of Mystery
\end{lstlisting}

When compiled with the following entity instances:

\begin{lstlisting}
Harry Potter is a Hero
Voldemort is a Villain
Dumbledore is a Mentor
The Gryffindor Sword is a Sword
Hogwarts is a Land of Adventure
The Chamber of Secrets is a Realm of Mystery
\end{lstlisting}

The full institution appears in Section~\ref{appendix:branch3} of Appendix~\ref{appendix:instal}.

The branching events are implemented in the \emph{initiates} and
\emph{terminates} clauses. In listing~\ref{lst:branch-init} below, we see that three possible events are given
permission to occur as part of Phase A: \emph{The Hero finds a Sword}, \emph{The
  Hero goes to the Land of Adventure}, and \emph{The Hero kills the Villain}. In
Phase B, two events corresponding to the possible branches are given permission
to occur: \emph{The Hero meets the Mentor} and \emph{The Hero goes to the Realm
  of Mystery}:

\begin{lstlisting}[label={lst:branch-init}, caption={Initiation events for the
branching trope in listing~\ref{lst:branch3}}]
intBranch3(R, S, T, U, V, W, X) initiates
    phase(branch3, phaseA),
    perm(find(R, U)),
    perm(go(R, X)),
    perm(kill(R, S)) if
        phase(branch3, active),
        object(U, sword),
        role(R, hero),
        place(X, landOfAdventure),
        role(S, villain);
intBranch3(R, S, T, U, V, W, X) initiates
    phase(branch3, phaseB),
    perm(meet(R, T)),
    perm(go(R, V)) if
        phase(branch3, phaseA),
        place(V, realmOfMystery),
        role(R, hero),
        role(T, mentor);
\end{lstlisting}

This means that during Phase A, there are three possible narrative branches,
which get closed off when any one of them occurs through the corresponding
termination event in the institution.

\subsection{Obligations}
\label{sec:instal-obl}
% TODO before / after snippets: 2
% TODO explanation: 2

\subsubsection{Obligations Without Deadlines or Consequences}
In our tropes, obligations have optional deadline events and consequences. At
the time of implementation, both of these were mandatory in InstAL, and so it
was necessary to add ``dummy'' events for both if they were not specified in TropICAL.
Returning to listing~\ref{lst:obl1} of section~\label{sec:obl-code}, which
specifies an obligation without a deadline or a consequence:

\begin{lstlisting}
The Hero goes Home
Then the Hero must go to the Land of Adventure
\end{lstlisting}

Line~\ref{line:obl1-oblev} of listing~\ref{lst:obl1-init} shows the
initiation rule for an obligation event with no deadline or consequence, and
listing~\ref{lst:obl1-fluent} shows the fluent declaration for the obligation
fluent. The \emph{intNoDeadline} and \emph{noViolation} events are dummy events
that never occur either inside or outside of the institution. There are there
only because InstAL always requires deadline and consequence events in
obligation fluents.
The first parameter of the obligation fluent specifies an institutional event
that must occur before the deadline. In this case, the event is
\emph{intGo(hero, landOfAdventure)}. Listing~\ref{lst:obl1-gen} shows how the
\emph{go(hero, landOfAdventure)} external event generates the \emph{intGo(hero,
  landOfAdventure} institutional event. Due to the fact that it is this external
event that triggers the obliged institutional event, it is given permission
to occur on line~\ref{line:obl1-oblev} of listing~\ref{lst:obl1-init}, along with
the obliged institutional event itself.

\begin{lstlisting}[label={lst:obl1-fluent}, caption={Fluent declaration for the
obligation event in listing~\ref{lst:obl1}}]
obligation fluent obl(intGo(Agent, PlaceName), intNoDeadline, noViolation);
\end{lstlisting}

\begin{lstlisting}[label={lst:obl1-init}, caption={Initiation rule for the
second (obligation) event of the trope in listing~\ref{lst:obl1}}, escapechar=|]
intObligation1(R, S, T) initiates
    phase(obligation1, phaseA),
    obl(intGo(R,T), intNoDeadline, noViolation), perm(go(R, T)), perm(intGo(R,T)), pow(intGo(R,T)) if|\label{line:obl1-oblev}|
        phase(obligation1, active),
        role(R, hero),
        place(T, landOfAdventure);
\end{lstlisting}

\begin{lstlisting}[label={lst:obl1-gen}, caption={Generation rule for the
obligation event}]
go(R, S) generates
    intGo(R,S) if
        role(R, hero),
        place(S, landOfAdventure);
\end{lstlisting}

The full InstAL institution that this generates appears in listing~\ref{lst:obl1-instal} of appendix~\ref{appendix:instal}.

\subsubsection{Obligations with Deadlines and Consequences}
The trope shown in listing~\ref{lst:obl2-trope} describes a situation where an
obligation with both a deadline and a consequence occurs.

\begin{lstlisting}[showstringspaces=false,label={lst:obl2-trope}, caption={Example of a trope
containing an obligation with both a deadline and a consequence}]
The Hero must go to the Land of Adventure before the Villain kills the Mentor
  Otherwise, the Villain may kill the Hero
\end{lstlisting}

Listing~\ref{lst:obl2-fluent} shows the fluent declaration of the obligation
event in listing~\ref{lst:obl2-trope}. The deadline event \emph{intKill(villain,
  mentor)} is an institutional event, and so is generated by the
\emph{kill(villain, mentor)} external event on
lines~\ref{line:obl2-deadline1} to~\ref{line:obl2-deadline2} of
listing~\ref{lst:obl2-gen}. Listing~\ref{lst:obl2-init} shows the obligation
fluent itself. As this obligation is the first event to occur in this trope,
it appears in the \emph{initially} clause of the InstAL code, rather than as an
initiation event.

If the violation event contained within this obligation were to occur, it would
enable the story to begin another course of events. In this case, it initiates
the permission of the villain to kill the hero. This is shown in the code of listing~\ref{lst:obl2-trope}.

\begin{lstlisting}[label={lst:obl2-fluent}, caption={Fluent declaration for an
obligation fluent with both deadline and consequence events}]
obligation fluent obl(intGo(Agent, PlaceName), intKill(Agent, Agent), violHeroGoLandOfAdventure);
\end{lstlisting}

\begin{lstlisting}[label={lst:obl2-init}, caption={An obligation fluent with
both a deadline and a consequence, translated from the trope in listing~\ref{lst:obl2-trope}}]
obl(intGo(R,U), intKill(S,T), violHeroGoLandOfAdventure), perm(go(R, U)), perm(intGo(R,U)), pow(intGo(R,U)) if role(R, hero), place(U, landOfAdventure);
\end{lstlisting}

\begin{lstlisting}[label={lst:obl2-viol}, caption={Permissions generated by the
violation event of the trope in listing~\ref{lst:obl2-trope}}]
violHeroGoLandOfAdventure initiates
    perm(kill(R, S)) if
        role(R, villain),
        role(S, hero);
\end{lstlisting}

\begin{lstlisting}[label={lst:obl2-gen}, caption={Generation events for the
trope in listing~\ref{lst:obl2-trope}}, escapechar=|]
go(R, U) generates
    intGo(R,U) if
        role(R, hero),
        place(U, landOfAdventure);
kill(S, T) generates|\label{line:obl2-deadline1}|
    intKill(S,T) if
        role(S, villain),
        role(T, mentor);|\label{line:obl2-deadline2}|
\end{lstlisting}

The full InstAL code listing for this institution appears in
listing~\ref{lst:obl2-instal} of appendix~\ref{appendix:instal}.

Obligations work best when both a deadline and consequence are specified.
Without any consequence for its violation, an agent has no real obligation to
carry out an action, and it serves the same role as a permission. For this
reason, it would be worth considering either making both deadline and
consequence events mandatory, or implementing some kind of default deadline and
consequence. A default deadline could be the next event in a sequence, for
example, and a default consequence could be the loss of health of a character,
or emotional deterioration in cases where an emotional model is used.

\subsection{Bridge Institution}
% TODO before / after snippets: 2
% TODO explanation of bridge institutions: 2
% TODO explanation: 2

In section~\ref{sec:subtrope-code}, we describe how \emph{subtropes} are
implemented in InstAL through the embedding of existing tropes inside of new
ones. This is translated into InstAL code through the use of a \emph{Bridge
  Institution}, a separate institution which is used to link two other
institutions together. The concept of a bridge institution is developed
by~\citet{bath45254} as a means of conflict resolution in interacting
institutions, and involves the creation of an intermediary institution to
coordinate events between two other institutions. The two institutions linked are referred to as the
\emph{source} and \emph{sink} institutions, where an event that occurs in the
source institution has the power to generate an event inside of the sink institution.

The \emph{Item Search} and \emph{Kill then Search} tropes
described in listings~\ref{lst:subtrope1} and~\ref{lst:subtrope2} can be
described as the \emph{sink} and \emph{source} of a bridge institution,
respectively. In this case, the first event of the sink institution (the
\emph{Item Search} trope) is given permission to occur when the final event of
the source institution (the \emph{Kill then Search} trope) happens.

\begin{lstlisting}[caption={Bridge institution for the \emph{Item Search} and
\emph{Kill then Search} tropes}, label={lst:bridge-inst},escapechar=|]
bridge killThenSearchItemSearch;

source killThenSearch;
sink itemSearch;

cross fluent ipow(killThenSearch, perm(chase(Agent, ObjectName)), itemSearch);|\label{line:cross1}|
cross fluent ipow(killThenSearch, phase(Trope, Phase), itemSearch);|\label{line:phase1}|

intStartItemSearch xinitiates phase(itemSearch, active);|\label{line:phase2}|
intStartItemSearch xinitiates perm(chase(R, S)) if|\label{line:crossperm1}|
        role(R, hero),
        object(S, macguffin);|\label{line:crossperm2}|

initially ipow(killThenSearch,  perm(chase(R, S)), itemSearch),
    ipow(killThenSearch, phase(itemSearch, active), itemSearch);
\end{lstlisting}

Listing~\ref{lst:bridge-inst} shows the bridge institution that links the
\emph{Item Search} and \emph{Kill then Search} institutions together. It starts
by defining \emph{killThenSearch} as the source institution, and
\emph{itemSearch} as the sink. Then a cross fluent for the first event in the
subtrope (which is \emph{itemSearch}, the sink institution) is declared in
line~\ref{line:cross1}. This line only states that this is a fluent that is to
be initiated in the sink institution, from the source institution.

Lines~\ref{line:crossperm1} to~\ref{line:crossperm2} describe the actual
sequence of events that lead to the \emph{The Hero chases the MacGuffin} event
being triggered inside the sink institution: when the institutional event
\emph{intStartItemSearch} occurs inside the source institution, it gives the
Hero permission to chase the MacGuffin in the sink institution. The same
mechanism is used on lines~\ref{line:phase1} and~\ref{line:phase2} to declare
and initiate the first phase (the \emph{active} phase) of the sink (\emph{Item
  Search}) institution from the source (\emph{Kill then Search}) institution.
The phase is initiated from the same institutional event inside the source
institution: the \emph{intStartItemSearch} event.

\begin{lstlisting}[label={lst:bridge-source}, caption={The generation event for
the final action in the \emph{Kill then Search} source institution}]
kill(R, S) generates
    intStartItemSearch if
        role(S, villain),
        role(R, hero),
        phase(killNSearch, phaseA);
\end{lstlisting}

Listing~\ref{lst:bridge-source} shows the generation event corresponding to the
final action that occurs in the \emph{Kill then Search} trope: \emph{The Hero
  kills the Villain}. The line after this event in the trope embeds the
\emph{Item Search} subtrope within this trope: \emph{Then the ``Item Search''
  trope happens}. For this reason, the \emph{intStartItemSearch} institutional
event is initiated, which further initiates the permission for the first event
in the \emph{Item Search} institution to happen through the bridge institution
in listing~\ref{lst:bridge-inst}.

Full listings for the generated InstAL code for these tropes, along with several
other examples, appear in Appendix~\ref{appendix:instal}.

\section{Answer Set Generation}
\label{sec:t-asp}
% TODO instalquery intro & explanation: 2
% TODO generate for hero's journey: 4
% TODO generate for evil empire: 4
% TODO generate for both: 4

Once the code has been translated from TropICAL into InstAL, it can be compiled
further into AnsProlog, an Answer Set Programming (ASP) language, using a
process described by~\citet{cliffe2007specifying}. We use the
Potassco project's Clingo~\citep{gebser2011potassco} solver to formally verify
that certain sequences of events conform to the tropes that form our
institutions. This solver can also be used to generate all of the possible
sequences of events that fit a story described using one or more tropes, by
generating all of the \emph{answer sets} that correspond to a set of institutions.

\section{Adding Constraints}
\label{sec:t-constraints}
% TODO ASP intro: 1
% TODO constraints explanation: 2
% TODO output: 2
When our answer sets (traces) are generated by the answer set solver, we want to
see sequences of events that can happen as part of a story with a set
of given tropes. However, the answer set solver may include some uninteresting
events as part of the grounding process, such as \emph{null} events, the dummy
events we used to replace unspecified deadlines and consequences for
obligations, and events that otherwise violate the rules in each institution.

Listing~\ref{lst:constraints} shows the AnsProlog rules used to constrain the
answer sets produced by the solver, so that they only consist of events that are
relevant to our stories. Line~\ref{line:viol} specifies that events are only
valid if they are not violation or null events. Line~\ref{line:inst} filters
out any event that has a \emph{null} institution. Line~\ref{line:valids} ignores
any answer sets where a valid event occurs after a sequence of non-valid events.
Lines~\ref{line:dead} and~\ref{line:dead2} ignore our dummy deadline events for
obligations where no deadline is specified. Finally, lines~\ref{line:null}
and~\ref{line:null2} remove any \emph{null} events at all from the answer sets.

\begin{lstlisting}[label={lst:constraints}, caption={Constraint rules to remove invalid and irrelevant events}, escapechar=|]
validEvent(I, In) :- instant(I), inst(In), event(E), occurred(E, In, I), not occurred(viol(_), In, I), E != null.|\label{line:viol}|
validEvent(I) :- validEvent(I, In), inst(In).|\label{line:inst}|
:- validEvent(I2), not validEvent(I), I < I2, instant(I), instant(I2).|\label{line:valids}|
:- instant(I), not validEvent(I), occurred(viol(_), In, I).|\label{line:instant}|
deadEvent :- observed(noDeadline(X), I, T).|\label{line:dead}|
:- deadEvent.|\label{line:dead2}|
nullInst :- occurred(X, null, T).|\label{line:null}|
:- nullInst.|\label{line:null2}|
\end{lstlisting}

The next section lists example answer set (trace) outputs of the solver with two
tropes, both separately and combined.

\section{Example Answer Sets (Traces)}
Answer Sets (also called ``traces'') are produced by the answer set solver when
given the institutions compiled as ASP code, along with a set of constraints.
The traces generated represent all of the \emph{possible} sequences of events
that may occur, given the described institutions. It is worth noting that by
default this also includes any event that violate any of the institutions, as
these violations are not prevented by the institutions. Instead, \emph{violation
events} are generated which carry consequences with them. By using a constraint
such as the one listed in~\ref{lst:constraints}, we can tell the answer set
solver to only generate sequences events that do not contain these violation
events. This is what we have done in this section. For visualisations of these
answer sets, refer to the diagrams in section~\ref{sec:story-vis}.

\subsection{Evil Empire}
Our first example trope is a simple sequence of events called the ``Evil Empire'':

\begin{lstlisting}[label={lst:evil-empire}, caption={The ``Evil Empire'' trope}]
``Evil Empire'' is a trope where:
  The Empire is a role
  The Hero is a role

  The Empire chases the Hero
  Then the Empire captures the Hero
  Then the Hero escapes
\end{lstlisting}

This trope, once compiled to InstAL, then AnsProlog, and run through Clingo,
produces four answer sets. The compiled institution is listed in
Appendix~\ref{appendix:evil-empire}, and all of the resulting traces appear in
Appendix~\ref{appendix:evil-empire-traces}. The listed traces are solved by
looking up to three events into the trope, as that is its maximum length.

Listing~\ref{lst:evil-trace} shows one of the four traces from the solver
output. This trace contains each event of the trope, happening one after
another. The other traces stop short of the final event, so that trace one has
zero events, trace two has only the first event, and trace three has the first
two events of the trope.

\begin{lstlisting}[label={lst:evil-trace}, caption={Example trace for the ``Evil
Empire'' trope}]
Answer Set 4:

Time Step 1:

holdsat(role(hero,hero),evilEmpire)
holdsat(role(empire,empire),evilEmpire)
holdsat(phase(evilEmpire,phaseA),evilEmpire)
holdsat(pow(intEvilEmpire(hero,empire)),evilEmpire)
holdsat(perm(null),evilEmpire)
holdsat(perm(intEvilEmpire(hero,empire)),evilEmpire)
holdsat(perm(capture(hero,empire)),evilEmpire)
holdsat(live(evilEmpire),evilEmpire)
occurred(intEvilEmpire(hero,empire),evilEmpire)
occurred(chase(hero,empire),evilEmpire)


Time Step 2:

holdsat(role(hero,hero),evilEmpire)
holdsat(role(empire,empire),evilEmpire)
holdsat(phase(evilEmpire,phaseB),evilEmpire)
holdsat(pow(intEvilEmpire(hero,empire)),evilEmpire)
holdsat(perm(null),evilEmpire)
holdsat(perm(intEvilEmpire(hero,empire)),evilEmpire)
holdsat(perm(escape(hero)),evilEmpire)
holdsat(live(evilEmpire),evilEmpire)
occurred(intEvilEmpire(hero,empire),evilEmpire)
occurred(capture(hero,empire),evilEmpire)


Time Step 3:

holdsat(role(hero,hero),evilEmpire)
holdsat(role(empire,empire),evilEmpire)
holdsat(pow(intEvilEmpire(hero,empire)),evilEmpire)
holdsat(perm(null),evilEmpire)
holdsat(perm(intEvilEmpire(hero,empire)),evilEmpire)
holdsat(live(evilEmpire),evilEmpire)
occurred(intEvilEmpire(hero,empire),evilEmpire)
occurred(escape(hero),evilEmpire)
\end{lstlisting}

The events that appear at the start of the listing (starting with the
\emph{holdsat} predicate) describe the \emph{fluents} that hold for a certain
institution at that particular time step in the answer set. After these
statements, the events that have \emph{occurred} in that institution appear.
These events can be both institutional or external events, which can be
distinguished due to the institutional event names beginning with ``int'' (such
as \emph{intEvilEmpire}, for example). External events usually trigger
institutional events with the institution name, which is why the \emph{chase},
\emph{capture} and \emph{escape} events all trigger the \emph{intEvilEmpire}
institutional event in listing~\ref{lst:evil-trace}.

\subsection{The Hero's Journey}

\begin{lstlisting}[showstringspaces=false]
``The Hero's Journey'' is a trope where:
  The Hero is a role
  The Villain is a role
  Home is a place
  The Evil Lair is a place

  The Hero goes to the Evil Lair
  Then the Hero kills the Villain
    Or the Villain escapes
  Then the Hero goes Home
\end{lstlisting}

The answer set solver produces six traces for this trope. The branching
alternatives (the Hero can either kill the Villain or the Villain can escape at
one point in the trope) add an extra three answer sets to the output. If this
trope were just three events long, without the branching alternatives, then only
three answer sets would be produced (corresponding with stories of one, two
and three events long). The fourth answer set produced is shown in
listing~\ref{lst:hero-trace} as an example.

\begin{lstlisting}[label={lst:hero-trace},caption={Example trace for the
``Hero's Journey'' trope}]
Answer Set 4:

Time Step 1:

holdsat(role(villain,villain),herosJourney)
holdsat(role(hero,hero),herosJourney)
holdsat(place(home,home),herosJourney)
holdsat(place(evilLair,evilLair),herosJourney)
holdsat(phase(herosJourney,phaseA),herosJourney)
holdsat(pow(intHerosJourney(hero,villain,evilLair,home)),herosJourney)
holdsat(perm(null),herosJourney)
holdsat(perm(kill(hero,villain)),herosJourney)
holdsat(perm(intHerosJourney(hero,villain,evilLair,home)),herosJourney)
holdsat(perm(escape(villain)),herosJourney)
holdsat(live(herosJourney),herosJourney)
occurred(intHerosJourney(hero,villain,evilLair,home),herosJourney)
occurred(go(hero,evilLair),herosJourney)


Time Step 2:

holdsat(role(villain,villain),herosJourney)
holdsat(role(hero,hero),herosJourney)
holdsat(place(home,home),herosJourney)
holdsat(place(evilLair,evilLair),herosJourney)
holdsat(phase(herosJourney,phaseB),herosJourney)
holdsat(pow(intHerosJourney(hero,villain,evilLair,home)),herosJourney)
holdsat(perm(null),herosJourney)
holdsat(perm(intHerosJourney(hero,villain,evilLair,home)),herosJourney)
holdsat(perm(go(hero,home)),herosJourney)
holdsat(live(herosJourney),herosJourney)
occurred(kill(hero,villain),herosJourney)
occurred(intHerosJourney(hero,villain,evilLair,home),herosJourney)


Time Step 3:

holdsat(role(villain,villain),herosJourney)
holdsat(role(hero,hero),herosJourney)
holdsat(place(home,home),herosJourney)
holdsat(place(evilLair,evilLair),herosJourney)
holdsat(pow(intHerosJourney(hero,villain,evilLair,home)),herosJourney)
holdsat(perm(null),herosJourney)
holdsat(perm(intHerosJourney(hero,villain,evilLair,home)),herosJourney)
holdsat(live(herosJourney),herosJourney)
occurred(intHerosJourney(hero,villain,evilLair,home),herosJourney)
occurred(go(hero,home),herosJourney)
\end{lstlisting}

So far, the \emph{Evil Empire} and \emph{Hero's Journey} tropes produce
unremarkable results. Due to their simple natures, they only produce a small
number of answer sets when run through a solver. The results are more
interesting when both tropes are input into the solver, however.

\subsection{The Hero's Journey \emph{and} The Evil Empire}

Using both institutions as inputs into the solver, combining the two tropes together, produces 124 answer sets. Again, the reason
that the solver produces so many answer sets is that it includes stories that
are less than the full length of every event from both tropes combined (which is
six events in this case). This is desirable when we are combining tropes, as
sometimes we need to incorporate events from a trope without needing it to run
its full course. In this example, if the Hero's Journey has come to an end, then
that would seem like a more suitable place to finish a story, rather than
waiting for all of the events in the Evil Empire trope to occur.

An example of a trace that runs the full six event length of both tropes put
together is shown in listing~\ref{lst:hero-evil-trace}.

\begin{lstlisting}[label={lst:hero-evil-trace},caption={Example trace for both the ``Evil Empire'' and
``Hero's Journey'' tropes combined together}]
Answer Set 70:

Time Step 1:

holdsat(role(villain,villain),herosJourney)
holdsat(role(hero,hero),herosJourney)
holdsat(role(empire,empire),herosJourney)
holdsat(place(home,home),herosJourney)
holdsat(place(evilLair,evilLair),herosJourney)
holdsat(phase(herosJourney,phaseA),herosJourney)
holdsat(pow(intHerosJourney(hero,villain,evilLair,home)),herosJourney)
holdsat(perm(null),herosJourney)
holdsat(perm(kill(hero,villain)),herosJourney)
holdsat(perm(intHerosJourney(hero,villain,evilLair,home)),herosJourney)
holdsat(perm(escape(villain)),herosJourney)
holdsat(live(herosJourney),herosJourney)
holdsat(role(villain,villain),evilEmpire)
holdsat(role(hero,hero),evilEmpire)
holdsat(role(empire,empire),evilEmpire)
holdsat(place(home,home),evilEmpire)
holdsat(place(evilLair,evilLair),evilEmpire)
holdsat(phase(evilEmpire,active),evilEmpire)
holdsat(pow(intEvilEmpire(hero,empire)),evilEmpire)
holdsat(perm(null),evilEmpire)
holdsat(perm(intEvilEmpire(hero,empire)),evilEmpire)
holdsat(perm(chase(hero,empire)),evilEmpire)
holdsat(live(evilEmpire),evilEmpire)
occurred(intHerosJourney(hero,villain,evilLair,home),herosJourney)
occurred(go(hero,evilLair),herosJourney)
occurred(null,evilEmpire)


Time Step 2:

holdsat(role(villain,villain),herosJourney)
holdsat(role(hero,hero),herosJourney)
holdsat(role(empire,empire),herosJourney)
holdsat(place(home,home),herosJourney)
holdsat(place(evilLair,evilLair),herosJourney)
holdsat(phase(herosJourney,phaseA),herosJourney)
holdsat(pow(intHerosJourney(hero,villain,evilLair,home)),herosJourney)
holdsat(perm(null),herosJourney)
holdsat(perm(kill(hero,villain)),herosJourney)
holdsat(perm(intHerosJourney(hero,villain,evilLair,home)),herosJourney)
holdsat(perm(escape(villain)),herosJourney)
holdsat(live(herosJourney),herosJourney)
holdsat(role(villain,villain),evilEmpire)
holdsat(role(hero,hero),evilEmpire)
holdsat(role(empire,empire),evilEmpire)
holdsat(place(home,home),evilEmpire)
holdsat(place(evilLair,evilLair),evilEmpire)
holdsat(phase(evilEmpire,phaseA),evilEmpire)
holdsat(pow(intEvilEmpire(hero,empire)),evilEmpire)
holdsat(perm(null),evilEmpire)
holdsat(perm(intEvilEmpire(hero,empire)),evilEmpire)
holdsat(perm(capture(hero,empire)),evilEmpire)
holdsat(live(evilEmpire),evilEmpire)
occurred(null,herosJourney)
occurred(intEvilEmpire(hero,empire),evilEmpire)
occurred(chase(hero,empire),evilEmpire)


Time Step 3:

holdsat(role(villain,villain),herosJourney)
holdsat(role(hero,hero),herosJourney)
holdsat(role(empire,empire),herosJourney)
holdsat(place(home,home),herosJourney)
holdsat(place(evilLair,evilLair),herosJourney)
holdsat(phase(herosJourney,phaseB),herosJourney)
holdsat(pow(intHerosJourney(hero,villain,evilLair,home)),herosJourney)
holdsat(perm(null),herosJourney)
holdsat(perm(intHerosJourney(hero,villain,evilLair,home)),herosJourney)
holdsat(perm(go(hero,home)),herosJourney)
holdsat(live(herosJourney),herosJourney)
holdsat(role(villain,villain),evilEmpire)
holdsat(role(hero,hero),evilEmpire)
holdsat(role(empire,empire),evilEmpire)
holdsat(place(home,home),evilEmpire)
holdsat(place(evilLair,evilLair),evilEmpire)
holdsat(phase(evilEmpire,phaseA),evilEmpire)
holdsat(pow(intEvilEmpire(hero,empire)),evilEmpire)
holdsat(perm(null),evilEmpire)
holdsat(perm(intEvilEmpire(hero,empire)),evilEmpire)
holdsat(perm(capture(hero,empire)),evilEmpire)
holdsat(live(evilEmpire),evilEmpire)
occurred(kill(hero,villain),herosJourney)
occurred(intHerosJourney(hero,villain,evilLair,home),herosJourney)
occurred(null,evilEmpire)


Time Step 4:

holdsat(role(villain,villain),herosJourney)
holdsat(role(hero,hero),herosJourney)
holdsat(role(empire,empire),herosJourney)
holdsat(place(home,home),herosJourney)
holdsat(place(evilLair,evilLair),herosJourney)
holdsat(pow(intHerosJourney(hero,villain,evilLair,home)),herosJourney)
holdsat(perm(null),herosJourney)
holdsat(perm(intHerosJourney(hero,villain,evilLair,home)),herosJourney)
holdsat(live(herosJourney),herosJourney)
holdsat(role(villain,villain),evilEmpire)
holdsat(role(hero,hero),evilEmpire)
holdsat(role(empire,empire),evilEmpire)
holdsat(place(home,home),evilEmpire)
holdsat(place(evilLair,evilLair),evilEmpire)
holdsat(phase(evilEmpire,phaseA),evilEmpire)
holdsat(pow(intEvilEmpire(hero,empire)),evilEmpire)
holdsat(perm(null),evilEmpire)
holdsat(perm(intEvilEmpire(hero,empire)),evilEmpire)
holdsat(perm(capture(hero,empire)),evilEmpire)
holdsat(live(evilEmpire),evilEmpire)
occurred(intHerosJourney(hero,villain,evilLair,home),herosJourney)
occurred(go(hero,home),herosJourney)
occurred(null,evilEmpire)


Time Step 5:

holdsat(role(villain,villain),herosJourney)
holdsat(role(hero,hero),herosJourney)
holdsat(role(empire,empire),herosJourney)
holdsat(place(home,home),herosJourney)
holdsat(place(evilLair,evilLair),herosJourney)
holdsat(pow(intHerosJourney(hero,villain,evilLair,home)),herosJourney)
holdsat(perm(null),herosJourney)
holdsat(perm(intHerosJourney(hero,villain,evilLair,home)),herosJourney)
holdsat(live(herosJourney),herosJourney)
holdsat(role(villain,villain),evilEmpire)
holdsat(role(hero,hero),evilEmpire)
holdsat(role(empire,empire),evilEmpire)
holdsat(place(home,home),evilEmpire)
holdsat(place(evilLair,evilLair),evilEmpire)
holdsat(phase(evilEmpire,phaseB),evilEmpire)
holdsat(pow(intEvilEmpire(hero,empire)),evilEmpire)
holdsat(perm(null),evilEmpire)
holdsat(perm(intEvilEmpire(hero,empire)),evilEmpire)
holdsat(perm(escape(hero)),evilEmpire)
holdsat(live(evilEmpire),evilEmpire)
occurred(null,herosJourney)
occurred(intEvilEmpire(hero,empire),evilEmpire)
occurred(capture(hero,empire),evilEmpire)


Time Step 6:

holdsat(role(villain,villain),herosJourney)
holdsat(role(hero,hero),herosJourney)
holdsat(role(empire,empire),herosJourney)
holdsat(place(home,home),herosJourney)
holdsat(place(evilLair,evilLair),herosJourney)
holdsat(pow(intHerosJourney(hero,villain,evilLair,home)),herosJourney)
holdsat(perm(null),herosJourney)
holdsat(perm(intHerosJourney(hero,villain,evilLair,home)),herosJourney)
holdsat(live(herosJourney),herosJourney)
holdsat(role(villain,villain),evilEmpire)
holdsat(role(hero,hero),evilEmpire)
holdsat(role(empire,empire),evilEmpire)
holdsat(place(home,home),evilEmpire)
holdsat(place(evilLair,evilLair),evilEmpire)
holdsat(pow(intEvilEmpire(hero,empire)),evilEmpire)
holdsat(perm(null),evilEmpire)
holdsat(perm(intEvilEmpire(hero,empire)),evilEmpire)
holdsat(live(evilEmpire),evilEmpire)
occurred(escape(hero),herosJourney)
occurred(viol(escape(hero)),herosJourney)
occurred(intEvilEmpire(hero,empire),evilEmpire)
occurred(escape(hero),evilEmpire)
\end{lstlisting}\mnote{Move to appendix?}

These trace outputs do not assist the reader in visualising the possible
sequences of events that they represent. For this reason, we suggest that the
reader refer to the visualisations produced by the StoryBuilder tool in section~\ref{sec:story-vis}.

\section{Extending for Legal Policies}
\label{sec:t-legal}
Though TropICAL is designed to describe non-linear stories in systems with
intelligent agents such as
interactive computer games, with minor changes it can also be adjusted to
describe legal contracts. Where a story describes the constraints on character
behaviour that any character can in theory break away from, a legal contract
performs a similar role by describing the expected behaviour of its parties and
the consequences for breaking those expectations.

In the case of contract law, though each individual contract between plaintiff and
defendant would be different, common patterns exist between contracts. For
example, contracts typically have a fixed duration and penalty for premature
termination. There are usually clauses in a contract that describe the
conditions for termination in the case of one of the parties performing certain
unpermitted actions. These are what we refer to as a ``policy'' throughout this section: an abstract description of
a commonly-occurring legal pattern that is analogous to a trope in a story.

In order to divide legal contracts into reusable components, the concept of tropes can easily be applied to capture fragments of the law. Some
examples are:

\begin{itemize}
\item \textbf{The Warranty}: A \emph{seller} sells an item to a \emph{buyer}. If
  the item is defective, then the \emph{buyer} has the right to return it within
  a certain period of time.
\item \textbf{The Lease}: A \emph{lessor} leases an item or property to a
  \emph{lessee}. The lessee is obliged to pay rental fees on time, and to keep the item
  or property in good condition. The lessor is obliged to perform maintenance
  and necessary repairs on the item or property.
\item \textbf{The Deposit}: A sum of money is given from one party to another
  with the understanding that it is to be returned upon expiration of a contract. 
\end{itemize}

As with tropes that contain sub-tropes, policies can have sub-policies. For
example, a ``sales contract'' policy could contain a ``warranty'' policy as a
sub-policy. This can be thought of as adding a clause to a contract. Similarly,
a ``lease'' policy could be a sub-policy of a ``tenancy agreement'' policy,
which keeps the spirit of the lease policy but adds the requirement for the
lessee to not make excessive noise, for example. These policies describe the
actions that both parties are permitted and obliged to carry out, in a sequence of events.

As an example of describing a legal policy in the manner of a story trope,
consider the subleasing of an object or property from a lessor to a lessee. In
such an arrangement, the following dispute could occur:

\begin{itemize}
  \item The \emph{lessor} leases property to the \emph{lessor}.
  \item The \emph{lessee} then subleases the property to a \emph{third party}.
  \item The \emph{lessor} cancels the lease contract, stating that the
    subleasing is a violation of the contract.
\end{itemize}

These policies can be expressed in terms of social norms, and translated into
TropICAL code:

\begin{lstlisting}[label={list:policies},caption={Example policies in TropICAL}]
``Lease'' is a policy where:
    The Lessor is a role
    The Lessor is a role
    The Thing is an object
    The Lessor leases the Thing to the Lessee
    Then the ``Maintenance of Confidence'' policy applies

``Sublease'' is a policy where:
    The Lessor is a role
    The Third Party is a role
    The Thing is an object
    The Lessor may sublease the Thing to a Third Party 

``Sublease Permission'' is a policy where:
    The Lessee is a role
    The Lessor is a role
    The Thing is an object
    The Lessee may ask permission to sublease the Thing
    If the Lessor gives permission to the Lessee:
      The ``Sublease'' policy applies

``Maintenance of Confidence'' is a policy where:
    The Lessee is a role
    The Lessor is a role
    The Due Date is an object
    The Lease is an object
    The Thing is an object
    The Lessee must pay the Lessor before the Due Date arrives
      Otherwise, the Lessor may cancel the Lease
    The Lessor must maintain the Thing
      Otherwise, the Lessee may cancel the Lease
\end{lstlisting}

The use of TropICAL to describe legal policies would be useful for lawyers that
want to validate or generate arguments from any given set of policies. This would be valuable as a tool
to aid lawyers in argument creation, for example.
Suppose a defense lawyer is trying to argue that her client
is not guilty. The system could generate traces (sequences of events) of a
given length, whose outcome does not result on the violation of any contract
or law. The lawyer would then be able to choose from amongst a list of
generated arguments to make their case. 

Adapting this idea to our \emph{sublease} example described above,
we can generate sequences of events where a lessee has
subleased a property, and find out whether or not their actions have been in
violation of a policy. We achieve this through the specification of constraints.
For example, a prosecution lawyer wishing to find all sequences of events where
both sublease event and violation events have occurred would specify this ASP
constraint.

\begin{lstlisting}
violEvent :- occurred(viol(X), I, T).
:- not violEvent.
subleaseEvent :- occurred(sublease(X, Y, Z), I, T), holdsat(role(X, lessee), I, T).
:- not subleaseEvent.
\end{lstlisting}

This constraint specifies that we want to generate all possible sequences where
the lessee has subleased something, and where a violation has occurred. Headless
rules in ASP (where the leftmost part of the rule is simply ``:-'') state what we
do \emph{not} want in generated answer sets, so in this example both headless
statements are double negatives.

The solver outputs several answer sets, containing event sequences (traces) of a
specified length. We can then parse these answer sets into a human-readable
``executive summary'' that can be used by lawyers. An example of such a summary
would be:

\begin{lstlisting}
Possibility 0:

The following occurred:
  Alice Leases Bob House
Then:
  Bob Subleases Charlotte House
  VIOLATION: Bob Subleases Charlotte House
\end{lstlisting}

Each summary lists a number of ``possibilities'', with each possibility
corresponding to an answer set produced by the solver, listing events and violations that occur
in a trace. To find an argument, a lawyer can simply read through the generated
possibilities to find one that suits her needs.

\section{Summary}
\label{sec:tropical-summary}
Returning to the requirements listed in section~\ref{sec:requirements} (listed
again here for the reader's convenience), we can now examine our language
specification to see if the requirements are met:

\begin{enumerate}[R1.]
  \item The software must be able to integrate into a
    multi-agent framework such as JASON
  \item The system must direct the behaviour of agents to fit
    a story description
  \item The user should be able to write and re-use
    existing story components (from a library)
  \item The components must be able to express sequences of events
  \item The components must be able to express branching
    (diverging) events
  \item The user should have the ability to nest existing
    components inside new components
  \item The user should be able to visualise the branches of the
    story that result in the addition or modification of components to the story.
  \item The story must tell the character agents what to do, but
    they should be free to break away from it in extreme circumstances, to add a
    degree of unpredictability
\end{enumerate}

Chapter~\ref{cha:agents} describes how our system integrates with the JASON
multi-agent framework. The output of an answer set solver such as Clingo can be
used to add beliefs to an agent's mental model to let it know what actions it is
permitted and obliged to carry out as part of a story. This fulfills requirement~\ref{req:agents}.

The use of permissions and obligations to describe story actions to agents in
either a weakly (permissions) or strongly (obligations) enforced way gives the
author two ways to direct agent behaviour. At times when the author wants agents
to conform tightly with a trope description, they can give the agent an
obligation with a strict consequence for non-compliance. This matches the needs
of requirement~\ref{req:direct}. Requirement~\ref{req:norm} states that authors
need a way to add some flexibility into a story, so that character agents are
given some freedom to break away from the story if needed. This is achieved
through the use of permissions to regulate agent actions.

Section~\ref{sec:seq-code} shows how sequences of events can be described in
TropICAL, fulfilling requirement~\ref{req:sequences}.
Section~\ref{sec:branch-code} does the same for branching story structures,
meeting requirement~\ref{req:branches}. A means of abstraction is
an important feature of our language, as stated in
requirement~\ref{req:subtropes}. Section~\ref{sec:subtrope-code} describes the
use of subtropes to meet the need for users to create their own abstractions.

As a text-based programming language, TropICAL addresses the requirements
mentioned above (numbers \ref{req:agents}, \ref{req:direct},
\ref{req:sequences}, \ref{req:branches}, \ref{req:subtropes} and
\ref{req:norm}). However, there are two requirements that cannot be fulfilled
through a text-based programming paradigm alone: selecting pre-existing story
components from a library (requirement~\ref{req:components}), and visualising
the branches of the story as it is modified (requirement~\ref{req:vis}). These
requirements are best addressed through an Interactive Development Environment
(IDE) with which we can add graphical features for the browsing, selection and
visualisation of tropes. The next chapter (Chapter~\ref{cha:storybuilder})
introduces \emph{StoryBuilder}, a browser-based tool that adds these features to
our TropICAL language.

% TODO outro: 1

% storybuilder

% HACK: predicted poms remaining: 51

\chapter{StoryBuilder: An Interface for Trope-based Interactive Story Creation}
% Think of it as an IDE: what kind of support do users want? Helps to visualise the complexity of the stories. Helps authors to construct the stories that they intend to construct

\label{cha:storybuilder}

% TODO intro: 2

\section{Use Cases}
% TODO intro: 1
% TODO list use cases: 3

\section{Requirements}
% TODO intro: 1
% TODO list requirements from use cases: 3

\section{Overview of Interface}

% TODO intro: 1
% TODO description of interface: 2
% TODO screens & explanation of editor: 2
% TODO screens & explanation of arranger: 2
% TODO screens & explanation of visualisations: 2

\section{Annotated User Stories}

% TODO intro: 1
% TODO user story 1 screens: 2
% TODO user story 1 explanation: 2
% TODO user story 2 screens: 2
% TODO user story 2 explanation: 2
% TODO user story 3 screens: 2
% TODO user story 3 explanation: 2

\section{Design Justification}
% Return to requirements, explain how they are met

% TODO intro: 1
% TODO explain how requirements have been met: 4

\section{Example Stories}
\label{sec:story-vis}
% Return to tests from previous section (though these could be the ``user
% stories'''), and put screenshots of visualisations

% TODO intro: 1
% TODO hero's journey screens: 2
% TODO hero's journey explanation: 2
% TODO evil empire screens: 2
% TODO evil empire explanation: 2
% TODO both screens: 2
% TODO both explanation: 2

% TODO outro: 1

% evaluation

% HACK predicted poms remaining: 21

\chapter{Evaluation}
% TODO intro: 2

\section{Methodology}
% TODO Explain the concept of ``thematic analysis''': 3

\section{Process}
% TODO explain the process: 3

\section{Questions}
% TODO list the questions asked: 3

\section{Transcripts}
% TODO transcript snippets: 2
% TODO transcript snippet explanations: 3
% Just choice quotes and snippets go here, but make sure that you forward link
% to the appendix

\section{Identified Themes}
% TODO intro: 1
% TODO list themes: 4

\section{Discussion}
\subsection{Comparison with a Propp-based Formalism Approach}

\subsection{Comparison with a Planner-Based Drama Manager Approach}

% conclusion

% HACK predicted poms remaining: 56

\chapter{Conclusions \& Future Work}
\label{cha:conclusions}

% TODO intro: 2

\section{Successes}

% TODO intro / summary: 2

\subsection{Trope Formalism}
% TODO explain success: 2

\subsection{TropICAL Syntax and Features}
% TODO explain success: 2
\subsection{Code Generated in InstAL}
% TODO explain success: 2
\subsection{Legal Policy Examples}
% TODO explain success: 2
\subsection{Answer Set Generation}
% TODO explain success: 2
\subsection{StoryBuilder Interface}
% TODO explain success: 2
\subsection{Story VIsualisation}
% TODO explain success: 2

\section{Potential Improvements}
% TODO intro / summary: 2

\subsection{Trope Formalism}
% TODO explain improvements: 2
\subsection{TropICAL Syntax and Features}
% TODO explain improvements: 2
\subsection{Code Generated in InstAL}
% TODO explain improvements: 2
\subsection{Legal Policy Examples}
% TODO explain improvements: 2
\subsection{Answer Set Generation}
% TODO explain improvements: 2
\subsection{StoryBuilder Interface}
% TODO explain improvements: 2
\subsection{Story VIsualisation}
% TODO explain improvements: 2

\section{Future Work}
\label{sec:future}
% TODO intro: 2

\subsection{Hierarchical Institutions}
% TODO explain: 3
\subsection{Expanded Use of TropICAL for Policy Descriptions}
% TODO explain: 3
\subsection{Role Hierarchies}
% TODO explain: 3
\subsection{Verifiable Contracts}
% TODO explain: 3
\subsection{IDE Support}
% TODO explain: 3
\subsection{Visual Coding Front End}
% TODO explain: 3

% TODO outro 2

\appendix
\chapter{Full TropICAL Examples}
\label{appendix:tropes}

\chapter{Generated InstAL Code}
\label{appendix:instal}

\section{Obligation Examples}
\label{appendix:obls}
\begin{lstlisting}[label={lst:obl1-instal},caption={An obligation event with no
deadline or consequence events}]
institution obligation1;
% TYPES ----------
type Identity;
type Agent;
type Role;
type Trope;
type Phase;
type Place;
type PlaceName;
type Object;
type ObjectName;

% FLUENTS ----------
fluent role(Agent, Role);
fluent phase(Trope, Phase);
fluent place(PlaceName, Place);
fluent object(ObjectName, Object);


% EXTERNAL EVENTS: Obligation 1 ----------
exogenous event go(Agent, PlaceName);
exogenous event noDeadline;

% VIOLATION EVENTS: Obligation 1 ----------
violation event violHeroGoLandOfAdventure;
violation event noViolation;

% INST EVENTS: Obligation 1 ----------

inst event intGo(Agent, PlaceName);
inst event intObligation1(Agent, PlaceName, PlaceName);
inst event intNoDeadline;

% OBLIGATION FLUENTS: Obligation 1 ----------
obligation fluent obl(intGo(Agent, PlaceName), intNoDeadline, noViolation);

% INITIATES: Obligation 1 ----------
intObligation1(R, S, T) initiates
    phase(obligation1, phaseA),
    obl(intGo(R,T), intNoDeadline, noViolation), perm(go(R, T)), perm(intGo(R,T)), pow(intGo(R,T)) if
        phase(obligation1, active),
        role(R, hero),
        place(T, landOfAdventure);
% TERMINATES: Obligation 1 ----------
intObligation1(R, S, T) terminates
    phase(obligation1, active),
    perm(go(R, S)) if
        phase(obligation1, active),
        role(R, hero),
        place(S, home);
intObligation1(R, S, T) terminates
    phase(obligation1, phaseA),
    obl(intGo(R,T), intNoDeadline, noViolation), perm(go(R, T)), perm(intGo(R,T)), pow(intGo(R,T)) if
        phase(obligation1, phaseA),
        role(R, hero),
        place(T, landOfAdventure);


% GENERATES: Obligation 1 ----------
go(R, S) generates
    intObligation1(R, S, T) if
        role(R, hero),
        place(S, home);
go(R, T) generates
    intObligation1(R, S, T) if
        role(R, hero),
        place(T, landOfAdventure);
go(R, S) generates
    intGo(R,S) if
        role(R, hero),
        place(S, landOfAdventure);

% INITIALLY: -----------
initially
    pow(intObligation1(R, S, T)) if role(R, hero), place(S, home), place(T, landOfAdventure);
initially
    perm(intObligation1(R, S, T)) if role(R, hero), place(S, home), place(T, landOfAdventure);
initially
    perm(go(R, S)) if role(R, hero), place(S, home);
initially
    phase(obligation1, active),
    role(hero, hero),
    place(landOfAdventure, landOfAdventure),
    place(home, home);
\end{lstlisting}

\begin{lstlisting}[label={lst:obl2-instal}, caption={An institution with an
obligation event containing a deadline and a consequence}]
institution obligation2;
% TYPES ----------
type Identity;
type Agent;
type Role;
type Trope;
type Phase;
type Place;
type PlaceName;
type Object;
type ObjectName;

% FLUENTS ----------
fluent role(Agent, Role);
fluent phase(Trope, Phase);
fluent place(PlaceName, Place);
fluent object(ObjectName, Object);


% EXTERNAL EVENTS: Obligation 2 ----------
exogenous event kill(Agent, Agent);
exogenous event go(Agent, PlaceName);
exogenous event noDeadline;

% VIOLATION EVENTS: Obligation 2 ----------
violation event violHeroGoLandOfAdventure;
violation event noViolation;

% INST EVENTS: Obligation 2 ----------
inst event intKill(Agent, Agent);
inst event intGo(Agent, PlaceName);
inst event intObligation2(Agent, Agent, Agent, PlaceName);
inst event intNoDeadline;

% OBLIGATION FLUENTS: Obligation 2 ----------
obligation fluent obl(intGo(Agent, PlaceName), intKill(Agent, Agent), violHeroGoLandOfAdventure);

% INITIATES: Obligation 2 ----------
violHeroGoLandOfAdventure initiates
    perm(kill(R, S)) if
        role(R, villain),
        role(S, hero);
% TERMINATES: Obligation 2 ----------
intObligation2(R, S, T, U) terminates
    phase(obligation2, active),
    obl(intGo(R,U), intKill(S,T), violHeroGoLandOfAdventure), perm(go(R, U)), perm(intGo(R,U)), pow(intGo(R,U)) if
        phase(obligation2, active),
        role(R, hero),
        place(U, landOfAdventure),
        role(T, mentor),
        role(S, villain),
        role(R, hero),
        role(S, villain);
intKill(S, T) terminates
    obl(intGo(R,U), intKill(S,T), violHeroGoLandOfAdventure), perm(go(R, U)), perm(intGo(R,U)), pow(intGo(R,U)) if
        role(R, hero),
        place(U, landOfAdventure),
        role(T, mentor),
        role(S, villain),
        role(R, hero),
        role(S, villain);


% GENERATES: Obligation 2 ----------
kill(S, T) generates
    intObligation2(R, S, T, U) if
        role(R, hero),
        role(S, villain);
go(R, U) generates
    intObligation2(R, S, T, U) if
        role(R, hero),
        place(U, landOfAdventure);
kill(R, S) generates
    intObligation2(R, S, T, U) if
        role(T, mentor),
        role(S, villain);
go(R, U) generates
    intGo(R,U) if
        role(R, hero),
        place(U, landOfAdventure);
kill(S, T) generates
    intKill(S,T) if
        role(R, hero),
        role(S, villain);
kill(R, S) generates
    intKill(R,S) if
        role(T, mentor),
        role(S, villain);

% INITIALLY: -----------
initially
    pow(intObligation2(R, S, T, U)) if role(R, hero), role(S, villain), role(T, mentor), place(U, landOfAdventure);
initially
    perm(intObligation2(R, S, T, U)) if role(R, hero), role(S, villain), role(T, mentor), place(U, landOfAdventure);
initially
    obl(intGo(R,U), intKill(S,T), violHeroGoLandOfAdventure), perm(go(R, U)), perm(intGo(R,U)), pow(intGo(R,U)) if role(R, hero), place(U, landOfAdventure);
initially
    pow(intKill(S, T)),
    phase(obligation2, active),
    role(hero, hero),
    role(villain, villain),
    role(mentor, mentor),
    place(landOfAdventure, landOfAdventure);
\end{lstlisting}

\section{Branching Trope Example}
\label{appendix:branch3}
\begin{lstlisting}
institution branch3;
% TYPES ----------
type Identity;
type Agent;
type Role;
type Trope;
type Phase;
type Place;
type PlaceName;
type Object;
type ObjectName;

% FLUENTS ----------
fluent role(Agent, Role);
fluent phase(Trope, Phase);
fluent place(PlaceName, Place);
fluent object(ObjectName, Object);


% EXTERNAL EVENTS: Branch 3 ----------
exogenous event kill(Agent, Agent);
exogenous event go(Agent, PlaceName);
exogenous event find(Agent, ObjectName);
exogenous event meet(Agent, Agent);
exogenous event noDeadline;

% VIOLATION EVENTS: Branch 3 ----------
violation event noViolation;

% INST EVENTS: Branch 3 ----------
inst event intKill(Agent, Agent);
inst event intFind(Agent, ObjectName);
inst event intGo(Agent, PlaceName);
inst event intMeet(Agent, Agent);
inst event intBranch3(Agent, Agent, Agent, ObjectName, PlaceName, PlaceName, PlaceName);
inst event intNoDeadline;

% INITIATES: Branch 3 ----------
intBranch3(R, S, T, U, V, W, X) initiates
    phase(branch3, phaseA),
    perm(find(R, U)),
    perm(go(R, X)),
    perm(kill(R, S)) if
        phase(branch3, active),
        object(U, sword),
        role(R, hero),
        place(X, landOfAdventure),
        role(S, villain);
intBranch3(R, S, T, U, V, W, X) initiates
    phase(branch3, phaseB),
    perm(meet(R, T)),
    perm(go(R, V)) if
        phase(branch3, phaseA),
        place(V, realmOfMystery),
        role(R, hero),
        role(T, mentor);
% TERMINATES: Branch 3 ----------
intBranch3(R, S, T, U, V, W, X) terminates
    phase(branch3, active),
    perm(go(R, W)) if
        phase(branch3, active),
        role(R, hero),
        place(W, home);
intBranch3(R, S, T, U, V, W, X) terminates
    phase(branch3, phaseA),
    perm(find(R, U)),
    perm(go(R, X)),
    perm(kill(R, S)) if
        phase(branch3, phaseA),
        object(U, sword),
        role(R, hero),
        place(X, landOfAdventure),
        role(S, villain);
intBranch3(R, S, T, U, V, W, X) terminates
    phase(branch3, phaseB),
    perm(meet(R, T)),
    perm(go(R, V)) if
        phase(branch3, phaseB),
        place(V, realmOfMystery),
        role(R, hero),
        role(T, mentor);


% GENERATES: Branch 3 ----------
find(R, U) generates
    intBranch3(R, S, T, U, V, W, X) if
        role(R, hero),
        object(U, sword);
go(R, V) generates
    intBranch3(R, S, T, U, V, W, X) if
        role(R, hero),
        place(V, realmOfMystery);
meet(R, T) generates
    intBranch3(R, S, T, U, V, W, X) if
        role(T, mentor),
        role(R, hero);
go(R, W) generates
    intBranch3(R, S, T, U, V, W, X) if
        role(R, hero),
        place(W, home);
go(R, X) generates
    intBranch3(R, S, T, U, V, W, X) if
        role(R, hero),
        place(X, landOfAdventure);
kill(R, S) generates
    intBranch3(R, S, T, U, V, W, X) if
        role(S, villain),
        role(R, hero);

% INITIALLY: -----------
initially
    pow(intBranch3(R, S, T, U, V, W, X)) if role(R, hero), role(S, villain), role(T, mentor), object(U, sword), place(V, realmOfMystery), place(W, home), place(X, landOfAdventure),
    perm(intBranch3(R, S, T, U, V, W, X)) if role(R, hero), role(S, villain), role(T, mentor), object(U, sword), place(V, realmOfMystery), place(W, home), place(X, landOfAdventure),
    perm(go(R, W)) if role(R, hero), place(W, home),
    phase(branch3, active),
    role(hero, hero),
    role(villain, villain),
    role(mentor, mentor),
    place(home, home),
    place(landOfAdventure, landOfAdventure),
    place(realmOfMystery, realmOfMystery),
    object(sword, sword);
\end{lstlisting}\mnote{Move this to the appendix}

\section{Parser Implementation}
\label{appendix:t-grammar}
% TODO intro: 1
This section of the appendix describes the technical implementation of the TropICAL parser.

% TODO instaparse description: 1
We use the Clojure programming language~\citep{clojure} with the Instaparse
parser generator library~\cite{instaparse} to create the language parser for
TropICAL. The method of parser creation with Instaparse is to describe the
parser using a syntax that resembles \emph{Extended Backus-Naur Form} (EBNF)
grammar, combined with regular expressions. According to the author, its
features are:

\begin{itemize}
\item Turns standard EBNF or ABNF notation for context-free grammars into an executable parser that takes a string as an input and produces a parse tree for that string.
\item No Grammar Left Behind: Works for any context-free grammar, including left-recursive, right-recursive, and ambiguous grammars.
\item Extends the power of context-free grammars with PEG-like syntax for lookahead and negative lookahead.
\item Supports both of Clojure's most popular tree formats (hiccup and enlive) as output targets.
\item Detailed reporting of parse errors.
\item Optionally produces lazy sequence of all parses (especially useful for diagnosing and debugging ambiguous grammars).
\item "Total parsing" mode where leftover string is embedded in the parse tree.
\item Optional combinator library for building grammars programmatically.
\item Performant.
\end{itemize}

Due to using this method of parser creation, our parser has two stages: first it
parses the entity declarations (roles, objects and places) at the top of the
trope description, then takes the parsed declarations and inserts them as
keywords in the grammar for the rest of the language, creating a new parser
specifically for those entity declarations. For example, if the author writes
``The Hero is a role'', ``The Castle is a place'' and ``The Sword is an
object'', then the \emph{role} 

% TODO simplified grammar from code: 2
% TODO format grammar: 2

The Extended Backus-Naur Form (EBNF) of the grammar is shown in
listings~\ref{lst:ebnf1} and~\ref{lst:ebnf2}. Listing~\ref{lst:ebnf1} describes
the syntax for the entity declarations at the top of each trope definition file,
as described in section~\ref{sec:dec-code}. Listing~\ref{lst:ebnf2} shows the
grammar for the rest of the trope definition files.

\begin{lstlisting}[showstringspaces=false,label={lst:ebnf1},caption={EBNF
grammar for the entity declarations in TropICAL}]
text = tropedef defs trope
<tropedef> = label <' is a ' ('trope' / 'policy') ' where:\\n'>
<defs> = (<whitespace> (chardef | objdef | placedef) <'\\n'?>)+
chardef = charname <' is a ' ('character' | 'role') '.'?>
objdef = objname <' is an object' '.'?>
placedef = placename <' is a place' '.'?>
<charname> = name
<objname> = name
<placename> = name
trope = (<whitespace> !(chardef | objdef | placedef) #'.*' <'\\n'>?)+
label = <'\"'> words <'\"'>
<whitespace> = #'\\s\\s'
<name> = [<'The ' / 'the '>] cwords
<words> = word (<' '> word)*
<cwords> = cword (<' '> ['of'<' '>] cword)*
<cword> = #'[A-Z][0-9a-zA-Z\\-\\_\\']*'
<the> = <'The ' | 'the '>
<word> = #'[0-9a-zA-Z\\-\\_\\']*'
\end{lstlisting}\mnote{I've taken this from the code verbatim. We need to meet and discuss how to simplify this.}

\begin{lstlisting}[showstringspaces=false,label={lst:ebnf2},caption={EBNF
grammar for the main trope definitions in TropICAL}]
trope = (<whitespace> sequence)+ <'\\n'?>
(str "label = '" label "'")
fluent = object <' is '> adjective
adjective = word
outcome =
(<'\\n' whitespace whitespace> (event | obligation | happens) (or? | if?) <'\\n'?>)+
happens =
<the?> subtrope <(' happens' / ' trope happens' / ' policy applies') '.'?>
block =
<the?> subtrope <' policy does not apply' / ' does not happen' / ' trope does not happen'> <'.'?>
sequence =
((fluent | event | norms | happens | block) (or* | if*)) | ((fluent | event | norms | happens | block) (<whitespace+ 'Then '> (block / norms / event / fluent / obligation / happens) (or* | if*))*)*
or =
<whitespace+ 'Or '> (fluent | event | norms)
if =
<whitespace+ 'If '> (fluent | event | norms)
action = !fverb verb [sp [particle sp] (character | object | place)] (crlf? | [sp particle? (character | object | place)] crlf?)
event = character sp action
<particle> = <'a' | 'at' | 'will' | 'of' | 'to'>
norms = permission | rempermission | obligation
fluent = (character | object | place) sp fverb sp [particle sp] (character | object | place) crlf?
fverb = (<'is'> sp 'at') | 'has'
violation = norms
(str "character = " (if (seq rs) rs "'nil'"))
(str "place = " (if (seq ps) ps "'nil'"))
(str "object = " (if (seq os) os "'nil'"))
subtrope = <'\'> words <'\"'>"
label = <'\'> words <'\"'>"
verb = words
rempermission = character <' may not '> action crlf?
permission = character <' may '> action crlf?
obligation = character <' must '> action (<' before '> deadline)? (<crlf whitespace+ 'Otherwise, '> <'the '?> violation)? <'.'?> crlf?
deadline = consequence
role-b = character
object-b = object
place-b = place
<crlf> = <'\\n'>
consequence = event
<whitespace> = #'\\s\\s'
<sp> = <' '>
<words> = word (sp word)*
<cwords> = cword (sp ['of' sp] cword)*
<cword> = #'[A-Z][0-9a-zA-Z\\-\\_\\']*'
<the> = <'The ' | 'the '>
<word> = #'[0-9a-zA-Z\\-\\_\\']*'
\end{lstlisting}\mnote{See above. Also, would it be best to move this to the appendix?}


\section{Parse Tree Examples}
\label{appendix:parse-tree}

% TODO describe intermediate data structure, hash-map structure
% TODO explain parse tree diagrams, hash-map notation (look up best way to do hmaps in latex)

This section of the appendix shows visualisations of the parse tree that results from the trope
code examples listed in sections~\ref{sec:dec-code} to~\ref{sec:subtrope-code}.

This is the parse tree output of the sequence of events in listing~\ref{lst:seq3}:

\vspace{7mm}
\centerline{\includegraphics[width=\textwidth]{seq2-tree.png}}
\vspace{7mm}

\begin{lstlisting}[showstringspaces=false]
{:label "Sequence 2", :events [{:role "Hero", :verb "go", :place "Land of Adventure"} {:role "Hero", :verb "find", :object "Sword"} {:verb "meet", :role-b "Villain", :role-a "Hero"} {:role "Hero", :verb "return", :place "Home"}], :roles ("The Hero" "The Villain"), :objects ("The Sword"), :locations ("Home" "The Land of Adventure")}
\end{lstlisting}

\vspace{7mm}
\centerline{\includegraphics[width=\textwidth]{branch2-tree.png}}
\vspace{7mm}

\begin{lstlisting}[showstringspaces=false]
{:label "Branch 2", :events [{:or [{:role "Hero", :verb "go", :place "Land of Adventure"} {:role "Hero", :verb "find", :object "Sword"} {:verb "kill", :role-b "Villain", :role-a "Hero"} {:role "Hero", :verb "return", :place "Home"}]}], :roles ("The Hero" "The Villain"), :objects ("The Sword"), :locations ("Home" "The Land of Adventure")}
\end{lstlisting}

\vspace{7mm}
\centerline{\includegraphics[width=\textwidth]{branch3-tree.png}}
\vspace{7mm}

\begin{lstlisting}[showstringspaces=false]
{:label "Branch 3", :events [{:role "Hero", :verb "go", :place "Home"} {:or [{:role "Hero", :verb "find", :object "Sword"} {:role "Hero", :verb "go", :place "Land of Adventure"} {:verb "kill", :role-b "Villain", :role-a "Hero"}]} {:or [{:verb "meet", :role-b "Mentor", :role-a "Hero"} {:role "Hero", :verb "go", :place "Realm of Mystery"}]}], :roles ("The Hero" "The Villain" "The Mentor"), :objects ("The Sword"), :locations ("Home" "The Land of Adventure" "The Realm of Mystery")}
\end{lstlisting}

\vspace{7mm}
\centerline{\includegraphics[width=\textwidth]{item-search-tree.png}}
\vspace{7mm}

\begin{lstlisting}[showstringspaces=false]
{:label "Item Search", :events [{:role "Hero", :verb "chase", :object "Macguffin"} {:or [{:role "Hero", :verb "find", :object "Macguffin"} {:role "Hero", :verb "go", :place "Home"}]}], :roles ("The Hero"), :objects ("The Macguffin"), :locations ("Home")}
\end{lstlisting}

\vspace{7mm}
\centerline{\includegraphics[width=\textwidth]{kill-then-search-tree.png}}
\vspace{7mm}

\begin{lstlisting}[showstringspaces=false]
{:label "Kill then Search", :events [{:role "Hero", :verb "go", :place "Away"} {:verb "kill", :role-b "Villain", :role-a "Hero"} {:subtrope "Item Search"}], :roles ("The Hero" "The Villain"), :objects (), :locations ("Away")}
\end{lstlisting}

\section{``Evil Empire'' Trope Example}
\label{appendix:evil-empire}

\subsection{Traces}
\label{appendix:evil-empire-traces}

\bibliographystyle{apalike}
\bibliography{thesis}

\end{document}
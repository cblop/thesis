\documentclass[11pt]{report}
\usepackage{baththesis}
\usepackage{graphicx}
\usepackage{caption}
\usepackage{amsmath}
\usepackage{subcaption}
\usepackage{verbatim}
\usepackage[inline]{enumitem}
%\usepackage{amsfonts}
\usepackage{url}
\usepackage{natbib}
\usepackage[hmargin=3cm,vmargin=2cm]{geometry}
%\usepackage{mathpazo}
%\usepackage{eulervm}
\usepackage[usenames,dvipsnames,svgnames,table]{xcolor}
\usepackage[]{todonotes}
\usepackage{tikz}
\usetikzlibrary{arrows}
\usepackage[]{todonotes}

\def\mnote#1{\todo[color=Goldenrod,size=\scriptsize]{Matt: #1}}
\def\jnote#1{\todo[color=CornflowerBlue,size=\scriptsize]{Julian: #1}}
\def\snote#1{\todo[color=WildStrawberry,size=\scriptsize]{Steve: #1}}
\definecolor{light-gray}{gray}{0.95}

\usepackage{listings}
\lstset{ %
   language=prolog,
%  frame=l,                     % adds a frame around the code
   basicstyle=\footnotesize\ttfamily,  % use courier
   breaklines=true,
   xleftmargin=0.5em,
   aboveskip=0.5em,
   belowskip=0.5em,
%  belowcaptionskip=5em,
   numbers=left,
   backgroundcolor=\color{light-gray},
   frame=single,
   framerule=0pt
}

\setlength{\jot}{0pt}
\def\mylabel#1{\tikz[remember picture]\node(#1){};}
\def\myref#1#2#3{\begin{tikzpicture}[remember picture]
\node[draw,rounded corners] (#2){\begin{minipage}{\textwidth}\raggedright#3\end{minipage}};
\draw[overlay,-triangle 45,thick,gray](#2.west)--(#1.west);
\end{tikzpicture}}

\title{Building Abstractable Story Components with Institutions and Tropes}
\author{Matt Thompson}
\degree{EngD Digital Media}
\department{Department of Computer Science}
\degreemonthyear{October 2016}
\norestrictions


\begin{document}
\maketitle
\clearpage
\tableofcontents
\clearpage

\chapter{Introduction}
\label{cha:introduction}
This report summarises the research carried out as part of the EngD Digital Media programme with the University of Bath and Sysemia Ltd.

The purpose of this report is to explain the academic research conducted during the course and its applicability to the needs of Sysemia Ltd. The main focus is to explore the state of the art in interactive and generative narrative and also to apply it in a real-world exhibition context.

Interactive narratives have the potential to transform two major fields: games
and training simulations. These are the applications for which the use of
interactive narrative can benefit Sysemia. The main part of the research examines the state of the art in interactive narrative primarily in computer games research, but then adapts techniques from that field for the purpose of interactive exhibitions for education. By implementing research ideas into `toy' game projects (such as an interactive Punch and Judy show in section \ref{}), I then take components from these projects and implement them into practical exhibition-based projects for Sysemia (such as the Every Object Tells a Story project described in section \ref{}).

\section{Interactive narrative for games}
Computer games are a new medium for artistic expression. The element of interactivity, combined with visual art, music and storytelling, allows the creation of ever more fantastic worlds. This combination of previous art forms isn't enough, though. Game creators are now experimenting with ways to make a narrative itself interactive. Imagine playing through a version of Romeo and Juliet where there is a possibility that the characters could escape their fates. Or playing a detective in a game where the story changes depending on how quickly you can piece together clues.

This is the new frontier that we are exploring with this research: the as-yet unconquered domain of \emph{interactive} storytelling.

We are defined by the choices we make throughout our lives. These choices form a story that describes our own personal history. If a player can watch somebody's life unfold and witness the choices that they made, and those that were forced upon them, they would have a deep understanding of how they came to be who they are.

This kind of deep understanding is only possible through experiencing a story where the decisions made have real consequences. Traditional fiction, and perhaps computer games, enable this to some extent. But the story is still experienced passively by the consumer in these media. A truly interactive narrative would go deeper, as though you had truly lived as another person.

For this reason, interactive narrative for games is worth exploring: it would
enable a player to truly understand other people's lives and situations, and why they became the people they are.

% Put a bit about what you did to explore the potential of interactive narrative
% for games

\section{Interactive narrative for education}
Stories are a powerful way for humans to understand and remember facts and events. Making these stories interactive could lead the way to even more effective learning methods.

\citet{schank1990tell} argues that stories may be a useful way for humans to better remember a series of events, when compared to simply reciting them as a list of facts. Mentalists and creative students use storytelling mnemonic techniques to help them memorise large lists of difficult-to-remember facts with perfect recall.

Training simulations make the use of stories taking place in immersive environments to help trainees to master new routines and techniques. These stories always follow a fixed pattern, however, unlike the mnemonic stories that people create to memorise information, which are highly personalised to the practitioner.

A narrative that dynamically reacts to the decisions of the audience could better enable them to remember information when compared with a static, unpersonalised narrative.

With this in mind, it is worth constructing an interactive narrative system and applying it to both entertainment and learning contexts. Some evaluation still needs to be carried out in order to determine how much more effective (if at all) interactive narrative could be for education, so this is proposed as part of this report.

Preliminary work has been done in collaboration with the Bishop's Palace in Wells to create an exhibition incorporating interactive narrative. Using a combination of Semantic Web data formats and an event calculus-based reasoner, the system will be able to infer future actions for recommendation to visitors. Description of the work done so far is in section \ref{}.

Both the game-based and education-based projects have a common thread: reasoning
over temporal data in order to predict future events, or constrain possible
future actions to conform to a narrative domain. Section \ref{}
describes the direction the research has taken, and how the use of techniques such as hierarchical institutions and the creation of an ontology for narrative might develop these ideas further.

\section{Outline}
% Outline of sections goes here
This report begins with a review of the literature in Section~\ref{cha:literature-review}. This review covers two main fields of research: narrative research from social science (Section~\ref{sec:narratology}), and interactive narrative research from computer science (Section~\ref{sec:implementations}).
The ``Institutions as Story Worlds'' chapter (\ref{cha:institutions}) describes the theory and research behind instituitions, and discusses the merit of their use for describing story, comparing their use against other logics and planner-based systems.
The ``Tropes as Story Components'' chapter (\ref{cha:tropes}), discusses the use
of \emph{tropes} as story components, and why they improve upon existing narrative formalisms.
Chapter~\ref{cha:tropes-and-institutions} describes how to use institutions to define tropes, and how to combine them to describe the story world for an interactive narrative. The section also describes the use of this technique to build a number of different narratives, contrasting the approach with using other techniques such as planners to build interactive narratives.
The final chapter looks back at the research done, evaluating its potential impact and discussing possible future work. 

\chapter{Literature Review}
\label{cha:literature-review}
This research covers a large number of fields of study, therefore an extensive
literature review covering these fields is needed. This section starts with a
look at the field of \emph{narratology}, or narrative theory, to gain some
insights into the themes and components that make up stories. Looking at
different formalisms that have been created for narrative and which themes and
motifs recur in stories should better inform the creation of techniques with
which to generate stories.

Additionally, computers have the potential to introduce a new type of storytelling. Though computer games offer increasingly immersive interactive worlds, the form of narrative they use remains much the same: linear. A player may be able to interact with elements of a game, but the story itself remains unchanged.
Chris Crawford refers to a new possible form of narrative as ``Interactive Storytelling'' \citep{crawford2012chris}. Though researchers and game designers have made great strides towards realising this new artform, little has been produced to capture the public imagination.

% why are you looking at classic narratology?
In order to better inform any implementation of interactive narrative, this
review begins with an examination of the field of classic narrative theory. Following from this is a look at the emerging research in interactive and generative narrative and their implementations.
As part of the examination of implementations of interactive narrative, this review especially focuses on agent-based systems. The section concludes with an overview of emotional models that can be used to model distinct characters using agents.

\section{Narratology}
\label{sec:narratology}
Narratology is a deep field with many sub-fields. This review examines the parts of it that might best inform the modelling of narrative by computers, as well as the construction of interactive narrative.

The first part of the overview of narratology examines research into
categorising different types of narrative, both traditional and experimental.
This draws from classic narratological texts, as well as work done on
``cybertext'' and experimental narrative in the interactive age. This
examination of recent research into non-linear narratives better informs how to better construct interactive narratives for games or simulations.

Structuralist formalisms of narrative attempt to explain how stories work by
dividing them into commonly occuring themes and motifs. This is a natural fit to
the modelling of narratives by computer, especially if using an ontology. This
overview of narratology starts with structuralism for this reason.

After the overview of the structuralists follows a section on the use of formal
logic for narrative modelling. The section ends with descriptions of other types
of story components, taxonomies and ontologies used in the literature.



\subsection{Types of Narrative}
% Look at Cybertext, etc, and try to explain how best to divide different types
% of story

\subsection{Structuralist Formalisms of Narrative}
% Propp, etc
Attempts to organise recurring themes, roles and motifs of narrative go back at least a century. The Aarne-Thompson tale-type index \citep{aarne1987types}, first published in 1910 and later refined by Stith Thompson in 1928 and 1961, is well known amongst folklorists as a classification and analysis method for traditional folktales and myths. Aarne-Thompson's index is a taxonomy of tale themes, arranging tales into categories such as \emph{animal tales} and \emph{jokes and anecdotes}, and then sub-categories (\emph{tales of magic} and \emph{numskull [sic] stories} being two examples). This taxonomy is only two levels deep however, and only serves as a useful way to categorise individual stories or tales. In order to break down and analyse components of tales, we must dig deeper.

In \emph{Structural Anthropology}, Claude L\'{e}vi-Strauss seeks to discover why myths and legends are so similar across cultures and history \citep{levi2008structural}. He concludes that there are global laws that govern the way in which people create stories, therefore these laws can be modelled as a set of rules for describing myths.

His theory is that myths describe opposing forces which are resolved through mediation. The example he gives in \emph{Structural Anthropology} describes how Native American legends often contain `trickster' characters in the form of ravens or coyotes. As scavenging animals, these tricksters symbolically act as mediators between life and death.

Like much of early narrative theory, there is no rigorous evaluation of L\'{e}vi-Strauss' ideas, leaving them feeling a little too opinionated and arbitrary. While interesting, L\'{e}vi-Strauss' ideas bring us no closer to developing a formal model of narrative structure. For that, we must go even further back in time, and turn to Vladimir Propp.

\subsubsection{Propp's Morphology of the Folktale}
Propp's seminal work ``The Morphology of the Folktale'' \citep{propp1968morphology}, though first published in 1928, is still a widely-used formalism for researchers and game designers looking to generate narratives procedurally. Propp identifies recurring characters and motifs in Russian folklore, distilling them down to a concise set of rules with which to describe stories.

In this formalism, characters have \emph{roles}, such as \emph{hero}, \emph{villain}, \emph{dispatcher}, \emph{false hero}, and more. Characters performing a certain role are able to perform a subset of \emph{story functions}, which are actions that make the narrative progress. For example, the \emph{dispatcher} might send the \emph{hero} on a quest, or the \emph{victim} may issue an \emph{interdiction} to the \emph{villain}, which is then \emph{violated}.

Propp defines a total of 31 distinct story functions, each of which is given a number and symbol in order to create a succinct way of describing entire stories. Examples of such functions are:

\begin{itemize}
  \item One of the members of a family absents himself from home: \emph{absentation}.
  \item An interdiction is addressed to the hero: \emph{interdiction}.
  \item The victim submits to deception and thereby unwittingly helps his enemy: \emph{complicity}.
  \item The villain causes harm or injury to a member of the family: \emph{villainy}.
\end{itemize}

Each of these functions can vary to some degree. For example, the \emph{villainy} function can be realised as one of 19 distinct forms of villainous deed, including \emph{the villain abducts a person}, \emph{the villain seizes the daylight}, and \emph{the villain makes a threat of cannibalism}.

These functions are enacted by characters following certain roles. Each role (or \emph{dramatis personae} in Propp's definition) has a \emph{sphere of action} consisting of the functions that they are able to perform at any point in the story. Propp defines seven roles that have distict spheres of action: \emph{villain}, \emph{donor}, \emph{helper}, \emph{princess}, \emph{dispatcher}, \emph{hero}, and \emph{false hero}.

Though Propp defines each \emph{dramatis personae} as being distinct (characters can only play one role at a time), it is simple to extend the idea to allow for overlapping roles. For example, a victim could also be a donor, so the set of functions they can perform would be the union of both the victim and donors' permitted function sets. Propp does not explore this possibility in \emph{The Morphology of the Folktale}, however.

\begin{figure}[!t]
\centerline{\includegraphics[height=0.4in]{propp1.png}}
\caption{One Propp function following another}\label{fig:propp1}
\end{figure}

\begin{figure}[!t]
\centerline{\includegraphics[height=0.6in]{propp2.png}}
\caption{Multiple simultaneous functions}\label{fig:propp2}
\end{figure}

In a typical story, one story function will follow another as the tale progresses in a sequential series of cause and effect (figure~\ref{fig:propp1}). However, Propp's formalism also allows for simultaneous story functions to be occuring at once (figure~\ref{fig:propp2}).

% What are the shortcomings of Propp? (i.e. lack of abstractability, etc)

\subsection{Describing Stories with Logic}
% Laure-Ryan did a bit of this

\subsection{Other Types of ``Story Component''}

\section{Implementations of Experimental Narrative}
\label{sec:implementations}

\subsection{Story Generation}
% TaleSpin, etc

\subsubsection{Generative Grammar}

\subsubsection{Author Modelling}
% This would also use planners

% But this is limited in interactivity, so in order to have characters we can
% interact with, we must use...

\subsection{Intelligent Agents as Characters}
% Facade, etc

\subsubsection{Character Modelling}

\subsubsection{Characters with Emotional Models}


% But characters in a story need to follow some kind of underlying plot
% mechanism, so...

\subsection{Governing Narrative in a Multi-Agent System}
\subsubsection{Planner-based Systems}
% Riedl, Young, etc

\chapter{Institutions as Story Worlds}
\label{cha:institutions}
\section{Describing Stories With Logic}
\subsection{Modal Logic and Kripke Structures}
\subsection{Deontic Logic and Norms}

\section{Norms and Institutions}
\label{sec:norms-and-institutions}

\section{Why Use Institutions for Interactive Narrative?}
\label{sec:why-use-institutions}
% Write about character freedom, regimentation vs regulation. Give story
% examples vs using a planner

\chapter{Tropes as Story Components}
\label{cha:tropes}
\section{Tropes: a ``Folksonomy'' of Story Components}
% Describe TVtropes

\section{Why Use Tropes?}
% Write about ability to abstract, give story examples vs Propp

\chapter{Describing Story Worlds with Tropes and Institutions}
\label{cha:tropes-and-institutions}

\chapter{Future Work}
\label{cha:future}


\bibliographystyle{apalike}
\bibliography{thesis}

\end{document}
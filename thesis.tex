\documentclass[11pt]{report}
\usepackage{baththesis}
\usepackage{graphicx}
\usepackage{caption}
\usepackage{amsmath}
\usepackage{subcaption}
\usepackage{verbatim}
\usepackage[inline]{enumitem}
%\usepackage{amsfonts}
\usepackage{url}
\usepackage{natbib}
\usepackage[hmargin=3cm,vmargin=2cm]{geometry}
%\usepackage{mathpazo}
%\usepackage{eulervm}
\usepackage[usenames,dvipsnames,svgnames,table]{xcolor}
\usepackage[]{todonotes}
\usepackage{tikz}
\usetikzlibrary{arrows}
\usepackage[]{todonotes}

\def\mnote#1{\todo[color=Goldenrod,size=\scriptsize]{Matt: #1}}
\def\jnote#1{\todo[color=CornflowerBlue,size=\scriptsize]{Julian: #1}}
\def\snote#1{\todo[color=WildStrawberry,size=\scriptsize]{Steve: #1}}
\definecolor{light-gray}{gray}{0.95}

\usepackage{listings}
\lstset{ %
   language=prolog,
%  frame=l,                     % adds a frame around the code
   basicstyle=\footnotesize\ttfamily,  % use courier
   breaklines=true,
   xleftmargin=0.5em,
   aboveskip=0.5em,
   belowskip=0.5em,
%  belowcaptionskip=5em,
   numbers=left,
   backgroundcolor=\color{light-gray},
   frame=single,
   framerule=0pt
}

\setlength{\jot}{0pt}
\def\mylabel#1{\tikz[remember picture]\node(#1){};}
\def\myref#1#2#3{\begin{tikzpicture}[remember picture]
\node[draw,rounded corners] (#2){\begin{minipage}{\textwidth}\raggedright#3\end{minipage}};
\draw[overlay,-triangle 45,thick,gray](#2.west)--(#1.west);
\end{tikzpicture}}

\title{Building Abstractable Story Components with Institutions and Tropes}
\author{Matt Thompson}
\degree{EngD Digital Media}
\department{Department of Computer Science}
\degreemonthyear{October 2016}
\norestrictions


\begin{document}
\maketitle
\clearpage
\tableofcontents
\clearpage

\chapter{Introduction}
\label{cha:introduction}
This report summarises the research carried out as part of the EngD Digital Media programme with the University of Bath and Sysemia Ltd.

The purpose of this report is to explain the academic research conducted during the course and its applicability to the needs of Sysemia Ltd. The main focus is to explore the state of the art in interactive and generative narrative and also to apply it in a real-world exhibition context.

Interactive narratives have the potential to transform two major fields: games
and training simulations. These are the applications for which the use of
interactive narrative can benefit Sysemia. The main part of the research examines the state of the art in interactive narrative primarily in computer games research, but then adapts techniques from that field for the purpose of interactive exhibitions for education. By implementing research ideas into `toy' game projects (such as an interactive Punch and Judy show in section \ref{}), I then take components from these projects and implement them into practical exhibition-based projects for Sysemia (such as the Every Object Tells a Story project described in section \ref{}).

\section{Interactive narrative for games}
Computer games are a new medium for artistic expression. The element of interactivity, combined with visual art, music and storytelling, allows the creation of ever more fantastic worlds. This combination of previous art forms isn't enough, though. Game creators are now experimenting with ways to make a narrative itself interactive. Imagine playing through a version of Romeo and Juliet where there is a possibility that the characters could escape their fates. Or playing a detective in a game where the story changes depending on how quickly you can piece together clues.

This is the new frontier that we are exploring with this research: the as-yet unconquered domain of \emph{interactive} storytelling.

We are defined by the choices we make throughout our lives. These choices form a story that describes our own personal history. If a player can watch somebody's life unfold and witness the choices that they made, and those that were forced upon them, they would have a deep understanding of how they came to be who they are.

This kind of deep understanding is only possible through experiencing a story where the decisions made have real consequences. Traditional fiction, and perhaps computer games, enable this to some extent. But the story is still experienced passively by the consumer in these media. A truly interactive narrative would go deeper, as though you had truly lived as another person.

For this reason, interactive narrative for games is worth exploring: it would
enable a player to truly understand other people's lives and situations, and why they became the people they are.

% Put a bit about what you did to explore the potential of interactive narrative
% for games

\section{Interactive narrative for education}
Stories are a powerful way for humans to understand and remember facts and events. Making these stories interactive could lead the way to even more effective learning methods.

\citet{schank1990tell} argues that stories may be a useful way for humans to better remember a series of events, when compared to simply reciting them as a list of facts. Mentalists and creative students use storytelling mnemonic techniques to help them memorise large lists of difficult-to-remember facts with perfect recall.

Training simulations make the use of stories taking place in immersive environments to help trainees to master new routines and techniques. These stories always follow a fixed pattern, however, unlike the mnemonic stories that people create to memorise information, which are highly personalised to the practitioner.

A narrative that dynamically reacts to the decisions of the audience could better enable them to remember information when compared with a static, unpersonalised narrative.

With this in mind, it is worth constructing an interactive narrative system and applying it to both entertainment and learning contexts. Some evaluation still needs to be carried out in order to determine how much more effective (if at all) interactive narrative could be for education, so this is proposed as part of this report.

Preliminary work has been done in collaboration with the Bishop's Palace in Wells to create an exhibition incorporating interactive narrative. Using a combination of Semantic Web data formats and an event calculus-based reasoner, the system will be able to infer future actions for recommendation to visitors. Description of the work done so far is in section \ref{}.

Both the game-based and education-based projects have a common thread: reasoning
over temporal data in order to predict future events, or constrain possible
future actions to conform to a narrative domain. Section \ref{}
describes the direction the research has taken, and how the use of techniques such as hierarchical institutions and the creation of an ontology for narrative might develop these ideas further.

\section{Outline}
% Outline of sections goes here


\chapter{Literature Review}
\label{cha:literature-review}

\section{Narratology}
\label{sec:narratology}

\subsection{Types of Narrative}
% Look at Cybertext, etc, and try to explain how best to divide different types
% of story

\subsection{Structuralist Formalisms of Narrative}
% Propp, etc

\subsection{Describing Stories with Logic}
% Laure-Ryan did a bit of this

\subsection{Other Types of ``Story Component''}

\section{Implementations of Experimental Narrative}
\label{sec:narratology}

\subsection{Story Generation}
% TaleSpin, etc

\subsubsection{Generative Grammar}

\subsubsection{Author Modelling}
% This would also use planners

% But this is limited in interactivity, so in order to have characters we can
% interact with, we must use...

\subsection{Intelligent Agents as Characters}
% Facade, etc

\subsubsection{Character Modelling}

\subsubsection{Characters with Emotional Models}


% But characters in a story need to follow some kind of underlying plot
% mechanism, so...

\subsection{Governing Narrative in a Multi-Agent System}
\subsubsection{Planner-based Systems}
% Riedl, Young, etc

\chapter{Institutions as Story Worlds}
\label{sec:institutions}

\section{Why Use Institutions?}
% Write about character freedom, regimentation vs regulation. Give story
% examples vs using a planner

\chapter{Tropes as Story Components}
\label{sec:tropes}
\section{Tropes: a ``Folksonomy'' of Story Components}
% Describe TVtropes

\section{Why Use Tropes?}
% Write about ability to abstract, give story examples vs Propp

\chapter{Describing Tropes as Institutions}
\label{sec:tropes-as-institutions}

\chapter{Future Work}
\label{sec:future}


\bibliographystyle{apalike}
\bibliography{thesis}

\end{document}
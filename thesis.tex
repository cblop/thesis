\documentclass[11pt]{report}
\usepackage{baththesis}
\usepackage{graphicx}
\usepackage{caption}
\usepackage{amsmath}
\usepackage{subcaption}
\usepackage{verbatim}
\usepackage[inline]{enumitem}
%\usepackage{amsfonts}
\usepackage{url}
\usepackage{natbib}
\usepackage[hmargin=3cm,vmargin=2cm]{geometry}
%\usepackage{mathpazo}
%\usepackage{eulervm}
\usepackage[usenames,dvipsnames,svgnames,table]{xcolor}
\usepackage[]{todonotes}
\usepackage{tikz}
\usetikzlibrary{arrows}
\usepackage[]{todonotes}

\def\mnote#1{\todo[color=Goldenrod,size=\scriptsize]{Matt: #1}}
\def\jnote#1{\todo[color=CornflowerBlue,size=\scriptsize]{Julian: #1}}
\def\snote#1{\todo[color=WildStrawberry,size=\scriptsize]{Steve: #1}}
\definecolor{light-gray}{gray}{0.95}

\usepackage{listings}
\lstset{ %
   language=prolog,
%  frame=l,                     % adds a frame around the code
   basicstyle=\footnotesize\ttfamily,  % use courier
   breaklines=true,
   xleftmargin=0.5em,
   aboveskip=0.5em,
   belowskip=0.5em,
%  belowcaptionskip=5em,
   numbers=left,
   backgroundcolor=\color{light-gray},
   frame=single,
   framerule=0pt
}

\setlength{\jot}{0pt}
\def\mylabel#1{\tikz[remember picture]\node(#1){};}
\def\myref#1#2#3{\begin{tikzpicture}[remember picture]
\node[draw,rounded corners] (#2){\begin{minipage}{\textwidth}\raggedright#3\end{minipage}};
\draw[overlay,-triangle 45,thick,gray](#2.west)--(#1.west);
\end{tikzpicture}}

\title{Building Abstractable Story Components with Institutions and Tropes}
\author{Matt Thompson}
\degree{EngD Digital Media}
\department{Department of Computer Science}
\degreemonthyear{October 2016}
\norestrictions


\begin{document}
\maketitle
\clearpage
\tableofcontents
\clearpage

\chapter{Introduction}
\label{cha:introduction}
This report summarises the research carried out as part of the EngD Digital Media programme with the University of Bath and Sysemia Ltd.

The purpose of this report is to explain the academic research conducted during the course and its applicability to the needs of Sysemia Ltd. The main focus is to explore the state of the art in interactive and generative narrative and also to apply it in a real-world exhibition context.

Interactive narratives have the potential to transform two major fields: games
and training simulations. These are the applications for which the use of
interactive narrative can benefit Sysemia. The main part of the research examines the state of the art in interactive narrative primarily in computer games research, but then adapts techniques from that field for the purpose of interactive exhibitions for education. By implementing research ideas into `toy' game projects (such as an interactive Punch and Judy show in section \ref{}), I then take components from these projects and implement them into practical exhibition-based projects for Sysemia (such as the Every Object Tells a Story project described in section \ref{}).

\section{Interactive narrative for games}
Computer games are a new medium for artistic expression. The element of interactivity, combined with visual art, music and storytelling, allows the creation of ever more fantastic worlds. This combination of previous art forms isn't enough, though. Game creators are now experimenting with ways to make a narrative itself interactive. Imagine playing through a version of Romeo and Juliet where there is a possibility that the characters could escape their fates. Or playing a detective in a game where the story changes depending on how quickly you can piece together clues.

This is the new frontier that we are exploring with this research: the as-yet unconquered domain of \emph{interactive} storytelling.

We are defined by the choices we make throughout our lives. These choices form a story that describes our own personal history. If a player can watch somebody's life unfold and witness the choices that they made, and those that were forced upon them, they would have a deep understanding of how they came to be who they are.

This kind of deep understanding is only possible through experiencing a story where the decisions made have real consequences. Traditional fiction, and perhaps computer games, enable this to some extent. But the story is still experienced passively by the consumer in these media. A truly interactive narrative would go deeper, as though you had truly lived as another person.

For this reason, interactive narrative for games is worth exploring: it would
enable a player to truly understand other people's lives and situations, and why they became the people they are.

% Put a bit about what you did to explore the potential of interactive narrative
% for games

\section{Interactive narrative for education}
Stories are a powerful way for humans to understand and remember facts and events. Making these stories interactive could lead the way to even more effective learning methods.

\citet{schank1990tell} argues that stories may be a useful way for humans to better remember a series of events, when compared to simply reciting them as a list of facts. Mentalists and creative students use storytelling mnemonic techniques to help them memorise large lists of difficult-to-remember facts with perfect recall.

Training simulations make the use of stories taking place in immersive environments to help trainees to master new routines and techniques. These stories always follow a fixed pattern, however, unlike the mnemonic stories that people create to memorise information, which are highly personalised to the practitioner.

A narrative that dynamically reacts to the decisions of the audience could better enable them to remember information when compared with a static, unpersonalised narrative.

With this in mind, it is worth constructing an interactive narrative system and applying it to both entertainment and learning contexts. Some evaluation still needs to be carried out in order to determine how much more effective (if at all) interactive narrative could be for education, so this is proposed as part of this report.

Preliminary work has been done in collaboration with the Bishop's Palace in Wells to create an exhibition incorporating interactive narrative. Using a combination of Semantic Web data formats and an event calculus-based reasoner, the system will be able to infer future actions for recommendation to visitors. Description of the work done so far is in section \ref{}.

Both the game-based and education-based projects have a common thread: reasoning
over temporal data in order to predict future events, or constrain possible
future actions to conform to a narrative domain. Section \ref{}
describes the direction the research has taken, and how the use of techniques such as hierarchical institutions and the creation of an ontology for narrative might develop these ideas further.

\section{Outline}
% Outline of sections goes here
This thesis begins with a review of the literature in Section~\ref{cha:literature-review}. This review covers two main fields of research: narrative research from social science (Section~\ref{sec:narratology}), and interactive narrative research from computer science (Section~\ref{sec:implementations}).
The ``Institutions as Story Worlds'' chapter (\ref{cha:institutions}) describes the theory and research behind instituitions, and discusses the merit of their use for describing story, comparing their use against other logics and planner-based systems.
The ``Tropes as Story Components'' chapter (\ref{cha:tropes}), discusses the use
of \emph{tropes} as story components, and why they improve upon existing narrative formalisms.
Chapter~\ref{cha:tropes-and-institutions} describes how to use institutions to define tropes, and how to combine them to describe the story world for an interactive narrative. The section also describes the use of this technique to build a number of different narratives, contrasting the approach with using other techniques such as planners to build interactive narratives.
The final chapter looks back at the research done, evaluating its potential impact and discussing possible future work. 

\chapter{Literature Review}
\label{cha:literature-review}
This research covers a large number of fields of study, therefore an extensive
literature review covering these fields is needed. This section starts with a
look at the field of \emph{narratology}, or narrative theory, to gain some
insights into the themes and components that make up stories. Looking at
different formalisms that have been created for narrative and which themes and
motifs recur in stories should better inform the creation of techniques with
which to generate stories.

Additionally, computers have the potential to introduce a new type of storytelling. Though computer games offer increasingly immersive interactive worlds, the form of narrative they use remains much the same: linear. A player may be able to interact with elements of a game, but the story itself remains unchanged.
Chris Crawford refers to a new possible form of narrative as ``Interactive Storytelling'' \citep{crawford2012chris}. Though researchers and game designers have made great strides towards realising this new artform, little has been produced to capture the public imagination.

% why are you looking at classic narratology?
In order to better inform any implementation of interactive narrative, this
review begins with an examination of the field of classic narrative theory. Following from this is a look at the emerging research in interactive and generative narrative and their implementations.
As part of the examination of implementations of interactive narrative, this review especially focuses on agent-based systems. The section concludes with an overview of emotional models that can be used to model distinct characters using agents.

\section{Narratology}
\label{sec:narratology}
Narratology is a deep field with many sub-fields. This review examines the parts of it that might best inform the modelling of narrative by computers, as well as the construction of interactive narrative.

The first part of the overview of narratology examines research into
categorising different types of narrative, both traditional and experimental.
This draws from classic narratological texts, as well as work done on
``cybertext'' and experimental narrative in the interactive age. This
examination of recent research into non-linear narratives better informs how to better construct interactive narratives for games or simulations.

Structuralist formalisms of narrative attempt to explain how stories work by
dividing them into commonly occuring themes and motifs. This is a natural fit to
the modelling of narratives by computer, especially if using an ontology. This
overview of narratology starts with structuralism for this reason.

After the overview of the structuralists follows a section on the use of formal
logic for narrative modelling. The section ends with descriptions of other types
of story components, taxonomies and ontologies used in the literature.

\subsection{Types of Narrative}
% Look at Cybertext, etc, and try to explain how best to divide different types
% of story

\subsection{Structuralist Formalisms of Narrative}
% Propp, etc
Attempts to organise recurring themes, roles and motifs of narrative go back at least a century. The Aarne-Thompson tale-type index \citep{aarne1987types}, first published in 1910 and later refined by Stith Thompson in 1928 and 1961, is well known amongst folklorists as a classification and analysis method for traditional folktales and myths. Aarne-Thompson's index is a taxonomy of tale themes, arranging tales into categories such as \emph{animal tales} and \emph{jokes and anecdotes}, and then sub-categories (\emph{tales of magic} and \emph{numskull [sic] stories} being two examples). This taxonomy is only two levels deep however, and only serves as a useful way to categorise individual stories or tales. In order to break down and analyse components of tales, we must dig deeper.

In \emph{Structural Anthropology}, Claude L\'{e}vi-Strauss seeks to discover why myths and legends are so similar across cultures and history \citep{levi2008structural}. He concludes that there are global laws that govern the way in which people create stories, therefore these laws can be modelled as a set of rules for describing myths.

His theory is that myths describe opposing forces which are resolved through mediation. The example he gives in \emph{Structural Anthropology} describes how Native American legends often contain `trickster' characters in the form of ravens or coyotes. As scavenging animals, these tricksters symbolically act as mediators between life and death.

Like much of early narrative theory, there is no rigorous evaluation of L\'{e}vi-Strauss' ideas, leaving them feeling a little too opinionated and arbitrary. While interesting, L\'{e}vi-Strauss' ideas bring us no closer to developing a formal model of narrative structure. For that, we must go even further back in time, and turn to Vladimir Propp.

\subsubsection{Propp's Morphology of the Folktale}
Propp's seminal work ``The Morphology of the Folktale'' \citep{propp1968morphology}, though first published in 1928, is still a widely-used formalism for researchers and game designers looking to generate narratives procedurally. Propp identifies recurring characters and motifs in Russian folklore, distilling them down to a concise set of rules with which to describe stories.

In this formalism, characters have \emph{roles}, such as \emph{hero}, \emph{villain}, \emph{dispatcher}, \emph{false hero}, and more. Characters performing a certain role are able to perform a subset of \emph{story functions}, which are actions that make the narrative progress. For example, the \emph{dispatcher} might send the \emph{hero} on a quest, or the \emph{victim} may issue an \emph{interdiction} to the \emph{villain}, which is then \emph{violated}.

Propp defines a total of 31 distinct story functions, each of which is given a number and symbol in order to create a succinct way of describing entire stories. Examples of such functions are:

\begin{itemize}
  \item One of the members of a family absents himself from home: \emph{absentation}.
  \item An interdiction is addressed to the hero: \emph{interdiction}.
  \item The victim submits to deception and thereby unwittingly helps his enemy: \emph{complicity}.
  \item The villain causes harm or injury to a member of the family: \emph{villainy}.
\end{itemize}

Each of these functions can vary to some degree. For example, the \emph{villainy} function can be realised as one of 19 distinct forms of villainous deed, including \emph{the villain abducts a person}, \emph{the villain seizes the daylight}, and \emph{the villain makes a threat of cannibalism}.

These functions are enacted by characters following certain roles. Each role (or \emph{dramatis personae} in Propp's definition) has a \emph{sphere of action} consisting of the functions that they are able to perform at any point in the story. Propp defines seven roles that have distict spheres of action: \emph{villain}, \emph{donor}, \emph{helper}, \emph{princess}, \emph{dispatcher}, \emph{hero}, and \emph{false hero}.

Though Propp defines each \emph{dramatis personae} as being distinct (characters can only play one role at a time), it is simple to extend the idea to allow for overlapping roles. For example, a victim could also be a donor, so the set of functions they can perform would be the union of both the victim and donors' permitted function sets. Propp does not explore this possibility in \emph{The Morphology of the Folktale}, however.

\begin{figure}[!t]
\centerline{\includegraphics[height=0.4in]{propp1.png}}
\caption{One Propp function following another}\label{fig:propp1}
\end{figure}

\begin{figure}[!t]
\centerline{\includegraphics[height=0.6in]{propp2.png}}
\caption{Multiple simultaneous functions}\label{fig:propp2}
\end{figure}

In a typical story, one story function will follow another as the tale progresses in a sequential series of cause and effect (figure~\ref{fig:propp1}). However, Propp's formalism also allows for simultaneous story functions to be occuring at once (figure~\ref{fig:propp2}).

% What are the shortcomings of Propp? (i.e. lack of abstractability, etc)

\subsection{Describing Stories with Logic}
% Laure-Ryan did a bit of this

\subsection{Other Types of ``Story Component''}

\section{Implementations of Experimental Narrative}
\label{sec:implementations}
% I've plenty of material, but it really needs reworking and extending

\subsection{Story Generation}
% TaleSpin, etc

\subsubsection{Generative Grammar}

\subsubsection{Author Modelling}
% This would also use planners

% But this is limited in interactivity, so in order to have characters we can
% interact with, we must use...

\subsection{Intelligent Agents as Characters}
% Write an introduction


\paragraph{The OZ project}
Carnegie Mellon University's OZ project \citep{mateas1999oz} is one of the first major research efforts towards creating interactive drama using agents as characters. A dramatic structure is given to the narrative by means of a \emph{drama manager}, which is able to see all of the actions occurring in the storyworld and can change anything in order to create a better experience for the user.

Ideas from the OZ project were later developed into what would become Fa\c{c}ade.
% Facade, etc
\paragraph{Fa\c{c}ade}
Mateas and Stern's \emph{Fa\c{c}ade} has players interact with the characters of the story through natural language. In this game, the player attends the party of a young couple (Grace and Trip) celebrating their wedding anniversary. As the course of events unfold however, the player learns that all is not as happy as it seems.

The player interacts with the characters by typing in natural language sentences, to which Grace and Trip respond. Though the characters are implemented through agents, the story is controlled using a drama manager. In all, their system consists of using NLP, a novel character authoring language and a novel drama manager to create an interactive narrative.

Several custom-designed languages were used to create the game, including a language called `A Behaviour Language' (ABL) for the agents and a special language for the sequencing of the beats. ABL represents situations as character goals, maintaining a tree of all the active goals and behaviours that are happening at any time.

In Fa\c{c}ade, the smallest unit of narrative action is called a \emph{story beat}, taken from McKee's book on authorial style for screenwriters \citep{mckee1997substance}. The simulation constantly monitors what the user is doing and how it may lead from the current story beat to another. Story beats have preconditions and effects on the state of the narrative, so it is the drama manager's job to work out when it makes sense to initiate a certain beat.

`Beats' have a very fine granularity, with 200 or so updating every minute of the simulation. They consist of a set of ABL behaviours, which advance the narrative yet still allow interaction to change to other beats. Only one beat can be active at a time.

A beat can have 5 types of goal:

\begin{enumerate}
  \item transition-in: characters express their intentions
  \item body: a dramatic question/situation is posed to the player
  \item local/global mix-in: react to the player before end of the beat
  \item wait-with-timeout: wait for the player's reaction
  \item transition-out: final reaction to the player's action in the beat
\end{enumerate}

A beat goal is a series of steps for an agent to perform, which can be:

\begin{itemize}
  \item staging (where to walk to, face)
  \item dialogue to speak
  \item where and how to gaze
  \item arm gestures to perform
  \item facial expression to perform
  \item head and face gestures to perform
  \item small arm and head emphasis motions triggered by dialogue (head nods, hand flourishes)
\end{itemize}

As an example, there is a behaviour called ``Fix\_Drinks'', which specifies a sequence of agent behaviours where the characters Grace and Trip have an argument while Trip asks the player what they would like to drink. If the player decides not to go along with the beat (in this case, by not choosing a drink), then the beat will be aborted and replaced with another.

Fa\c{c}ade has become popular as a game outside of academia, with playthroughs of the game reaching millions of views on Youtube. This shows the promise of interactive narrative as being a unique and engaging new form of entertainment. Unfortunately, no other implementation of interactive narrative seems to have captured the public imagination since the release of Fa\c{c}ade.

Fa\c{c}ade's popularity seems to reinforce Crawford's assertion (section \ref{sec:media}) that interactive narratives must be social in nature. The gameplay comes entirely from the conversations and interactions between Grace, Trip and the player. Much of the excitement comes from the social consequences of certain conversation paths or actions. By modelling characters as agents, Mateas et al have created a truly interactive experience. However, by also using a drama manager to manage the agents, they have used these agents to tell a story.

How might these agents be made more convincing? Outside of writing rules for their behaviour consisting of character goals and beliefs, how might an author create truly unique and idiosyncratic characters? To address the question, I next examine different types of emotional models in psychology, and how each might be used to model characters as agents.

\subsubsection{Character Modelling}

\subsubsection{Characters with Emotional Models}
% Intro: not done very much?

\subsubsection{Emotional models}\label{sec:emotional-models}
% How is this useful for narrative?
Usually it would seem odd to want to model emotion as part of a computational process. Emotion is such a seemingly irrational set of behaviours that they are easy to dismiss as `human imperfections'. However, as \citet{marsella2014} observe, emotions may have a useful role to play in communication, so long as they are displayed at appropriate times.

For example, anger prepares the human body to fight by increasing the manufacture of adrenaline. Fear similarly triggers the `fight or flight' response, alerting the senses for danger and preparing the body to react.

In order to model human emotions using agents, we must first find a suitable psychological model to use. Marsella et al describe three main types of emotional model:

\begin{enumerate}
 \item \textbf{Discrete} emotional models, which claim that humans have a set of innate, pre-defined emotional states which people may enter and leave.
 \item \textbf{Dimensional} models of emotion, describing the spectrum of emotions as being points somewhere in continuous space. Implementations typically use two or three dimensions for simplicity.
 \item \textbf{Appraisal} theories of emotion take an agent's mental processes into account. Their emotional state is derived from whether or not their goals have been achieved, and what effects current events are having on their circumstances, for example.
\end{enumerate}

% Give examples of concrete models for each type.
\subsubsection{`Basic' emotions}
Ekman first made a case for discrete, biologically-determined emotions, based on evidence from research into facial expressions \citep{ekman1992argument}. He describes emotions as being \emph{basic}, in two senses of the word: \emph{i.} that there are a number of distinct emotions, each with its own different characteristics, and \emph{ii.} that these emotions were developed through evolution for specific functions.

Ekman argues that these evolved emotions share nine characteristics:

\begin{enumerate}
  \item Distinctive universal signals
  \item Presence in other primates
  \item Distinctive physiology
  \item Distinctive universals in antecedent events
  \item Coherence among emotional response
  \item Quick onset
  \item Brief duration
  \item Automatic appraisal
  \item Unbidden occurrence
\end{enumerate}

These characteristics are shared by all of the `basic' emotions as observed in humans and primates.

Discrete models of emotion suggest that there is a neural basis for emotion. For example, Armony et al describe how the amygdala in the brain is responsible for conditioned fear responses  and create a neural network to model it \citep{armony1997computational}.

Using a discrete model of emotion for agent-based characters would be relatively simple. Each basic emotion could have its own distinct set of behaviours as postconditions, and triggering circumstances as preconditions.

However, a more fluid approach could be useful when modelling emotions with agents. It would be impossible to say that an agent is \emph{angry and approaching furious} using a discrete theory of emotion. Nuanced levels of emotion and even combinations of several emotions add an extra level of texture to a character. Dimensional and appraisal theories of emotion address this challenge.

\subsubsection{Russell's circumplex model of emotion}\label{sec:circumplex}
\begin{figure}[!t]
\centerline{\includegraphics[height=3in]{circumplex.png}}
\caption{Russell's circumplex model of emotion} \label{fig:circumplex}
\end{figure}

Russell's circumplex model of emotion is a well-known dimensional model \citep{russell1980circumplex}. In this case, the dimensional variables are \emph{valence} (how agreeable or otherwise a situation is to an agent) and \emph{arousal} (how excited an agent is).

Russell's original paper proposes a model similar to that shown in figure \ref{fig:circumplex}, where the $x$ axis is a person's valence level and the $y$ axis is their arousal level. He argues that the full range of human emotions lie as points along these axes. Eight such examples are shown in fig. \ref{fig:circumplex}.

This model is very easy to adapt to human-like agents. \citet{ahn2012nvc} adapt this model by adding a third dimension, dominance, to create conversational agents in a 3D environment. This `dominance' dimension was first proposed in Mehrabian and Russell's original work \citep{mehrabian1974approach}, but later removed due to being perceived as the consequences of the \emph{effects\/} of emotion \citep{russell1980circumplex}, rather than being a component of emotion itself. Like Ahn et al, I found it useful to add the dominance-submission dimension, and so left it in my emotional model. This is the approach I take in creating my Punch and Judy simulation, and so it is described in more detail in section \ref{sec:emotion}.

\subsubsection{Appraisal theory}
Appraisal theories of emotion lend well to simulation with agents, due to their taking a person's beliefs, desires and intentions into account with respect to external events. Emotions arise when an event occurs and a person internally \emph{appraises} its consequences with respect to their beliefs, desires and intentions. This fits well with the popular BDI architecture for intelligent agents.

Different methods of appraisal may be used in order to produce emotions. Gratch and Marsella use decision theoretic plans \citep{gratch2004domain}, but other approaches could include reactive plans, Markov-decision processes, or detailed cognitive models.

Though the Punch and Judy simulation described in section \ref{sec:punchjudy} uses a dimensional model of emotion, an appraisal-based model would be worth investigating due to its tight coupling with belief desire intention psychological models used in agents. I describe my intention to explore this area further in section \ref{sec:fappraisal}.


% But characters in a story need to follow some kind of underlying plot
% mechanism, so...

\subsection{Governing Narrative in a Multi-Agent System}
\subsubsection{Planner-based Systems}
% Riedl, Young, etc

\chapter{Institutions as Story Worlds}
\label{cha:institutions}

% Intro about justification/inspiration for using institutions for stories

\section{Describing Stories With Logic}
\subsection{Modal Logic and Kripke Structures}
\subsection{Deontic Logic and Norms}

\section{Norms and Institutions}
\label{sec:norms-and-institutions}

An institution describes a set of `social' norms describing the permitted and obliged behaviour of interacting agents. Noriega's `Fish Market' thesis~\cite{noriega1999agent} describe how an institutional model can be used to regiment the actions of agents in a fish market auction. Several~\cite{artikis2009specifying,fornara2007agent,cardoso2007institutional} extend this idea to build systems where institutions actively regulate the actions of agents, while still allowing them to decide what to do. We build on the work of Cliffe et al.~\cite{cliffe2007specifying} and Lee et al.~\cite{lee2013decoupling} to adapt it for the world of narrative, using an institutional model to describe the story world of Punch and Judy in terms of Propp's story moves and character roles, through which the actors acquire powers and permissions appropriate to the character and the story function in which they are participating.

Institutional models use concepts from deontic logic to provide obligations and permissions that act on interacting agents in an environment. By combining this approach with Propp's concepts of \emph{roles} and \emph{story moves}, we describe a Propp-style formalism of Punch and Judy in terms of what agents are \emph{obliged} and \emph{permitted} to do at certain points in the story.

For example, in one Punch and Judy scene, a policeman enters the stage and attempts to apprehend Punch. According to the rules of the Punch and Judy world, Punch has an obligation to kill the policeman by the end of the scene (as this is what the audience expects to happen, having seen other Punch and Judy shows). The policeman has an obligation to try his best to catch Punch. Both agents have permission to be on the stage during the scene. The policeman only has permission to chase Punch if he can see him (Punch is obliged to hide from him at the start of the scene).

The permissions an agent has, on the one hand, constrain the choices of actions available to them at any given moment. Obligations, on the other hand, affect the goals of an agent. Whether or not an agent actively tries to fulfil an obligation depends on their emotional state.

\subsection{Institution example}
To illustrate the application of institutional modelling, we here continue the `sausages and crocodile' scene example from section~\ref{sec:pjexample}, taking the Propp story functions and describing them in an institutional model.  We define our institution in terms of \emph{fluents}, \emph{events}, \emph{powers}, \emph{permissions} and \emph{obligations}, following~\cite{cliffe2007specifying}, to which the interested reader is referred for the full details of the formal model, including the generate ($\cal G$) and consequence ($\cal C$) relations, which are only described here in sufficient depth for the model being presented.

\subsubsection{Fluents}
These are properties that may or may not hold true at some instant in time, and that change over the course of time. \emph{Institutional events} are able to \emph{initiate} or \emph{terminate} fluents at points in time. A fluent could describe whether a character is currently on stage, the scene of the story that is currently being acted out, or whether or not the character is happy at that moment in time.
Domain fluents ($\mathcal{D}$) describe domain-specific properties that can hold at a certain point in time. In the Punch and Judy domain, these can be whether or not an agent is on stage, or their role in the narrative: % (equation~\ref{eq:domain}).
\begin{align*}
   \mathcal{D} &= \left\{\mathtt{onstage, hero, villain, victim, donor, item}\right\} %\label{eq:domain}
\end{align*}

Institutional fluents consist of (institutional) \emph{powers}, \emph{permissions} and \emph{obligations}.
% check your facts on this one
An \textbf{institutional power} ($\mathcal{W}$) describes whether or not an external event has the authority to generate a meaningful institutional event. Taking an example from Propp's formalism, an \emph{absentation\/} event can only be generated by an external event brought about by a \emph{donor\/} character (such as their leaving the stage). Therefore, any characters other than the donor character would not have the institutional power to generate an \emph{absentation\/} institutional event when they leave the stage.
The possible empowerments (institutional events) from Propp used in Punch and Judy are:
\begin{align*}
  \mathcal{W} =&\left\{\mathtt{pow(introduction), pow(interdiction), pow(give),}\right.\\ %\nonumber\\
               &\left. {} \mathtt{pow(absentation), pow(violation), pow(return)}\right\} %\label{eq:power}
\end{align*}

\subsubsection{Permissions} ($\mathcal{P}$) are associated with external actions that agents are permitted to do at a certain instant in time. These can be thought of as the set of \emph{socially permitted\/} actions available to an agent. While it is possible for an agent to perform other actions, societal norms usually discourage them from doing so.
% PJ examples
For example, it would not make sense in the world of Punch and Judy if Punch were to give the sausages to the Policeman. It is always Joey who gives the sausages to Punch. Also, it would be strange if Joey were to do this in the middle of a scene where Punch and Judy are arguing. We make sure agents' actions are governed so as to allow them only a certain subset of permitted actions at any one time. The set of permission fluents is:
\begin{align*}
\mathcal{P} =& \left\{\mathtt{perm(leavestage), perm(enterstage), perm(die), perm(kill),}\right.\nonumber\\
             &\left. {} \mathtt{perm(hit), perm(give), perm(fight)}\right\} %\label{eq:perm}
\end{align*}

\subsubsection{Obligations} ($\mathcal{O}$) are institutional facts that contain actions agents \emph{should} do before a certain deadline. If the action is not performed in time, a \emph{violation event} is triggered, which may result in a penalty being incurred. While an agent may be obliged to perform an action, it is entirely their choice whether or not they actually do so. They must weigh up whether or not pursuing other courses of action is worth accepting the penalty that an unfulfilled obligation brings.

% replace with sausages obligation
Anybody who has seen a Punch and Judy show knows that at some point Joey tells Punch to guard some sausages, before disappearing offstage. Joey's departure is modelled in the institution as the \emph{absentation\/} event. It could also be said that Joey has an obligation to leave the stage as part of the \emph{absentation} event, otherwise the story function is violated. This can be described in the institution as:
\begin{align*}
  \mathcal{O} =& \left\{\text{obl}(\mathtt{leavestage, absentation, viol(absentation)})\right\}%\label{eq:obl}
\end{align*}
The first argument is the external event that must be triggered according to the obligation, the second argument is the institutional deadline event, and the third argument is the violation event which is triggered if the obligation is not fulfilled before the deadline. 

\subsubsection{Events}
% actually 3 kinds, including violation events
Cliffe's model specifies three types of \textbf{event}: \emph{external events} (or `observed events', $\mathcal{E}_{obs}$), \emph{institutional events} ($\mathcal{E}_{instevent}$) and \emph{violation events} ($\mathcal{E}_{viol}$). Examples of each are given in Figure~\ref{fig:events}.
\emph{External events} are observed to happen in the agents' environment, which can \emph{generate} \emph{institutional events} which occur only within the institional model, leading to the \emph{initiation} or \emph{termination} of (domain) fluents, permissions, obligations or institutional powers.
An external event could be an agent leaving the stage, an agent hitting another, or an agent dying. Internal events include narrative events such as scene changes, or the triggering of Propp story functions such as \emph{absentation} or \emph{interdiction} (described in section~\ref{sec:propp}). \emph{Violation} is the name of a Propp story function, and is included as an internal event, although it has no relation to the violation events of an institution.
Violation events occur when an agent has failed to fulfil an obligation before the specified deadline. These can be implemented in the form of a penalty, by decreasing an agent's health, for example.

\begin{figure}[!t]
\begin{align}
  \mathcal{E}_{obs} =& \left\{\mathtt{startshow, leavestage, enterstage, die, give,}\right.\nonumber\\
  &\left. {} \mathtt{harmed, hit, fight, kill, escape}\right\}\label{eq:eobs}\\
  \mathcal{E}_{instevent} =& \left\{\mathtt{introduction, interdiction, receipt, absentation,}\right.\nonumber\\
                         &\left. {} \mathtt{violation, return, struggle, defeat, complicity,}\right.\nonumber\\
                         &\left. {} \mathtt{victory, escape}\right\}\label{eq:einst}\\
  \mathcal{E}_{viol} =& \left\{\mathtt{viol(introduction), viol(interdiction), viol(receipt),}\right.\nonumber\\
 &\left. {} \mathtt{viol(absentation), viol(violation), viol(return),}\right.\nonumber\\
 &\left. {} \mathtt{viol(struggle), viol(defeat), viol(complicity)}\right.\nonumber\\
 &\left. {} \mathtt{viol(victory), viol(escape)}\right\}\label{eq:viol}
\end{align}
\caption{External, institutional and violation events for Punch and Judy} \label{fig:events}
\end{figure}

% internal and external

\subsubsection{Event Generation and Consequences}
An \textbf{event generation} function, $\mathcal{G}$, describes how events
($\mathcal{E}$, usually external, but can also be internal) %\mnote{Added $\mathcal{E}$ explanation here}
can generate other (usually institutional) events, conditional upon the current institutional state ($\cal X$). This is the counts-as relation.  For example, if an agent leaves the stage while the \emph{interdiction} event holds, they trigger the \emph{leavestage} event. This combination generates the \emph{absentation} institutional event (rule~\ref{eq:absentation}). Further examples appear in figure~\ref{fig:gen}.

Event generation functions follow a $\langle \mathtt{preconditions} \rangle \rightarrow \{\mathtt{postconditions}\}$ format. The preconditions consist of a set of fluents that hold at that time, along with an event to have occurred. The postconditions are the events that are generated. The generation functions are used to generate internal, institutional events from external events.

Consider the Punch and Judy scenario described in section~\ref{sec:pjexample}. There are seven institutional events (story functions) that occur during this scene: \emph{interdiction}, \emph{complicity}, \emph{receipt} (from Propp's \emph{receipt of a magical agent}) \emph{absentation}, \emph{violation}, \emph{struggle}, \emph{return}.
These institutional events are all generated by external events. The \emph{interdiction} is generated when Joey tells Punch to protect the sausages. Punch agreeing amounts to \emph{complicity}. Joey \emph{gives} punch the sausages (\emph{receipt}), then leaves the stage (\emph{absentation}). The crocodile eating the sausages is a \emph{violation} of Punch's oath, the agents fight (\emph{struggle}), then Joey enters the stage again (\emph{return}).

It is desirable that these story functions occur in this sequence in order for a satisfying narrative to emerge. Agents may decide to perform actions that diverge from this set of events, but the institution is guiding them towards the most fitting outcome for a \emph{Punch and Judy} world. For this reason, a currently active story function can be the precondition for event generation. For example, the \emph{receipt} event may only be triggered if an agent externally performs a \emph{give} action \textbf{and} if the \emph{complicity} event currently holds (rule~\ref{eq:receipt}).
Examples of event generation function for this scenario, complete with preconditions, are listed in rules~\ref{eq:gfirst}--\ref{eq:glast} (Figure~\ref{fig:gen}).

\begin{figure}[!t]
\abovedisplayskip=0pt
\abovedisplayshortskip=0pt
$\mathcal{G(X, E)}:\left\{\mbox{%
{\begin{minipage}[c]{0.85\textwidth}
% \vspace{-2.1em}\begin{align}
\begin{align}
\langle \emptyset,\mathit{tellprotect}\mathtt{(donor, villain, item)} \rangle%\nonumber\\
             %         &\qquad\qquad\qquad
& \rightarrow \left\{\mathit{interdiction}\right\}\label{eq:gfirst}\\
                      \langle \{\mathit{interdiction}\}, \mathit{agree}\mathtt{(villain)}) \rangle %\nonumber\\
            %          &\qquad\qquad\qquad
& \rightarrow \left\{\mathit{complicity}\right\}\\
                      \langle \emptyset, \mathit{give}\mathtt{(donor, villain, item)}) \rangle %\nonumber\\
        %              &\qquad\qquad\qquad
& \rightarrow \left\{\mathit{receipt}\right\}\label{eq:receipt}\\
                      \langle \{\mathit{interdiction}\}, \mathit{leavestage}(\mathtt{donor}) \rangle %\nonumber\\
              %        &\qquad\qquad\qquad
& \rightarrow \left\{\mathit{absentation}\right\}\label{eq:absentation}\\
                      \langle \{\mathit{interdiction}\}, \mathit{harmed}(\mathtt{item}) \rangle %\nonumber\\
         %             &\qquad\qquad\qquad
& \rightarrow \left\{\mathit{violation}\right\}\\
                      \langle \{\mathit{interdiction, absentation}\},
                      \mathit{enterstage}(\mathtt{donor}) \rangle %\nonumber\\
              %        &\qquad\qquad\qquad
& \rightarrow \left\{\mathit{return}\right\}\\
                      \langle \emptyset, \mathit{hit}(\mathtt{donor, villain}) \rangle %\nonumber\\
%                      &\qquad\qquad\qquad
& \rightarrow \left\{\mathit{struggle}\right\}\label{eq:glast}
\end{align}
\end{minipage}}}\right.$
\caption{Event generation in the sausage scene} \label{fig:gen}
\end{figure}

\textbf{Consequences} consist of fluents, permissions and obligations that are \emph{initiated} ($\mathcal{C}^{\uparrow}$) or \emph{terminated} ($\mathcal{C}^{\downarrow}$) by institutional events. For example, the institutional event \emph{receipt} initiates the donor agent's permission to leave the stage, triggering the \emph{absentation} event (rule~\ref{eq:initgive}). When the \emph{interdiction} event is currently active and a \emph{violation} event occurs, the interdiction event is terminated (\ref{eq:interm}). Rules~\ref{eq:cfirst}--\ref{eq:clast} in Figures~\ref{fig:init} and~\ref{fig:term} describe the initiation and termination of fluents in the Punch and Judy sausages scene detailed in section~\ref{sec:pjexample}.

\begin{figure}[!t]
\abovedisplayskip=0pt
\abovedisplayshortskip=0pt
$\mathcal{C^{\uparrow}(X, E)}:\left\{\mbox{%
\begin{minipage}[c]{0.85\textwidth}
\begin{align}
    \langle \emptyset, \mathtt{interdiction} \rangle %\nonumber\\
                                 % &\qquad\qquad
&\rightarrow \{\text{active}(\mathtt{interdiction}), \nonumber\\&\qquad\text{perm}(\mathtt{give(donor, villain, item)})\}\label{eq:cfirst}\\
                                 \langle \emptyset, \mathtt{receipt} \rangle % \nonumber\\
                                 % &\qquad\qquad
&\rightarrow \{\text{perm}(\mathtt{leavestage(donor)})\}\label{eq:initgive}\\
                                 \langle\{\mathit{active(absentation)}\}, \nonumber\\\mathtt{enterstage(villain)} \rangle %\nonumber\\
                                 %&\qquad\qquad
&\rightarrow \{\text{obl}(\mathtt{eat(villain, sausages),} \nonumber\\&\qquad\qquad\mathtt{return, viol(violation)})\}\label{eq:obl1}\\
                                 \langle\{\mathit{active(interdiction)}\}, \nonumber\\\mathtt{leavestage(donor)} \rangle % \nonumber\\
                                 % &\qquad\qquad
&\rightarrow \{\text{obl}(\mathtt{enterstage(donor),}\nonumber\\&\qquad\qquad\mathtt{eat(villain, sausages),}\nonumber\\&\qquad\qquad\mathtt{viol(return)})\}\label{eq:obl2}\\
                                 \{\mathit{active(interdiction)}\},\nonumber\\ \mathtt{violation} \rangle %\nonumber\\
                                 % &\qquad\qquad
&\rightarrow \{\text{perm}(\mathtt{enterstage(dispatcher)})\}\\
                                 \langle\{\mathit{active(absentation),}\nonumber\\\mathit{active(violation)}\},\nonumber\\ \mathtt{return} \rangle %\nonumber\\
                                 %&\qquad\qquad
&\rightarrow \{\text{perm}(\mathtt{hit(donor, villain)})\}
\end{align}
\end{minipage}}\right.$
\caption{Fluent initiation in the sausage scene} \label{fig:init}
\medskip
\abovedisplayskip=0pt
\abovedisplayshortskip=0pt
$\mathcal{C^{\downarrow} (X, E)}:\left\{\mbox{%
\begin{minipage}[c]{0.85\textwidth}
\begin{align}
\langle \emptyset, \mathtt{interdiction} \rangle %\nonumber\\
                                   %&\qquad\qquad
&\rightarrow \{\text{perm}(\mathtt{give(donor, villain, item)})\}\\
                                   \langle \{\mathit{active(interdiction)}\},\nonumber\\ \mathtt{absentation} \rangle %\nonumber\\
                                   %&\qquad\qquad
&\rightarrow \{\text{perm}(\mathtt{leavestage(donor)})\}\\
                                   \langle \{\mathit{active(interdiction)}\},\nonumber\\ \mathtt{violation} \rangle %\nonumber\\
                                   %&\qquad\qquad
&\rightarrow \{\mathit{active(interdiction)}\}\label{eq:interm}\\
                                   \langle \{\mathit{active(absentation),}\nonumber\\\mathit{ active(violation)}\},\nonumber\\ \mathtt{return} \rangle %\nonumber\\
                                   %&\qquad\qquad
&\rightarrow \{\mathit{active(absentation)}\}\label{eq:clast}
\end{align}
\end{minipage}}\right.$
\caption{Fluent termination in the sausage scene} \label{fig:term}
\end{figure}%\mnote{Added \emph{active(interdiction)} to top of fig. \ref{fig:init}}

\section{Why Use Institutions for Interactive Narrative?}
\label{sec:why-use-institutions}
% Write about character freedom, regimentation vs regulation. Give story
% examples vs using a planner

\chapter{Tropes as Story Components}
\label{cha:tropes}
\section{Tropes: a ``Folksonomy'' of Story Components}
% Describe TVtropes

\section{Why Use Tropes?}
% Write about ability to abstract, give story examples vs Propp

\chapter{Describing Story Worlds with Tropes and Institutions}
\label{cha:tropes-and-institutions}

\chapter{Future Work}
\label{cha:future}


\bibliographystyle{apalike}
\bibliography{thesis}

\end{document}
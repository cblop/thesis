\chapter{Tropes as Story Components}
\label{cha:tropes}
As previously described in chapter~\ref{cha:introduction}, tropes are
patterns that appear throughout various different media. Once one is familiar
with a trope, it becomes easy to identify its use in any story. Take, for
example, the \emph{Hero's Journey} trope first described in
chapter~\ref{cha:introduction}. It is a template which is repeated so often in
many different media, stories and contexts that it is instantly recognised even
by those that are completely unfamiliar with the concept of tropes.

% Bit about difference between tropes and cliches

In this section we examine the concept of a ``trope'', deconstructing examples
to demonstrate widely-recognised trope patterns, and exploring tropes that
operate at different levels of abstraction within a story. At the end of the
section we identify a formal definition of a trope, and how it fits within the
wider context of a story.

% TODO NUMBER
% TODO list of examples where tropes appear from jurisin paper
% TODO TVtropes screenshot figure
% TODO describe the periodic trope groups
\section{Tropes: a ``Folksonomy'' of Story Components}
The existence of a website called ``TV Tropes''~\cite{tvtropes} makes the discovery of example
tropes very simple. Tv Tropes is a wiki for tropes, containing over [NUMBER]
trope descriptions, along with the media that they appear in. For example, the
``Evil Empire'' trope appears in [EXAMPLES FROM JURISIN].

Figure~\ref{fig:tvtropes} shows a screenshot of the ``Evil Empire'' page on the
website. It clearly shows the description of the trope at the top of the page,
along with instances of its use across different media at the bottom.

A large and highly active community of users and contributors exists around TV
Tropes. In addition to creating content for and curating the content on the
website, they also work to create useful ways to visualise the usage of tropes
in stories. For example, The Periodic Table of
Storytelling~\cite{periodicTableOfStorytelling} is a visualisation of tropes as
``elements'' in the ``molecules'' of a story. The table itself
(fig.~\ref{fig:periodic-table}) arranges the tropes into different ``groups''
according to the part of a story that they operate on. [DESCRIBE THE GROUPS].
The story is then visualised as a molecule composed from tropes, linked together as
atoms (fig.~\ref{fig:trope-molecules}).

This visualisation demonstrates the core idea of our use of tropes as reusable
story components, but the ``molecule'' metaphor breaks down for a couple of
reasons. Firstly, linking tropes together as atoms in a molecule does not
communicate the different levels of abstraction at which tropes operate. The
``Hero's Journey'' trope, for example, would describe the narrative arc as a
whole, while the ``Comeuppance'' trope would describe just a single scene in the
story. The metaphor is also not ideal because it presents orthogonal concepts
together in a story with no indication of which part of a narrative they affect.
A ``scoundrel sidekick'' could be linked together with a ``breaking the fourth
wall'' trope, even though one trope relates to a certain character, and the
other may describe a single line of dialogue or action that occurs at a specific
point in the story. 

Taking this visualisation as inspiration, we develop our concept of tropes as
logical, reusable components for the formal description of stories. Importantly,
we develop a way to nest tropes within other tropes as subtropes as a way to
describe tropes acting at different levels of abstraction.

% TODO rip the whole bit from the lit review? Consider it at least
\section{Why Use Tropes?}
% Write about ability to abstract, give story examples vs Propp
% This is pretty much covered in the lit review

\section{Using Tropes with Modal Logic}
In section~\ref{sec:pjexample}, we described the ``sausages'' scene from
\emph{Punch and Judy} by combining Propp's story functions with modal operators
to create Kripke structures to visualise the paths through the scene. In order
to compare the expressiveness of tropes as a story formalism, this section
describes the same scene, but instead using tropes in place of Propp functions.

% ACTUALLY DO THIS!

\section{Describing Tropes as Institutions} % 1/2
\label{sec:tropes-as-insts}
Rather than strictly telling our story characters what to do to conform to a
story arc, we govern their behaviours with \emph{social institutions}, as
described in Section~\ref{sec:institutions}. An institution describes a set of `social' norms describing the permitted and obliged behaviour of interacting agents. Noriega's `Fish Market' thesis~\cite{noriega1999agent} describes how an institutional model can be used to regiment the actions of agents in a fish market auction.~\cite{cliffe2007specifying},~\cite{lee2013decoupling} extend this idea to build systems where institutions actively regulate the actions of agents, while still allowing them to decide what to do. Adapting this idea to the world of narrative, we use an institutional model to describe the tropes that occur within a story world.

Institutional models use concepts from deontic logic to provide obligations and permissions that act on interacting agents in an environment. By combining this approach with the idea of tropes, we can create a narrative model in terms of what agents are \emph{obliged} and \emph{permitted} to do at certain points in the story. In this way, the tropes are described as social norms which govern the character agents of a story, where an institution describes the norms that govern a certain trope, and a story is a collection of tropes.

In order to describe story tropes in terms of social norms, we break them down into three components:

\begin{compactenum}
\item characters, which instantiate roles
\item objects, which instantiate types
\item places, which instantiate locations
\end{compactenum}

Characters' actions are described in terms of permissions and obligations. For example, a character in a certain role \emph{may} go to the cinema, or a character \emph{must} buy a ticket before the movie begins, otherwise they will not see it. Note that an obligation (which says that a character \emph{must} do something) can have a deadline (``before the movie begins'') and a consequence (``they will not see the movie''). These are both optional in our system.

Returning to the tropes described in the introduction, we can express them in terms of social norms like this:

\begin{compactitem}
  \item \textbf{The Hero's Journey}: The hero \emph{must} leave home when they receive the call to adventure. Then the hero \emph{may} kill the villain. Once this is done, the hero \emph{may} return home.
  \item \textbf{The Evil Empire}: The villain has an empire, and \emph{may} kill the hero.
  \item \textbf{MacGuffin}: The hero \emph{must} search for an object. However, the hero \emph{may} find it.
  \item \textbf{Chekhov's Gun}: If a weapon appears in the beginning, it \emph{must} be used before the end of the story.
\end{compactitem}

Describing tropes in terms of permissions and obligations is enough for us to be able to specify them as social norms, but also we need to be able to determine which norms hold at any point of a story. For this, we use a \emph{Answer Set Programming} (ASP) approach to describe our tropes in order to use an answer-set solver. We do this with the aid of \emph{InstAL}~\cite{cliffe2007specifying}, the Institution Action Language, a language for describing social institutions, which compiles to AnsProlog. This allows us to use trope models and an ASP solver to determine which norms hold after agent or player actions have occurred in the story world.

In InstAL, external events trigger institutional (internal) events. External events are the actions of the character agents in their environment. For example, an agent playing the role of Luke Skywalker in a Star Wars game may pick up a Lightsaber. Since Luke Skywalker is a hero character, and a Lightsaber is a type of weapon, this would trigger an institutional event where a hero has picked up a weapon. Institutional events initiate and terminate fluents inside the institution, which may describe the institutional state, and which permissions and obligations currently hold. So when Luke Skywalker picks up a Lightsaber, the institutional event could initiate his permission to use the weapon, or an obligation to go to the land of adventure.
For more details on InstAL, social institutions, and the formalism in figures~\ref{fig:events} and~\ref{fig:term}, refer to~\citep{cliffe2007specifying}.

Figure~\ref{fig:events} lists some external ($\mathcal{E}_{external}$) and institutional (internal, $\mathcal{E}_{internal}$) events for the \emph{Hero's Journey} trope. A wide range of external events such as \emph{go}, \emph{meet}, \emph{kill}, \emph{escape} generate the \emph{intHerosJourney} internal event, but only if the external event meets certain criteria. These criteria could be whether or not an agent fulfils a certain role, for example. Figure~\ref{fig:gen} shows examples of such internal event generation ($\mathcal{G}$). In the first example (rule~\ref{eq:tatooine}), the \emph{intHerosJourney} event is generated when Luke Skywalker goes to Tatooine, but only if Luke has the role of \emph{hero}, and Tatooine's location is \emph{home}.
Figure~\ref{fig:init} shows how internal events initiate ($\mathcal{C^{\uparrow}}$) fluents and norms (permissions and obligations) in a trope. Because the \emph{Hero's Journey} trope has several stages, this example only shows the first two phases of the trope (this is explained further in the ``Sequencing'' part of the ``TropICAL: a DSL for Tropes'' section). Rule~\ref{eq:phaseb} shows how the \emph{intHerosJourney} internal event initiates the hero's permission to kill the villain, an obligation for the hero to go to the Land of Adventure before the villain kills the victim, and the next phase (phase C) of the \emph{Hero's Journey} trope. These fluents are only initiated if the \emph{intHerosJourney} internal event happens while the trope is in phase B (when \emph{phase(herosJourney, phaseB)} holds), however.
Fluent termination ($\mathcal{C^{\downarrow}}$) works in a similar manner to initiation, with permissions and obligations for previous trope phases being terminated once the next phase of a trope has been entered. Examples for the first two phases of the \emph{Hero's Journey} trope are shown in figure~\ref{fig:term}.

While InstAL allows us to express tropes as social institutions, it would be difficult to use for non-programmers who are unfamiliar with logic programming paradigms. In order for story authors to be able to create their own tropes, a much more user-friendly language is needed. This is the motivation for TropICAL, the domain specific language we describe in the next section.

\begin{figure}[!t]\small
\begin{align}
  \mathcal{E}_{external} = & \left\{\begin{array}{c}
\mathtt{go(Agent, Place)},\\
\mathtt{meet(Agent, Agent)},\\
\mathtt{kill(Agent, Agent)}\\
\mathtt{escape(Agent)}
\end{array}
\right\}\label{eq:eobs}\\
  \mathcal{E}_{internal} = &\left\{\begin{array}{l}
\mathtt{intHerosJourney(Agent,}\\
\;\;\mathtt{Agent, Agent, Place, Place)}
\end{array}\right\}
\label{eq:einst}
\end{align}
\caption{External and institutional events ($\mathcal{E}$) for the \emph{Hero's Journey} trope} \label{fig:events}
\end{figure}

\begin{figure*}[!t]
\small
%------------------------------------------------------------------------
The generation relation $\mathcal{G}$ for trope state $\mathcal{X}$ and external event $\mathcal{E}$ in the \emph{Hero's Journey} trope:\\ 
$\mathcal{G(X, E)}:\left\{\mbox{%
\begin{minipage}[c]{0.85\textwidth}
\begin{align}
\begin{array}{r}
\langle \{\mathit{role(lukeSkywalker, hero)},\\
\mathit{location(tatooine, home)}\},\\
\mathit{go}\mathtt{(lukeSkywalker, tatooine)} \rangle
\end{array}
&\rightarrow
\begin{array}{l}
\{\mathit{intHerosJourney}\mathtt{(lukeSkywalker,}\\\;\;\;\;\mathtt{R, S, tatooine, T)}\}
\end{array}\label{eq:tatooine}\\
\begin{array}{r}
                                 \langle \{\mathit{role(lukeSkywalker, hero)},\\
  \mathit{role(obiWan, dispatcher)}\},\\
\mathit{meet}\mathtt{(lukeSkywalker, obiWan)} \rangle
\end{array}
&\rightarrow
\begin{array}{l}
\{\mathit{intHerosJourney}\mathtt{(lukeSkywalker,}\\\;\;\;\;\mathtt{obiWan, R, S, T)}\}
\end{array}\label{eq:hsnth}
\end{align}
\end{minipage}
}\right.$\smallskip

%------------------------------------------------------------------------
The fluent initiation relation $\mathcal{C^{\uparrow}}$ for trope state $\mathcal{X}$ and internal event $\mathcal{E}$ in the \emph{Hero's Journey} trope:\\
$\mathcal{C^{\uparrow}(X, E)}:\left\{\mbox{%
\begin{minipage}[c]{0.85\textwidth}
\begin{align}
\begin{array}{r}
                                 \langle \{\mathit{phase(herosJourney, phaseA)}\},\\
  \mathit{intHerosJourney(hero, dispatcher, villain},\\
\mathit{home, landOfAdventure)}\} \rangle
\end{array}
&\rightarrow
\left\{\begin{array}{l}
\text{perm}(\mathtt{meet(hero, dispatcher)})\\
\text{phase}(\mathtt{herosJourney, phaseB})
\end{array}\right\}\label{eq:cfirst}\\
\begin{array}{r}
                                 \langle\{\mathit{phase(herosJourney, phaseB)},\\
  \mathit{intHerosJourney(hero, dispatcher, villain},\\ \mathit{home, landOfAdventure)}\} \rangle\\
  \end{array}
&\rightarrow \left\{
\begin{array}{l}
  \text{perm}(\mathtt{kill(hero, villain)}) \\
\text{obl}(\mathtt{go(hero, landOfAdventure),} \\
\mathtt{kill(villain, victim)},\\
\mathtt{viol(story, end)})\\
\text{phase}(\mathtt{herosJourney, phaseC})
\end{array}
\right\}\label{eq:phaseb}
\end{align}
\end{minipage}
}\right.$\smallskip

%------------------------------------------------------------------------

The fluent termination relation $\mathcal{C^{\downarrow}}$ for trope state $\mathcal{X}$ and internal event $\mathcal{E}$ in the \emph{Hero's Journey} trope:\\
$\mathcal{C^{\downarrow} (X, E)}:\left\{\mbox{%
\begin{minipage}[c]{0.85\textwidth}
\begin{align}
\begin{array}{r}
                                 \langle \{\mathit{phase(herosJourney, phaseA)}\},\\
  \mathit{intHerosJourney(hero, dispatcher, villain},\\ \mathit{home, landOfAdventure)}\} \rangle
  \end{array}
&\rightarrow \left\{
\begin{array}{l}\text{perm}(\mathtt{go(hero, home)}),\\
\text{phase}(\mathtt{herosJourney, phaseA})
\end{array}
\right\}\label{eq:crt}\\
\begin{array}{r}
                                 \langle\{\mathit{phase(herosJourney, phaseB)},\\
  \mathit{intHerosJourney(hero, dispatcher, villain},\\ \mathit{home, landOfAdventure)}\} \rangle
  \end{array}
&\rightarrow \left\{
\begin{array}{l}
\text{perm}(\mathtt{meet(hero, dispatcher)})\\
\text{phase}(\mathtt{herosJourney, phaseB})
\end{array}
\right\}\label{eq:crst}
\end{align}
\end{minipage}
}\right.$
\caption{Generation events, and fluent initation and termination for the Hero's Journey trope}
\label{fig:gen}
\label{fig:init}
\label{fig:term}
\end{figure*}%\mnote{Added \emph{active(interdiction)} to top of fig. \ref{fig:init}}

\section{Punch and Judy as Tropes}

\section{TropICAL: a DSL for Tropes} % 1
\label{sec:tropical}

We propose \tropical\ (the TROPe Interactive Chronical Language) as a DSL for describing tropes in a constrained natural language, which we compile to InstAL~\cite{cliffe2007specifying}, through which process we capture the events that can occur and the consequent state changes, and from which a model is constructed in ASP.  The model, when given an event trace, delivers the evolution of the trope state, including crucially, the addition or removal of permission associations between actors and actions and the addition of obligations as consequences of actors' actions.  The syntax of \tropical\ is heavily influenced by the Inform 7~\cite{reed2010creating} language for interactive fiction, with the tropes being expressed in constrained natural language mostly conforming to Attempto Controlled English (ACE)~\cite{fuchs1996attempto}. \tropical\ shares similar aims to Inform~7 in that it aims to make interactive fiction authoring accessible to non-programmer story authors, however its focus is on authoring and combining tropes written in terms of roles, in contrast to complete stories in terms of actual characters.\jnote{revised a bit} This section describes the syntax and semantics of \tropical, as well as sketching its compilation to InstAL.

% In order for a trope description language to be useful, it must have the following features:\jnote{origin? requirements from nowhere}
The features of our trope description language are designed to be able to express the events, permissions and obligations of social institutions while addressing the shortcomings of planner and drama manager-based approaches:\mnote{Explained where features are from, more justification after enum}
\begin{compactenum}[R1.]
\item\label{perms} A way to express what certain characters are \textbf{permitted} to do at a given time.\footnote{An alternative approach would be to specify prohibitions, such that anything not prohibited is permitted, whereas we currently specify permissions, such that anything not permitted is prohibited.  This latter convention is the default semantics of InstAL, it is however straightforward from a technical point of view to adopt the alternative, as demonstrated in \cite{DBLP:conf/atal/KingLVDJPR15}.}
\item\label{obls} A way to express \textbf{obligations} on characters, with \emph{deadlines} and \emph{penalties} if the obligations are not fulfilled.
\item\label{seqs} A way of \textbf{sequencing} events in a trope.
\item\label{conds} A way to express \textbf{conditionals}, so that some events may occur only if others have.
\item\label{branches} A way to have \textbf{branches} in a trope, so that only one of two events may occur.
\item\label{embed} A way to \textbf{embed} sub-tropes inside of parent tropes.
\end{compactenum} % \jnote{no mention of power.  Can undertand if intentional omission}\mnote{yes, was intentional for space}
Requirements~R\ref{perms} to~R\ref{seqs} allow \tropical\ to describe the permissions, obligations and sequences of events that occur in social institutions, while R\ref{conds} and~R\ref{branches} add the ability to specify alternative paths through a story, as planner-based systems are able to do when combined with formalisms such as Propp's. Finally, requirement~R\ref{embed} enables us to nest tropes to go beyond the capabilities of structuralist formalisms of narrative, addressing the limitations described in the ``Related Work'' section of this paper.\mnote{Added this justification, but needs checking.}\jnote{looks better}  The TropICAL language satisfies these technical requirements, while being easy to learn for non-programmers,\jnote{can we claim that is supported by the evaluation?} especially those familiar with the Inform 7 language (as is supported by the evaluation). It supports the expression of the above features in the following ways: 

% Replace with ``hostage situation''?
\begin{figure}[!t]
% \begin{lstlisting}[float=t!,caption={The ``Hero's Journey'' trope in TropICAL},label=lst:hero]
\begin{lstlisting}
"The Hero's Journey" is a Trope where:
  The Hero is at Home
  Then the Hero meets the Dispatcher
  Then the "Quest" trope happens
  Then the Hero returns Home
\end{lstlisting}%
\caption{The ``Hero's Journey'' trope in TropICAL\label{lst:hero}}
\medskip
% \begin{lstlisting}[float=t!,caption={The ``Evil Empire'' trope in TropICAL},label=lst:evil]
\begin{lstlisting}
"The Evil Empire" is a Trope where:
  The Villain has an Empire
  The Empire is a Weapon
  The Villain has a Victim
  The Villain may kill the Victim
  The Villain may kill the Hero
\end{lstlisting}%
\caption{The ``Evil Empire'' trope in TropICAL\label{lst:evil}}
\medskip
% \begin{lstlisting}[float=t!,caption={The ``Quest'' trope in TropICAL},label=lst:quest]
\begin{lstlisting}
"Quest" is a Trope where:
  Then the Hero must go to the Land Of Adventure before the Villain kills the Victim
    Otherwise, the Story ends
  When the Hero goes to the Land Of Adventure:
    The Hero may rescue the Victim
    The Hero may kill the Villain
      Or the Villain may escape
\end{lstlisting}
\caption{The ``Quest'' trope in TropICAL\label{lst:quest}}
\end{figure}

\begin{compactdesc}
% \subsection{Permissions}
\item[Permissions:]
Permissions on characters can be described by making statements in the simple present form, such as ``The Hero finds a weapon''. When compiled to InstAL code, these statements are equivalent to giving a character permission to do something. In this case, the Hero would have permission to find a weapon at that point in the trope.
The reason that this statement is translated to a permission is so that character agents can at any time have multiple permitted actions from several active tropes. It makes sense to make permission the ``default'' norm, rather than obligation, to allow the agents as much freedom as possible within the constraints of the story. If an author wants to make sure an agent carries out a particular action, they would specify it as an obligation instead (``The Hero \emph{must} find a weapon'').

% \subsection{Obligations}
\item[Obligations:]
Fig.~\ref{lst:quest} shows an example of an obligation with both a deadline and a violation event (both of which are optional). This obligation states that the hero must go to the Land of Adventure before the villain kills the victim, otherwise the story ends. In this case, the story ending is a particularly harsh penalty for the violation of the \emph{Quest} trope. Alternative violation events could be reduction of the hero's health, or the death of the victim.

% \subsection{Sequencing}
\item[Sequencing:]
As well as specifying permissions and obligations, it is frequently necessary to be able to express that certain events can only occur in a certain order. The \emph{Then} keyword means that the succeeding statement can only occur once the previous event has occurred. This is the means of implementing the \emph{phases} described in the ``Ordering Events in Tropes'' part of this section. In the \emph{Hero's Journey} (Fig.~\ref{lst:hero}) example, the Hero only has permission to return home (line 5) once everything in the \emph{Quest} trope has happened (Fig.~\ref{lst:quest}).
In some tropes, such as the \emph{Evil Empire} trope (Fig.~\ref{lst:evil}), the permissions and obligations described do not always need to occur in a certain sequence. In this case the trope serves the purpose of describing certain themes and characters in a part of a story, or events that may occur at any time, rather than in a specific order. The \emph{Evil Empire} trope only needs a villain with an empire to be present to fight the hero, so all of its permissions and obligations will apply from the beginning of the story.\jnote{beginning of which trope?}\mnote{Changed to story. Should I explain that tropes will be grouped into scenes?}\jnote{tricky, there's already enough going on; if we can manage without, might be safer}

% \subsection{Branching}
\item[Branching:]
Tropes may also contain branching paths where one or more events may take place. This is expressed in TropICAL with the \emph{Or} operator. Lines 6 and 7 of the \emph{Quest} trope in Fig.~\ref{lst:quest} express two alternatives: the Hero may kill the Villain or the Villain may escape. In this case, both permissions will hold at the same time, but both will be terminated once either permitted event has occurred. This makes it impossible for both events to happen in the story.

% \subsection{Conditionals}
\item[Conditionals:]
Conditionals are another way to create branching paths in a narrative, by allowing certain actions to occur if a particular trope state is reached.\jnote{not an adequate explanation, besides when is controlled by trope state, not events}\mnote{Better?}\jnote{yes, although there is the technical issue that noninertials cannot initate inertials and I don't know how this is implemented} For this purpose we add the \emph{When} keyword, to express that when a specified state or event occurs, then some norms will be activated. In our \emph{Quest} example, we see an example of this on line 4, stating what the hero may do once in the Land of Adventure. This is similar to the ``Simple Present Statement'' example, except for the addition of the \emph{When} keyword. In much the same way, the statement after the \emph{When} keyword is a permission that holds on a character during the trope, except that it is also used to describe the consequences of certain events occurring. In this case, it states that when the Hero goes to\jnote{enters or is?  I think it's a fact not an action}\mnote{I've changed the code snippet to an action}\jnote{as long as this is implemented via an institutional event (initiating the perm) rather than using InstAL's when, it should be fine} the Land of Adventure, they may either kill the Villain, or the Villain may escape.

% \subsection{Embedding Tropes Within Tropes}
\item[Embedding Tropes Within Tropes:]
Tropes can be embedded inside other tropes by simply writing \emph{The X trope happens}.
Line 4 of the \emph{Hero's Journey} trope (Fig.~\ref{lst:hero}) shows an example of embedding one trope inside another. In this case, the \emph{Quest} trope occurs at a certain point in the \emph{Hero's Journey}, once the hero has met the dispatcher character. Because this is sequenced using the \emph{Then} keyword, these events must occur in the specified order, and the norms described in the \emph{Quest} trope cannot hold until the specified point in the trope. However, if the \emph{Then} keyword is omitted (\emph{The ``Quest'' trope happens}), this means that the norms contained inside the embedded trope apply from the start of its containing trope. In our example, this would mean that the Hero would be free to embark on a quest at any time, rather than waiting to first meet a dispatcher character.
% Rather than compiling all the tropes into one institution, they are compiled into separate institutions and coordinated using a ``bridge'' institution. This technique, described by~\citep{bath45254}, allows institutions to generate events inside of other institutions while keeping each one separate. The cross-institution event generation logic is described in the bridge institution.\jnote{are you actually using bridges now?  if so, need to revise the future work text to be consistent}
\end{compactdesc}

\subsection{Building a Text Adventure with Tropes} % 1/2
\label{sec:tropes-text}

\begin{figure}[!t]
%% \begin{align}
%%   \texttt{You are Luke Skywalker}\qquad
%%   &\texttt{Luke Skywalker is the Hero}\nonumber\\
%%   \texttt{Tatooine is Home}\qquad
%%   &\texttt{Obi Wan is the Dispatcher}\nonumber\\
%%   \texttt{Space is The Land of Adventure}\qquad
%%   &\texttt{Darth Vader is the Villain}\nonumber\\
%%   \texttt{The Imperial Forces are the Empire}\qquad
%%   &\texttt{Princess Leia is the Victim}\nonumber\\
%% \end{align}
%% \begin{align}
%%   &\texttt{> meet obi wan} \nonumber\\
%% &&\text{Luke Skywalker must go to space before Darth Vader kills Princess Leia}\nonumber\\
%%   &&\text{Darth Vader may kill Luke Skywalker}\nonumber\\
%%   &&\text{Darth Vader may kill Princess Leia}\label{eq:text2}\\
%%   &\texttt{> go to space} \nonumber\\
%% &&\text{Luke Skywalker may rescue Princess Leia}\nonumber\\
%% &&\text{Luke Skywalker may kill Darth Vader}\nonumber\\
%%   &&\text{Darth Vader may kill Luke Skywalker} \label{eq:text3}\\
%%   &\texttt{> rescue princess leia} \nonumber\\
%% &&\text{Luke Skywalker may kill Darth Vader}\nonumber\\
%%   &&\text{Darth Vader may kill Luke Skywalker} \label{eq:text4}\\
%%   &\texttt{> kill darth vader} \nonumber\\
%% &&\text{Luke Skywalker may go to Tatooine} \label{eq:text5}
%% \end{align}
\fbox{
\begin{minipage}{0.95\columnwidth}
\begin{flushleft}\small
You are Luke Skywalker\\
Luke Skywalker is the Hero\\
Tatooine is Home\\
Obi Wan is the Dispatcher\\
Space is The Land of Adventure\\
Darth Vader is the Villain\\
The Imperial Forces are the Empire\\
Princess Leia is the Victim
\end{flushleft}

\begin{flushleft}\tt\small
> meet obi wan\\\smallskip
\hfill\begin{minipage}{0.95\columnwidth}\flushright\rm\noindent
Luke Skywalker must go to space before Darth Vader kills Princess Leia\\
Darth Vader may kill Luke Skywalker\\
Darth Vader may kill Princess Leia
\end{minipage}\\
> go to space\\\smallskip
\hfill\begin{minipage}{\columnwidth}\flushright\rm
Luke Skywalker may rescue Princess Leia\\
Luke Skywalker may kill Darth Vader\\
Darth Vader may kill Luke Skywalker
\end{minipage}
> rescue princess leia\\\smallskip
\hfill\begin{minipage}{\columnwidth}\flushright\rm
Luke Skywalker may kill Darth Vader\\
Darth Vader may kill Luke Skywalker
\end{minipage}
> kill darth vader\\\smallskip
\hfill\begin{minipage}{\columnwidth}\flushright\rm
Luke Skywalker may go to Tatooine
\end{minipage}
\end{flushleft}
\end{minipage}
}
\caption{Text adventure interface example for a story built with TropICAL and StoryBuilder; user input in {\tt typewriter} and trope state in Roman} \label{fig:text}
\end{figure}

In practice, an interactive story author writes their story tropes using the TropICAL language. The TropICAL code is compiled into InstAL institutions, then to AnsProlog code to be used in an ASP solver. When a player or agent action occurs, we run one cycle of the ASP solver\footnote{We use clingo, available from \url{http://potassco.sourceforge.net}, accessed 20160608.}, with that action to determine the permissions and obligations that hold on each character at the next point in the story.\jnote{incorrect}\mnote{better?}\jnote{yes, revised further}
TropICAL is designed to be used inside \textbf{StoryBuilder}, a browser-based user interface for interactive story creation. StoryBuilder allows authors to write their own tropes, select pre-written tropes and combine these tropes into a story timeline. The interface provides drop-down boxes to select characters, objects and places to fulfil abstract roles, object types and locations within the selected tropes. Once completed, the InstAL institutions are generated and the game can be played, with player and agent actions passed to the answer-set solver as they occur.

Figure~\ref{fig:text} shows how a user interacts with a text adventure game\footnote{This simplistic interface is used purely for the purpose of compact illustration of the interaction in this paper.} built using TropICAL and StoryBuilder. At the start of the game, the user may choose which character they would like to be (in this case, Luke Skywalker). Once a user performs an action by typing it into the text parser (as shown on the left side of the figure), this is sent as an event to the ASP solver, which updates the norms that apply to both the player and the character agents. The agents are then free to carry out their plans with these new norms in mind. The player can attempt to perform any action, but non-permitted actions result in a a message such as ``You cannot do that''.

% In the text adventure, the tropes are used as a means of governing the character agents in the story. The agents themselves must still have goals and dialogue to fit their character roles. For this reason, figure~\ref{fig:text} only lists the permissions and obligations that are governing the agents at each step.\jnote{do not follow}

\subsection{Making Tropes Abstractable}

% Explain use of bridge institutions to nest tropes
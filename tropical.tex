% HACK: predicted poms remaining: 75

\chapter{TropICAL: A DSL for Story Tropes}
\label{cha:tropical}
% DONE introduction: 1
% DONE motivation: 1

This chapter describes the design and implementation of TropICAL (the TROPe
Interactive Chronical Action Language), our Domain Specific Language for
describing story tropes. The purpose of this language is to allow for the
creation of interactive narratives through the description of tropes in a
constrained natural language. The tropes created through the language are
designed to be reusable components that can go into a ``library'', from which
story authors can choose the tropes best suited to the particular story that they
are creating.

The motivation for the creation of TropICAL is the lack of any methods of
creating interactive narrative that are suitable for non-programmers to use. The
systems described in the literature review in
chapter~\ref{cha:literature-review}, such as any using the Mimesis
architecture~\cite{young2004architecture}, a drama manager and planner such as
in Fa\c{c}ade's system~\cite{mateas2003faccade}, or linear logic as in Ceptre's
system~\ref{martens2015ceptre}, all require the story author to be familiar with
planner-based systems or the description of formal logics. The purpose of
TropICAL is to greatly reduce the barrier to the creation of interactive stories
by allowing authors to describe the components of their story using constrained
natural language. In fact, many authors using our system may not even need to
write their own tropes using TropICAL, and will instead simply select the tropes
that they need for their story from a pre-created ``library'' of tropes. This
process is facilitated through the ``StoryBuilder'' user interface described in
chapter~\ref{cha:storybuilder}

This section describes TropICAL's constrained natural
language syntax in section~\ref{sec:t-constrained}, the features of the language in section~\ref{sec:t-features}, 
the formal grammar used to
describe its syntax in section~\ref{sec:t-grammar}, how ``nested
tropes'' are created in section~\ref{sec:t-nested}, the intermediate data structure used between parsing and
code generation steps (section~\ref{sec:t-intermediate}), its compilation
to InstAL code (section~\ref{sec:t-codegen}), some samples of generated InstAL
institutions (section ~\ref{sec:t-tropes}), answer set
generation (section~\ref{sec:t-asp}), and finally its extension for the description of legal policies
(section~\ref{sec:t-legal}).


\section{Controlled Natural Language Syntax}
\label{sec:t-constrained}
% DONE intro: 1
% TODO describe ACE: 2
% TODO describe Inform 7: 1

TropICAL uses a \emph{Controlled Natural Language} (also referred to as
\emph{Constrained Natural Language}) syntax, which means that it superficially
resembles natural language English, but is only a subset.

There are two types of Controlled Natural Language (CNL). The first type is \emph{naturalistic} languages
such as ASD Simplified Technical English~\cite{asd2007simplified}, designed to
be used in documentation for technical industries such as aerospace or defense,
or Ogden Basic English~\cite{ogden1944basic}, a simplified language for teaching
English as a second language. This type of Controlled Natural Language merely
describe a subset of the parent language. The other type of CNL is
\emph{formalistic}, with a formal
syntax and semantics, which can be mapped to rules in other formal languages
such as first-order logic. Attempto Controlled English~\cite{fuchs1996attempto}
is an example of this formal type of CNL, and the one which forms the basis of
our syntax for TropICAL.

\subsection{Attempto Controlled English}

Attempto Controlled English (ACE) is a controlled natural language that is also
a formal language, and was created by the University of Zurich in 1995. It is
still under development, with version 6.7 of the language announced in 2013. ACE
has been used in a wide variety of fields, such as software specifications


\section{Features}
\label{sec:t-features}
% TODO intro: 1
% TODO list features: 1
% TODO tech used: 1
% TODO relate features to the requirements: 2


\section{EBNF Grammar}
\label{sec:t-grammar}
% TODO intro: 1
% TODO instaparse description: 1
% TODO simplified grammar from code: 2
% TODO format grammar: 2

\section{Nested Tropes via Bridge Institution}
\label{sec:t-nested}
% TODO why nested tropes: 1
% TODO explanation of bridge institutions: 2

\section{Intermediate Data Structure}
\label{sec:t-intermediate}
% TODO why do we have an intermediate data structure? (possible other uses,
% different parsers): 1
% TODO describe the hash map: 2

\section{InstAL Code Generation}
\label{sec:t-codegen}
% TODO intro: 1
% TODO tech discussion: 1

\subsection{Initiation}
% TODO before / after snippets: 2
% TODO explanation: 2

\subsection{Termination}
% TODO before / after snippets: 2
% TODO explanation: 2

\subsection{Generation}
% TODO before / after snippets: 2
% TODO explanation: 2

\subsection{Initial Conditions}
% TODO before / after snippets: 2
% TODO explanation: 2

\subsection{Bridge Institution}
% TODO before / after snippets: 2
% TODO explanation: 2

\section{Generated Trope Samples}
\label{sec:t-tropes}
% TODO intro for heros journey trope: 1
% TODO run through generated code: 3

% TODO intro for evil empire trope: 1
% TODO run through generated code: 3

\section{Adding Constraints}
\label{sec:t-constraints}
% TODO ASP intro: 1
% TODO constraints explanation: 2
% TODO output: 2

\section{Answer Set Generation}
\label{sec:t-asp}
% TODO instalquery intro & explanation: 2
% TODO generate for hero's journey: 4
% TODO generate for evil empire: 4
% TODO generate for both: 4

\section{Extending for Legal Policies}
\label{sec:t-legal}
% TODO intro: 1
% TODO explanation from JURIX: 2

\subsection{Example Policies}
% TODO examples from JURIX: 2

% TODO outro: 1